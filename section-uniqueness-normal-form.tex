In this section we prove a weak form of confluence: normal form for
all closed terms of $\CTlambda$ of type $\N$ is unique.
By the weak normalization result proved in Section~\ref{section-weak-normalization},
we will deduce: for all closed terms $t$ of $\CTlambda$ of type $\N$,
there is $n \in \N$ such that $t \safeReducesAst n$
and all normal form of $t$ are equal to $n$. 

We define binary relations $\sim_A$ (for each type $A$) on well-typed terms with type $A$.
\begin{definition}
  We define binary relations $\sim_A$ for \emph{closed} terms of $\CTlambda$
  by induction on the type $A$. 
  \begin{enumerate}
  \item
    $t_1 \sim_N t_2$ if and only if for all normal forms $t_1',t_2':\N$, if $t_1 \safeReducesAst t_1'$
    and $t_2 \safeReducesAst t_2'$ then $t_1' = t_2'$.
  \item
    $t_1 \sim_{A\rightarrow B} t_2$ if and only if for all closed $a_1,a_2:A$ if
    $a_1 \sim_A a_2$ then $t_1(a_1) \sim_B t_2(a_2)$.
  \end{enumerate}
  We write $a_1,\ldots,a_n \sim_{A_1,\ldots,A_n} a'_1,\ldots,a'_n$ if $a_i\sim_{A_i} a'_i$ holds
  for all $i\in\{1,\ldots,n\}$. 
  
  We also define $t_1[\vec{x}] \sim_A t_2[\vec{x}]$ for \emph{any} terms of $\CTlambda$,
  with $\vec{x}:\vec{A}\vdash t_i[\vec{x}]:B$, where $i\in\{1,2\}$, as follows. 
  \begin{enumerate}
  \item[3.]
    $t_1[\vec{x}] \sim_A t_2[\vec{x}]$ if and only if
    for all closed terms $\vec{a_1}:\vec{A}$ and $\vec{a_2}:\vec{A}$,
    if $\vec{a_1}\sim_{\vec{A}}\vec{a_2}$ then $t_1[\vec{a_1}] \sim_B t_2[\vec{a_2}]$. 
  \end{enumerate}  
  
\end{definition}

It is easily shown that $\sim_A$ is symmetric by the definition. 
We will prove that, for all closed terms $t$ of $\CTlambda$ with type $A$, we have $t \sim_A t$.
If $A=\N$ this means that the normal form of all closed terms of $\CTlambda$ of type $\N$ is unique,
which is our goal.

\begin{definition}[Confluent and U-counterexample]
  Let $t$ be a term of $\CTlambda$. 
  \begin{enumerate}
  \item
    Assume that $t$ is a closed term with type $N$. 
    We say that $n$ is a {\em value} of $t$ if $t \safeReducesAst n$ ($t$ safely reduces to $n$). 
  \item
    We say that $t$ is {\em confluent} if $t \sim_A t$.
  \item
    Assume $\vec{x}:\vec{D}\vdash t:\vec{A}\rightarrow\N$. 
    We say a quadraple $(\vec{d},\vec{a},\vec{d'},\vec{a'})$ is a {\em U-counterexample} to $t$
    if and only if
    both $(\vec{d},\vec{a})$ and $(\vec{d'},\vec{a'})$ are value assignments for $t$, 
    $\vec{d}\sim_{\vec{D}}\vec{d'}$, $\vec{a}\sim_{\vec{A}}\vec{a'}$, and
    $t[\vec{d}](\vec{a}) \not\sim_N t[\vec{d'}](\vec{a'})$. 
  \end{enumerate}
\end{definition}

By the weak normalization result proved in Section~\ref{section-weak-normalization}, all 
closed term $t$ of $\CTlambda$ of type $\N$ have a value. If $t$ is confluent, then the value is unique.
Hence we have the following. 

\begin{lemma}
  If $t:\N$ is a confluent closed term, then $t$ has a unique normal form.
\end{lemma}

By using the weak normalization result, 
two closed terms $t$ and $u$ equivalent with respect to $\sim_\N$, namely $t \sim_\N u$,
have a common numeral as their unique normal form. 

\begin{lemma}\label{lem:uniq_NF}
  Let $t_1$ and $t_2$ be closed terms in $\CTlambda$. 
  Assume $t_1 \sim_\N t_2$. 
  \begin{enumerate}
  \item\label{lem:uniq_NF1}
    There exists $n \in \Num$ such that $t_1 \safeReducesAst n$ and $t_2 \safeReducesAst n$.
    Moreover, for each $i\in \{1,2\}$, if $t_i \safeReducesAst u$ and $u$ is a normal form, then $u = n$. 
  \item\label{lem:uniq_NF2}
    If $t_1 \safeReduces t'_1$ and $t_2 \safeReduces t'_2$, then $t'_1 \sim_\N t'_2$. 
  \end{enumerate}
  
\end{lemma}
\begin{proof}
  We show point~\ref{lem:uniq_NF1}. 
  By Corollary~\ref{cor:WN_typeN}, there are $n_1,n_2 \in \Num$ such that
  $t_1 \safeReducesAst n_1$ and $t_2\safeReducesAst n_2$.
  Then $n_1=n_2$ holds by $t_1\sim_\N t_2$. 
  If $t_i \safeReducesAst u$ and $u$ is a normal form, then $u = n_2$ by $t_1\sim_\N t_2$.

  Next we show point~\ref{lem:uniq_NF2}.
  It is enough to show the statement: $t_1\sim_\N t_2$ and $t_1 \safeReduces t'_1$ implies $t'_1 \sim_\N t_2$,
  because we can prove point~\ref{lem:uniq_NF2} by using the statement twice with the symmetricity of $\sim_\N$. 
  For showing the statement, assume that $t_1\sim_\N t_2$ and $t_1 \safeReduces t'_1$. Our goal is $t'_1 \sim_\N t_2$. 
  To show it, take arbitrary normal forms $v'_1$ and $v_2$ such that $t'_1 \safeReducesAst v'_1$ and $t_2 \safeReducesAst v_2$.
  By point~\ref{lem:uniq_NF1}, there is $n\in\Num$ for $t_1$ and $t_2$ that satisfies the properties as described.
  Hence we have $v'_1 = n = v_2$ by using the property with $t_1 \safeReduces t'_1 \safeReducesAst v'_1$ and $t_2 \safeReducesAst v_2$. 
  Therefore $t'_1 \sim_\N t_2$ is obtained. 
\end{proof}

%For confluent closed term $t$ of $\CTlambda$ of type $\N$, "safe" reduction preserves the value. Indeed,
%if $t \reduces u$ then $u$ is a closed term of $\CTlambda$ of type $\N$, therefore $u$ has a value
%$m$, which is also a value of $t$. By confluence of $t$ we conclude that $n=m$.

%If $t$ is as above, with value $n$, and $t \reduces \Succ (u)$, then $n>0$, 
%$u$ is a confluent closed term $t$ of $\CTlambda$ of type $\N$, and the value of $u$ is $n-1$.
%Indeed,if $u \reduces u'$ and $u'$ is normal, then $t \reduces \Succ (u')$, therefore $\Succ (u')=n$ 
%by confluence of $t$, and we conclude that $n>0$ and $u'=n-1$. Thus, the normal form of $u$ is unique.

We have $t \sim_A t$ if and only if there is no U-counterexample for $t$.
This is shown by the following lemma.
\begin{lemma}\label{lem:uniq_value}
  Let $t$ be a term of $\vec{x}:\vec{D} \vdash t[\vec{x}]:\vec{A}\rightarrow\N$. 
  Let $\vec{d},\vec{a},\vec{d'},\vec{a'}$ be vectors of closed terms.
  Assume that
  $\vec{d}\sim_{\vec{D}}\vec{d'}$ and $\vec{a}\sim_{\vec{A}}\vec{a'}$ implies $t[\vec{d}](\vec{a}) \sim_N t[\vec{d'}](\vec{a'})$. 
  Then $t[\vec{x}]$ is confluent. 
\end{lemma}
\begin{proof}
  Take any $\vec{d},\vec{a},\vec{d'},\vec{a'}$ such that $\vec{d}\sim_{\vec{D}}\vec{d'}$ and $\vec{a}\sim_{\vec{A}}\vec{a'}$.
  Then $t[\vec{d}](\vec{a}) \sim_N t[\vec{d'}](\vec{a'})$ holds by the assumption.
  We have $t[\vec{d}] \sim_{\vec{A}\rightarrow\N} t[\vec{d'}]$ by the definition of $\sim_{\vec{A}\rightarrow\N}$.
  Hence $t[\vec{x}] \sim_{\vec{A}\rightarrow\N} t[\vec{x}]$ holds by the definition. 
\end{proof}

Taking contraposition of this lemma, there exists a U-counterexample for non-confluent $t$. 
Using this, we prove the following theorem.

\begin{theorem}
  Assume $\Pi:\Gamma\vdash t:A$ (hence $t \in \WTyped$). 
  If $t$ is not confluent,
  then $t \not\in \GTC$, that is, $\Pi$ does not satisfy the global trace condition. 
\end{theorem}
\begin{proof}
  Assume that $t$ is not confluent. 
  Let $\Gamma = \vec{x}:\vec{D}$ and $A = \vec{A}\rightarrow\N$. 
  By Proposition \ref{prop:trace_assign}.\ref{prop:trace_assign2} it is enough to prove that
  there are an infinite path $\pi=(e_1, e_2, \ldots)$ of $\Pi$
  and a trace-compatible assignment $\rho$ for $\pi$.
  
  By induction on $i \in \N$, for each $i$, we construct a path $l_i = (e_1,\ldots,e_{i-1})$
  and value assignments $\vec{v_i} = (\vec{d_i},\vec{a_i})$ and $\vec{v'_i} = (\vec{d'_i},\vec{a'_i})$
  for $t_i = t \restr l_i$, which is a subterm of $t$, such that 
  both $(\vec{v_1},\ldots,\vec{v_i})$ and $(\vec{v'_1},\ldots,\vec{v'_i})$ are
  trace-compatible assignments for the node $(t_1, \ldots, t_i)$.
  We have to find $(l_i,\vec{d_i},\vec{a_i},\vec{d'_i},\vec{a'_i})$ such that:
  \begin{enumerate}
  \item[(i)]
    For each $i\ge 1$, $l_i$ is a node of $\Pi$ and
    $\Label(\Pi,l_i) = \vec{x_i}:\vec{D_i}\vdash t_i:\vec{A_i}\rightarrow\N$. 
  \item[(ii)]
    $t_i$ is not confluent whose U-counterexample is $(\vec{d_i},\vec{a_i},\vec{d'_i},\vec{a'_i})$. 
  \item[(iii)]
    $((\vec{d_1},\vec{a_1}),\ldots,(\vec{d_i},\vec{a_i}))$ and
    $((\vec{d'_1},\vec{a'_1}),\ldots,(\vec{d'_i},\vec{a'_i}))$ are trace-compatible assignments for
    $(t_1,\ldots,t_i)$. 
  \end{enumerate}

  For the case $i=1$,
  by Lemma~\ref{lem:uniq_value},
  there exist closed values $\vec{d_1},\vec{d'_1}:\vec{D}$ and $\vec{a_1},\vec{a'_1}:\vec{A}$ such that
  $\vec{d_1}\sim_{\vec{D}}\vec{d'_1}$, $\vec{a_1}\sim_{\vec{A}}\vec{a'_1}$ and 
  $t[\vec{d_1}]\vec{a_1} \not\sim_{\N} t[\vec{d'_1}]\vec{a'_1}$. 
  Then we take $((),\vec{d_1},\vec{a_1},\vec{d'_1},\vec{a'_1})$ for this case.
  Points (i), (ii), and (iii) are immediately shown. 

  Next, assume that $(l_i,\vec{d_i},\vec{a_i},\vec{d'_i},\vec{a'_i})$ is already constructed. 
  Then we define $(l_{i+1},\vec{d_{i+1}},\vec{a_{i+1}},\vec{d'_{i+1}},\vec{a'_{i+1}})$ by the case analysis
  on the last rule for the node $l_i$ in $\Pi$. 

  \begin{enumerate}
  \item
    The case of $\weak$, namely
    $\Label(\Pi,l_i) = \Gamma'\vdash t:\vec{A_i}\rightarrow\N$
    is obtained from $\Gamma\vdash t:\vec{A_i}\rightarrow\N$, where
    $\Gamma = x_1:D_1,\ldots,x_n:D_n$, $\Gamma' = x'_1:D'_1,\ldots,x'_m:D'_m$, and $\Gamma \subseteqsim \Gamma'$
    with an injection $\phi:\{1,\ldots,n\}\to\{1,\ldots,m\}$. 
    We have $x_i = x'_{\phi(i)}$ and $D_i = D'_{\phi(i)}$ for all $i \in \{1,\ldots,n\}$.
    By the induction hypothesis and (ii), we have 
    $t[d_{i,\phi(1)}/x_1,\ldots,d_{i,\phi(n)}/x_n]\vec{a_i} = t[\vec{d_i}/\vec{x'}]\vec{a_i} \not\sim_\N t[\vec{d'_i}/\vec{x'}]\vec{a'_i} = t[d'_{i,\phi(1)}/x_1,\ldots,d'_{i,\phi(n)}/x_n]\vec{a'_i}$, 
    where $\vec{d_i} = d_{i,1}\ldots d_{i,m}$ and $\vec{d'_i} = d'_{i,1}\ldots d'_{i,m}$.
    Then we define $l_{i+1} = l_i \conc (1)$ taking the unique child node of $l_i$ in $\Pi$, and we
    define
    $\vec{d_{i+1}} = (d_{i,\phi(1)}\ldots d_{i,\phi(n)})$, 
    $\vec{d'_{i+1}} = (d'_{i,\phi(1)}\ldots d'_{i,\phi(n)})$, 
    $\vec{a_{i+1}} = \vec{a_i}$, and $\vec{a'_{i+1}} = \vec{a'_i}$. 
    We obtain (i), (ii), and (iii) for $i+1$, as expected.
    We also have (iv) since $((\vec{d_i},\vec{a_i}),(\vec{d_{i+1}},\vec{a_{i+1}}))$
    and $((\vec{d'_i},\vec{a'_i}),(\vec{d'_{i+1}},\vec{a'_{i+1}}))$
    are trace compatible for $(t_i,t_{i+1})$. 

  \item
    The case of $\var$-rule, namely $\Label(\Pi,l_i) = \Gamma\vdash x:D$ for some $x:D \in \Gamma$
    cannot be, because $t_i = x$ is confluent. 

  \item
    The case of $0$-rule, namely $\Label(\Pi,l_i) = \Gamma\vdash 0:\N$, cannot be.
    Indeed, $t_i = 0$ is confluent. 

  \item  
    The case of $\Succ$-rule, namely $\Label(\Pi,l_i) = 
    \Gamma\vdash t_i: \N = \Gamma\vdash \Succ(u): \N$ for some $u$ 
    is obtained from our assumptions on
    $\Gamma\vdash u: \N$. In this case $\vec{a_i}$ is empty, $t_{i+1}=u$, and
    by the induction hypothesis $\Succ(u)[\vec{d_i}]:\N$ is not confluent. 
    Then, by the definition of confluent, $u[\vec{d_i}] =t_{i+1}[\vec{d_i}] $ is not confluent. 
    We define $l_{i+1} = l_i \conc (1)$ taking the unique child of $l_i$ in $\Pi$. 
    We also define $\vec{d_{i+1}} = \vec{d_i}$ and $\vec{a_{i+1}} = ()$.  
    We obtain (i), (ii), (iii), and (iv) for $i+1$, as expected.

  \item
    The case of the $\apnotvar$-rule, namely 
    $\Label(\Pi,l_{i+1}) = \Gamma\vdash f[\vec{x}](u[\vec{x}]): \vec{A}\rightarrow\N$,
    for some $f$ and $u$, 
    is obtained from $\Gamma\vdash f[\vec{x}]: B \rightarrow \vec{A}\rightarrow\N$ 
    and $\Gamma\vdash u[\vec{x}]: B$, where $u$ is not a variable.
    By the induction hypothesis,
    $f[\vec{d_i}](u[\vec{d_i}])\vec{a_i} \not\sim_\N f[\vec{d'_i}](u[\vec{d'_i}])\vec{a'_i}$
    We argue by case on the statement: \emph{$u[\vec{d_i}] \sim_B u[\vec{d'_i}]$}.

    \begin{enumerate}
    \item
      We first consider the subcase that \emph{$u[\vec{d_i}] \sim_B u[\vec{d'_i}]$}.
      We define $b,b':B$ by $b = b' = n\in\Num$ such that $u[\vec{d_i}] \safeReducesAst n$ if $B=\N$,
      by $b = u[\vec{d_i}]$ and $b' = u[\vec{d'_i}]$ otherwise.
      By lemma \ref{lem:total_value}.\ref{lem:total_value1}, $b$ and $b'$ are values.
      We also have $b \sim_B b'$ by the assumption of this case.
      Then define $l_{i+1}=l_i\conc(1)$, the index of $f[\vec{x}]$,
      and define $\vec{d_{i+1}} = \vec{d_i}$, $\vec{d'_{i+1}} = \vec{d'_i}$,
      $\vec{a_{i+1}} = b,\vec{a_i}$, and $\vec{a'_{i+1}} = b',\vec{a'_i}$.
      Then, by the induction hypothesis, $\vec{d_{i+1}} \sim_{\vec{D}} \vec{d'_{i+1}}$ and
      $\vec{a_{i+1}} \sim_{B,\vec{A}} \vec{a'_{i+1}}$. 
      Using Lemma \ref{lem:total_value}.\ref{lem:total_value1} and \ref{lem:total_value}.\ref{lem:total_value2}
      we obtain (i), (ii), (iii) for $i+1$, as expected. 
      We also have (iv) since the connections 
      from $(\vec{d_i},\vec{a_i})$ to $(\vec{d_{i+1}},\vec{a_{i+1}}) = (\vec{d_i},b,\vec{a_i})$
      and
      from $(\vec{d'_i},\vec{a'_i})$ to $(\vec{d'_{i+1}},\vec{a'_{i+1}}) = (\vec{d'_i},b,\vec{a'_i})$
      are trace-compatible: all connected 
      $\N$-argument of $t_i=f(u)[\vec{x}]$ and $t_{i+1}[\vec{x}] = f[\vec{x}]$ are the same,
      because the only fresh arguments of $f[\vec{x}]$ 
      are $b$ and $b'$, and no argument of $f(u)[\vec{x}]$ is connected to them.
    \item
      Next we consider the subcase that \emph{$u[\vec{d_i}] \not\sim_B u[\vec{d'_i}]$}.
      Let $B = \vec{C}\rightarrow\N$. 
      By lemma \ref{lem:uniq_value},
      there are sequences of values $\vec{c}, \vec{c'}:\vec{C}$ such that
      $u[\vec{d_i}]\vec{c} \not\sim_\N u[\vec{d_i}]\vec{c'}$. 
      Define $l_{i+1}= l_i \conc (2)$ taking the child $u[\vec{x}]$,
      and define $\vec{d_{i+1}} = \vec{d_i}$, $\vec{a_{i+1}} = \vec{c}$, 
      $\vec{d'_{i+1}} = \vec{d'_i}$, and $\vec{a'_{i+1}} = \vec{c'}$. 
      We obtain (i), (ii), (iii) for $i+1$, as expected.
      We also have (iv) since the connections from 
      $(\vec{d_i},\vec{a_i})$ to $(\vec{d_{i+1}},\vec{a_{i+1}}) = (\vec{d_i},\vec{c})$
      and
      $(\vec{d'_i},\vec{a'_i})$ to $(\vec{d'_{i+1}},\vec{a'_{i+1}}) = (\vec{d'_i},\vec{c'})$
      are trace compatible: all connected $\N$-argument of $t_{i}=f(u)[\vec{x}]$ and $t_{i+1}=u[\vec{x}]$ are 
      in $\vec{d_i}$ and therefore are the same.
    \end{enumerate}

  \item
    The case of $\apvar$-rule, namely 
    $\Label(\Pi,l_i) 
    = 
    \Gamma \vdash f[\vec{x}](x): \vec{A}\rightarrow\N$ is obtained from
    $\Gamma \vdash f[\vec{x}]: D,\vec{A} \rightarrow \N$,
    where $\Gamma=x_1:D_1,\ldots,x_m:D_m$ and $x_k:D_k = x:D\in\Gamma$. 
    By the induction hypothesis, $f[\vec{d_i}]d_{i,k}\vec{a_i} \not\sim_\N f[\vec{d'_i}]d'_{i,k}\vec{a'_i}$. 
    where $\vec{d_i} = (d_{i,1},\ldots,d_{i,m})$ and $\vec{d'_i} = (d'_{i,1},\ldots,d'_{i,m})$. 
    We define $l_{i+1}=l_i\conc(1)$ as the unique child of $l_i$ in $\Pi$. 
    We also define $\vec{d_{i+1}} = \vec{d_i}$ and $\vec{a_{i+1}} = d_{i,k},\vec{a_i}$
    and define $\vec{d'_{i+1}} = \vec{d'_i}$ and $\vec{a'_{i+1}} = d'_{i,k},\vec{a'_i}$. 
    We obtain (i), (ii), and (iii) for $i+1$, as expected.
    We also have (iv) since the connections from
    $(\vec{d_i},\vec{a_i})$ to $(d_{i+1},\vec{a_{i+1}}) = (\vec{d_i},d_{i,k},\vec{a_i})$
    and
    $(\vec{d'_i},\vec{a'_i})$ to $(d'_{i+1},\vec{a'_{i+1}}) = (\vec{d'_i},d'_{i,k},\vec{a'_i})$
    are trace compatible: all connected $\N$-arguments in $\vec{d_{i}},\vec{a_{i}}$ and 
    $\vec{d_{i+1}},\vec{a_{i+1}}$ are the same. Similarly
    the connections of $\N$-arguments in $\vec{d'_i},\vec{a'_i}$ and $\vec{d'_{i+1}},\vec{a'_{i+1}}$ are the same.
    The only differences are that, if $D_k = \N$, the values $d_{i,k}$ and $d'_{i,k}$ of type $\N$
    for the variable $x_k^\N$ in $f[x_1,\ldots,x_k,\ldots,x_m](x_k)$ are duplicated to the values
    $d_{i,k}$ and $d'_{i,k}$ of the first unnamed argument of $ f[\vec{x}]$, respectively. 

  \item
    The case of $\lambda$-rule, namely
    $\Label(\Pi,l_i) = 
    \Gamma\vdash \lambda x^A.u[\vec{x},x] : A, \vec{A} \rightarrow \N$ is obtained from
    $\Gamma,x:A\vdash t_{i+1}[\vec{x},x^A]:\vec{A}\rightarrow\N$, 
    where $t_{i+1}[\vec{x},x]=u[\vec{x},x]$.
    By the induction hypothesis, $(\lambda x.(u[\vec{d_i},x]))a\vec{a_*} \not\sim_\N (\lambda x.(u[\vec{d'_i},x]))a'\vec{a'_*}$, 
    where $\vec{a_i} = a,\vec{a_*}$ and $\vec{a'_i} = a',\vec{a'_*}$.    
    Then, by Lemma~\ref{lem:uniq_NF}.\ref{lem:uniq_NF2}, $t_{i+1}[\vec{d_i},a]\vec{a_*}\not\sim_\N t_{i+1}[\vec{d'_i},a']\vec{a'_*}$.
    We define $l_{i+1}=l_i\conc(1)$ as the unique child node of $l_i$,
    and define $\vec{d_{i+1}} = \vec{d_i},a$ and $\vec{a_{i+1}} = \vec{a_*}$,
    and also define $\vec{d'_{i+1}} = \vec{d'_i},a'$ and $\vec{a'_{i+1}} = \vec{a'_*}$.
    We obtain (i), (ii), (iii) for $i+1$, as expected.
    We also have (iv) since the connections from 
    $(\vec{d_i},\vec{a_i}) = (\vec{d_i},a,\vec{a_*})$ to $(\vec{d_{i+1}},\vec{a_{i+1}}) = (\vec{d_i},a,\vec{a_*})$
    and
    $(\vec{d'_i},\vec{a'_i}) = (\vec{d'_i},a',\vec{a'_*})$ to $(\vec{d'_{i+1}},\vec{a'_{i+1}}) = (\vec{d'_i},a',\vec{a'_*})$
    are trace-compatible.
    We note that the values $a$ in $(\vec{d_i},\vec{a_i})$ and $a'$ in $(\vec{d'_i},\vec{a'_i})$ are
    the first unnamed arguments of $t_i[\vec{x}]$, and, in $(\vec{d_{i+1}},\vec{a_{i+1}})$ and $(\vec{d_{i+1}},\vec{a_{i+1}})$,
    they are moved to the values of the last variable of type $A$ of $t_{i+1}[\vec{x},x]=u[\vec{x},x]$.

  \item  
    The case of $\cond$-rule, namely
    $\Label(\Pi,l_i) = \Gamma\vdash C\rightarrow \cond(f[\vec{x}],g[\vec{x},x]):\N,\vec{A}\rightarrow\N$
    is obtained from 
    $\Gamma\vdash f[\vec{x}]:\vec{A}\rightarrow\N$
    and
    $\Gamma\vdash g[\vec{x}]:\N,\vec{A}\rightarrow\N$. 
    By the induction hypothesis, $\cond(f[\vec{d_i}],g[\vec{d_i},x])m\vec{a_*} \not\sim_\N \cond(f[\vec{d'_i}],g[\vec{d'_i},x])m\vec{a'_*}$, 
    where $\vec{a_i} = m,\vec{a_*}$ and $\vec{a'_i} = m',\vec{a'_*}$ with $m,m' \in \Num$.
    Since $m,m'$ are normal and $m \sim_\N m'$, we have $m = m'$. 
    We argue by cases on $m$.
    \begin{enumerate}
    \item
      We first consider the \emph{subcase $m=m'=0$}.
      Define $l_{i+1}=l_i\conc(1)$, namely the first child node of $l_i$ whose term is $f[\vec{x}]$. 
      We define $\vec{d_{i+1}} = \vec{d_i}$ and $\vec{a_{i+1}} = \vec{a_*}$,
      and also define $\vec{d'_{i+1}} = \vec{d'_i}$ and $\vec{a'_{i+1}} = \vec{a'_*}$. 
      We obtain (i), (ii), (iii) for $i+1$, as expected. 
      We also have (iv) since the connection from 
      $(\vec{d_i},\vec{a_i}) =(\vec{d_i},0,\vec{a_*})$ to
      $(\vec{d_{i+1}},\vec{a_{i+1}}) = (\vec{d_i},\vec{a_*})$ is trace compatible: 
      each argument of $t_i$ is connected to some equal argument of $t_{i+1}[\vec{x}]=f[\vec{x}]$,
      the first argument of $t_i[\vec{x}]$ disappears but it is connected to no $\N$-argument in $f[\vec{x}]$.
    \item
      Next we consider the \emph{subcase $m=m'=\Succ(m_*)$}. 
      Define $l_{i+1}=l_i\conc(2)$, namely the second child node of $l_i$ whose term is $g[\vec{x}]$. 
      We define $\vec{d_{i+1}} = \vec{d_i},m_*$ and $\vec{a_{i+1}} = \vec{a_*}$,
      and also define $\vec{d'_{i+1}} = \vec{d'_i},m_*$ and $\vec{a'_{i+1}} = \vec{a'_*}$. 
      We obtain (i), (ii), (iii) for $i+1$, as expected.
      We also have (iv) since the connections from 
      $(\vec{d_i},\vec{a_i}) =(\vec{d_i},\Succ(m_*),\vec{a_*})$ to $(\vec{d_{i+1}},\vec{a_{i+1}}) = (\vec{d_i},m_*,\vec{a_*})$
      and
      $(\vec{d'_i},\vec{a'_i}) =(\vec{d'_i},\Succ(m_*),\vec{a'_*})$ to $(\vec{d'_{i+1}},\vec{a'_{i+1}}) = (\vec{d'_i},m_*,\vec{a'_*})$      
      are trace compatible.
      We note that, in both trace-compatible assignments, the first unnamed argument $m = \Succ(m_*)$ of $t_i[\vec{x}]$ 
      which is connected the first unnamed argument $m_*$ of $g[\vec{x}]$, \emph{decreasing by $1$ from $m$}.
      This is indeed as we expected for trace-compatible assignments since this $i$ is a progress point of them. 
    \end{enumerate}
  \end{enumerate}
  
  By the above construction, we have an infinite path $\pi=(e_1,e_2,\ldots)$ in $\Pi$ and
  two trace compatible assignments
  $\rho = ((\vec{d_1},\vec{a_1}),(\vec{d_2},\vec{a_2}),\ldots)$
  and
  $\rho' = ((\vec{d'_1},\vec{a'_1}),(\vec{d'_2},\vec{a'_2}),\ldots)$ for $\pi$,
  as we wished to show. 
  
\end{proof}

As a corollary from the theorem we will conclude: 
if $t$ satisfies the global trace condition, then there is no such $\sigma$, therefore $t \sim t$, 
as we wished to show.

\begin{corollary}\label{cor:UniqueNF_typeN}
  \begin {enumerate}
  \item
    Assume $\Gamma\vdash t:A$. Then $t \in \GTC$ implies that $t$ is confluent. 
  \item
    For any closed $t:\N$, there is numeral $n\in\Num$ such that $t\safeReducesAst n$.
    Moreover, for any normal form $u$ such that $t \safeReducesAst u$, we have $u=n$. 
  \end{enumerate}
\end{corollary}


%We prove that for any term $ (t \sim t)$ with counter-example $\vec{a},\vec{d} \sim_0 \vec{b},\vec{e}$
%and $\neg (t[\vec{a}](\vec{d}) \sim_0  t[\vec{b}](\vec{e}))$ we can find some immediate
%subterm $t'$ and with a counter-example $\vec{a'},\vec{d'} \sim_0 \vec{b'},\vec{e'}$
%and $\neg (t'[\vec{a'}](\vec{d'}) \sim_0  t'[\vec{b'}](\vec{e'}))$, \emph{compatible with trace
%condition}.


%21:09 20/03/2024
