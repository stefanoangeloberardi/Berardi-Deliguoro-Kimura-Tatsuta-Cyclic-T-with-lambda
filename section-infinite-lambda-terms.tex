

\section{The set of infinite $\lambda$-terms}
We define the set $\LAMBDA$ of infinite circular terms, the subset $\WTyped$
of well-typed terms and a reduction relation for them. Infinite terms are labeled binary trees, therefore we
have to define binary trees first, and before them the lists on $\{1,2\}$ and more related notions.

\subsection{Lists and Binary Trees}

\begin{definition}[Lists on $\{1,2\}$]
\begin{enumerate}
\item
We denote with $\List(\{1,2\})$ the set of all lists of elements of $\{1,2\}$: $(),(1),(1),(1,1),(1,2),\ldots$. 
We call the elements of $\List(\{1,2\})$ just lists for short.
\item
If $i_1, \ldots, i_n \in \{1,2\}$, we write $(i_1, \ldots, i_n)$
for the list with elements $i_1, \ldots, i_n$. 
\item
We write $\nil$ for the empty list, or $()$. 
\item
If $l=(i_1, \ldots, i_n)$, $m=(j_1, \ldots, j_m)$ and $l, m \in \List(\{1,2\})$ we define
$$
l \conc m = (i_1, \ldots, i_n,\ j_1, \ldots, j_m)\in \List(\{1,2\})
$$
%If $I \subseteq \List(\{1,2\})$
%we write $l \conc I = \{l \conc m | m \in I\}$. 
\item
The prefix order $\le$ on lists
is defined by $l \le m$ if and only if $m = l \conc m'$ for some $m' \in  \List(\{1,2\})$.
\end{enumerate}
\end{definition}

Binary trees are represented by set of lists.

\begin{definition}[Binary trees]
\begin{enumerate}
\item 
A binary tree, just a tree for short, is any set $T\subseteq \List(\{1,2\})$ 
of lists including the empty list and closed by prefix: $\nil \in T$ and for all $l, m \in \List(\{1,2\})$
if $m \in T$ and $l \le m$ then $l \in T$. 
\item
$\nil$ is called the root of $T$, any $l \in T$ is called a node of
$T$ and if $l \in T$, then any $l \conc (i) \in T$ is called a child of $l$ in $T$.
\item
A node $l$ of $T$ is a leaf of $T$ if $l$ is an maximal element of $T$ w.r.t. the prefix order (i.e.,
if $l$ has no children). 
\item
If $T$ is a tree and $l \in T$ we define $T \restr l = \{m \in \List(\{1,2\}) | l \conc m \in T\}$.
$T \restr l$ is a tree and we say that $T \restr l$ is a subtree of $T$ in the node $l$,
and for each $m \in T \restr l$ we say that $l \conc m$ is the corresponding node in $T$.
\item
If $l=(i)$ we say that $T_l$ is an immediate sub-tree of $T$.
\item
A binary tree labeled on a set $L$ is any pair $\setT = (T, \phi)$ with 
$\phi:T \rightarrow L$. We call $\universe{\setT}=T$ the set of nodes of $\setT$.
For all $l \in T$ we call $\Label(\setT,l) = \phi(l)$ the label of $l$ in $\setT$.
\item
We define $\setT \restr l = (T \restr l, \phi_l)$ and $\phi_l(m) = l \conc m$. 
We call any $\setT \restr l$ a labeled subtree of $\setT$
in the node $l$, an immediate sub-tree if $l=(i)$. 
\end{enumerate}
\end{definition}

The labeling of a node $m \in \universe{( \setT \restr l )}$ 
is the same label of the corresponding node $l \conc m$ of $\universe{\setT}$.
 
Let $n = 0, 1, 2$. Assume $\setT_1, \ldots,\setT_n$ are trees labeled on $L$ and $l \in L$.
Then we define $$\setT = l(\setT_1, \ldots, \setT_n)$$ as the unique tree labeled on $L$
with the root labeled $l$ and with:
$\setT \restr (i) = \setT_i$ for all $1 \le i \le n$.


\subsection{Types, Terms and Contexts}
Now we define the types of $\LAMBDA$, then the terms and the contexts of $\LAMBDA$.

\begin{definition}[Types of $\LAMBDA$]
\mbox{}
\begin{enumerate}

\item
The types of $\LAMBDA$ are: the type $\N$ of natural numbers, an infinite list 
$\alpha,\beta,\ldots$ of type variables, and with $A,B$ also  $A \rightarrow B$.
We call them simple types, just \emph{types} for short. 

\item 
$\Type$ is the set of simples types.

\item
We suppose be given a set $\var$, consisting of all pairs $x^T=(x,T)$ 
of a variable name $x$ and a type $T \in \Type$.
\end{enumerate}
\end{definition}

Terms of $\LAMBDA$ are labeled binary trees.


\begin{definition}[Terms of $\LAMBDA$]
The terms of $\LAMBDA$ 
are all binary trees $t$ with set of labels $L=\{x^T, \lambda x^T., \ap, 0, \Succ, \cond\}$
such that:
\begin{enumerate}
\item 
a node $l$ labeled $x^T$ or $0$ is a leaf of $t$ (no children). 
\item
a node $l$ labeled $\lambda x^T.$ or $\Succ$ has a unique child $l \conc (1)$. 
\item
a node labeled $\ap$, $\cond$ has two children $l \conc (1)$, $l \conc (2)$.
\end{enumerate}
We say that $t$ is a sub-term of $u$ if $t$ is a labeled sub-tree of $u$,
an immediate subterm if it is an immediate sub-tree.
\end{definition}
 
Assume  $l \in L = \{x^T, \lambda x^T., \ap, 0, \Succ, \cond\}$.
We  use the operation $l( \setT_1 \ldots \setT_n )$ on labeled trees to define new terms in $\LAMBDA$.
If $t, u \in \LAMBDA$ then $x^T, 0, \lambda x^T.t,\ap(t,u),\Succ(t),\cond(t,u) \in \LAMBDA$ 
are terms with the root labeled $l$, and with immediate subterms among $t$, $u$. 
Conversely, each $v \in \LAMBDA$ is in one of the forms:
$x^T, 0, \lambda x^T.t$, $\ap(t,u)$, $\Succ(t)$, $\cond(t,u)$ for some $t, u \in \LAMBDA$.

Now we define the regular terms $\LAMBDA$.

%11:55 22/04/2024

\begin{definition}[Regular terms of $\LAMBDA$]
\mbox{}
\begin{enumerate}

\item
We write $\SubTerm(t)$ for the (finite or infinite) set of subterms of $t$. 
Subterms are coded by the nodes of $t$, different nodes can code the same subterm. 

\item
$\Reg$ is the set of terms $t \in \LAMBDA$ such that $\SubTerm(t)$ is finite.
We call the terms of $\Reg$ the \emph{regular terms}.

%\item
%We write $\Tree(t)$ for the tree of all chains
%$(t_1, \ldots, t_n)$ with $t_1=t$ and weakly increasing by $\sqsubset_1$: 
% for all $(i+1) \le n$ we have $t_{i+1}=t_i$ or $t_{i+1} \sqsubset_1 t_i$.

\item
As usual, we abbreviate $\ap(t,u)$ with $t(u)$.

\item
When $t = \Succ ^n(0)$ for some natural number $n \in \Nat$
we say that $t$ is a numeral. We write $\Num$ for the set of numeral in $\LAMBDA$.

\item
A variable $x^T$ is free in $t$ if there is some $l \in t$ labeled $x^T$ in $T$, 
and \emph{no $m < l$ labeled $\lambda x^T.$} in $t$. 
$\FV(t)$ is the set of free variables of $t$.
\end{enumerate}
 
\end{definition}

%We use two different names for the operation $\ap(t,u)$: 
%we call it $\ap$ when $u$ is not a variable and $\apvar$ when $u$ is a variable. 

$\Num$ is the representation inside $\LAMBDA$ of the set $\Nat$ of natural numbers.
All numerals are finite trees of $\Lambda$. 
All finite well-typed typed $\lambda$-terms 
we can define with the rules above are finite terms of $\LAMBDA$.
Regular terms can be represented by the subterm relation restricted to $\SubTerm(t)$:
this relation defines a graph with possibly cicles. Even if $\SubTerm(t)$ is finite, $t$ can be an infinite tree. 

We do not assume implicit renaming of bound variables, namely $\alpha$-equivalence
because the typing system (given later) does not have renaming rule as a primitive rule. 

\begin{Eg}
\label{example-regular-infinite}
An example of regular term which is an infinite tree: the term $t = \cond(0,t) \in \LAMBDA$. 
The set $\SubTerm(t)=\{t,0\}$ of subterms  of $t$ is finite, therefore $t$ is a regular term.
However, $t$ is an infinite tree (it includes itself as a subtree). 
The immediate sub-term chains of $t$ are all $(t,t,t,\ldots,t)$ and $(t,t,t,\ldots,0)$.
There is a unique infinite sub-term chain, which is $(t,t,t,\ldots)$. 
\end{Eg}

In order to define the type of an infinite term, we first 
define contexts and sequences for any term of $\LAMBDA$.
A context is a list of type assignments to variables: our variables already have a type superscript,
so in fact a type assignment $x^T:T$ is \emph{redundant} for our variables (not for our terms).
Yet, we add an assignment relation $x^T:T$ for uniformity with the notation $x:T$ 
in use in Type Theory.

%13:32 22/04/2024
%11:17 29/04/2024

\begin{definition}[Contexts of $\LAMBDA$]
\mbox{}
\begin{enumerate}

\item
A  context of $\LAMBDA$ is any finite list $\Gamma = ({x_1}^{A_1}:A_1, \ldots, x_n^{A_n}:A_n)$ 
of \emph{pairwise distinct variables}, each assigned to its type superscript $A_1, \ldots, A_n \in \Type$. 

\item
We denote the empty context with $\nil$. We write $\Ctxt$ for the set of all contexts.

\item
A sequent is the pair of a context $\Gamma$ and a type $A$, which we write as $\Gamma \vdash A$.
We write $\Seq = \Ctxt \times \Type$ for the set of all sequents.

\item 
A typing judgement is the list of a context  $\Gamma$, a term $t$ and a type $A$, 
which we write as $\Gamma \vdash t:A$.
We write $\Stat = \Ctxt \times \LAMBDA \times \Type$ for the set of all typing judgements.

\item
We write $\FV(\Gamma) = \{ {x_1}^{A_1}, \ldots, {x_n}^{A_n} \}$.
We say that $\Gamma$ is a context for $t \in \LAMBDA$ and we write $\Gamma \vdash t$ 
if $\FV(t) \subseteq \FV(\Gamma)$.

\item
If $\Gamma = ({x_1}^{A_1}:A_1, \ldots, x_n^{A_n}:A_n)$ ,
$\Gamma' = (x'_1:A'_1, \ldots, x'_n:A'_{n'})$ are context of $\LAMBDA$, then we
write 
\begin{enumerate}
\item
$\Gamma \ \subseteqsim \ \Gamma'$ \ if for all $(x^A:A)$:  \ 
$(x^A:A) \in \Gamma  \ \Rightarrow  \  (x^A:A)\in\Gamma'$
\item
$\Gamma \sim \Gamma'$  \  if for all $(x^A:A)$:  \ 
$(x^A:A) \in \Gamma  \ \Leftrightarrow  \  (x^A:A)\in\Gamma'$
\end{enumerate}


\item
If $\Gamma$ is a context of $\LAMBDA$, then $\Gamma\setminus\{x^T:T\}$ is the context obtained
by removing $x_i^{A_i}:A_i$ from $\Gamma$ for all $x_i^{A_i}=x^T$. 
If $x \not \in \FV(\Gamma)$ then $\Gamma\setminus\{x^T:T\} = \Gamma$.

\end{enumerate}
\end{definition}

We have $\Gamma \subseteqsim \Gamma'$ if and only if
if there is a (unique) map $\phi:\{1,\ldots,n\} \rightarrow \{1,\ldots,n'\}$
such that $x_{i}=x'_{\phi(i)}$ and $A_{i}=A'_{\phi(i)}$ for all $i \in \{1,\ldots,n\}$.
We have  $\Gamma \sim \Gamma'$ if and only if  $\Gamma \subseteqsim \Gamma'$
and  $\Gamma \supseteqsim \Gamma'$ if and only if $\Gamma$, $\Gamma'$ are permutation
each other if and only if the map $\phi$ above is a bijection. We do not identify two contexts
which are one a permutation of the other, therefore we will need a rule to move from one to the other.


%12:38 17/04/2024
%15:37 17/04/2024

%From the context for a term we can define a context for each subterm of the term.


%21:26 19/04/2024

%\begin{definition}[Inherited Contexts of $\LAMBDA$]
%
%Given any context $\Gamma$, any $t \in \Lambda$ and any subterm chain 
%$\pi = (t_1, \ldots, t_n) \in \Tree(t)$, we define a unique inherited context for $t_n$ in $\pi$.
%The inherited context is obtained by repeatedly adding $x^T:T$ to the context whenever we
%cross a term $t_i = \lambda x^T.u_i$, while simultaneously removing $x^T:T$ from the previous
%context, if it was there.
%%08:00 20/04/024
%
%\begin{enumerate}
%
%\item
%$t$ has inherited context $\Gamma$.
%
%\item
%Any binder on $x$ subtracts the variable $x$ from the context of its \emph{last} argument:
%if $t = \lambda x^T.u, \cond(f,g)$ has inherited context $\Delta$, 
%then $u$ and $g$ have context $\Delta \setminus \{x^T:T\}, x^T$, 
%while $f$ has  inherited context $\Delta$.
%
%\item
%In any other case the context of a term and of the immediate subterm are the same:
%ff $t=\Succ(u), f(a)$ have inherited context $\Delta$,
% then $u,f,a$ have  inherited context $\Delta$.
%\end{enumerate}
%We abbreviate \emph{`` inherited context from $\Gamma$''} with \emph{context}
%when $\Gamma$ is fixed.
%\end{definition}



%The scope of the binder $\lambda x^T.t$ is $t$.
%The scope of the binder $\cond(f,g)$ is $g$ ($f$ is \emph{not} in the scope of $\cond(f,g)$).

\Daisuke{begin definition of substitution}\\

We define the term $t[u/x]$ obtained by substituting the infinite term $u$ for the free variable $x$ in $t$ 
avoiding variable captures.
In order to define the capture avoiding substitution
under the assumption of not having implicit $\alpha$-equivalence, 
we implicitly fix a well-order on variables
and define the simultaneous substitution,
written $t[u_1/x_1,\ldots,u_n/x_n]$ or $t[\vec{u}/{\vec{x}}]$,
namely it satisfies the following:
\begin{itemize}
\item
  $x_i[u_1/x_1,\ldots,u_n/x_n] = u_i$ and $y[\vec{u}/\vec{x}] = y$, where $y\not\in\vec{x}$. 
\item
  $(f(a))[\vec{u}/\vec{x}] = f[\vec{u}/\vec{x}](a[\vec{u}/\vec{x}])$.
\item
  $(\lambda z^T.b)[\vec{u}/\vec{x}] =
  \left\{
  \begin{array}{ll}
    \lambda z^T.(b[\Restrict{(\vec{u}/\vec{x})}{\overline{z}}]),
    &
    \text{if $z \not\in \FV(\vec{u})$},
    \\
    \lambda z'^T.(b[\Restrict{(\vec{u}/\vec{x})}{\overline{z}},z'/z]),
    &
    \text{otherwise},
  \end{array}
  \right.$
  \\
  where
  $\Restrict{(\vec{u}/\vec{x})}{\overline{z}}$ is the one obtained by removing $u/z$ (for some $u$) from $\vec{u}/\vec{x}$,
  and 
  $z'$ is the least variable (with respect to the well-order)
  such that $z' \not\in \FV(b,\vec{u})$.
\item
  $0[\vec{u}/\vec{x}] = 0$.
\item
  $\Succ(t)[\vec{u}/\vec{x}] = \Succ(t[\vec{u}/\vec{x}])$.
\item
  $\cond(f,g)[\vec{u}/\vec{x}] = \cond(f[\vec{u}/\vec{x}],g[\vec{u}/\vec{x}])$.
\end{itemize}

Formally we define the substitution as follows.
\begin{definition}[Substitution for terms of $\LAMBDA$]
  A set $\theta=\{u_1/x_1,\ldots,u_n/x_n\}$ is called a {\em substitution}
  if $x_1,\ldots,x_n$ are pairwise distinct variables.
  It is called {\em renaming} if $u_1,\ldots,u_n$ are pairwise distinct variables. 
  We define $\theta(y)$ by $u_i$ if $y=x_i$, and by $y$ otherwise.
  We write $\theta,\theta'$ if $\theta\cup\theta'$ is a substitution.   
  The substitution $\Restrict{\theta}{\overline{x}}$ is defined 
  as the one obtained by removing $u/x$ (for some $u$) from $\theta$. 
  
  Let $t$ be a term of $\LAMBDA$, which is the labeled tree $(T_t,\phi_t)$,
  and $\theta$ be $\{u_1/x_1,\ldots,u_n/x_n\}$, where each $u_i$ is $(T_{u_i},\phi_{u_i})$. 
  Then the term $t[\theta]$ is $(T,\phi)$, where 
  $T = T_t \cup \{l\conc m \mid \text{$\phi_t(l) = x_i \in \FV(t)$ and $m\in T_{u_i} $}\}$.
  For each $l\conc m$ in the latter part, $\phi(l\conc m) = \phi_{u_i}(m)$.  
  For $l\in T_t$, we inductively define $\phi(l)$ and
  $\theta_l$ (such that $\theta_l = \theta',\theta_{ren}$, where $\theta'\subseteq \theta$ and $\theta_{ren}$ is a renaming) as follows. 
  \begin{itemize}
  \item
    $\theta_{\nil} = \theta$.
  \item
    If $\phi_t(l) = y^T \not\in\FV(t)$, then $\phi(l) = \theta_l(y^T)$.
  \item
    If $\phi_t(l) = 0$, then $\phi(l) = 0$. 
  \item    
    If $\phi_t(l) = \Succ$, then $\phi(l) = \Succ$ and $\theta_{l\conc (1)} = \theta_l$. 
  \item    
    If $\phi_t(l) \in \{\ap,\cond\}$, then $\phi(l) = \phi_t(l)$
    and $\theta_{l\conc (1)} = \theta_{l\conc(2)} = \theta_l$.
  \item
    If $\phi_t(l) = \lambda z^T.$ and $z\not\in\FV(\vec{u})$,
    then $\phi(l) = \lambda z^T.$ and $\theta_{l\conc(1)} = \Restrict{(\theta_l)}{\overline{z}}$. 
  \item
    If $\phi_t(l) = \lambda z^T.$ and $z\in\FV(\vec{u})$,
    then $\phi(l) = \lambda z'^T.$ and $\theta_{l\conc(1)} = \Restrict{(\theta_l)}{\overline{z}},z'/z$, where $z'$ is the least variable such that $z' \not\in \FV(t\restr l\conc(1),\vec{u})$.
  \end{itemize}
  We often abbreviate $t[\{u_1/x_1,\ldots,u_n/x_n\}]$ by $t[u_1/x_1,\ldots,u_n/x_n]$.
  For a context $\Gamma = (x_1:A_1,\ldots,x_n:A_n)$ and a renaming $\theta$,
  we write $\Gamma[\theta]$ for $(\theta(x_1):A_1,\ldots,\theta(x_n):A_n)$ if it is a context. 
\end{definition}

\Daisuke{end definition of substitution}


\subsection{Typing Rules}
We define typing rules for terms of $\LAMBDA$ and the subset $\WTyped$ of well-typed terms.
We consider a term well-typed if typing exists and it is unique for the term and for all its subterms.

The typing rules are the usual ones but for %the conditional binder $\cond$, and for 
the \emph{typing rule $\ap$ for an application} $t(u)$, which we split in two sub-rules, $\apvar$
and $\apnotvar$, according if $u$ is a variable or is not a variable.
%$\apvar$ corresponds to an $\eta$-expansion and it introduces a global variable name $x^T$
%for the first argument of $t$. 
%As we said, 
We need to insert this extra information $t$ in typing because it is important for checking termination
of the computation of $t(u)$, as we will explain later.

Remark that we introduced a unique \emph{term notation}
for application: we write $\ap(t,u)$ no matter if $u$ is a variable or not, but we we split 
the typing rule for $\ap$ into two sub-cases, $\apvar$ and $\apnotvar$.

%We have a single structural rule  $\struct_f$, which can be used for:
% weakening, variable permutation and variable renaming. 
We have a a single structural rule $\weak$ for extending a context $\Gamma$ to a context 
$\Gamma' \supseteqsim \Gamma$. When $\Gamma' \sim \Gamma$ (when 
$\Gamma'$, $\Gamma$ are permutation each other), 
the rule $\weak$ can be used for variable permutation.
Variable renaming for a term $t[\vec{x}]$ can be obtained by writing 
$(\lambda \vec{x}.t[\vec{x}])(\vec{x'})$. 
Therefore we do not assume having a primitive rule for renaming. 
The lack of a renaming rule is a 
\emph{difference with the circular syntax for inductive \underline{proofs}}.

\Daisuke{start}
The reason of not having implicit $\alpha$-conversion mentioned in the previous section 
comes from this assumption. 
For example, if we identify the $\alpha$-equivalent terms $\lambda x^A.x$ and $\lambda z^A.z$,
the left derivation is a correct proof, but the right one is not ($(z:A,z:A)$ is not a context).
That is, the proof structure does not closed under identity of the $\alpha$-equivalence. 
\begin{center}
  $\infer{
    z:A \vdash \lambda x^A.x:A\to A
  }{
    \infer{
      z:A, x:A \vdash x:A
    }{}
  }$
  \qquad
  $\infer{
    z:A \vdash \lambda z^A.z:A\to A
  }{
    \infer{
      z:A, z:A \vdash z:A
    }{}
  }$  
\end{center}
The provability of sequents is closed under the variable renaming and the $\alpha$-equivalence.
We will discuss this point later. 
\Daisuke{end}

%11:56 29/04/2024

\begin{definition}[Typing rules of $\LAMBDA$]
Assume $\Gamma = {x_1}^{A_1}:A_1, \ldots, x_n^{A_n}:A_1$ is a context. 
Let $p=0,1,2$.
%$\Delta = y_1:B_1, \ldots, y_n:B_m$ are sequents of length $n$, $m$ respectively. Suppose
%$f:\{1, \ldots, n\} \rightarrow \{1, \ldots, m\}$ is any injection, compatible with types
%in $\Gamma$, $\Delta$: we assume $A_i = B_{f(i)}$ for all $1 \le i \le n$.

\begin{enumerate}
%\item
%$\struct_f$-rule.
%If $t: \Gamma \vdash T$ then $t[ y_{f(1)}/x_1, \ldots,  y_{f(p)}/x_p]:\Delta \vdash T$
\item
A rule is a list of $p+1$ typing judgements: 
$\Gamma \vdash t_1:A_1, \ldots, \Gamma \vdash t_p:A_p, \Gamma \vdash t : A$.
The first $p$ sequents are the premises of the rule (at most two in our case), 
the last sequent is the conclusion of the rule.
\item
We read the rule above: \emph{``if $\Gamma \vdash t_1:A_1, \ldots, \Gamma \vdash t_p:A_p$
then $\Gamma \vdash t : A$"}.
\end{enumerate}

We list the typing rules of $\LAMBDA$: 
the rules $\apvar$ and $\apnotvar$ are restricted.

\begin{enumerate}

\item
$\weak$-rule (Weakening + Exchange).
If $\Gamma \vdash t:T$ and $\Gamma \subseteqsim \Gamma'$
then $\Gamma' \vdash t : T$

\item
$\var$-rule.
If $x^A \in \Gamma$ then $\Gamma \vdash x^A:A$.



\item
$\lambda$-rule.
If $\Gamma, x^A:A \vdash b: B$
then $ \Gamma \vdash \lambda x^A.b :A \rightarrow B$.

\item
$\apvar$-rule.
If $\Gamma \vdash f: A \rightarrow B$ then $\Gamma \vdash f(x^A) :  B$,
provided  $(x^A:A)\in  \Gamma$.

\item
$\apnotvar$-rule.
If $\Gamma \vdash f:A \rightarrow B$ and $\Gamma \vdash a:A$
then $\Gamma \vdash f(a) : B$, provided $a$ is \emph{not} a variable 

\item
$0$-rule.
$\Gamma \vdash 0: \N$

\item
$\Succ$-rule.
If $\Gamma \vdash t:\N$ then $\Gamma \vdash \Succ (t):\N$.

\item
$\cond$-rule.
If $\Gamma \vdash  f :T$ and  $\Gamma \vdash g : \N \rightarrow T$ 
then $\Gamma \vdash \cond(f,g) : \N \rightarrow T$.
\end{enumerate}
We abbreviate $\nil \vdash  t:A$ ($t:A$ in the empty context) with $\vdash t:A$. 
We write the set of rules of $\LAMBDA$ as
\[
\Rule = 
\{r \in \Seq \cup \Seq^2 \cup \Seq^3 | r \mbox{ instance of some rule of }\LAMBDA\}
\]
\end{definition}

A rule is uniquely determined from its conclusion, provided we know whether the rule is a weakening or not.
Two rules are equal if they have the same assumptions and the same conclusion.

\begin{proposition}[Rules and subterms]
\label{proposition-rules-subterms}
Assume $r, r' \in \Rule$ have conclusion $\Gamma \vdash t:A$, $\Gamma' \vdash t':A'$
respectively.
\begin{enumerate}
%%%%%%%%%%%%%%%%%%%%%%%%%%%%%%%%
% NO THIS IS WRONG ASSUMPTIONS OF weak ARE NOT UNIQUE. 
% Anyway we do not need this point.
%%%%%%%%%%%%%%%%%%%%%%%%%%%%%%%%
%\item
%If $r, r'$ are weakening rules and $\Gamma \vdash t:A = \Gamma' \vdash t':A'$ then $r = r'$.
%%%%%%%%%%%%%%%%%%%%%%%%%%%%%%%%
\item
If $r, r'$ are non-weakening rules and $\Gamma \vdash t:A = \Gamma' \vdash' t':A'$ then $r = r'$.
\item
If $r$ is not a weakening and the premises of $r$ are some 
$\Gamma_1 \vdash t_1:A_1, \ldots, \Gamma_p \vdash t_p:A_p$,
then the list of immediate subterms of $t$ is exactly $t_1, \ldots, t_p$.
\end{enumerate}
\end{proposition}

From the typing rules we define the proofs that a term is well-typed. 
Proofs are binary trees on the set $\Rule$, each node is associated to a typing judgement which is
a consequence of the typing judgements associated to the children node under some rule $r \in \Rule$
of $\LAMBDA$.

%12:55 29/04/2024

The next definition is the formal definition of proof.
We mention that we will introduce a restriction we call \emph{Almost-left-finite} on proof trees. 
The restriction Almost-left-finite is used to prevent trivial proofs. 
For instance, we want to forbid a proof cyclically switching the variables $x$ and $y$ in the context of a term,
by an infinite nesting of $\weak$-rule (Weakening + Exchange  rule):
 \[
  \infer[\weak]{
    (x^T:T, y^U:U) \prove t : A
  }{
    \infer[\weak]{
    (y^U:U,  x^T:T) \prove t : A
  }{\infer*{}{}}
  }
  \]
Another example: we want to forbid an infinite left nesting of applications, as in a proof of 
$t_{n}:A^n \rightarrow A$ where $t_{n} = t_{n+1}(t_0)$:
 \[
  \infer[\apnotvar]{
   \prove t_n : A^n \rightarrow A
  }{
    \infer[]{
   \prove t_{n+1} : A^{n+1} \rightarrow A
  }{\ldots}
   &
   \infer[]{
   \prove t_{0} : A
  }{\ldots}
  }
  \]
These proof trees are meaningless and should be ruled out. The Almost-Left-Finite condition will ask
that all leftmost branches in any sub-proof are finite, say, that no infinite leftmost branch made of
$\weak$-rules or of $\apnotvar$ exists. 
The idea is that when the main operation in a programming language is application, 
then the leftmost branch uniquely  determines the type of the term and therefore it should be finite.
%12:54 03/06/2024


\begin{definition}[Well-typed term of $\LAMBDA$]
Assume $\Pi=(T,\phi)$ is a binary tree labeled on $\Rule$.
Assume $\Gamma$ is a context of $t$ (i.e, $\FV(t) \subseteq \Gamma$) and $A \in \Type$ 

Then we write $\Pi: \Gamma \vdash t:A$, and we say that $\Pi$ is a proof of $\Gamma \vdash t:A$ if:

\begin{enumerate}
\item 
  $\phi(\nil) = \Gamma \vdash t:A$.
\item
  Assume that
  \begin{enumerate}
  \item
    $l \in T$ is labeled with $\phi(l) = \Delta \vdash u: B$.
  \item
    The children of $l$ in $T$ are labeled 
$
\phi(l \conc (1)) = \Delta_1 \vdash u_1: B_1, 
\ldots, 
\phi(l \conc (p)) = \Delta_p \vdash u_p: B_p
$
  \end{enumerate}
  Then for some rule $r \in \Rule$ of $\LAMBDA$ we have
  \[
  \infer[r]{
    \Delta \prove u : B
  }{
    \Delta_1  \prove u_1:B_1
    &
    \cdots
    &
    \Delta_p  \prove u_p:B_p
  }
  \]
\item
A path $\pi$ in a proof tree $\Pi$ is \emph{almost-leftmost} if there are only finitely many rules with two
 premises in $\pi$ such that $\pi$ does \emph{not} choose the leftmost premise.
\item
A proof $\Pi$ is Almost-left-finite if all almost-lefmost paths $\pi$ in $\Pi$ are finite
\end{enumerate}

\end{definition}

We can check that a proof is almost-left-finite if and only if all leftmost paths of all sub-proofs are finite.

In the case of a regular proof, we can decide in polynomial time in the number of sub-proofs whether a
proof is almost-left-finite. Here is the algorithm. Suppose a possibly infinite proof has $n$ subproofs.
For each sub-proof we follow the leftmost path, after $n$ steps either we reach some rule without
premises (either a $\var$-rule or a $0$-rule), or we loop. Then:
\begin{enumerate}
\item
If for all sub-proofs we stop then the proof is almost-left-finite
\item
If for some sub-proof we loop, then there is infinite leftmost path and the proof is \emph{not} almost-left-finite.
\end{enumerate}
We estimate the computation time to $O(n)$ for each of the $n$ sub-proofs, 
and the total computation time to $O(n^2)$.
\\

Eventually we define the well-typed terms of $\LAMBDA$ using almost-left-finite proofs.

\begin{definition}[Well-typed term of $\LAMBDA$]
\mbox{}
\begin{enumerate}
\item
$\Gamma \vdash t:A$ is true if and only if $\Pi:\Gamma \vdash t:A$ for some $\Pi$.
\item
$t \in \LAMBDA$ is well-typed if and only if 
$\Pi:\Gamma \vdash t:A$ for some \emph{almost-left-finite} proof $\Pi$,
for some $A$ and some context $\Gamma \supseteqsim \FV(t)$.
\item
$\WTyped$ is the set of well-typed $t \in \LAMBDA$.
\end{enumerate}
%A proof $\Pi:\Gamma \vdash  t :T$ is canonical if all $\weak$-rules
%\begin{enumerate}
%\item
% either follow some $\lambda$-rule for a term $\lambda x^T.b$, and they introduce $x^T$ in the context,
%\item
%or follow some $\cond$-rule for a term $\cond(f,g)$, and and they introduce $x^\N$ in the context,
%\end{enumerate}
\end{definition}

Well-typed terms are closed by substitution.

We prove that the type assigned to a term by an Almost-Left-Finite proof is \emph{unique}: 
if two Almost-Left-Finite proofs assign two types $T$, $T'$ to the same possibly infinite term $t$,
then $T = T'$.

\begin{proposition}[Uniqueness of almost-left-finite type]
If $t \in \LAMBDA$, $\Pi:\Gamma \vdash t:A$ and  $\Pi':\Gamma' \vdash t:A'$ are two almost-left-finite
proofs then $A = A'$.
\end{proposition}

%08:56 19/06/2024
\begin{proof}
By induction on the sum of the length of the leftmost branch in $\Pi$ and $\Pi'$. These lengths are finite
because $\Pi$ and $\Pi'$ are assumed to be almost-left-finite. Let $r$ be the last rule of $\Pi$ and $r'$ be the
last rule of $\Pi'$.
\begin{enumerate}
\item
Assume $r = \weak$. Then $\Pi$ is obtained from a proof $\Pi_1:\Gamma_1 \vdash t:A$. 
By induction hypothesis on $\Pi_1$ and $\Pi'$ we conclude that $A = A'$.

\item
Assume $r \not = \weak$ and $r' = \weak$. 
Then $\Pi'$ is obtained from a proof $\Pi'_1:\Gamma_1 \vdash t:A$. 
By induction hypothesis on $\Pi$ and $\Pi'_1$ we conclude that $A = A'$.

\item
Assume  $r \not = \weak$ and $r' \not = \weak$. Then both $r$ and $r'$ are the typing rule for
the outermost constructor of $t$ and for some
$h=0,1,2$ the proof $\Pi$ has premises $\Pi_1, \ldots, \Pi_h$
and the proof $\Pi'$ has premises $\Pi'_1, \ldots, \Pi'_h$ (for the same $h$). 
We distinguish one case for each constructor.

\begin{enumerate}
\item
Assume $t = x^T$. Then $r=r'=\var$-rule and $A = A' = T$.

\item
Assume $t =0^\N$. Then $r=r'=0$-rule and $A = A' = \N$.

\item
Assume $t =\Succ(u)$. Then $r=r'=\Succ$-rule  and $\Pi_1:\Gamma \vdash u: \N$
and $\Pi_2:\Gamma' \vdash u:\N$ and $A = A' = \N$.

\item
Assume $t = f(u)$. Then $r=r'=\ap$-rule and $\Pi_1:\Gamma \vdash f:U \rightarrow A$
and $\Pi_2:\Gamma' \vdash f:U' \rightarrow A'$. By induction hypothesis on $\Pi_1$ and
$\Pi'_1$ we deduce that $(U \rightarrow A) = (U' \rightarrow A')$, in particular that $A = A'$.

\item
Assume $t = \lambda x^T.b$. Then $r=r'=\lambda$-rule and 
$\Pi_1:\Gamma \vdash b:B$
and $\Pi_2:\Gamma' \vdash b:B'$. By induction hypothesis on $\Pi_1$ and
$\Pi'_1$ we deduce that $B=B'$, in particular that $A = T \rightarrow B = T \rightarrow B' = A'$.

\item
Assume $t = \cond(f,g)$. Then $r=r'=\cond$-rule and $\Pi_1:\Gamma \vdash f:A$
and $\Pi_2:\Gamma' \vdash f:A'$. By induction hypothesis on $\Pi_1$ and
$\Pi'_1$ we deduce that $A = A'$.

\end{enumerate}
\end{enumerate}
\end{proof}


We provide some examples of well-typed and not well-typed terms.
%23:30 23/04/2024
Some term in $\LAMBDA$ has no type, like the application $0(0)$ of the non-function $0$. 

\begin{Eg}
We provide a term in $\LAMBDA$ having more than one type.
Let $t=u(0)$ and $u=\cond(t,u)$. $t$ has an infinite lefmost branch $t,u,t,u, \ldots$, therefore
$t$ is not almost-left-finite and it has no \emph{almost-left-finite typing proof}.
Unicity fails and we can prove $\vdash t:A$ for all types $A \in \Type$. 
The subterms of $t$ are $\{t, u, 0\}$ and a proof $\Pi: \emptyset \vdash t:A$ is
\[
\infer[\apnotvar]{
  \prove t : A
}{
  \infer[\cond]{
    \prove u:\N \rightarrow A
  }{
    \infer*{
      \prove t : A
    }{}
    &
    \quad
    &
    \infer*{
      \prove u : \N \rightarrow A
    }{} 
  }
  &
  \quad
  &
  \infer[0]{
    \prove 0:\N
  }{}
}
\]
Formally, the typing proof $\Pi=(T,\phi):\vdash t:A$ 
is defined by $\universe{\Pi}=\universe{t}$ and:
\begin{enumerate}
\item
$\Label(\Pi,l)=(\vdash A)$ \ \ \ \ \ \ \ for all $l \in \universe{t}$ such that  $t \restr l = t$,
\item
$\Label(\Pi,l) = (\vdash \N \rightarrow A)$  for all $l \in \universe{t}$ such that $t \restr l = u$,
\item
 $\Label(\Pi,l)=(\vdash\N)$ \ \ \ \ \ \ \  for all $l \in \universe{t}$ such that $t \restr l = 0$. 
\end{enumerate}
%09:06 24/04/2024
\end{Eg}


A term with at least two types has the leftmost branch infinite, as it is the case for $t$ above.
We proved that if \emph{all} subterms of a term have the leftmost branch finite, then the term 
has at most one type.


\emph{Claim (existence of a polynomial-time typing algorithm)}.
We claim without proof that if a term $t$ is regular then we can decide if an almost-left-finite proof 
$\Pi:\Gamma \vdash A$ exists. If an almost-left-finite proof $\Pi$ exists 
then at least one of almost-left-finite proofs is regular and we can compute it. 
We first check whether the term is almost-left-finite in time $O(n^2)$. If it is not we reject the term.
Otherwise we start the standard recursive typing algorithm, until a type has been inferred consistently
for each of the finitely many subterms, or some type inconsistency has been found.
In the first case we accept the term and we return the time, in the second case we reject the term
and we return no type.
The computation requires quadratic time in the size of the graph representing the term.


%10:26 20/04/2024
%13:49 29/04/2024


\subsection{Reduction Rules}
Our goal is to provide a set of well-formed term for $\LAMBDA$ and interpret them as partial functionals.
Some terms, those satisfying the global trace condition (to be introduced later) will be total functionals.
Our first step is to provide reduction rules for $\LAMBDA$.

%18:14 27/03/2024

\begin{definition}[reduction rules for $\LAMBDA$]
\mbox{}
\begin{enumerate}

\item
$\reduces_\beta$: $(\lambda x^A.b)(a) \reduces_\beta b[a/x]$

\item 
$\reduces_\cond$: $\cond(f,g)(0) \reduces_\cond f$ and
$\cond(f,g)(\Succ (t)) \reduces_\cond g(t)$.

\item
$\reduces$ is the contraction of one $\beta$ or $\cond$ redex: 
the context closure of $\reduces_\beta$ and $\reduces_\cond$.

\item
$\reduces^*$ is the transitive closure of $\reduces$ (finite sequence of reductions).


\item
The $\cond$-depth of a node $l=(i_1, \ldots, i_n) \in t$ 
is the number of $1 \le h < n$ such that $(i_1, \ldots, i_h)$ has label $\cond$
and $i_{h+1} = 2$
(the number of $\cond$ such that $l$ occurs in the right-hand-side of such $\cond$).

\item
The $n$-safe level of $t$ is the set of nodes of $t$ of depth $\le n$.

\item
$t =_{n,\cond} u$, or \quotationMarks{$t$,$u$ are $n$-safe equal} is equality of $t$, $u$ after removing 
all subterms whose $\cond$-depth is $>n$.
%$t \sim u$ if and only if there is some $v \in \LAMBDA$ such that $t \reduces v$ and $u \reduces v$.

\item
We say that $t \reduces_{n,\safe} u$, or that $t$ $n$-safely reduces to $u$,  
if we reduce $t$ to $u$ by contracting a single redex in the $n$-safe level of $t$ 
(i.e. of $\cond$-depth $\le n$).
%We call \emph{unsafe} a reduction inside any $\cond(f,g)$.

\item
A term is $n$-safe-normal if all its redexes (if any) have $\cond$-depth $>n$.

We abbreviate \quotationMarks{$0$-safe normal} with just \quotationMarks{safe normal}.
\end{enumerate}
\end{definition}

%09:19 24/04/2024
%14:00 29/04/2024

\begin{Eg}
Let $u = \cond. (0, (\lambda z.u)(z) )$, where we omitted the type superscript
of $z$ because it is irrelevant. Then $u$ is $0$-safe-normal (safe normal for short), because
all redexes in $u$ are of the form  $(\lambda z.u)(z)$ and are in the right-hand-side of a $\cond$. 
However $u$ has infinitely many nodes which are $\beta$-redexes. 
Indeed, the tree form of $u$ has the following branch:
\begin{center}
  $u$, 
  \quad
  $(\lambda z.u)(z)$, 
  \quad
  $\lambda z.u$, 
 \quad
  $u$, 
 \quad $\ldots$
\end{center}
This branch is cyclic, infinite,
and it includes infinitely many $\beta$-redexes, all equal to the same term $(\lambda z.u)(z)$.
\end{Eg}

Through branches crossing the right-hand-side
of some $\cond$ we will-express fixed-point equations.
Reductions on fixed-point equations can easily loop.
However, in order to produre the output of a computation we only need 
to reduce closed terms of type $\N$ to a numeral. We will prove that for this taks we only
need $0$-safe reductions, those in the the right-hand-side
of \emph{no} $\cond$, and that $0$-safe reductions strongly normalize. 

\begin{Eg}
This is an example of safe $\cond$-reductions from a term $v(n)$ to a normal form. 
Assume $n \in \Num$ is any numeral and $v = \cond(1, v)$. There are only finitely many reductions
from $v(n)$, and they are all $0$-safe. $v(n)$ $\cond$-reduces to $v(n-1)$, 
then we loop: $v(n-1)$ $\cond$-reduces to $v(n-2)$ and so forth.
After $n$ $\cond$-reductions we get $v(0)$. With one last $\cond$-reduction we get $1$ and we stop. 
Thus, the term $v(n)$ strongly normalizes to $1$ in $(n+1)$-steps, in fact $v(n)$ has a unique reduction path,
having length $(n+1)$ and terminating in $1$.
\end{Eg}

There are terms without a normal form, but we can get as close as we want to a normal form
In fact, we will prove that for all $n$, all terms have some $n$-safe normal form, that is, 
with no redexes of $\cond$-depth $\le n$. 
The $n$-safe normal form of a term is unique up $n$-safe equality: any two $n$-safe normal forms have the same $n$-safe level, though they may be different in some level of $\cond$-depth $>n$.

We will also prove that terms have a limit normal form, 
obtained by taking the limit of the $n$-safe normal forms of the term, and that this limit normal form is unique.

\newpage
