% lmcs, 2018.5.2, 2019.9.9, 2020.2.10



\newif\ifdraft \draftfalse
\drafttrue % put % for lipics


\long\def\Stefano#1{{{\color{red}{SB: #1}}}}
\long\def\Makoto#1{{{\color{blue}{MT: #1}}}}
\long\def\Daisuke#1{{{\color{green}{DK: #1}}}}

\ifdraft

\documentclass{article}
\usepackage{mystyle}
\usepackage{amsfont,amssymb,amsmath,graphicx,color}
\usepackage{mymath,proof,latexsym}
\usepackage{xcolor}
\usepackage[pdftex,outline]{contour}

\A4page
%\usepackage{tikzpicture}
\usepackage{graphicx}

\else

\documentclass{lmcs}
\usepackage{hyperref}
\usepackage{graphicx}

\fi



%%%%%%%%%%%%%%%%%%%%%%%%%%%%%%%%%%%%
% WE ADD THE COMMANDS theorem lemma proposition corollary definition % 
%%%%%%%%%%%%%%%%%%%%%%%%%%%%%%%%%%%%

\newtheorem{theorem}{Theorem}[section]
\newtheorem{lemma}[theorem]{Lemma}
\newtheorem{proposition}[theorem]{Proposition}
\newtheorem{corollary}[theorem]{Corollary}
\newtheorem{definition}[theorem]{Definition}

\newenvironment{proof}[1][Proof]{\begin{trivlist}
\item[\hskip \labelsep {\bfseries #1}]}{\end{trivlist}}
\newenvironment{example}[1][Example]{\begin{trivlist}
\item[\hskip \labelsep {\bfseries #1}]}{\end{trivlist}}
\newenvironment{remark}[1][Remark]{\begin{trivlist}
\item[\hskip \labelsep {\bfseries #1}]}{\end{trivlist}}



\begin{document}
\sloppy 
\hbadness=10000
\vbadness=10000

\input{macro}

\ifdraft

\title{CT-$\lambda$, a cyclic simply typed $\lambda$-calculus with fixed points}

\author{Stefano Berardi }
\date{}

\else

\title[Equivalence]
{
???
}

\author[S. Berardi]{Stefano Berardi}
\address{Universit\`{a} di Torino,
Torino, Italy}
\email{stefano@unito.it}


%% mandatory lists of keywords 
\keywords{
proof theory
inductive definitions
Brotherston-Simpson conjecture
cyclic proofs
}
\fi

\maketitle

\begin{abstract}
We explore the possibility of having a circular syntax
directly for $\lambda$-abstraction and simple types, instead of interpreting
 $\lambda$-abstraction through combinators and inserting a  circular syntax afterwards.
We introduce our circular syntax as a fixed point operator defined by cases.
Our motivation for using binders instead of combinators is that this way of introducing circular syntax 
could be more familiar for researchers working in the field of Type Theory.

We prove the expected results for this circular simply typed $\lambda$-calculus, which we call $\CTlambda$: 
every closed term of type $\N$ reduces to some numeral;
strong normalization for all reductions sequences outside all fixed points; 
 strong normalization for ``fair'' reductions; Church-Rosser property;
and the equivalence with G\"{o}del system $\systemT$. 

Besides, terms of $\CTlambda$ are much shorter than the equivalent terms written in $\systemT$. 
\end{abstract}

\iffalse
key words: 
proof theory,
inductive definitions,
Brotherston-Simpson conjecture,
cyclic proofs,
Martin-Lof's system of inductive definitions,
infinite Ramsey theorem
Podelski-Rybalchenko termination theorem
size-change termination theorem
\fi

\input{section-introduction}

\newpage
\input{section-infinite-lambda-terms}

\newpage
\input{section-trace-infinite-lambda-terms}

\newpage
\input{section-circular-system-CTlambda}

\newpage

\section{An example of \quotationMarks{interactive use} of $\CTlambda$}
Our thesis is that we can have two very similar definitions $f_1$, $f_2$ 
of the same map $F:\N,\N \rightarrow \N$
by two regular well-typed terms of $\LAMBDA$. However, $f_1$ does not satisfies the Global
Trace Condition, therefore $f_1 \not \in \CTlambda$ is \emph{not} automatically recognized as total, 
while $f_2$ satisfies the Global Trace condition and is automatically recognized as total.
\\

Why does it happen? The global trace condition for $\CTlambda$ follows the trace of some
type-$\N$variables, for instance it can follow the trace of the input $x^\N$ for $g(x)$, but it cannot
follow the trace of a non-variable $u$ used as input for $g(u)$. This means that for the
global trace algorithm can follow the trace of the local name $x$ of $u$ in $(\lambda x^\N.g(x))(u)$,
while cannot follow the trace of $u$ itself in $g(u)$. If $x$ is infinitely decreasing the global trace algorithm
can deduce termination in the first case, cannot deduce termination in the second case.

In a sense, the $\GTC$-algorithm is \emph{interactive} for $\CTlambda$. When we believe that 
$u:\N$ is infinitely decreasing in every infinite computation, and that this fact is crucial to insure termination,
we should to assign to $u$ a local name $x^\N$. In this way the $\GTC$-algorithm can follow the trace of 
$x^\N$ in any computation. This way of using the global trace algorithm 
is similar to the way of using the Size-Change-Termination algorithm in Neil Jones.


\subsection{The Ackermann Function}
The Ackermann function is a good example if we want to test the global trace 
condition in a case in which it is difficult to prove the convergence of a map. We check that the
global trace condition can prove that a definition of Ackermann function in $\LAMBDA$ 
is convergent, provided we use local names to denote the sub-expressions  ofour definition 
which are infinitely progressing toward $0$.
\\

The Ackermann function $\Ack:\N,\N\rightarrow \N$ is a map whose convergence can be
 proved in$\PA$, but only if we use induction over formulas with at least two unbounded quantifiers. 
For a similar reason, $\Ack$ can be defined in system $\systemT$, but only if we use primitive recursion over
terms whose type has degree $ \ge 2$. $\Ack$ grows faster than all primitive recursive maps.


Here is a fixed point definition of $\Ack$. We will express it in $\CTlambda$ using a nested conditional.
\begin{align*}
  \Ack(0,n) &= n+1
  \\
  \Ack(m+1,0) &= \Ack(m,1)
  \\
  \Ack(m+1,n+1) &= \Ack(m,\Ack(m+1,n))
\end{align*}

We can prove that $\Ack$ is convergent by induction on the lexicographic order on $(m,n)$. 
In the next subsections we try to prove convergence of $\Ack$ using $\GTC$.


\subsection{A Term Representation of Ackermann \emph{without} the Global Trace Condition}

We give a first representation $\AckA \in \LAMBDA$ of the Ackermann function. 
In this article, we sometimes abbreviate $f(x)(y)$ with $f(x,y)$, in this way we also improve readability.

\begin{definition}[$\AckA$]
  We define $\AckA \in \LAMBDA$ by the following infinite regular term .
  \[
  \AckA = \lambda \redM\bluen.\Cond{\Suc{\bluen}}{\lambda m'.\Cond{\AckA(m',1)}{\lambda n'.\AckA(m',\AckA(\Suc{m'},n'))}(\bluen)}(\redM)
  \]
\end{definition}

We check that $\AckA$ almost-left-finite. 
Any infinite path in $\AckA$ must move infinitely many times from $\AckA$ to $\AckA$, and whenever we 
do so we move to the right-hand-side of a $\cond$. Therefore any infinite path in $\AckA$ is 
not almost-leftmost, by taking the contrapositive we prove that all almost-leftmost paths in $\AckA$
are finite. $\AckA$ has (unique) type $\N,\N\rightarrow \N$.
\\

\noindent{\bf Checking the fixed point equation for $\AckA$}
\begin{align*}
  \AckA(0,n)
  &\mapsto
  \Cond{\Suc{n}}{\lambda m'.\Cond{\AckA(m',1)}{\lambda n'.\AckA(m',\AckA(\Suc{m'},n'))}(n)}(0)
  \\
  &\mapsto
  \Suc{n}
\end{align*}
%%%%%%%
\begin{align*}
  \AckA(\Suc{m},0)
  &\mapsto
  \Cond{\Suc{0}}{\lambda m'.\Cond{\AckA(m',1)}{\lambda n'.\AckA(m',\AckA(\Suc{m'},n'))}(0)}(\Suc{m})
  \\
  &\mapsto
  (\lambda m'.\Cond{\AckA(m',1)}{\lambda n'.\AckA(m',\AckA(\Suc{m'},n'))}(0))(m)
  \\
  &\mapsto
  \Cond{\AckA(m,1)}{\lambda n'.\AckA(m',\AckA(\Suc{m'},n'))}(0)
  \\
  &\mapsto
  \AckA(m,1)
\end{align*}
%%%%%%%
\begin{align*}
  \AckA(\underline{\Suc{m}},\Suc{n})
  &\mapsto
  \Cond{\Suc{\Suc{n}}}{\lambda m'.\Cond{\AckA(m',1)}{\lambda n'.\AckA(m',\AckA(\Suc{m'},n'))}(\Suc{n})}(\underline{\Suc{m}})
  \\
  &\mapsto
  (\lambda m'.\Cond{\AckA(m',1)}{\lambda n'.\AckA(m',\AckA(\Suc{m'},n'))}(\Suc{n}))(\underline{m})
  \\
  &\mapsto
  \Cond{\AckA(m,1)}{\lambda n'.\AckA(m,\AckA(\Suc{\underline{m}},n'))}(\Suc{n})
  \\
  &\mapsto
  (\lambda n'.\AckA(m,\AckA(\Suc{\underline{m}},n')))(n)
  \\
  &\mapsto
  \AckA(m,\AckA(\Suc{\underline{m}},n))
\end{align*}


The global trace algorithm will fail to prove that $\AckA$ is terminating. The reason is in the last case. 
The term $\underline{\Suc{m}}$ of the first line and $\Suc{\underline{m}}$ of the last line are slightly 
different: $\underline{\Suc{m}}$ is decomposed into $\underline{m}$ by the $\text{cond}$-reduction, 
then $\Suc{\underline{m}}$ is constructed by substituting $m'$ of $\Suc{m'}$ by $\underline{m}$. 
When we loop from the rightmost occurrence of $\AckA$ 
in the picture above back to the root of $\AckA$, the input $\Suc{m}$ of $\AckA$ is sent without any change to the first argument of $\AckA$. The globale trace algorithm fails to notice this: 
in order to notice it, the two terms $\Suc{m}$ should have the same local name $\redM$.
As a consequence, the trace of $\Suc{m}$ is cut in this point and some infinitely progressing trace
disappears.



\subsection{$\AckA:\N\to\N\to\N$ is not in $\GTC$}

\begin{claim}
  $\AckA:\N\to\N\to\N$ is well-typed, but $\AckA \not \in \GTC$.
\end{claim}

In the following, weakening is implicitly applied. 

\begin{center}

{\scriptsize
  \hspace{-3cm}
  $\infer{
    \vdash \AckA:\redN\to\goldN\to\N\ (\dagger)
  }{
    \infer[\Rapv]{
      m:\redN, n:\goldN \vdash \Cond{\Suc{n}}{\lambda m'.\Cond{\AckA(m',1)}{\lambda n'.\AckA(m',\AckA(\Suc{m'},n'))}(n)}(m): \N
    }{
      \infer[\Rcond]{
        m:\redN, n:\goldN \vdash \Cond{\Suc{n}}{\lambda m'.\Cond{\AckA(m',1)}{\lambda n'.\AckA(m',\AckA(\Suc{m'},n'))}(n)}: \redN\to\N
      }{
        \infer{
          n:\goldN \vdash \Suc{n}: \N
        }{
          \infer{
            n:\goldN \vdash n: \N
          }{}
        }
        &
        \infer{
          n:\goldN \vdash \lambda m'.\Cond{\AckA(m',1)}{\lambda n'.\AckA(m',\AckA(\Suc{m'},n'))}(n): \redN\to\N
        }{
          \infer[\Rapv]{
            n:\goldN, m':\redN \vdash \Cond{\AckA(m',1)}{\lambda n'.\AckA(m',\AckA(\Suc{m'},n'))}(n): \N
          }{
            \infer[\Rcond]{
              n:\goldN, m':\redN \vdash \Cond{\AckA(m',1)}{\lambda n'.\AckA(m',\AckA(\Suc{m'},n'))}: \goldN\to\N
            }{
              \infer{
                m':\redN \vdash \AckA(m',1):\N
              }{
                \infer[\Rapv]{
                  m':\redN \vdash \AckA(m'):\N\to\N
                }{
                  \infer{
                    m':\redN \vdash \AckA:\redN\to\N\to\N
                  }{
                    \deduce{
                      \vdash \AckA:\redN\to\N\to\N
                    }{(\dagger1)}
                  }
                }
                &
                \infer{
                  \vdash 1:\N
                }{}
              }
              &
              \infer{
                m':\redN \vdash \lambda n'.\AckA(m',\AckA(\Suc{m'},n')): \goldN\to\N
              }{
                \infer{
                  m':\redN, n':\goldN \vdash \AckA(m',\AckA(\Suc{m'},n')): \N
                }{
                  \infer[\Rapv]{
                    m':\redN \vdash \AckA(m'): \N\to\N
                  }{
                    \infer{
                      m':\redN \vdash \AckA: \redN\to\N\to\N
                    }{
                      \deduce{
                        \vdash \AckA: \redN\to\N\to\N
                      }{(\dagger2)}
                    }
                  }
                  &
                  \infer[\Rapv]{
                    m':\redN, n':\goldN \vdash \AckA(\Suc{m'},n'): \N
                  }{
                    \infer[\RapNv]{
                      m':\redN, n':\goldN \vdash \AckA(\Suc{m'}): \goldN\to\N
                    }{
                      \deduce{
                        \vdash \AckA:\N\to\goldN\to\N
                      }{
                        (\dagger3)
                      }
                      &
                      \infer{
                        m':\redN \vdash \Suc{m'}:\N
                      }{
                        \infer{
                          m':\redN \vdash m':\N
                        }{}
                      }
                    }
                  }
                }
              }
            }
          }
        }
      }
    }
  }$  
}

\end{center}

The problem is with the rightmost rule $\RapNv$ in the picture above. 
The same value $\Suc{m'} = m$ is send to the first argument of $\AckA$, but this fact 
is not reported in the trace.
This cuts the trace chasing of $\redN$ (red color). A similar situation arises with the leftmost rule
$\RapNv$  in the picture above: the trace chasing of $\goldN$ (gold color) is cut.

As a consequence, the infinite path $(\dagger)\rightsquigarrow(\dagger1)\rightsquigarrow(\dagger)
\rightsquigarrow(\dagger3)\rightsquigarrow(\dagger)\rightsquigarrow(\dagger1)
\rightsquigarrow(\dagger)\rightsquigarrow(\dagger3)\rightsquigarrow\cdots$ 
does not contain any infinite trace, and with more reason any infinitely progressing trace. 



\subsection{Second term representation: $\AckB$}

We give the second representation $\AckB$ of Ackermann function. In this case we inform
the global trace algorithm that the local term $\Suc{m'}$ is equal to $m$ in any computation
by writing $m$ in the place of $\Suc{m'}$.

\begin{definition}[$\AckB$]
  We define $\AckB$ by the following infinite regular term of $\LAMBDA$.
  \[
  \AckB = \lambda \redM\bluen.
\Cond{\Suc{\bluen}}{\lambda m'.\Cond{\AckB(m',1)}{\lambda n'.\AckB(m',\AckB(\redM,n'))}(\bluen)}(\redM)
  \]
\end{definition}

Again, the term $\AckB$ is almost-left-finite and has type $\N, \N \rightarrow \N$.
\\

\noindent{\bf Checking the fixed point equation for $\AckB$}

We check only the last case. 
\begin{align*}
  \AckB(\underline{\Suc{m}},\Suc{n})
  &\mapsto
  \Cond{\Suc{\Suc{n}}}{\lambda m'.\Cond{\AckB(m',1)}{\lambda n'.\AckB(m',\AckB(\underline{\Suc{m}},n'))}(\Suc{n})}(\underline{\Suc{m}})
  \\
  &\mapsto
  (\lambda m'.\Cond{\AckB(m',1)}{\lambda n'.\AckB(m',\AckB(\underline{\Suc{m}},n'))}(\Suc{n}))(\underline{m})
  \\
  &\mapsto
  \Cond{\AckB(m,1)}{\lambda n'.\AckB(m,\AckB(\underline{\Suc{m}},n'))}(\Suc{n})
  \\
  &\mapsto
  (\lambda n'.\AckB(m,\AckB(\underline{\Suc{m}},n')))(n)
  \\
  &\mapsto
  \AckB(m,\AckB(\underline{\Suc{m}},n))
\end{align*}

The point of $\AckB$ is that the global trace algorithm by checking the
traces discover that $\underline{\Suc{m}}$ 
at the first line and the one at the last line are exactly the same, thanks to the fact that
they have the same local name $m$.




\subsection{$\AckB:\N\to\N\to\N$ is in $\GTC$}

\begin{proposition}
  $\AckB:\N\to\N\to\N$ has a proof $\Pi$ that satisfies $\GTC$.
\end{proposition}

\begin{proof}
Below is a proof $\Pi$ of $\AckB:\N\to\N\to\N$. 
In the following, weakening is implicitly applied. 


\begin{flushright}

{\scriptsize
  \hspace{-3cm}
  $\infer{
    \vdash \AckB:\redblueN\to\goldN\to\N\ (\dagger)
  }{
    \infer[\Rapv]{
      m:\redblueN, n:\goldN \vdash \Cond{\Suc{n}}{\lambda m'.\Cond{\AckB(m',1)}{\lambda n'.\AckB(m',\AckB(m,n'))}(n)}(m): \N
    }{
      \infer[\Rcond]{
        m:\redN, n:\goldN \vdash \Cond{\Suc{n}}{\lambda m'.\Cond{\AckB(m',1)}{\lambda n'.\AckB(m',\AckB(m,n'))}(n)}: \blueN\to\N
      }{
        \infer{
          n:\goldN \vdash \Suc{n}: \N
        }{
          \infer{
            n:\goldN \vdash n: \N
          }{}
        }
        &
        \infer{
          m:\redN, n:\goldN \vdash \lambda m'.\Cond{\AckB(m',1)}{\lambda n'.\AckB(m',\AckB(m,n'))}(n): \blueN\to\N
        }{
          \infer[\Rapv]{
            m:\redN, n:\goldN, m':\blueN \vdash \Cond{\AckB(m',1)}{\lambda n'.\AckB(m',\AckB(m,n'))}(n): \N
          }{
            \infer[\Rcond]{
              m:\redN, n:\goldN, m':\blueN \vdash \Cond{\AckB(m',1)}{\lambda n'.\AckB(m',\AckB(m,n'))}: \goldN\to\N
            }{
              \infer{
                m':\blueN \vdash \AckB(m',1): \N
              }{
                \infer[\Rapv]{
                  m':\blueN \vdash \AckB(m'): \N\to\N
                }{
                  \infer{
                    m':\blueN \vdash \AckB: \blueN\to\N\to\N
                  }{
                    \deduce{
                      \vdash \AckB: \blueN\to\N\to\N
                    }{(\dagger1)}
                  }
                }
                &
                \infer{
                  \vdash 1:\N
                }{}
              }
              &
              \infer{
                m:\redN, m':\blueN \vdash \lambda n'.\AckB(m',\AckB(m,n')): \goldN\to\N
              }{
                \infer{
                  m:\redN, m':\blueN, n':\goldN \vdash \AckB(m',\AckB(m,n')): \N
                }{
                  \infer[\Rapv]{
                    m':\blueN \vdash \AckB(m'): \N\to\N
                  }{
                    \infer{
                      m':\blueN \vdash \AckB: \blueN\to\N\to\N
                      }{
                      \deduce{
                        \vdash \AckB: \blueN\to\N\to\N
                      }{(\dagger2)}
                    }
                  }
                  &
                  \infer[\Rapv]{
                    m:\redN, n':\goldN \vdash \AckB(m,n'): \N
                  }{
                    \infer{
                      m:\redN, n':\goldN \vdash \AckB(m): \goldN\to\N
                    }{
                      \infer[\Rapv]{
                        m:\redN \vdash \AckB(m): \goldN\to\N
                      }{
                        \infer{
                          m:\redN \vdash \AckB: \redN\to\goldN\to\N
                        }{
                          \deduce{
                            \vdash \AckB: \redN\to\goldN\to\N
                          }{(\dagger3)}
                        }
                      }
                    }
                  }
                }
              }
            }
          }
        }
      }
    }
  }$
}

\end{flushright}

\vspace{1cm}

We check that this proof $\Pi$ satisfies the global trace condition.
We first note that:

\begin{itemize}
\item
  The path $(\dagger)\rightsquigarrow(\dagger1)$ contains a progressing blue trace $\tau_1 = (\redblueN,\redblueN,\blueN,\ldots,\blueN)$. The gold trace is cut in $(\dagger1)$.
\item
  The path $(\dagger)\rightsquigarrow(\dagger2)$ contains a progressing blue trace $\tau_2 = (\redblueN,\redblueN,\blueN,\ldots,\blueN)$. The gold trace is cut in $(\dagger2)$.
\item
  The path $(\dagger)\rightsquigarrow(\dagger3)$ contains a progressing gold trace 
$\tau'_3 = (\goldN,\ldots,\goldN)$ and a non-progressing red trace 
$\tau_3 = (\redblueN,\redblueN,\redN,\ldots,\redN)$. 
\end{itemize}

Each trace $\tau_i$, blue or red, can be composed with each trace $\tau_j$. 

The first steps of each $\tau_i$ are in common and we write them with the blue and red colors superposed.
Then the blue and the red traces fork.

Take an infinite path $\pi$ from this proof $\Pi$ 
in order to prove that it includes some infinitely progressing trace.

\begin{enumerate}
\item
If $\pi$ passes through $(\dagger1)$ or $(\dagger2)$ infinitely many times,
we take its infinitely progressing trace by combining $\tau_1$, $\tau_2$, and $\tau_3$,
according if we pass through$(\dagger1)$ or $(\dagger2)$ or $(\dagger3)$. 

\item
If $\pi$ passes through $(\dagger1)$ and $(\dagger2)$ only finitely many times,
namely it eventually becomes a loop of $(\dagger)$ and $(\dagger3)$,
from this point on we take its infinitely progressing trace by repeating $\tau'_3$ for infinitely many times.
\end{enumerate}

\end{proof}



  
%\end{document}
%
%\infer{}{}
%
%\lambda mn.\Cond{\Suc{n}}{\lambda m'.\Cond{\Ack(m')(1)}{\lambda n'.\Ack(m')(\Ack(\Suc{m'})(n'))}(n)}(m)


\newpage

\section{Subject Reduction for Well-Typed Infinite Lambda Terms}
\label{section-subject-reduction}

We show the subject reduction for well-typed terms of $\LAMBDA$,
and also show the global trace condition is preserved by reductions. 
We first introduce some auxiliary notations for proofs of them.

Let $X$ be a set of variables.
We write $\Gamma_X$ be $\{x^T:T \in \Gamma \mid x^T \in X \}$.
Let $\Gamma$ and $\Delta$ be contexts of $\LAMBDA$.
We say that $\Gamma$ and $\Delta$ are {\em consistent} if 
$\Gamma_{\FV(\Gamma)\cap\FV(\Delta)} \sim \Delta_{\FV(\Gamma)\cap\FV(\Delta)}$. 
The merged context $\Gamma\sharp\Delta$
is defined by $\Gamma\conc\Delta_{\FV(\Delta)\setminus\FV(\Gamma)}$
if $\Gamma$ and $\Delta$ are consistent, and is undefined otherwise. 

Let $S_1 = \Gamma_1\vdash t_1:\vec{B_1}\rightarrow N$
and $S_2 = \Gamma_2\vdash t_2:\vec{B_2}\rightarrow N$ be sequents. 
Let $k_1$ and $k_2$ be indexes of $N$-arguments of $t_1$ and $t_2$, respectively. 
Then we write $(k_1,S_1) \simIndex (k_2,S_2)$ (or $(k_1,t_1) \simIndex (k_2,t_2)$ for short) if $k_1$ and $k_2$ are those of named arguments with the same name,
or $k_1$ and $k_2$ are those of unnamed arguments at the same position
in $\vec{B_1}$ and $\vec{B_2}$, respectively. 
Note that the index equivalent to $k_1$ is unique (if it exists), namely 
$(k_1,t_1) \simIndex (k_2,t_2)$ and $(k_1,t_1) \simIndex (k'_2,t_2)$ implies $k_2=k'_2$. 

\begin{lemma}
  Assume $\Pi:\Gamma\vdash t:A$.
  Then $\Pi':\Gamma_{\FV(t)}\vdash t:A$ for some $\Pi'$.
  Moreover if the global trace condition holds for $\Pi$, then it also holds for $\Pi'$. 
\end{lemma}
\begin{proof}
  Let $\Pi$ be $(T,\phi)$.
  For each $l \in T$, we write $\Gamma_l\vdash t_l:A_l$
  for the conclusion of $\phi(l) \in \Rule$. 
  Then we construct $\Pi'=(T,\phi')$,
  whose set of nodes is the same as that of $\Pi$. 
  For each $l\in T$, we define $\phi'(l)$ and $\Gamma'_l$ that satisfies
  the following requirements:
  \begin{itemize}
  \item[(a)]
    $\Gamma'_l\vdash t_l:A_l$ is the conclusion of $\phi'(l) \in \Rule$,
  \item[(b)]
    $\Restrict{(\Gamma_l)}{\FV(t_l)} \subseteqsim \Gamma'_l \subseteqsim \Gamma_l$, and
  \item[(c)]
    if $\tilde{k},t_{l\conc(i)}$ is the successor of $k,t_l$ in $\Pi$
    and $k$ is an index of some unname argument
    or a named argument of some name $z\in\FV(t_l)$, 
    then there are $\tilde{k'}$ and $k'$ such that
    $\tilde{k'},t_{l\conc(i)}$ is the successor of $k',t_l$ in $\Pi'$
    and $(k,t_l) \simIndex (k',t_l)$. 
  \end{itemize}
  We define $\Gamma'_{\nil} = \Restrict{\Gamma}{\FV(t)}$.
  Then it satisfies (b) since $t_\nil = t$. 
  Next, assuming the induction hypothesis that $\Gamma'_l$ that satisfies (b)
  is already defined, 
  we define $\phi'(l)$ and $\Gamma_{l\conc(i)}$, 
  for each $i$ such that $l\conc(i)\in T$, 
  that satisfies (a), (b), and (c). 
  It is done by the case analysis of $\phi(l)$.

  The case that $\phi(l) = \Gamma_l\vdash x:A_l$,
  which is an instance of $(\var)$-rule with $x:A_l\in\Gamma_l$. 
  Then define $\phi'(l) = \Gamma'_l\vdash x:A_l$. This is an instance of $(\var)$-rule
  because $\{x:A_l\} = \Restrict{(\Gamma_l)}{\{x\}} \subseteqsim \Gamma'_l$ by (b).

  The case that $\phi(l) = (\Gamma_{l\conc(1)}\vdash t_l:A_l,\Gamma_{l}\vdash t_l:A_l)$,
  which is an instance of $(\weak)$-rule with
  $t_{l\conc(1)} = t_l$, $A_{l\conc(1)} = A_{l}$, and
  $\Gamma_{l\conc(1)}\subseteqsim \Gamma_l$.
  Let $\psi$ be the unique map that determines $\Gamma_{l\conc(1)}\subseteqsim \Gamma_l$.
  Then define $\Gamma'_{l\conc(1)}$ such that
  $\Gamma'_{l\conc(1)}\subseteqsim \Gamma'_l$ determined by
  the induced map from $\psi$ restricting the range to $\FV(\Gamma'_l)$.
  Note that $\FV(\Gamma'_{l\conc(1)}) = \FV(\Gamma'_l)\cap\FV(\Gamma_{l\conc(1)})$. 
  Then it satisfies (b) since $\FV(t_l)\subseteq \FV(\Gamma'_l) \cap \FV(\Gamma_{l\conc(1)}) = \FV(\Gamma'_{l\conc(1)}) \subseteq \FV(\Gamma_{l\conc(1)})$ by (b) for $\Gamma'_l$.
  Define $\phi'(l) = (\Gamma'_{l\conc(1)}\vdash t_l:A_l,\Gamma'_{l}\vdash t_l:A_l)$
  as $(\weak)$-rule.
  Hence the requirement (a) holds.
  We can also show (c): if $k$ is an index in $\Gamma_l\vdash t_l:A_l$ for a name
  $z\in\FV(t_{l\conc(1)}) = \FV(t_l)$, then we can take an index $k'$
  in $\Gamma'_l\vdash t_l:A_l$ for $z$ by $\FV(t_l)\subseteq\FV(\Gamma'_l)$ by (b). 
  An index $\tilde{k'}$ for the name $z$ can be taken
  from $\Gamma'_{l\conc(1)}\vdash t_l:A_l$
  since $z \in \FV(t_l) \subseteq \FV(\Gamma'_{l\conc(1)})$. 
  
  The case that $\phi(l)$ is an instance of $(\lambda)$-rule
  whose conclusion is $\Gamma_l\vdash\lambda z^C.b:C,\vec{B}\to N$
  with $A_l=C,\vec{B}\to N$. 
  Define $\Gamma'_{l\conc(1)} = \Gamma'_l,z:C$.
  This is possible since $\Gamma_l,z:C$ is consistent
  and $\Gamma'_l\subseteqsim \Gamma_l$ by (b) for $\Gamma'_l$.
  Then define
  $\phi'(l) = (\Gamma'_{l},z:C\vdash b:\vec{B}\to N, \Gamma'_l\vdash \lambda z.b:C,\vec{B}\to N)$ as an instance of $(\lambda)$-rule.
  Then we have (a). 
  We also have (b) for $\Gamma'_{l\conc(1)}$
  since $\FV(b) \subseteq \FV(\lambda z.b) \cup \{z\} \subseteq \FV(\Gamma'_l)\cup\{z\} = \FV(\Gamma'_{l\conc(1)}) \subseteq \FV(\Gamma_l)\cup\{z\} = \FV(\Gamma_{l\conc(1)})$.
  The requirement (c) holds: 
  if $k$ is an index of some named argument $y\in\FV(\lambda z.b)$ in $\Gamma_l$,
  then $k'$ can be taken as the index of $y$ in $\Gamma'_l$ by (b) for $\Gamma_l$.
  If $k$ is an index of some unnamed argument in $C,\vec{B}$,
  then $k'$ can be taken as the index of some unnamed argument. 
  In both cases, their successors $\tilde{k}$ and $\tilde{k'}$ are uniquely
  determined by $k$ and $k'$, respectively. 
  
  The case that $\phi(l)$ is an instance of $(\apvar)$-rule
  whose conclusion is $\Gamma_l\vdash f(x^B):A_l$. 

  
  The case that $\phi(l)$ is an instance of $(\apnotvar)$-rule
  whose conclusion is $\Gamma_l\vdash f(b^B):A_l$. 

  The case that $\phi(l)$ is an instance of $(0)$-rule
  whose conclusion is $\Gamma_l\vdash 0:N$ with $A_l=N$. 

  The case that $\phi(l)$ is an instance of $(\Succ)$-rule
  whose conclusion is $\Gamma_l\vdash \Succ(t_{l\conc(1)}):N$ with $A_l=N$.


  The case that $\phi(l)$ is an instance of $(\cond)$-rule
  whose conclusion is $\Gamma_l\vdash \cond(f,g):N\to B$ with $A_l=N\to B$.

  
\end{proof}



\begin{lemma}[Substitution lemma]
  Assume that $\Pi_u: \Gamma \vdash u:A$ and $\Pi_t:\Gamma,x:A \vdash t:B$ hold.
  Then there exists $\Pi^*$ such that $\Pi^*:\Gamma \vdash t[u/x]:B$. 
  Moreover, if $\Pi_u$ and $\Pi_t$ satisfy the global trace condition,
  then $\Pi^*$ also satisfies it. 
\end{lemma}
\begin{proof}


\end{proof}


\begin{theorem}[Subject reduction]
  Assume that $\Pi_t:\Gamma\vdash t:A$ and $t\reduces u$.
  Then there exists $\Pi_u$ such that $\Pi_u:\Gamma\vdash u:A$ holds. 
  Moreover, if $\Pi_t$ satisfies the global trace condition,
  then $\Pi_u$ also satisfies it. 
\end{theorem}
\begin{proof}


\end{proof}


\newpage


\newpage

\section{Weak Church-Rosser for leveled safe reductions}
\label{section-n-safe-church-rosser}

In this section we prove a result for the $n$-safe part of a term, which we call 
\quotationMarks{\emph{unicity of the $n$-safe part of the safe normal form up to $\nequal{n}$}}. By this we mean:
for all $t, u, v \in \GTC$, if $t \nsafeReduces{n} u$ and $t \nsafeReduces{n} v$ and $u$, $v$ are $n$-safe-normal 
then $u \nequal{n} v$ holds. 

Our first idea (wrong) is to prove a full Church-Rosser property for $\LAMBDA$: 
for all $t,u,v \in \LAMBDA$, if $t \reduces u$ and $t \reduces v$ then for some $w \in \LAMBDA$
we have $u \reduces w$ and $v \reduces w$. This property is false: for some $t \in \LAMBDA$, finding a 
common reduction of $u$, $v$ takes infinitely many steps. This even in the case $t \in \CTlambda$,
as the next example shows.

\begin{Eg}[Failure of Church-Rosser for $\CTlambda$]
Let $b = \cond(x^{\N},b):\N \rightarrow \N$ a normal form
and $t = (\lambda x^{\N}.b)(r):\N \rightarrow \N$, 
where $r = (\lambda x^{\N}.x^{\N})(3)$ is some redex. 

We have $b[r/x](n) \reduces r \reduces 3$ for all numerals $n$, 
therefore $t$ and $\lambda \_.3$ are extensionally equal, however $t \not \reduces \lambda \_.3$. 
We have $t \in \CTlambda$. Indeed, 
\begin{enumerate}
\item
$t$ is regular by construction.
\item
We have $t \in \GTC$, because the unique infinite path of $t$ is 
$t, \lambda x^{\N}.b, b, b, b, \ldots$, and the
unique unnamed argument of $b:\N \rightarrow \N$ in the path progresses infinitely many times.
\end{enumerate}

Now consider the reductions: $t \reduces b[r/x^\N]$ and $t \reduces  (\lambda x^{\N}.b)(3)$.
We expect $b[3/x^\N]$ as common normal form. But we have $b[r/x^\N] = \cond(r,b[r/x^\N]$,
that is, we have replicated the redex $r$ infinitely many times in $b[r/x^\N]$. Therefore to reduce 
$b[r/x^\N]$ to $b[3/x^\N]$ takes infinitely many steps, and for \emph{no finite reduction we have}
$b[r/x^\N] \reduces b[3/x^\N]$. 

We proved that Church-Rosser is false for $\CTlambda$.
\end{Eg}

In the following we will work in the $n$-safe level of $t\in \GTC$. 
The global trace condition guarentees the finiteness of the $n$-safe level for each $n$. 
\begin{lemma}
  For any $n\ge 0$, the set of the $n$-safe level of $t\in \GTC$ is finite.
\end{lemma}
\begin{proof}
  Let $t\in \GTC$. We show the finiteness of any $n$-level of $t$ by proof of contradiction.
  Assume that the finiteness fails. 
  Then take the least number $n$ such that the $n$-safe level of $t$ is infinite.
  
  The case of $n=0$.
  Since the binary tree obtained from $t$ restricting to the $0$-safe level of $t$ contains infinite nodes,
  we can take an infinite path $t,t_1,t_2,\ldots$ of the tree by using K\"{o}nig's lemma.
  Then this infinite path does not contain a progressing trace since any $t_i$
  cannot be the right subterm of $\cond$. This contradicts $t\in \GTC$.
  
  The case of $n>0$. By the leastness of $n$, for any $n'<n$, all $n'$-level of $t$ are finite.
  We can also take an infinite path from 
  the binary tree that is a restriction of $t$ with nodes until the $n$-safe level of $t$. 
  The path has the form $\vec{t_0},\vec{t_1},\ldots,\vec{t_n}$,
  where each $\vec{t_i}$ is a sequence of nodes of the $i$-safe level,
  and only $\vec{t_n}$ is infinite.
  Then this infinite path does not contain a progressing trace as in the case of $n=0$. 
  This also contradicts $t\in \GTC$.
  
  Hence we have the finiteness of the $n$-safe level of $t\in\GTC$, as we wished. 
\end{proof}

We write $\Lv{n}{t}$ for the number of the $n$-safe level of $t\in\GTC$. 

We define a $n$-safe level context $\Lctx{n}$
with holes (written $\cdot$) that will be filled with terms in $\LAMBDA$,
and whose positions are at the $n$-safe level. 
\begin{definition}[Safe level context]
  We define $\Lctx{-1} = \cdot$. 
  For $n \ge 0$, the $n$-safe level context, written $\Lctx{n}$, is inductively defined as
  follows:
  \[
  \Lctx{n} ::= x^T \mid 0 \mid \Lctx{n}\Lctx{n} \mid \lambda x^T.\Lctx{n}
  \mid \Suc{\Lctx{n}} \mid \Cond{\Lctx{n}}{\Lctx{n-1}}. 
  \]
\end{definition}
Multiple holes may appear in a context and each holes are distinguished.
The resulting term obtained by filling $k$-holes in $\Lctx{n}$
with terms $t_1,\ldots,t_k$ is written $\Lctx{n}[t_1,\ldots,t_k]$. 
Note that filling holes with terms is not substitution,
but just putting terms at the positions of holes, namely, for example,
the result of filling the unique hole in $\lambda x.\Cond{0}{\cdot}$ with $x$
is $\lambda x.\Cond{0}{x}$. 

The finiteness of the $n$-safe level of $t\in\GTC$ enables 
to split $t$ by a $n$-safe level context and terms that apper at level $>n$
as stated in the next lemma. 

\begin{lemma}\label{lem:split_context}
  \begin{enumerate}
  \item\label{lem:split_context1}
    For any $\Lctx{n}$, there uniquely exists
    $(\Lctx{0},\Lctx{n-1}^{1},\ldots,\Lctx{n-1}^{k})$ such that
    $\Lctx{0}$ has $k$-holes and
    $\Lctx{n} = \Lctx{0}[\Lctx{n-1}^1,\ldots,\Lctx{n-1}^k]$, namely $\Lctx{n}[\vec{t_1},\ldots,\vec{t_k}] = \Lctx{0}[\Lctx{n-1}^1[\vec{t_1}],\ldots,\Lctx{n-1}^k[\vec{t_k}]]$ holds for any $\vec{t_1},\ldots,\vec{t_k}$. 
  \item\label{lem:split_context2}
    Let $t\in\GTC$ and $n\ge 0$.
    There uniquely exists $(\Lctx{n},t_1,\ldots,t_k)$
    such that $\Lctx{n}$ has $k$-holes, $t = \Lctx{n}[t_1,\ldots,t_k]$,
    and all $t_1,\ldots,t_k$ appear at the $(n+1)$-safe level of $t$.
  \end{enumerate}
\end{lemma}
\begin{proof}
  The point~\ref{lem:split_context1} is shown by induction on the construction of $\Lctx{n}$.
  If $\Lctx{n}$ is $x^T$ or $0$, then define $\Lctx{0}$ by the same one as $\Lctx{n}$. 
  For the case of $\Lctx{n} = \lambda x^T.\Lctx{n}'$, 
  we have a unique $(\Lctx{0}',\overrightarrow{\Lctx{n-1}'})$ such that 
  $\Lctx{n}' = \Lctx{0}'[\overrightarrow{\Lctx{n-1}'}]$.
  Then $(\lambda x^T.\Lctx{0}',\overrightarrow{\Lctx{n-1}'})$
  satisfies the expected condition for $\Lctx{n}$ of this case. 
  The cases of $\Lctx{n} = \Lctx{n}'\Lctx{n}''$ and $\Lctx{n} = \Suc{\Lctx{n}'}$
  are shown similarly by using the induction hypothesis.
  We show the case $\Lctx{n} = \Cond{\Lctx{n}'}{\Lctx{n-1}''}$. 
  By the induction hypothesis, there exists a unique
  $(\Lctx{0}',\overrightarrow{\Lctx{n-1}'})$ such that 
  $\Lctx{n}' = \Lctx{0}'[\overrightarrow{\Lctx{n-1}'}]$.
  Then $(\Cond{\Lctx{0}'}{\cdot}, \overrightarrow{\Lctx{n-1}'}, \Lctx{n-1}'')$
  satisfies the expected condition for $\Lctx{n}$ of this case. 
  
  The point~\ref{lem:split_context2} is shown
  by induction on $n$ using \ref{lem:split_context1}.

  We first show the case of $n=0$ by induction on the number of the $0$-safe level of $t$.
  If $t = x^T$ or $t = 0$, it is shown by taking $\Lctx{0}$ as $t$.
  If $t = \lambda x^T.t'$, by the induction hypothesis,
  there uniquely exists $(\Lctx{0}',\vec{t'})$ that satisfies the condition for $t'$. 
  Then $(\lambda x^T.\Lctx{0}',\vec{t'})$ satisfies
  the expected condition for $\lambda x^T.t'$. 
  If $t =t't''$ or $t = \Suc{t'}$, it is also shown by the induction hypothesis. 
  If $t = \Cond{t'}{f}$, by the induction hypothesis, 
  there uniquely exists $(\Lctx{0}',\vec{t'})$ that satisfies the condition for $t'$. 
  Then $(\Cond{\Lctx{0}'}{\cdot},\vec{t'},f)$ satisfies
  the expected condition for $\Cond{t'}{f}$. 

  Then we show the case of $n>0$.
  By the result of the case of $n=0$, there uniquely exists $(\Lctx{0},t_1,\ldots,t_k)$
  that satisfies $t = \Lctx{0}[t_1,\ldots,t_k]$ and
  each $t_i$ appears at the $1$-safe level of $t$.
  For each $i$, by applying the induction hypothesis to $t_i$,
  we have unique $(\Lctx{0}^i,\vec{t'_i})$
  satisfies $t_i = \Lctx{0}^i[\vec{t'_i}]$ and $\vec{t'_i}$ appear
  at the $(n-1)$-safe level of $t_i$. 
  Then $(\Lctx{0}[\Lctx{n-1}^1,\ldots,\Lctx{n-1}^k],\vec{t'_1},\ldots,\vec{t'_k})$
  satisfies the expected condition for $t$.
  Its uniqueness is obtained by using the uniqueness of $\Lctx{0}$ and $\Lctx{n-1}^i$,
  and the point \ref{lem:split_context1}. 
\end{proof}

We define the notion \quotationMarks{$n$-safe equality} that intuitively means
two terms are approximately equal excepting for deeper safe levels more than $n$. 

\begin{definition}[$n$-safe equal]
  $t_1 \nequal{n} t_2$, or \quotationMarks{$t_1$ and $t_2$ are $n$-safe equal}
  is defined by after possibly renaming the bound variables of $t_1$ and $t_2$, 
  they have the forms $\Lctx{n}[\vec{u}_1]$ and $\Lctx{n}[\vec{u}_2]$
  with some terms $\vec{u}_1$ and $\vec{u}_2$. 
\end{definition}

The $n$-safe equality is necessary to fill \quotationMarks{gaps}
after $n$-safe reductions from the same term.
For example, let $t$ be $(\lambda x^{\N\to\N}.\Cond{0}{x})(IS)$,
where $I = \lambda z^{\N\to\N}.z$ and $S = \lambda n^\N.\Suc{n}$.
For the two $0$-safe reducuctions 
$t \nsafeReducesAst{0} \Cond{0}{S}$ and $t \nsafeReducesAst{0} \Cond{0}{IS}$, 
we have $\Cond{0}{IS} \nequal{0} \Cond{0}{S}$
since $\Cond{0}{IS} \nsafeReduces{0} \Cond{0}{S}$ does not hold. 

The main theorem of this section is a weak form of Church-Rosser
of the $n$-safe reductions that holds up to the $n$-equalities. 

\begin{theorem}[Weak Church-Rosser of $n$-safe reduction modulo $n$-safe equality]
  Let $t\in \GTC$.
  If $t_1 \nsafeReducesAstL{n} t \nsafeReducesAst{n} t_2$, 
  then there exist $t'_1$ and $t'_2$ such that
  $t_1 \nsafeReducesAst{n} t'_1 \nequal{n} t'_2 \nsafeReducesAstL{n} t_2$. 
\end{theorem}

This theorem will be proved by a variant of the parallel reduction technique. 

\begin{definition}[Safe parallel reduction]
  Let $n\ge -1$.
  We define the $n$-safe parallel reduction relation $\nsafePReduces{n}$ on the $\GTC$ terms.
  The $\nsafePReduces{-1}$ is defined by $t\nsafePReduces{-1} t$ for all $t\in\GTC$. 
  For $n\ge 0$, the relation $\nsafePReduces{n}$ is inductively defined as follows: 
  \begin{itemize}
  \item[(id)]
    $t \nsafePReduces{n} t$.
  \item[$(\Succ)$]
    If $t \nsafePReduces{n} t'$, then $\Suc{t} \nsafePReduces{n} \Suc{t'}$.
  \item[$(\lambda)$]
    If $t \nsafePReduces{n} t'$, then $\lambda x^T.t \nsafePReduces{n} \lambda x^T.t'$.
  \item[(ap)]
    If $f \nsafePReduces{n} f'$ and $a \nsafePReduces{n} a'$,
    then $f(a) \nsafePReduces{n} f'(a')$.
  \item[$(\cond)$]
    If $a \nsafePReduces{n} a'$ and $f \nsafePReduces{n-1} f'$,
    then $\Cond{a}{f} \nsafePReduces{n} \Cond{a'}{f'}$.
  \item[$(\beta)$]
    If $t \nsafePReduces{n} t'$ and $u \nsafePReduces{n} u'$,
    then $(\lambda x^T.t)u \nsafePReduces{n} t'[u'/x]$.
  \item[$(\cond\,0)$]
    If $a \nsafePReduces{n} a'$, 
    then $\Cond{a}{f}(0) \nsafePReduces{n} a'$.
  \item[$(\cond\,\Succ)$]
    If $f \nsafePReduces{n-1} f'$ and $u \nsafePReduces{n} u'$, 
    then $\Cond{a}{f}(\Suc{u}) \nsafePReduces{n} f'(u')$.
  \end{itemize}
\end{definition}

The relation $\nsafePReduces{-1}$ is a special case to simplify
the rules $(\cond)$ and $(\cond\,\Succ)$.
For $n\ge 0$, if $t$ is reduced to $u$ by the $n$-safe parallel reduction,
redexes that appear in the $n$-safe level of $t$ can be reduced. 
For example, $\Cond{I0}{\Cond{I0}{IS}} \nsafePReduces{n} \Cond{0}{\Cond{0}{S}}$
holds if $n=2$, but does not hold if $n=1$. 
The case (id) cannot be replaced by $x^T \nsafePReduces{n} x^T$ and $0 \nsafePReduces{n} 0$ expecting (id) can derived from them, but it is not the case in our setting
because we are working on the infinite terms. 

The relation $\nsafePReduces{n}$ satisfies the following basic properties. 
\begin{lemma}\label{lem:parallel_basic}
  Let $t, u\in \GTC$. The following properties hold. 
  \begin{enumerate}
  \item\label{lem:parallel_basic1}
    $n < n'$ and $t \nsafePReduces{n} u$ implies $t \nsafePReduces{n'} u$ (monotonicity). 
  \item\label{lem:parallel_basic2}
    If $t \nsafeReduces{n} u$, then $t \nsafePReduces{n} u$. 
  \item\label{lem:parallel_basic3}
    If $t \nsafePReduces{n} u$, then $t \nsafeReducesAst{n} u$. 
  \end{enumerate}
\end{lemma}
\begin{proof}
  The point~\ref{lem:parallel_basic1} is shown by induction on $\nsafePReduces{n}$.
  We show the point~\ref{lem:parallel_basic2}.
  
  

\end{proof}


\newpage
\section{Terms with the Global Trace Condition are Finite for Safe Reductions}

\label{section-finite-safe-reductions}
In this section we prove that for all $n \in \N$, every infinite reduction sequence $\pi$ 
from some term of $\GTC$ includes
only finitely many \quotationMarks{safe} reduction steps.    
In particular,
this implies that in $\GTC$ the reduction sequences made of only \quotationMarks{$n$-safe} reductions 
are finite. That is, this implies strong normalization for \quotationMarks{$n$-safe} reduction steps:
no matter how we reduce within the safe level of a term, eventually we obtain some safe normal form.
\\

We introduce the property of being \quotationMarks{finite for $n$-safe reductions}.

\begin{definition}
\label{definition-finite-n-safe-reduction}
Assume $t \in \CTlambda$. We say that $t$ is \emph{finite for $n$-safe reductions} if all infinite
reduction paths include only finitely many $n$-safe reductions.
\end{definition}

In the following, we explicitly write $t[x_1,\ldots,x_n]$,
when each free variable in a term $t$ is some $x_i$, 
and, under this notation, we also write $t[a_1,\ldots,a_n]$ instead of $t[a_1/x_1,\ldots,a_n/x_n]$. 

It is enough to consider the case $n=0$. 
We abbreviate \quotationMarks{$0$-safe} with \quotationMarks{safe}.
We have to prove that all terms in $\GTC$ are finite for safe reductions, for all $n \in \N$.
For, we define total well-typed terms by induction on the type, as in Tait's normalization proof.

%We recall that $u \in \LAMBDA$ is \emph{finite for safe reduction} 
%if and only if all infinite reduction sequences from $t$ include only finitely many "safe" reduction steps
%(Def. \ref{definition-safe-trunk}).

\begin{definition}[Total well-typed terms of $\LAMBDA$]
\label{definition-total-term}
Let $t \in \WTyped$ and $t : T$.
We define \quotationMarks{$t$ is total of type $T$} by induction on $T$

\begin{enumerate}
\item
Let $T$ be any atomic type: $T = \N$ or $=\alpha$ for some type variable $\alpha$.
Then $t$ is total of type $T$ if and only if $t$ is finite for safe reductions, for all $n \in \N$.

\item
Let $T$ be any arrow type $A \rightarrow B$.
Then $f$ is total of type $T$ if and only if for all total $a$ of type $A$ we have $f(a)$ total of type $B$.
\end{enumerate}

A \emph{total assignment} $[\vec{x}/\vec{v}]$ 
is any assignment of total terms to variables of the same type.

A term $t$ is \emph{total by substitution} if and only if:
$t[\vec{x}/\vec{v}]$ is total for all total assignments $[\vec{x}/\vec{v}]$
\end{definition}

By definition, any total term is well-typed, in particular it has exactly one type. 

We define a well-founded relation predecessor relation on terms finite for safe reductions and of type $\N$.

\begin{definition}[The $\Succ$-order]
Assume $t, u \in \WTyped$ and $t, u :\N$ (possibly open terms).
Then:
$$
(u \prec t) \Leftrightarrow (t \reduces^* \Succ(u))
$$
\end{definition}


\begin{Eg}
If $t = \Succ^2(x)$ then $x \prec \Succ(x) \prec t$. 
If $t = \Succ(t)$ then $t \prec t$ and $\prec$ is \emph{not} well-founded on $t$:
\end{Eg}

We did not prove Church-Rosser yet, 
therefore we ignore whether all decreasing sequences of $\prec$ have the same length. 

However, we can prove that if $t$ is a term finite for safe reductions, 
then any $\prec$-decreasing sequence from $t$ terminates.


\begin{lemma}[The $\prec$-order]
\label{lemma-prec-order}
Assume $n \in \N$, $t \in \WTyped$ has type $\N$, and $t$ is finite for safe reductions .

\begin{enumerate}
\item
\label{lemma-prec-order-01}
There is no infinite sequence 
$\sigma: t = t_0 \reduces \Succ(t_1) \reduces \Succ^2(t_2) \reduces \ldots$

\item
\label{lemma-prec-order-02}
$\prec$ is well-founded on total terms of type $\N$.
\end{enumerate}
\end{lemma}


\begin{proof}
\begin{enumerate}
\item
%\label{lemma-prec-order-01}
$t$ is finite for safe-reductions, therefore
$\sigma$ only has finitely many safe-reductions. 
Thus, from some $k\in\N$ there are no more safe reductions from
$\Succ^k(t_k)$. This implies that for some $\cond$-free term 
$u$ and some terms $f_1, g_1, \ldots, f_m, g_m$ we have
$t_n = u[\cond(f_1,g_1), \ldots, \cond(f_m,g_m)]$ and all reductions from $t_k$ on are inside
some $g_1, \ldots, g_m$. This means for all $h \in \N$, $h \ge k$ we  have
$\Succ^{h}(t_{h}) =  u[\cond(f_1,g'_1), \ldots, \cond(f_n,g'_m)]$ for some 
$g'_1, \ldots, g'_m$. This implies that first $h$ symbols of $u$ are $\Succ$.
This is a contradiction when $h$ is larger than the number of symbols in $u$.

\item
%\label{lemma-prec-order-02}
Assume for contradiction that there is some infinite sequence
$\ldots t_n \prec \ldots \prec t_2 \prec t_1 \prec t_0$
from some $t_0:\N$ total. By definition, $t_0$ is finite for safe reductions.
Then there is some infinite sequence 
$\sigma: t = u_0 \reduces \Succ(u_1) \reduces \Succ^2(u_2) \reduces \ldots$,
contradicting point \ref{lemma-prec-order-01} above.
\end{enumerate}
\end{proof}


We will prove that if $t$ is not total, then we can assign total terms to  the sub-terms of $t$
in an infinite path of a proof $\Pi : \Gamma \vdash t: A$ in a way compatible with traces.


\begin{definition}[Trace-compatible Assignment]
\label{definition-trace-compatible}
Assume $\pi  = (\Gamma_1 \vdash t_1:A_1, \ldots, \Gamma_n \vdash t_n:A_n, \ldots)$ 
is any finite or infinite branch of a typing proof $\Pi$
and $\vec{v} = (\vec{v_1}, \ldots, \vec{v_n}, \ldots)$ 
is any sequence of total assignments, one on each $t_i$. 
$\vec{v}$ is \emph{trace-compatible} in an index $i$ of $\pi$  
if and only if it satisfies the following condition:

  for all $j$  index of an $\N$-argument of $t_i$, 
  all $k$ index of an $\N$-argument of $t_{i+1}$, 
  if $j$ is connected to $k$ then:
 \begin{enumerate}
 \item
 if $j$ progresses to $k$ then $v_j \prec v_k$ 
 \item
 if $j$ does not progress to $k$ then $v_j = v_k$.
 \end{enumerate}
$\vec{v}$ is \emph{trace-compatible} if it is trace-compatible in all $i$.
\end{definition}

Now we will prove that if an infinite branch $\pi$ of a proof tree $\Pi:\Gamma \vdash t:A$ 
has a trace-compatible assignment made of total terms, 
then all traces $\sigma$ of $\pi$ progress only finitely many times and the term $t$ is not in $\GTC$
($t$ does not satisfy the global trace condition).



\begin{proposition}[Trace assignment]
\label{prop:trace_assign-finiteness}
Assume $\Pi:\Gamma \vdash t:A$ and there is some infinite path $\pi$ of $\Pi$ for which we have
some \emph{total} trace-compatible assignment  $\rho$ to $\pi$. 
\begin{enumerate}
\item
\label{prop:trace_assign-finiteness1}
Any trace $\sigma$ in $\pi$ progresses only finitely many times.
\item
\label{prop:trace_assign-finiteness2}
$t \not \in \GTC$.
\end{enumerate}
\end{proposition}



\begin{proof}
\begin{enumerate}
\item
%\label{prop:trace_assign-finiteness1}
By definition of trace-compatible assignment, if at step $i \in \N$ the trace $\sigma$ progresses, 
then $\sigma(i+1)\prec \sigma(i)$, and if $\sigma$ does not progress, 
then $\sigma(i+1) = \sigma(i)$
The assignment is made of total terms, therefore
 $\prec$ is well-founded by lemma \ref{lemma-prec-order}.\ref{lemma-prec-order-02}.
Thus, any  trace in $\pi$ progresses only finitely many times, as we wished to show.

\item
%\label{prop:trace_assign-finiteness2}
By point \ref{prop:trace_assign-finiteness1} above, no trace $\sigma$ 
from any argument in any term of the branch $\pi$ of $\Tree(t)$ progresses infinitely many times.
We assumed that $\pi$ is an infinite path in $\Pi$.
By definition of $\GTC$, we conclude that $t \not \in \GTC$. 
\end{enumerate}
\end{proof}

%16:46 04/09/2024

We now continue with our Tait's-style proof of Strong Normalization.
We check that total terms are closed by reductions, by application and by variables.
Any total term is finite for safe reductions.



\begin{lemma}\label{lem:total_value-finiteness}
Assume $t,u,f,a \in \WTyped$, $n \in \N$ and $A,B,T$ are types.

  \begin{enumerate}
  \item
\label{lem:total_value-finiteness1}
    Let $t:A$ and $t \reduces^* u$.
    If $t$ is total, then $u$ is total.

  \item
\label{lem:total_value-finiteness2}
    If $f:A \rightarrow B$ and $a:A$ are total terms, then $f(a)$ is total.

  \item
\label{lem:total_value-finiteness2bis}
    $x^T:T$ is total.

 \item
\label{lem:total_value-finiteness2ter}
  If $t$ is total then $t$ is  finite for safe reductions.

  \item
\label{lem:total_value-finiteness3}
    Let $T$ be any atomic type, and $t[\vec{x}]:\vec{A}\rightarrow T$ be a term
    whose all free variables are $\vec{x}:\vec{B}$.

    If for all total $\vec{u}:\vec{B}$, $\vec{a}:\vec{A}$ the term 
    $t[\vec{u}]\vec{a}:\N$ is \emph{finite for safe reductions}, then
    the term $t[\vec{x}]$ is \emph{total by substitution}.
  \end{enumerate}

\end{lemma}




\begin{proof}
\begin{enumerate}

\item
%\label{lem:total_value-finiteness1}
  We show \emph{point \ref{lem:total_value-finiteness1}}  by induction on $A$. 
  We assume that $t:A$ and $t \reduces ^*u$.
    and $t$ is total, in order to prove that $u$ is total.
 By the subject reduction property, $u$ has type $A$.
\begin{enumerate}
\item
  We show the \emph{base case}, namely when $A =\N,\alpha$ is an atomic type.
  By the assumption, $t$ is total.
  By definition of $t$ total for $T$ atomic, all infinite 
  reductions from $t$ only include finitely many safe reductions, for all $n \in \N$.
  In particular, all infinite reductions $\sigma: t \reduces \ldots \reduces 
  u \reduces u_1 \reduces u_2 \ldots$ 
  passing through $u$ only include finitely many safe reductions. We conclude that
  all infinite reductions 
  $\sigma': u \reduces u_1 \reduces u_2 \ldots$  from $u$
  only include finitely many safe reductions. From $u:A =\N,\alpha$ we conclude that  $u$ is total.
\item
  We show the \emph{induction case}, namely when $A = (A_1\rightarrow A_2)$.
  Take any arbitrary total term $a:A_1$ in order to prove that $u(a):A_2$ is total. 
  Then we have $t(a) \reduces^* u(a)$ and 
  $t(a):A_2$ is total by the assumption that $t$ is total.
  Hence $u(a)$ is total by $t(a) \reduces^* u(a)$ and the induction hypothesis on $A_2$. 
  We conclude that $u:A_1\rightarrow A_2$ is total. 
\end{enumerate}

  \item
%\label{lem:total_value-finiteness2}
If $f:A \rightarrow B$, $a:A$ are total  terms, then $f(a)$  is total by definition of total.

\item
%\label{lem:total_value-finiteness3}
We prove that $x^T:T$ is total. 
We actually prove a little more, 
that for all total $\vec{a}:\vec{A}$, if $T = \vec{A} \rightarrow U$ then $x(\vec{a}):U$
is total. The thesis follows if we take $\vec{a} = \nil$. We argue by induction on $U$. 

\begin{enumerate}
\item
Assume $U$ is atomic. Then every reduction on $x(\vec{a}):U$
takes place in $\vec{a}$. By definition of total
for an atomic type we have to prove that in all infinite reduction sequences from $x(\vec{a}):U$ 
there are only finitely many safe reduction. All reductions on $x(\vec{a})$ take place on $\vec{a}$,
and since each $a_i$ in $\vec{a}$ is total only finitely many safe reductions are possible, as we wished.
\item
Assume $U = (A_1 \ldots A_2)$. By definition of total
for an arrow type we have to prove that for all total $a$ we have  $x(\vec{a},a):A_2$ total.
This follows by induction hypothesis on $A_2$.
\end{enumerate}

\item
%\label{lem:total_value-finiteness4}
\emph{We assume that  $t:U$ is total in order to prove that $t$ is finite for safe reductions},
for all $n \in \N$.
We argue by induction on $U$.
\begin{enumerate}
\item
If $U$ is atomic then the thesis is true by definition of total.
\item
Suppose $U = (A_1 \rightarrow A_2)$. By point \ref{lem:total_value-finiteness3} above,
$x^{A_1}:A_1$ is total, therefore $t(x):A_2$ is total and by induction hypothesis on $A_2$
any infinite reduction sequence from $t(x)$ only includes finitely many safe reductions. 
Any infinite reduction sequence 
$\sigma: t = t_0 \reduces_1 t_1 \reduces_1 t_2 \reduces_1 \ldots$  from $t$ 
can be raised to an infinite reduction sequence 
$\tau: t = t_0(x) \reduces_1 t_1(x) \reduces_1 t_2(x) \reduces_1 \ldots$ from $t(x)$
while preserving the fact that a reduction is safe, because $t$ occurs in no $\cond$ in $t(x)$.
We conclude that $\sigma$ only includes finitely many safe reductions. 
\end{enumerate}

\item  
%\label{lem:total_value-finiteness5}
We show that $t[\vec{u}]$ is total by induction on the number $|\vec{A}|$ of
elements of $\vec{A}$.
\begin{enumerate}
\item
  The \emph{base case} $|\vec{A}| = 0$ is immediately shown by the assumption.
\item
  We show the \emph{induction case}. Let $\vec{A} = A_0,\vec{A'}$.
  Take arbitrary values $\vec{u}:\vec{B}$, $\vec{a'}:\vec{A'}$, and $a_0:A_0$. 
  By the assumption, we have that $t[\vec{u}]a_0\vec{a'}:\N$ is total  for all 
  vector of total terms $\vec{a'}$. 
  Then $t[\vec{u}]a_0:\vec{A'}\rightarrow\N$ is total for all total $a_0:A_0$
  by the induction hypothesis on $\vec{A'}\rightarrow \N$.
  By definition of total we deduce that $t[\vec{u}] : A_0,\vec{A'}\rightarrow\N$ is total.
\end{enumerate}
We conclude that $t[\vec{x}]$ is total by substitution, as we wished to show.

\end{enumerate}
\end{proof}

%10:28 19/04/2024
%17:32 24/04/2024

Let $\Pi=(T,\phi)$ be a proof of $\Gamma\vdash t:A$ and $l_n$ be a node of $\Pi$, that is, a list  
of integers $l_n=(e_1, \ldots, e_{n-1}) \in \universe{\Pi}$ for some $n \in \N$.
We write $\Gamma_{n}\vdash t_{n}:A_{n} = \Label(\Pi,l_n)$ for the sequent
 labelling the node $l_n$. We want to prove that all terms of $\GTC$ are total by substitution.
If we consider the substitution of a variable with itself 
(a variable is a total term by \ref{lem:total_value-finiteness}.\ref{lem:total_value-finiteness3}),
we will deduce that all all terms of $\GTC$ are total, 
hence finite for safe reductions, hence strongly normalizing for safe reductions.

One problem in the proof of the theorem 
is that reduction does not commute with the second argument of substitution. 
That is, if $a \reduces^* a'$ we cannot deduce that $v[a] \reduces^* v[a']$. 
The reason is that we could have infinitely many free occurrences of $a$ in $v[a]$, and it could take
an infinite number of steps to reduce each of these $a$ to $a'$.

However, we can prove a weaker property, expressed by the following Lemma.


\begin{lemma}[Safe-infinite reductions and Substitution]
 \label{lemma-safe-infinite-substitution}
 Assume $a \reduces^* a'$ and there is some safe-infinite reduction $\sigma$ from $v[a']$.
 Then there is some safe-infinite reduction from $v[a]$.
\end{lemma}

\begin{proof}
We prove that if $v[a'] \reduces w$ then for some $z$ we have $w=z[a']$ and $v[a] \reduces^+ z[a]$,
and if $v[a'] \reduces z[a']$ is a safe then the last reduction in $v[a] \reduces^+ z[a]$ is safe.
The proof is by cases on the reduction $v[a'] \reduces w$.  We argue by cases.

\begin{enumerate}

\item
Assume the reduction $v[a'] \reduces w$ is on a redex of the form $r[a']$, with both the left
and right hand-side of $r$ not included in any $a'$ obtained by substitution. 
In this case we have $v[a'] = e[r[a'],a']
\reduces e[s'[a],a']$, and
$r[a'] = \cond(f[a'],g[a'])(0) \reduces f[a'] = s[a']$ or 
$r[a'] = \cond(f[a'],g[a'])(S(t[a])) \reduces g[a'](t[a']) = s[a']$
or $r[a'] = (\lambda x.t[a',x])(u[a']) \reduces t[a',u[a']] = s[a']$. 
We define $z[a'] = e[s'[a],a']$. 
In all three sub-cases we have $r[a] \reduces s[a]$ and therefore $v[a] \reduces e[r[a],a] = z[a]$
if $v[a'] \reduces z[a']$ is safe then $v[a] \reduces z[a]$ is safe.

\item
Assume the reduction $v[a'] \reduces w$ is on a redex of the form $r$, with either the left
or right hand-side of $r$ not included in some $a'$. In this case either $r$ is included in some $a'$ 
obtained by substitution, or the left or right hand side of $r$ is equal to some $a'$ obtained by substitution.
We cannot have both the left and right hand side equal to $a$, because the two sides of any redex 
have a different type.

In this case there is a single $a'$ whose value matters, that is, $v[a] = v_0[a][a]$, with the first $a$ denoting
the single occurrence of $a$ we are speaking about. We choose $v'[.] = v_0[a'][.]$, then we have
$v'[a']=v_0[a'][a']=v[a']$ by definition, and $v[a] = v_0[a][a] \reduces^* v_0[a'][a] = v'[a]$ 
because $a \reduces^* a'$ and there is a single occurrence of $a$ in $v_0[a][.]$. 

The reduction on $v'[a']$ has both the left
and right hand-side of $r$ not included in any $a'$ obtained by substitution, because this unique
$a'$ is not part of $v'[.]$. We conclude from the previous case applied to $v'[a]$, $v'[a']$.

\end{enumerate}
Then by induction on $n' \in \N$ we prove that if $v[a'] \reduces^{n'} w$ ($n$ steps) 
then for some $z$, some $n \ge n'$ we have $w=z[a']$ and $v[a] \reduces^{n} z[a]$,
with the number of safe reductions in  $v[a'] \reduces^{n'} z[a']$ less or equal than the number
of safe reductions in $v[a] \reduces^{n} z[a]$.

\end{proof}

The Lemma above will be enough to prove our Main Theorem.

%08:13 05/09/2024

\begin{theorem}[Main Theorem]
\label{theorem-main-finite-safe-reduction}
  Assume $\Pi:\Gamma\vdash t : A$ (hence $t\in \WTyped$).
  If $t$ is \emph{not} total by substitution, then $t \not \in \GTC$, i.e.:
  there is some infinite path $\pi = (e_1, e_2, \ldots)$ of $\Pi$ with no infinite progressing trace. 
\end{theorem}

%19:01 16/04/2024
%18:11 30/04/2024
%13:39 04/06/2024


\begin{proof}
  Assume that $t$ is not total by substitution. 
  Let $\vec{x}:\vec{D}$ and $\vec{A}\rightarrow U$ for some \emph{atomic} $U$ 
 be $\Gamma$ and $A$, respectively.
  By Proposition \ref{prop:trace_assign-finiteness}.\ref{prop:trace_assign-finiteness2} it is enough to prove that
  $\Pi$ has some infinite branch $\pi=(e_1, e_2, \ldots)$ 
  and some \emph{total} trace-compatible assignment $\rho$ for $\pi$.
  By case \ref{lem:total_value-finiteness3} of Lemma~\ref{lem:total_value-finiteness},
  there exist total terms $\vec{a}:\vec{A}$ and $\vec{d}:\vec{D}$ for which there is
  an infinite reduction $\sigma$ from $t[\vec{d}]\vec{a}$ having infinitely many safe reductions.
  By induction on $i \in \Nat$, for each $i$, we construct a path 
  $l_i = (e_1,\ldots,e_{i-1}) \in \universe{\Pi}$
  and a total assignment $\vec{v_i} = (\vec{d_i},\vec{a_i})$ such that
  $(\vec{v_1},\ldots,\vec{v_i})$ is a total trace-compatible assignment for the node $l_i$. 

%  We have to find some $(l_i,\vec{d_i},\vec{a_i})$ such that:
%  \begin{itemize}
%  \item[(i)]
%%15:38 19/04/2024
%   $l_i = (e_1, \ldots, e_{i-1}) \in \universe{\Pi}$ 
%     %and $\Label(\Pi,l_i) = \vec{x_{i}}:\vec{D_{i}}\vdash t_{i} : \vec{A_{i}}\rightarrow\N$; 
%%  \item[(ii)]
%%    $t_{i}$ is not total by substitution;
%  \item[(ii)]
%    $\vec{d_{i}}:\vec{D_i}$ and $\vec{a_i}:\vec{A_i}$ are total terms
%    such that $t_{i}[\vec{d_i}]\vec{a_i}$ is not total;
%  \item[(iii)]
%    the total assignment 
%    $\vec{d_1},\vec{a_1}$, \ldots, $\vec{d_i},\vec{a_i}$) is trace-compatible for $l_i$.
%    %\Daisuke{mynote:write this more clearly}
%  \end{itemize}
  
 We recall that \quotationMarks{trace compatible in $i$} means: 
 if $i-1$ is a progress point, namely if $t_{i-1}=\cond(f,g)$ and $e_i=2$ 
    and $t_i=g$,
    then $\vec{a_i} = a',\vec{a'}$ 
    and $a'$, the first unnamed argument of $\cond(f,g):\N \rightarrow A$, reduces to $\Succ(a'')$
     while $\vec{d_i} = \vec{d_i},a''$, 
     the last named argument of $g$ is $1$ unit smaller.
  In all other cases two corresponding arguments are equal.

  We first define $(l_1,\vec{d_1},\vec{a_1})$ for the root node $t$ of $\Pi$.
  In this case $l_1 = \nil$ 
   and $(\vec{d},\vec{a})$ are total terms such that $t[\vec{d}](\vec{a})$ is not total.
  Trace compatibility is vacuous because the branch $l_1$ does not contain two nodes.
  %Points (i), (ii), (iii), (iv) are immediate.

  Next, assume that $(l_i,\vec{d_i},\vec{a_i})$ is already constructed.
  Then we define $(l_{i+1},\vec{d_{i+1}},\vec{a_{i+1}})$ by the case analysis on
  the last rule for the node $l_i$ in $\Pi$. 

%%%%%%%%%%%%%%%%%%%%%%%%%%%%%%%%
% APPARENTLY THE CASE $\struct(f)$ CAN BE SIMPLIFIED TO WEAKENING-Stefano
%%%%%%%%%%%%%%%%%%%%%%%%%%%%%%%%



\begin{enumerate}

\item
%WEAK
  The case of $\weak$, namely
  $\Label(\Pi,l_i) = \Gamma'\vdash t_{i+1}:\vec{A_i}\rightarrow\N$
  is obtained from the induction hypothesis for
  $\Gamma\vdash t_{i}:\vec{A_i}\rightarrow\N$. We have:

\begin{enumerate}
\item
 $t_i = t_{i+1}$. 
\item
  $\Gamma = x_1:D_1,\ldots,x_n:D_n$, $\Gamma' = x'_1:D'_1,\ldots,x'_m:D'_m$, and $\Gamma \subseteqsim \Gamma'$
  with an injection $\phi:\{1,\ldots,n\}\to\{1,\ldots,m\}$ between contexts. 
\item
  $x_i = x'_{\phi(i)}$ and $D_i = D'_{\phi(i)}$ for all $i \in \{1,\ldots,n\}$.
\end{enumerate}

  By the induction hypothesis, 
  $t_i[d_{i,\phi(1)}/x_1,\ldots,d_{i,\phi(n)}/x_n]\vec{a_i} 
   = t_i[d_{i,1}/x'_1,\ldots,d_{i,m}/x'_m]\vec{a_i}$ is not total,
  where $\vec{d_i} = d_{i,1}\ldots d_{i,m}$.
  Then we define $l_{i+1} = l_i\conc(1)$ taking the unique child node of $l_i$ in $\Pi$, and we
  also define $\vec{d_{i+1}}$ and $\vec{a_{i+1}}$ by $d_{i,\phi(1)}\ldots d_{i,\phi(n)}$
  and $\vec{a_i}$, respectively. This is an assignment with total terms.
  $((\vec{d_i},\vec{a_i}),(\vec{d_{i+1}},\vec{a_{i+1}}))$
  is trace compatible for $(t_i,t_{i+1})$: if two arguments are connected then they are assigned
  to the same term. 

\item
%VAR
  The case of $\var$-rule, namely $\Label(\Pi,l_i) = \Gamma\vdash x:D$ 
  for some $x_{i,k}:D_{i,k} = x:D \in \Gamma$
  cannot be, because $t_i [\vec{d_i}/\vec{x_i}] = d_{i,k}$ is total  by assumption on $\vec{d}$.
  
\item
%0-RULE
  The case of $0$-rule, namely $\Label(\Pi,l_i) = \Gamma\vdash 0:\N$, 
  cannot be. Indeed, $t_i = 0$ is total because $0$ is a numeral.

%12:58 06/06/2024

\item 
%SUCC 
  The case of $\Succ$-rule, 
namely $\Label(\Pi,l_i) = \Gamma\vdash t_i: \N = \Gamma\vdash \Succ(u): \N$
  for some $u$ is obtained from our assumptions on
  $\Gamma\vdash u: \N$. In this case $\vec{a_i}$ is empty, $t_{i+1}=u$, and
  by the induction hypothesis $\Succ(u)[\vec{d_i}]:\N$ has an infinite reduction with
  infinitely many safe reductions
  $\Succ(u) \reduces  \Succ(u_1) \reduces \Succ(u_2) \reduces \ldots$,
  all taking place on $u$.
  Then, by the definition of total, $u[\vec{d_i}] =t_{i+1}[\vec{d_i}] $ is not total, because the
 $u \reduces_1  u_1 \reduces_1 u_2 \reduces_1 \ldots$ is an  an infinite reduction with
  infinitely many safe reductions.

  We define $e_{i}=1$, 
  the index of the unique child node of $l_{i+1}=l_i\conc(1)$ in the $\Succ$-rule, and
  we also define $\vec{d_{i+1}} = \vec{d_i}$ and $\vec{a_{i+1}} = ()$. 
  This is an assignment with total terms and 
  trace compatible for $(t_i,t_{i+1})$: if two arguments are connected then they are assigned
  to the same term. 

%13:05 06/06/2024


\item
  The case of the $\apnotvar$-rule, namely 
  $\Label(\Pi,l_i) = \Gamma\vdash t_i: \vec{A}\rightarrow\N$, 
  with $t_i = f[\vec{x}](u[\vec{x}])$ for some $f$ and $u$.
  The premises of the $\apnotvar$-rule
   are $\Gamma\vdash f[\vec{x}]: B \rightarrow \vec{A}\rightarrow\N$ 
  and $\Gamma\vdash u[\vec{x}]: B$, where $u$ is \underline{not} a variable.
  By the induction hypothesis, there is some infine reduction from
  $t_i[\vec{d_i}]\vec{a_i} = f[\vec{d_i}](u[\vec{d_i}])\vec{a_i}:\N$ with infinitely many safe-reductions.
  We argue by cases on the statement: \emph{$u[\vec{d_i}]:B$ is total}.


%11:04 05/06/2024



\begin{enumerate}
\item
  We first consider the subcase: \emph{$u[\vec{d_i}]:B$ is total}.
  We define $b = u[\vec{d_i}]$, then $l_{i+1}=l_i\conc(1)$, taking the first premise of the rule,
  and we define $\vec{d_{i+1}} = \vec{d_i}$ and $\vec{a_{i+1}} = b,\vec{a_i}$. 
  This is an assignment with total values, and providing an infinite reduction with infinitely many safe
  reductions, as expected. 
  The connection from 
  $(\vec{d_i},\vec{a_i})$ to $(\vec{d_{i+1}},\vec{a_{i+1}}) = (\vec{d_i},b,\vec{a_i})$ is
  trace-compatible: all connected 
  $\N$-argument of $t_{i}=f(u)[\vec{x}]$ and $t_{i+1}[\vec{x}] = f[\vec{x}]$ are the same,
  because the only fresh argument of $f[\vec{x}]$ 
  is $b$ and no argument of $f(u)[\vec{x}]$ is connected to it.


\item
  Next we consider the subcase that \emph{$u[\vec{d_i}]:B$ is not total}.
  Let $B = \vec{C}\rightarrow\N$.
  By lemma \ref{lem:total_value-finiteness}.\ref{lem:total_value-finiteness3}
  there is a sequence of values $\vec{c}:\vec{C}$ and an infinite reductions from 
  $u[\vec{d_i}]\vec{c}:\N$ with infinitely many safe reductions.
  Define $l_{i+1}= l_i \conc (2)$ taking the second premise of the rule,
  and define $\vec{d_{i+1}} = \vec{d_i}$ and $\vec{a_{i+1}} = \vec{c}$. 
  This is an assignment with total terms providing an infinite reductions with 
  infinitely many safe reductions, as expected.
  The connection from 
  $(\vec{d_i},\vec{a_i})$ to $(\vec{d_{i+1}},\vec{a_{i+1}}) = (\vec{d_i},\vec{c})$ is
  trace compatible: all connected $\N$-argument of $t_{i} = f(u)[\vec{x}]$ and $t_{i+1}=u[\vec{x}]$ are 
  in $\vec{d_i}$ and therefore are the same, and no unnamed arguments in $\vec{a_i}$
  and $\vec{a_{i+1}}$ are connected each other.
 \end{enumerate}



\item
  The case of $\apvar$-rule, namely 
  $\Label(\Pi,l_i) 
  = 
  \Gamma \vdash f[\vec{x}](x): \vec{A}\rightarrow\N$ is obtained from
  $\Gamma \vdash f[\vec{x}]: D,\vec{A} \rightarrow \N$,
  where $\Gamma=x_1:D_1,\ldots,x_m:D_m$ and $x:D\in\Gamma$, therefore
  $x:D = x_k:D_k$ for some $k \in [1,m]$. 
  By the induction hypothesis, there is an infinite reduction from $f[\vec{d_i}]d_{i,k}\vec{a_i}:\N$
  with infinitely many safe reductions,
  where $\vec{d_i} = (d_{i,1},\ldots,d_{i,m})$. 
  Define $l_{i+1}=l_i\conc(1)$ as the unique child of $l_i$ in $\Pi$, and
  also define $\vec{d_{i+1}} = \vec{d_i}$ and $\vec{a_{i+1}} = d_{i,k},\vec{a_i}$. 
  This is an assignment with total terms providing an infinite reductions with 
  infinitely many safe reductions, as expected.
  The connection from
  $(\vec{d_i},\vec{a_i})$ to
  $(\vec{d_{i+1}},\vec{a_{i+1}}) = (\vec{d_i},d_{i,k},\vec{a_i})$
  is trace compatible: all connected $\N$-arguments in $\vec{d_i},\vec{a_i}$ and 
  $\vec{d_{i+1}},\vec{a_{i+1}}$ are the same.  
  The only difference between the two assignments
  is that, if $D_k = \N$, the value $d_{i,k}$ of type $\N$ for the variable $x_k$
  in $t_i[\vec{x}]=f[x_1,\ldots,x_k,\ldots,x_m](x_k)$ is duplicated to the term $d_{i,k}$ 
  assigned to the first unnamed argument of $ f[\vec{x}]$. 

%08:40 05/09/2024

\item
  The case of $\lambda$-rule, namely when a sequent
  $\Label(\Pi,l_{i+1}) = 
    \Gamma\vdash t_i : A, \vec{A} \rightarrow \N$ with $t_i = \lambda x^A.u[\vec{x},x]$
  is obtained from
  $\Gamma,x:A\vdash t_{i+1}[\vec{x},x^A]:\vec{A}\rightarrow\N$, 
  where $t_{i+1}[\vec{x},x]=u[\vec{x},x]$.
  By the induction hypothesis we have an infinite reduction sequence $\sigma$ from
  $t_i[\vec{d_i}]\vec{a_i} = (\lambda x.(u[\vec{d_i},x]))\vec{a_i}:\N$ with infinitely many safe
  reductions,
  where $\vec{a_i} = a,\vec{b}$. 
  Define $l_{i+1}=l_i\conc(1)$ as the unique child of $l_i$ in $\Pi$,
  and $\vec{d_{i+1}} = a,\vec{d_i}$  and $\vec{a_{i+1}} = \vec{b}$. 
  This is an assignment with total terms.
    The connection from 
  $(\vec{d_i},\vec{a_i})$ to $(\vec{d_{i+1}},\vec{a_{i+1}}) = (\vec{d_i},a, \ \vec{b})$ is
  trace compatible: all connected $\N$-argument of 
  $t_{i}=\lambda x^A.u[\vec{x},x]$ and $t_{i+1}=u[\vec{x},x]$ are 
  the same, except for the first unnamend argument $a$ of $\vec{a_i}$ which is moved to
  the last named argument of $u[\vec{x},x]$, with name $x$.
  
  We have to prove that there is some infinite reduction from 
  $t_{i+1}[\vec{d_{i+1}}](\vec{a_{i+1}})$ with infinitely
  many safe reductions. We argue by case.

\begin{enumerate}
\item
 Suppose all reductions in the infinite reduction $\sigma$ 
  are inside $\lambda x.(u[\vec{d_i},x])$ or $\vec{a_i}$.
 Then there are finitely many safe reductions in $\vec{a_i}$ because $\vec{a_i}$ is total.
 Thus, there are infinitely many safe reductions on the part of $\sigma$
  taking place on $\lambda x.(u[\vec{d_i},x])$.
  We conclude that there is an infinite reduction from $u[\vec{d_i},x]$ with infinitely many safe reduction.
  This is true for $u[\vec{d_i},a]$, too, because reductions and safe reductions are closed by substitution
  of $x$ with $a$.

%09:39 05/09/2024
%non è vero che a-->a' implica v[a] ---> v[a'] ma è vero che se da v[a'] ci sono infinite safe-reductions
% allora da v[a] ci sono infinite safe-reductions
\item
 Suppose there is some reduction in $\sigma$ contracting the first $\beta$-redex.
 Then  $(\lambda x.(u[\vec{d_i},x]))a\vec{b}$ reduces first to some
 $ (\lambda x.v[x])a'\vec{b'}$, then to $v[a']\vec{b'}$, with: 
 $u[\vec{d_i},x] \reduces^* v[x]$ and $a\reduces^* a'$ and $\vec{b} \reduces^* \vec{b'}$.
 Then the reduction sequence $\sigma$ continues 
  with some infinite reduction $\sigma'$ from $v[a']\vec{b'}$, including infinitely many safe reductions. 
 From $u[\vec{d_i},x] \reduces v[x]$ and $\vec{b} \reduces^* \vec{b'}$ we deduce that 
$$
        		t_{i+1}[\vec{d_i},a]\vec{b}  
\ \ \ 		=
\ \ \     u[\vec{d_i},a]\vec{b} 
\ \ \    \reduces^*
\ \ \    v[a]\vec{b}
\ \ \    \reduces^*
 \ \ \   v[a]\vec{b'}
$$ 
 We proved that there is a reduction $\sigma'$ including infinitely many safe reductions from  $v[a']\vec{b'}$.
 From $a \reduces^* a'$ and Lemma \ref{lemma-safe-infinite-substitution} 
 we deduce that there is a reduction including infinitely many safe reductions from  $v[a]\vec{b'}$.
 We conclude that the same holds from  $t_{i+1}[\vec{d_i},a]\vec{b}$.
\end{enumerate}

%13:55 06/06/2024


\item
  The case of $\cond$-rule, namely a sequent
  $\Label(\Pi,l_i) = \Gamma\vdash t_i:\N,\vec{A}\rightarrow\N$
  having $t_i[\vec{x}] = \cond(f[\vec{x}],g[\vec{x}])$,
  and obtained from 
  $\Gamma\vdash f[\vec{x}]:\vec{A}\rightarrow\N$
  and
  $\Gamma\vdash g[\vec{x}]:\N,\vec{A}\rightarrow\N$. 
  By the induction hypothesis, there is some infinite reduction sequence $\sigma$ from
  $t_i[\vec{d_i}]a \vec{b} = \cond(f[\vec{d_i}],g[\vec{d_i}])a\vec{b}:\N$
  with infinitely many safe reductions,
  where $\vec{a_i} = a,\vec{b}$. 
  We argue by case.

\begin{enumerate}
\item
  \emph{Suppose the reduction sequence $\sigma$ never contracts the leftmost $\cond$.}

  $a$, $\vec{b}$ are total, only finitely many reduction on them are possible.
  Then infinitely many safe reductions take place on $f[\vec{d_i}]$ or $g[\vec{d_i}]$. 
  All these safe reduction are in $g[\vec{d_i}]$, because inside $g[\vec{d_i}$ 
  a reduction is in the right-hand-side of some $\cond$, therefore it is not safe.
  The safe-infinite reduction from $g[\vec{d_i}]$ can be raised to an infinite reduction from 
  $g[\vec{d_i}](\vec{b})$, with the same safe reductions, therefore with infinite safe reduction.
  In this case we take $g[\vec{x}],\vec{d_i},\vec{b}$ as next step of our path in $\Pi$:
  we choose $l_{i+1} = l_i \conc (1)$ and $\vec{d_{i+1}} = \vec{d_i}$,
  $\vec{a_{i+1}} = \vec{b}$.  This is an assignment with total terms.

  We have trace-compatibility because: 
  each argument of $t_i$ is connected to some equal argument of $t_{i+1}[\vec{x}]=f[\vec{x}]$,
  the first argument of $t_i[\vec{x}]$ disappears but it is connected to no $\N$-argument in $f[\vec{x}]$.

\item
  \emph{Suppose the reduction sequence contracts the leftmost $\cond$-redex at some step.}

  Then $t_i[\vec{d_i}]a \vec{b} \reduces \cond(u',v')a"\vec{b'}$, with $a"=0, \Succ(a')$,
  Then  in the first sub-case $\cond(u',v')a" \reduces u' \vec{b'}$, in the second sub-case
  $\cond(u,v)a" \reduces v' a' \vec{b'}$, 
  with $u \reduces^* u'$ and $v \reduces^* v'$ and $\vec{b} \reduces^* \vec{b'}$.
  After this $\cond$-reduction we have a reduction sequence with infinitely many safe reductions.
  We prove our thesis by sub-cases on $a"$.
   
%19:37 05/09/2024

\begin{enumerate}
\item
  \emph{Suppose $a" = 0$}.
   In this case we take $g[\vec{x}],\vec{d_i},\vec{b}$ as next step of our path in $\Pi$:
  we choose $l_{i+1} = l_i \conc (1)$ and $\vec{d_{i+1}} = \vec{d_i}$,
  $\vec{a_{i+1}} = \vec{b}$. This is an assignment with total terms.
  $f[\vec{d_{i+1}}](\vec{b})$ reduces to $u \vec{b'}$, then we have infinitely many safe reductions.

  We have trace-compatibility because: 
    each argument of $t_i$ is connected to some equal argument of $t_{i+1}[\vec{x}]=f[\vec{x}]$,
    the first argument of $t_i[\vec{x}]$ disappears but it is connected to no $\N$-argument in $f[\vec{x}]$.
%12:53 05/06/2024

 \item
  \emph{Suppose $a" = \Succ(a')$}. 
  We choose $l_{i+1} = l_i \conc (2)$ and $\vec{d_{i+1}} = \vec{d_i}$,
  $\vec{a_{i+1}} = a',\vec{b}$. This is an assignment with total terms: $a'$ is total because
  all reduction sequences from $a'$ can be raised from a reduction sequences from $a'' = \Succ(a')$
  with the same safe-reduction. From $a \reduces^* a''$ we deduce that
  there are finitely many safa-reductions from $a''$, therefore finitely many from $a'$.
 
  We have trace-compatibility because: 
  each argument of $t_i[\vec{x}]$ is connected to some equal argument of 
    $t_{i+1}[\vec{x}]=g[\vec{x}]$,
    but for the first unnamed argument $a$ of $t_i[\vec{x}]$ 
    which is connected the first unnamed argument of $g[\vec{x}]$.
    This is fine because in the second premise of a $\cond$ 
    the trace progresses and we have $a' \prec a$, because $a \reduces  a" = \Succ(a')$.

  \end{enumerate}
 \end{enumerate}
\end{enumerate}

By the above construction, we have an infinite path $\pi = (e_1,e_2,\ldots)$ in $\universe{\Pi}$
and a trace-compatible assignment, as we wished to show.

%  Since $\Pi$ satisfies the global trace condition, $\vec{e}$ contains a progressing trace
%  $(k_{a},k_{a+1},\ldots)$, where, for each $a\le i$, $k_i$ is an $\N$-argument of $t_i$, 
%  and $k_{i+1},t_{i+1}$ is the successor of $k_i,t_i$. 
%  Let $n_i$ be an numeral in $\vec{d_i},\vec{a_i}$ at index $k_i$ for each $a\le i$.
%  Then the sequence $(n_a,n_{a+1},\ldots$ decreases at each progressing point.
%  This means that it decreases infinitely many times
%  since $(k_{a},k_{a+1},\ldots)$ has infinitely many progressing point.
%  Finally we have a contradiction. 
  
\end{proof}

%14:09 06/06/2024

From this theorem we derive the strong normalization result for safe reductions on terms of $\GTC$
(on all terms which satisfy the global trace condition).

\begin{corollary}\label{cor:SN_GTC}
  \begin {enumerate}
  \item
   Assume  $\Gamma\vdash t:A$. Then $t \in \GTC$ implies that $t$ is totaland all reduction sequences
   from $t$ include only finitely many $0$-safe reductions.
  
\item
   Assume  $\Gamma\vdash t:A$. Then $t \in \GTC$ implies that $t$ strongly $0$-safe-normalizes
   \end{enumerate}
\end{corollary}

Safe-normalization on closed terms of $\GTC$ of type $\N$ produce a numeral:

\begin{proposition}[Closed Safe-Normal terms of Type $\N$]
If $t \in \GTC$, $t$ is closed and $0$-safe-normal then $t \in \Num$ ($t$ is a numeral)
\end{proposition}


\newpage


\newpage

\section{Infinite Church-Rosser for Infinite Lambda Terms}
\label{section-safe-church-rosser}

%15:12 16/04/2024
In this section prove a result for the safe part of a term, which we call 
\quotationMarks{\emph{unicity of the safe part of the safe normal form}}. By this we mean 
if $t,u,v \in \LAMBDA$ and $t \reduces u$ and $t \reduces v$ and $u, v$ are safe-normal 
then $u$, $v$ are equal outside the right-hand side of all $\cond$-sub-terms. 

%%%%%%%%%%%%%%%%%%%%%%%%%%%
%\ldots\ldots\ldots\ldots\ldots\ldots\ldots\ldots\ldots\ldots
%\\
%\bfColor{red}{(Here we should fill this part by adapting the proof of Church Rosser for the type $\N$- Stefano)}
%\ldots\ldots\ldots\ldots\ldots\ldots\ldots\ldots\ldots\ldots
%%15:46 24/04/2024
%%%%%%%%%%%%%%%%%%%%%%%%%%%

Our first idea is to prove a full Church-Rosser property for $\LAMBDA$: 
if $t,u,v \in \LAMBDA$ and $t \reduces u$ and $t \reduces v$ then for some $w \in \LAMBDA$
we have $u \reduces w$ and $v \reduces w$. However this property is false: for some $t$, finding a 
common reduction of $u,v$ takes infinitely many steps, even for terms of $\CTlambda$,
as the next example shows.

\begin{Eg}
Let $b = \cond(x^{\N},b):\N \rightarrow \N$ 
and $t = (\lambda x^{\N}.b)(r)$, with $r = (\lambda x^{\N}.x^{\N})(3)$ some redex. 

We first remark that  $t \in \CTlambda$. Indeed, 
\begin{enumerate}
\item
$t$ is regular by construction.
\item
We have $t \in \GTC$, because the unique infinite path of $t$ is 
$t, \lambda x^{\N}.b, b, b, b, \ldots$, and the
unique unnamed argument of $b:\N \rightarrow \N$ in the path progresses infinitely many times.
\end{enumerate}

Now consider the reductions: $t \reduces b[r/x^\N]$ and $t \reduces  (\lambda x^{\N}.b)(3)$.
We expect $b[3/x^\N]$ as common reduction. But we have $b[r/x^\N] = \cond(r,b[r/x^\N]$,
that is, we have replicated the redex $r$ infinitely many times in $b[r/x^\N]$. Therefore to reduce 
$b[r/x^\N]$ to $b[3/x^\N]$ takes infinitely many steps, and for \emph{no finite reduction we have}
$b[r/x^\N] \reduces b[3/x^\N]$.
\end{Eg}

In order to recover Church-Rosser we have to consider a more general notion of reduction $\reduces_X$, 
which allow to reduce \emph{infinitely many redexes in one step}: 
all those in a \emph{decidable} set $X$ of redexes of $t$. Then we
will prove that $\reduces_X$ is confluent: namely, we will prove that 
if $t \reduces_X u$ and $t \reduces_Y v$, 
then for some $w, Z, T$ we will have $u \reduces_Z w$ and $v  \reduces_T w$.
We call this property Infinite Church-Rosser.
Infinite Church-Rosser implies unicity of the safe part of the safe normal form.

Our first problem is that infinite reductions can easily loop, therefore Infinite Church-Rosser is stated for the
set $\LAMBDA_\bot$ of terms with possibly \emph{undefinite} subterms. 
This is not an obstacle as we will see.
We formally state Infinite Church-Rosser as follows. 
\quotationMarks{\emph{For all $t,u,v \in \LAMBDA_\bot$, all decidable sets $X$ of redexes of $t$, 
if $t \reduces_X u$ and $t \reduces_Y v$, 
then for some $w \in \LAMBDA_\bot$, some decidable set $Z, T$ of redexes of $u, v$
we have $u \reduces_Z w$ and $v  \reduces_T w$ and $t \reduces{X \cup Y} w$}}.

%08:00 10/06/2024

We represent a decidable set $X$ of positions of redexes by a map 
$\phi_X:\universe{t} \rightarrow \{\True,\False\}$ 
such that $l \in X$ if and only if $\phi_X(l) =\True$. 
We have to precise how $\phi_X$ changes when redexes in $X$ are moved or duplicated.


\begin{definition}[Substitution, subterms and labels]
\label{definition-substitution-label}
Suppose $X$ is a decidable set of redexes of $t$ and $Y$ a decidable set of redexes of $u$
\begin{enumerate}
\item
$Z = X[Y/x]$ is a a decidable set of redexes of $t[u/x]$ defined as:
for all $l \in \universe{t}$, $m \universe{u}$ 
if $l$ is a free occurrence of $x$ we set $\phi_Z(l \conc m) = \phi_Y(m)$,
we set $\phi_Z(l) = \phi_X(l)$ otherwise.
otherwise.
\item
If $n=1,2$ and $t = c(t_1)\ldots(t_n)$ and $X$ a decidable set of redexes of $t$,
then for all $1 \le i \le n$ we define a set $X_i$ of redexes in $t_i$ by $\phi_{X_i}(l) = \phi_X((i) \conc l)$. 
\item
If $n=1,2$ and $t = c(t_1)\ldots(t_n)$ and for all $1 \le i \le n$ $X_i$ is a decidable set of redexes of $t_i$,
then we define a set $X$ of redexes in $t$ by $\phi_X(\nil)=\False$
and $\phi_X((i) \conc l) = \phi_{X_i}(l)$. 
\end{enumerate}
\end{definition}


%%%%%%%%%%%%%%%%%%%%%%%%%%%%
%   \ldots\ldots\ldots\ldots\ldots\ldots
%%%%%%%%%%%%%%%%%%%%%%%%%%%%


We define $t \reduces_X u$ as the limit of a map $\rho(t,X,n)$ for $n \rightarrow \infty$. 
$\rho(t,X,n)$ start with the undefined value $\bot$, then either holds the value $\bot$ forever,
or at some step the root of $\rho(t,X,n)$ becomes some constructor of $\LAMBDA$ and never
changes again. At each step, if $t$ itself is a redex in $X$ then we reduce it, 
if $t$ is not a redex in $X$ then we move to the subterms of $t$. 
In both case we update $X$ accordingly to some set of labels $Y$.
We update any other set $Z$ of labels in $t$ in the same way to some $\sigma(t,X,n,Z)$,
where $Y = \sigma(t,X,n,X)$.

\begin{definition}
Assume $t \in \LAMBDA_\bot$ and $X,Z$ are decidable sets of redexes of $t$.

We set $\rho(t,X,0)=\bot$. Now assume  $\phi_X(t) = \True$. Then we set:

\begin{enumerate}
\item
If $t = (\lambda x^T.b)(a)$, then $\rho(t,X,n+1) = \rho(b[a/x],Y,n)$ with $Y = (X_1)_1[X_2/x]$.
We set $\sigma(t,X,n,Z) = (Z_1)_1[Z_2/x]$.
\item
If $t = \cond(f,g)(0)$, then $\rho(t,X,n+1) = \rho(f,Y,n)$ with $Y = (X_1)_1$.
We set $\sigma(t,X,n,Z) = (Z_1)_1$.
\item
If $t = \cond(f,g)(\Succ(u))$, then $\rho(t,X,n+1) = \rho(g(u),Y,n)$ with $Y = \ap((X_1)_2, (X_2)_1)$.
We set $\sigma(t,X,n,Z) =  \ap((Z_1)_2, (Z_2)_1)$
\end{enumerate}

Assume  $\phi_X(t) = \False$ 
and $t=c(t_1)\ldots(t_h)$ for some $h=0,1,2$ some $t_1, \ldots, t_h \in \LAMBDA_\bot$.
Then we set 
$$
\rho(t,X,n+1) = c(\rho(t_1,X_1,n)\ldots(\rho(t_h,X_h,n))
$$
We define $\rho(t,X) = \lim_{n \rightarrow \infty} \rho(t,X,n)$,  
$\sigma(t,X,Z) = \lim_{n \rightarrow \infty} \sigma(t,X,n,Z)$
and $t \reduces_X \rho(t,X)$.
\end{definition}


\begin{theorem}[Infinite Church-Rosser]
\label{theorem-infinite-church-rosser}
For all $t,u,v \in \LAMBDA_\bot$, all decidable sets $X$ of redexes of $t$:
if $t \reduces_X u$ and $t \reduces_Y v$, 
then for some $w \in \LAMBDA_\bot$, for some sets
$Z=\sigma(t,X,n,Y)$, $T = \sigma(t,Y,n,X)$ of redexes of $u, v$
we have $u \reduces_Z w$ and $v  \reduces_T w$ and $t \reduces_{X \cup Y} w$.
\end{theorem}

\begin{proof}
We have to prove that $\rho(\rho(t,X),Y))  = \rho(t,X \cup Y)$.
We prove that for all $n,m \in \N$
there is some $p=n+m \in \N$ such that  $\rho(\rho(t,X,n),Y,m))  = \rho(t,X \cup Y,p)$,
and conversely that for all $p \in \N$ there are $n,m\in\N$ such that $p=n+m$ and
$\rho(\rho(t,X,n),Y,m))  = \rho(t,X \cup Y,p)$.
\end{proof}

We define the safe trunk of a term as the part of the term which we can normalize with safe reductions only.
In the rest of this section we will
prove that Infinite Church-Rosser implies that if the safe trunk exists then it is unique. 

Infinite Church-Rosser and Safe strong Normalization together imply that after finitely many steps
all safe reductions reach the same safe trunk.

\begin{definition}[Safe Trunk of a term]
\label{definition-safe-trunk}
Assume $t \in \LAMBDA$.
\begin{enumerate}
\item
The safe trunk of $t$ is any expression $u[\cond(f_1,\cdot), \ldots, \cond(f_n,\cdot)]$
such that  for some $g_1, \ldots, g_n$ we have $v = u[\cond(f_1,g_1), \ldots, \cond(f_n,g_n)]$
\emph{safe normal} and $t \reduces v$.
\item
$t$ is \emph{finite for safe reduction} if and only if all infinite reduction sequences from $t$ 
include only finitely many \quotationMarks{safe} reduction steps.  
\end{enumerate}
\end{definition}


\begin{lemma}[Safe Trunk of a term]
\label{lemma-safe-trunk}
Assume $t$ is finite for safe reductions.
If the  safe-trunk of $t$ exists then it is unique. 
\end{lemma}


\begin{proof}
Assume $t$ is finite for safe reductions in order to prove that the safe-trunk of $t$ is unique.

Assume that $u[\cond(f_1,\cdot), \ldots, \cond(f_n,\cdot)]$ and
$u'[\cond(f'_1,\cdot), \ldots, \cond(f'_{n'},\cdot)]$ are safe-trunks for $t$, in order to prove
that $u=u'$ and $n=n'$. 

Then for some $g_1, \ldots,g_n$ and some $g'_1, \ldots,g'_n$ we have that 
$v = u[\cond(f_1,g_1), \ldots, \cond(f_n,g_n)]$ and 
$v' = u'[\cond(f'_1,g'_1), \ldots, \cond(f'_n,g'_{n'})]$ 
are safe-normal forms of $t$ and all $\cond$-expressions shown are maximal. 
The decomposition of each safe-normal form $v$ is therefore unique:
if $v = u"[\cond(f"_1,g"_1), \ldots, \cond(f"_n,g"_{n"})]$ then $u=u"$ and $n=n"$
and $f_1=f"_1$, \ldots, $f_n=f"_n$.
Each reduction from $v$, $v'$ takes place in some $g_1, \ldots,g'_{n'}$. 

By Infinite Church-Rosser  
(theorem \ref{theorem-infinite-church-rosser}) we deduce that $v$ and $v'$ are confluent
outside the right-hand-side of $\cond$-expressions, therefore
for some $v"$ we have $v \reduces_Z v"$ and $v' \reduces_T v'"$. 
Since the reductions on $v, v'$ take place in some $g_1, \ldots,g'_{n'}$, 
we deduce that $v" = u[\cond(f_1,g"_1), \ldots, \cond(f_n,g"_n)]$
and $v'" = u[\cond(f'_1,g'"_1), \ldots, \cond(f'_n,g'"_{n'})]$ for some $g"_1, \ldots,g'"_{n'}$.
From the unicity of the decomposition of $v"$
with maximal $\cond$-subterms we conclude that $u=u'$ and $n=n'$
and and $f_1=f'_1$, \ldots, $f_n=f'_n$, as wished.

\end{proof}


 %DRAFT


\begin{thebibliography}{99}

\bibitem{2021-Anupam-Das}
A Circular Version of G\"{o}del's T and its abstraction complexity.
Anupam Das. arXiv:2012.14421v2 [cs.LO] 16 Jan 2021.

\end{thebibliography}





%\section{Appendix: Weak normalization for closed terms of $\GTC$ of type $\N$}
\label{section-weak-normalization}
%\Daisuke{WN start}
\bfColor{red}{This section is now subsumed in the section about Strong Normalization for safe-reductions}
(\S \ref{section-finite-safe-reductions}). We only maintain
it as an example of proof using the global trace condition.
\\



In this section we proved that every closed term of type $\N$ of $\GTC$
 (that is, well-typed and with the global trace condition) normalizes with finitely many "safe" steps
to some numeral $n \in \Num$. In the next section we proved that this numeral is unique.
\\

In the following, we explicitly write $t[x_1,\ldots,x_n]$,
when each free variable in  a term $t$ is some $x_i$, 
and, under this notation, we also write $t[a_1,\ldots,a_n]$ instead of $t[a_1/x_1,\ldots,a_n/x_n]$. 
We recall that $\Num$, the set of all numerals, is the set of all terms of the form
$\Succ^n(0)$ for some $n\in \Nat$.

We define closed total terms as in Tait's normalization proof, and we simultaneously 
define a subset of them, the values.
The only difference between values and closed total terms 
is that if a value has type $\N$ then it is a numeral.

Here is an informal definition. Values of type $\N$ are exactly the numerals. 
Total terms $t$ of type $\N$ are
the closed terms of type $\N$ in $\GTC$ evaluating in at least one reduction path to a numeral. 
Hence total terms $t$ of type $\N$ include values of type $\N$. 
Closed total terms and values of type $t:\alpha$ (with $\alpha$ type variable) of $\GTC$ coincide, 
both are all closed terms.
Closed total terms and values of type $t:A \rightarrow B$ of $\GTC$ coincide, 
both are all the terms mapping values to total terms. 

An open term is total if all substitutions of free variables with values produce a \emph{closed total} term.

Below is the formal definition of values, closed total terms and total terms.

\begin{definition}[Values and total term]
  We define values and closed total terms on values by induction on types. 
  \begin{enumerate}
  \item
    A closed term $t:\N$ is a \emph{value} if and only if $t \in \Num$ ($t$ is some numeral).
  \item
    A closed term $t:\N$ of $\GTC$ is \emph{closed total}
    if and only if $t \safeReducesAst u$ for some $u\in \Num$.
  \item
    Closed terms and values $t$ of $\GTC$ of type $\alpha$ (type variable) coincide, 
    both are all closed terms of $\GTC$.
   \item
    Closed terms and values $t$ of $\GTC$ of type $A\rightarrow B$ coincide:
    both are the set of $t \in \GTC$ such that $t(a)$ is closed total for any value $a:A$.
   \end{enumerate}
   A term $t[\vec{x}]:C$ (i.e., whose free variables are in $\vec{x}:\vec{A}$), is \emph{total} in $\GTC$
   if and only if $t[\vec{a}]$ is closed total for any vector $\vec{a}$ of values of type $\vec{A}$.
\end{definition}

Assume $t[\vec{x}]:\vec{A}\rightarrow\N$ is a well-typed term of $\LAMBDA$ in the context 
 $\Gamma = \vec{x}:\vec{B}$.
 A \emph{value assignment} $\vec{v}$ for $t$ in $\Gamma$ is any vector 
$\vec{v}=\vec{u},\vec{a}:\vec{B}\vec{A}$ of closed \emph{values}. 
We can assign values to  the sub-terms of $t$
in a path of a proof $\Pi : \Gamma \vdash t: A$ in a way compatible with traces. 

\begin{definition}[Trace-compatible Assignment]
Assume $\pi = (t_1, \ldots, t_n) \in \Tree(t)$ is any chain of immediate subterms from $t_1=t$,
and $\vec{v} = (\vec{v_1}, \ldots, \vec{v_n})$ 
 is any vector of value assignments, one for each $t_i$. 
$\vec{v}$ is \emph{trace-compatible} in $i$  if and only if it satisfies the following condition:

  for all $j$  index of an $\N$-argument of $t_i$ assigned to $\Succ^{a}(0)$, 
  all $k$ index of an $\N$-argument of $t_{i+1}$ assigned to $\Succ^{b}(0)$, 
  if $j$ is connected to $k$ then:
 \begin{enumerate}
 \item
 if $j$ progresses to $k$ then $a=b+1$
 \item
 if $j$ does not progress to $k$ then $a=b$.
 \end{enumerate}
$\vec{v}$ is \emph{trace-compatible} if it is trace-compatible in all $i$.
\end{definition}

If an infinite path has a trace-compatible assignment, then all traces of the path progress only finitely many times.

\begin{proposition}[Trace assignment]
\label{prop:trace_assign}
Assume $\Pi:\Gamma \vdash t:A$ and $\pi$ is an infinite path of $\Pi$ and
$\rho$ is a trace-compatible assignment to $\pi$. Assume $n \in \Nat$.
\begin{enumerate}
\item
\label{prop:trace_assign1}
If an $\N$-argument $j$ of some $t_i \in \pi$ has value $\Succ^n(0)$ in $\rho$, then a trace
from $j$ progresses at most $n$ times.
\item
\label{prop:trace_assign2}
$t \not \in \GTC$.
\end{enumerate}
\end{proposition}

\begin{proof}
\begin{enumerate}
\item
%\label{prop:trace_assign1}
By definition of trace-compatible assignment, whenever the trace progresses, 
the numeral $n$ associated to the progressing argument decreases by $1$.
Whenever the trace does not progress, the numeral $n$ remains the same.
Thus, a trace from $j$ progresses at most $n$ times, as we wished to show.
\item
%\label{prop:trace_assign2}
By point \ref{prop:trace_assign1} above, 
no trace from any argument in any term of the branch $\pi$ of $\Tree(t)$ progresses infinitely many times.
We assumed that there is an infinite path $\pi$ in $\Pi$.
By definition of $\GTC$, we conclude that $t \not \in \GTC$. 
\end{enumerate}
\end{proof}

Closed total terms are closed by safe reductions and by application. 

\begin{lemma}\label{lem:total_value}
  \begin{enumerate}
  \item\label{lem:total_value1}
    Let $t:A$ be a closed term and $t \safeReduces u$.
    If $u$ is total, then so is $t$.
  \item\label{lem:total_value2}
    Let $f:A \rightarrow B$ and $a:A$ be closed \emph{total} terms.
    Then so is $f(a)$.
  \item\label{lem:total_value3}
    Let $t[\vec{x}]:\vec{A}\rightarrow\N$ be a term,
    whose all free variables are $\vec{x}:\vec{B}$,
    and $\vec{u}:\vec{B}$ and $\vec{a}:\vec{A}$ be closed values.
    If all $t[\vec{u}]\vec{a}:\N$ are total, then $t[\vec{x}]$ is total. 
  \end{enumerate}
\end{lemma}
\begin{proof}
\begin{enumerate}

\item
  We show \emph{point \ref{lem:total_value1}}  by induction on $A$. 
By the subject reduction property, $u$ has type $A$.
\begin{enumerate}
\item
  We show the first \emph{base case}, namely when $A =\N$.
  By the assumption, $u$ is closed total.
  By definition of $u$ closed total, 
  we have $t \safeReduces u \safeReducesAst n$ for some $n\in\Num$. 
  Hence $t:\N$ is total.
\item
  We show the second \emph{base case}, namely when $A =\alpha$.
  Both $t$ and $u$ are closed terms of type $\alpha$, therefore both are closed total terms.
\item
  We show the \emph{induction case}, namely when $A = (A_1\rightarrow A_2)$.
  Take arbitrary closed value $a:A_1$. Then we have $t(a) \safeReduces u(a)$ and 
  $u(a):A_2$ is total by the assumption that $u$ is total.
  Hence by the induction hypothesis $t(a)$ is total. 
  We obtain that $t:A_1\rightarrow A_2$ is total. 
\end{enumerate}

  \item
We show \emph{point \ref{lem:total_value2}} by case reasoning.
Assume that$f:A \rightarrow B$, $a:A$ are closed total terms, in order to prove
that $f(a)$  is closed total.

Assume $A$ is a variable type or an arrow type. 
Then $a:A$ is a value because values and closed total terms of type $A$
coincide, therefore $f(a):A$ is closed total by definition of closed total.

Assume $A=\N$. Then $a \safeReduces n$ for some $n \in \Num$ by definition of closed total.
$f(n)$ is closed total by definition of closed total. $f(a)$ is closed total by $f(a) \safeReduces f(n)$,
$f(n)$ closed total and point \ref{lem:total_value1} above.

\item  
We show \emph{point \ref{lem:total_value3}} by induction on the number $|\vec{A}|$ of
elements of $\vec{A}$.
\begin{enumerate}
\item
  The \emph{base case} $|\vec{A}| = 0$ is immediately shown by the definition.
\item
  We show the \emph{induction case}. Let $\vec{A} = A_0,\vec{A'}$.
  Take arbitrary values $\vec{u}:\vec{B}$, $\vec{a'}:\vec{A'}$, and $a_0:A_0$. 
  By the assumption, we have that $t[\vec{u}]a_0\vec{a'}:\N$ is closed total for all 
  vector of values $\vec{a'}$. 
  Then $t[\vec{u}]a_0:\vec{A'}\rightarrow\N$ is total 
  by the induction hypothesis on $\vec{A'}\rightarrow \N$.
  By definition $t[\vec{u}] : A_0,\vec{A'}\rightarrow\N$ is closed total,
  and so by definition $t[\vec{x}]$ is total, as we wished to show.
\end{enumerate}

\end{enumerate}
\end{proof}

%10:28 19/04/2024
%17:32 24/04/2024

Let $\Pi=(T,\phi)$ be a proof of $\Gamma\vdash t:A$ and $e$ be a node of $\Pi$, that is, a list  
$l_n=(e_1, \ldots, e_{n-1}) \in \universe{\Pi} = T$ for some $n \in \N$.
We write $\Gamma_{n-1}\vdash t_{n-1}:A_{n-1} = \Label(\Pi,l_n)$ for the sequent
 labelling the node $l_n$. We want to prove that all terms of $\GTC$ are total.

\begin{theorem}
  Assume $\Pi:\Gamma\vdash t:A$ (hence $t\in \WTyped$).
  If $t$ is \emph{not} total, then $t \not \in \GTC$, i.e.:
  there is some infinite path $\pi = (e_1, e_2, \ldots)$ of $\Pi$ with no infinite progressing trace. 
\end{theorem}

%19:01 16/04/2024
%18:11 30/04/2024
%13:39 04/06/2024

\begin{proof}
  Assume that $t$ is not total. 
  Let $\vec{x}:\vec{D}$ and $\vec{A}\rightarrow\N$ be $\Gamma$ and $A$, respectively.
  By Proposition \ref{prop:trace_assign}.\ref{prop:trace_assign2} it is enough to prove that
  $\Pi$ has some infinite path $\pi=(e_1, e_2, \ldots)$ 
  and some trace-compatible assignment $\rho$ for $\pi$.
  By \ref{lem:total_value3} of Lemma~\ref{lem:total_value},
  there exist closed values $\vec{a}:\vec{A}$ and $\vec{d}:\vec{D}$ such that
  $t[\vec{d}]\vec{a}$ is not total. 
  By induction on $i \in \N$, for each $i$, we construct a path $l_i = (e_1,\ldots,e_{i-1})$
  and a value assignment $\vec{v_i} = (\vec{d_i},\vec{a_i})$ such that
  $(\vec{v_1},\ldots,\vec{v_i})$ is a trace-compatible assignment for the node $l_i$. 
  We have to find some $(l_i,\vec{d_i},\vec{a_i})$ such that:
  \begin{itemize}
  \item[(i)]
%15:38 19/04/2024
   $l_i = (e_1, \ldots, e_{i-1}) \in \universe{\Pi}$, 
     and $\Label(\Pi,l_i) = \vec{x_{i}}:\vec{D_{i}}\vdash t_{i} : \vec{A_{i}}\rightarrow\N$; 
  \item[(ii)]
    $t_{i}$ is not total;
  \item[(iii)]
    $\vec{d_{i}}:\vec{D_i}$ and $\vec{a_i}:\vec{A_i}$ are closed values
    such that $t_{i}[\vec{d_i}]\vec{a_i}$ is not total;
  \item[(iv)]
    the value assignment 
    $\vec{d_1},\vec{a_1}$, \ldots, $\vec{d_i},\vec{a_i}$) is trace-compatible in $l_i$.
    %\Daisuke{mynote:write this more clearly}
  \end{itemize}
  
 We recall that \quotationMarks{trace compatible in $i$} means: 
 if $i-1$ is a progress point, namely if $t_{i-1}$ is of the form $\cond(f,g)$ and $e_i=2$ 
    (hence $t_i$ is $g$),
    then $\vec{a_i} = \Succ(m'),\vec{a'}$ 
     (the first unnamed argument of $\cond(f,g):\N \rightarrow A$ is $S(m')$)
     while $\vec{d_i} = \vec{d_i},m'$ 
     (the last named argument of $g$ is $1$ unit smaller).

  We first define $(l_1,\vec{d_1},\vec{a_1})$ for the root node $t$ of $\Pi$.
  In this case $l_1 = \nil$ 
   and $(\vec{d},\vec{a})$ are values such that $t[\vec{d}](\vec{a})$ is not total.
  Points (i), (ii), (iii), (iv) are immediate.

  Next, assume that $(l_i,\vec{d_i},\vec{a_i})$ is already constructed.
  Then we define $(l_{i+1},\vec{d_{i+1}},\vec{a_{i+1}})$ by the case analysis on
  the last rule for the node $l_i$ in $\Pi$. 

%%%%%%%%%%%%%%%%%%%%%%%%%%%%%%%%
% APPARENTLY THE CASE $\struct(f)$ CAN BE SIMPLIFIED TO WEAKENING-Stefano
%%%%%%%%%%%%%%%%%%%%%%%%%%%%%%%%

\begin{enumerate}
\item
  The case of $\weak$, namely
  $\Label(\Pi,l_i) = \Gamma'\vdash t_i:\vec{A_i}\rightarrow\N$
  is obtained from $\Gamma\vdash t_i:\vec{A_i}\rightarrow\N$, where
  $\Gamma = x_1:D_1,\ldots,x_n:D_n$, $\Gamma' = x'_1:D'_1,\ldots,x'_m:D'_m$, and $\Gamma \subseteqsim \Gamma'$
  with an injection $\phi:\{1,\ldots,n\}\to\{1,\ldots,m\}$. 
  We have $x_i = x'_{\phi(i)}$ and $D_i = D'_{\phi(i)}$ for all $i \in \{1,\ldots,n\}$.
  By the induction hypothesis and (ii), 
  $t_i[d_{i,\phi(1)}/x_1,\ldots,d_{i,\phi(n)}/x_n]\vec{a_i} 
   = t_i[d_{i,1}/x'_1,\ldots,d_{i,m}/x'_m]\vec{a_i}$ is not total,
  where $\vec{d_i} = d_{i,1}\ldots d_{i,m}$.
  Then we define $l_{i+1} = l_i\conc(1)$ taking the unique child node of $l_i$ in $\Pi$, and we
  also define $\vec{d_{i+1}}$ and $\vec{a_{i+1}}$ by $d_{i,\phi(1)}\ldots d_{i,\phi(n)}$
  and $\vec{a_i}$, respectively. 
  We obtain (i), (ii), and (iii) for $i+1$, as expected.
  We also have (iv) since $((\vec{d_i},\vec{a_i}),(\vec{d_{i+1}},\vec{a_{i+1}}))$
  is trace compatible for $(t_i,t_{i+1})$. 

\item
  The case of $\var$-rule, namely $\Label(\Pi,l_i) = \Gamma\vdash x:D$ for some $x_{i,k}:D_{i,k} = x:D \in \Gamma$
  cannot be, because $t_i [\vec{d_i}/\vec{x_i}] = d_{i,k}$ is closed total by assumption on $\vec{d}$.
  
\item
  The case of $0$-rule, namely $\Label(\Pi,l_i) = \Gamma\vdash 0:\N$, 
cannot be. Indeed, $t_i = 0$ is total because $0$ is a numeral.

\item  
  The case of $\Succ$-rule, namely $\Label(\Pi,l_i) = \Gamma\vdash t_i: \N = \Gamma\vdash \Succ(u): \N$
  for some $u$ is obtained from our assumptions on
  $\Gamma\vdash u: \N$. In this case $\vec{a_i}$ is empty, $t_{i+1}=u$, and
  by the induction hypothesis $\Succ(u)[\vec{d_i}]:\N$ is not closed total (it does not reduce to a numeral).
  Then, by the definition of closed total, $u[\vec{d_i}] =t_{i+1}[\vec{d_i}] $ is not closed total
 (it does not reduce to a numeral).
  We define $e_{i}=1$, 
  the index of the unique child node of $l_{i+1}=l_i\conc(1)$ in the $\Succ$-rule, and
  we also define $\vec{d_{i+1}} = \vec{d_i}$ and $\vec{a_{i+1}} = ()$. 
  We obtain (i), (ii), (iii), and (iv) for $i+1$, as expected.

\item
  The case of the $\apnotvar$-rule, namely 
  $\Label(\Pi,l_i) = \Gamma\vdash f[\vec{x}](u[\vec{x}]): \vec{A}\rightarrow\N$, 
  for some $f$ and $u$, is obtained from $\Gamma\vdash f[\vec{x}]: B \rightarrow \vec{A}\rightarrow\N$ 
  and $\Gamma\vdash u[\vec{x}]: B$, where $u$ is not a variable.
  By the induction hypothesis, $f[\vec{d_i}](u[\vec{d_i}])\vec{a_i}:\N$ is not total.
  We argue by case on the statement: \emph{$u[\vec{d_i}]:B$ is closed total}.

\begin{enumerate}
\item
  We first consider the subcase that \emph{$u[\vec{d_i}]:B$ is closed total}.
  We define $b:B$ by $b = n\in\Num$ such that $u[\vec{d_i}] \safeReducesAst n$ if $B=\N$,
  by $b = u[\vec{d_i}]$ otherwise. By lemma \ref{lem:total_value}.\ref{lem:total_value1}, $b$ is a value.
  Then define $l_{i+1}=l_i\conc(1)$, the index of $f[\vec{x}]$,
  and define $\vec{d_{i+1}} = \vec{d_i}$ and $\vec{a_{i+1}} = b,\vec{a_i}$. 
  Using Lemma \ref{lem:total_value}.\ref{lem:total_value1} and \ref{lem:total_value}.\ref{lem:total_value2}
  we obtain (i), (ii), (iii) for $i+1$, as expected. 
  We also have (iv) since the connection from 
  $(\vec{d_i},\vec{a_i})$ to $(\vec{d_{i+1}},\vec{a_{i+1}}) = (\vec{d_i},b,\vec{a_i})$ is
  trace-compatible: all connected 
  $\N$-argument of $t_{i}=f(u)[\vec{x}]$ and $t_{i+1}[\vec{x}] = f[\vec{x}]$ are the same,
  because the only fresh argument of $f[\vec{x}]$ 
  is $b$ and no argument of $f(u)[\vec{x}]$ is connected to it.
\item
  Next we consider the subcase that \emph{$u[\vec{d_i}]:B$ is not closed total}.
  Let $B = \vec{C}\rightarrow\N$.
  By lemma \ref{lem:total_value}.\ref{lem:total_value3}
  there is a sequence of values $\vec{c}:\vec{C}$ such that $u[\vec{d_i}]\vec{c}:\N$ is not total.
  Define $l_{i+1}= l_i \conc (2)$ taking the child $u[\vec{x}]$,
  and define $\vec{d_{i+1}} = \vec{d_i}$ and $\vec{a_{i+1}} = \vec{c}$. 
  We obtain (i), (ii), (iii) for $i+1$, as expected.
  We also have (iv) since the connection from 
  $(\vec{d_i},\vec{a_i})$ to $(\vec{d_{i+1}},\vec{a_{i+1}}) = (\vec{d_i},\vec{c})$ is
  trace compatible: all connected $\N$-argument of $t_{i}=f(u)[\vec{x}]$ and $t_{i+1}=u[\vec{x}]$ are 
  in $\vec{d_i}$ and therefore are the same.
 \end{enumerate}

\item
  The case of $\apvar$-rule, namely 
  $\Label(\Pi,l_i) 
  = 
  \Gamma \vdash f[\vec{x}](x): \vec{A}\rightarrow\N$ is obtained from
  $\Gamma \vdash f[\vec{x}]: D,\vec{A} \rightarrow \N$,
  where $\Gamma=x_1:D_1,\ldots,x_m:D_m$ and $x_k:D_k = x:D\in\Gamma$. 
  By the induction hypothesis, $f[\vec{d_i}]d_{i,k}\vec{a_i}:\N$ is not total,
  where $\vec{d_i} = (d_{i,1},\ldots,d_{i,m})$. 
  Define $l_{i+1}=l_i\conc(1)$ as the unique child of $l_i$ in $\Pi$, and
  also define $\vec{d_{i+1}} = \vec{d_i}$ and $\vec{a_{i+1}} = d_{i,k},\vec{a_i}$. 
  We obtain (i), (ii), and (iii) for $i+1$, as expected.
  We also have (iv) since the connection from
  $(\vec{d_i},\vec{a_i})$ to
  $(\vec{d_{i+1}},\vec{a_{i+1}}) = (\vec{d_i},d_{i,k},\vec{a_i})$
  is trace compatible: all connected $\N$-arguments in $\vec{d_i},\vec{a_i}$ and 
  $\vec{d_{i+1}},\vec{a_{i+1}}$ are the same.  
  The only difference between the two assignments
  is that, if $D_k = \N$, the value $d_{i,k}$ of type $\N$ for the variable $x_k$
  in $t_i[\vec{x}]=f[x_1,\ldots,x_k,\ldots,x_m](x_k)$ is duplicated to the value $d_{i,k}$ 
  of the first unnamed argument of $ f[\vec{x}]$. 

\item
  The case of $\lambda$-rule, namely
  $\Label(\Pi,l_{i+1}) = 
    \Gamma\vdash \lambda x^A.u[\vec{x},x] : A, \vec{A} \rightarrow \N$ is obtained from
  $\Gamma,x:A\vdash t_{i+1}[\vec{x},x^A]:\vec{A}\rightarrow\N$, 
  where $t_{i+1}[\vec{x},x]=u[\vec{x},x]$.
  By the induction hypothesis, $(\lambda x.(u[\vec{d_i},x]))a\vec{a_*}:\N$ is not total,
  where $\vec{a_i} = a,\vec{a_*}$.
  Then, by Lemma~\ref{lem:total_value}, $t_{i+1}[\vec{d_i},a]\vec{a_*}$ is not total. 
  We define $l_{i+1} = l_i \conc (1)$ as the index of unique parent node of $l_{i}=(e_1,\ldots,e_{i-1})$,
  and we also define $\vec{d_{i+1}} = \vec{d_i},a$ and $\vec{a_{i+1}} = \vec{a_*}$. 
  We obtain (i), (ii), (iii) for $i+1$, as expected.
  We also have (iv) since the connection from 
  $(\vec{d_i},\vec{a_i}) = (\vec{d_i},a,\vec{a_*})$ 
  to $(\vec{d_{i+1}},\vec{a_{i+1}}) = (\vec{d_i},a,\vec{a_*})$
  is trace-compatible: all connected argument in $\vec{d_i},\vec{a_i}$ and 
  $\vec{d_{i+1}},\vec{a_{i+1}}$ are the same. The only difference between the two assignments
  is that the value $a$ of the first unnamed argument of $t_i[\vec{x}]$ is moved to the value $a$
  of the last variable of type $A$ of $t_{i+1}[\vec{x},x]=u[\vec{x},x]$.
  This is a kind of opposite of the movement we have in the $\apvar$.

\item
  The case of $\cond$-rule, namely
  $\Label(\Pi,l_i) = \Gamma\vdash \cond(f[\vec{x}],g[\vec{x}]):\N,\vec{A}\rightarrow\N$
  is obtained from 
  $\Gamma\vdash f[\vec{x}]:\vec{A}\rightarrow\N$
  and
  $\Gamma\vdash g[\vec{x}]:\N,\vec{A}\rightarrow\N$. 
  By the induction hypothesis, $\cond(f[\vec{d_i}],g[\vec{d_i}])m\vec{a_*}:\N$ is not total,
  where $\vec{a_i} = m,\vec{a_*}$ and $m \in \Num$. We argue by cases on $m$.
  \begin{enumerate}
  \item
    We first consider the \emph{subcase $m=0$}.
    Define $l_{i+1}=l_i\conc(1)$, namely the first child node of $l_i$ whose term is $f[\vec{x}]$,
    and define $\vec{d_{i+1}} = \vec{d_i}$ and $\vec{a_{i+1}} = \vec{a_*}$. 
    We obtain (i), (ii), (iii) for $i+1$, as expected. 
    We also have (iv) since the connection from 
    $(\vec{d_i},\vec{a_i}) =(\vec{d_i},0,\vec{a_*})$ to
    $(\vec{d_{i+1}},\vec{a_{i+1}}) = (\vec{d_i},\vec{a_*})$ is trace compatible: 
    each argument of $t_i$ is connected to some equal argument of $t_{i+1}[\vec{x}]=f[\vec{x}]$,
    the first argument of $t_i[\vec{x}]$ disappears but it is connected to no $\N$-argument in $f[\vec{x}]$.
  \item
    Next we consider the \emph{subcase $m=\Succ(m')$}. 
    Define $l_{i+1}=l_i\conc(2)$, namely the second child node of $l_i$ whose term is $g[\vec{x}]$,
    and define $\vec{d_{i+1}} = \vec{d_i},m'$ and $\vec{a_{i+1}} = \vec{a_*}$. 
    We obtain (i), (ii), (iii) for $i+1$, as expected.
    We also have (iv) since the connection from 
    $(\vec{d_i},\vec{a_i})
    =(\vec{d_i},\Succ(m'),\vec{a_*})$ to $(\vec{d_{i+1}},\vec{a_{i+1}}) = (\vec{d_i},m',\vec{a_*})$
    is trace compatible: 
    each argument of $t_i[\vec{x}]$ is connected to some equal argument of 
    $t_{i+1}[\vec{x}]=g[\vec{x}]$,
    but for the first unnamed argument of $t_i[\vec{x}]$ 
    which is connected the first unnamed argument of $g[\vec{x}]$.
    This is fine because in the second premise of a $\cond$ 
    the trace progress. This requires that \emph{the value $m=\Succ(m')$ of the argument decreases by $1$}, 
    as indeed it is the case: the new value is $m'$.
  \end{enumerate}

\end{enumerate}

By the above construction, we have an infinite path $\pi = (e_1,e_2,\ldots)$ in $\universe{\Pi}$
and a trace-compatible assignment, as we wished to show.

%  Since $\Pi$ satisfies the global trace condition, $\vec{e}$ contains a progressing trace
%  $(k_{m},k_{m+1},\ldots)$, where, for each $m\le i$, $k_i$ is an $\N$-argument of $t_i$, 
%  and $k_{i+1},t_{i+1}$ is the successor of $k_i,t_i$. 
%  Let $n_i$ be an numeral in $\vec{d_i},\vec{a_i}$ at index $k_i$ for each $m\le i$.
%  Then the sequence $(n_m,n_{m+1},\ldots$ decreases at each progressing point.
%  This means that it decreases infinitely many times
%  since $(k_{m},k_{m+1},\ldots)$ has infinitely many progressing point.
%  Finally we have a contradiction. 
  
\end{proof}

From this theorem we derive the weak normalization result: 
every closed term of type $\N$ is closed total, therefore by definition it reduces to some numeral for at least
one reduction path. 

\begin{corollary}\label{cor:WN_typeN}
  \begin {enumerate}
  \item
   Assume  $\Gamma\vdash t:A$. Then $t \in \GTC$ implies that $t$ is total.
  \item
    For any closed $t:\N$, there is numeral $n\in\Num$ such that $t\safeReducesAst n$. 
  \end{enumerate}
\end{corollary}

%14:56 04/06/2024

%\Daisuke{WN end}        %SUBSUMED IN THE SN SECTION

%In this section we prove a weak form of confluence: normal form for
all closed terms of $\CTlambda$ of type $\N$ is unique.
By the weak normalization result proved in Section~\ref{section-weak-normalization},
we will deduce: for all closed terms $t$ of $\CTlambda$ of type $\N$,
there is $n \in \N$ such that $t \safeReducesAst n$
and all normal form of $t$ are equal to $n$. 

We define binary relations $\sim_A$ (for each type $A$) on well-typed terms with type $A$.
\begin{definition}
  We define binary relations $\sim_A$ for \emph{closed} terms of $\CTlambda$
  by induction on the type $A$. 
  \begin{enumerate}
  \item
    $t_1 \sim_N t_2$ if and only if for all normal forms $t_1',t_2':\N$, if $t_1 \safeReducesAst t_1'$
    and $t_2 \safeReducesAst t_2'$ then $t_1' = t_2'$.
  \item
    $t_1 \sim_{A\rightarrow B} t_2$ if and only if for all closed $a_1,a_2:A$ if
    $a_1 \sim_A a_2$ then $t_1(a_1) \sim_B t_2(a_2)$.
  \end{enumerate}
  We write $a_1,\ldots,a_n \sim_{A_1,\ldots,A_n} a'_1,\ldots,a'_n$ if $a_i\sim_{A_i} a'_i$ holds
  for all $i\in\{1,\ldots,n\}$. 
  
  We also define $t_1[\vec{x}] \sim_A t_2[\vec{x}]$ for \emph{any} terms of $\CTlambda$,
  with $\vec{x}:\vec{A}\vdash t_i[\vec{x}]:B$, where $i\in\{1,2\}$, as follows. 
  \begin{enumerate}
  \item[3.]
    $t_1[\vec{x}] \sim_A t_2[\vec{x}]$ if and only if
    for all closed terms $\vec{a_1}:\vec{A}$ and $\vec{a_2}:\vec{A}$,
    if $\vec{a_1}\sim_{\vec{A}}\vec{a_2}$ then $t_1[\vec{a_1}] \sim_B t_2[\vec{a_2}]$. 
  \end{enumerate}  
  
\end{definition}

It is easily shown that $\sim_A$ is symmetric by the definition. 
We will prove that, for all closed terms $t$ of $\CTlambda$ with type $A$, we have $t \sim_A t$.
If $A=\N$ this means that the normal form of all closed terms of $\CTlambda$ of type $\N$ is unique,
which is our goal.

\begin{definition}[Confluent and U-counterexample]
  Let $t$ be a term of $\CTlambda$. 
  \begin{enumerate}
  \item
    Assume that $t$ is a closed term with type $N$. 
    We say that $n$ is a {\em value} of $t$ if $t \safeReducesAst n$ ($t$ safely reduces to $n$). 
  \item
    We say that $t$ is {\em confluent} if $t \sim_A t$.
  \item
    Assume $\vec{x}:\vec{D}\vdash t:\vec{A}\rightarrow\N$. 
    We say a quadraple $(\vec{d},\vec{a},\vec{d'},\vec{a'})$ is a {\em U-counterexample} to $t$
    if and only if
    both $(\vec{d},\vec{a})$ and $(\vec{d'},\vec{a'})$ are value assignments for $t$, 
    $\vec{d}\sim_{\vec{D}}\vec{d'}$, $\vec{a}\sim_{\vec{A}}\vec{a'}$, and
    $t[\vec{d}](\vec{a}) \not\sim_N t[\vec{d'}](\vec{a'})$. 
  \end{enumerate}
\end{definition}

By the weak normalization result proved in Section~\ref{section-weak-normalization}, all 
closed term $t$ of $\CTlambda$ of type $\N$ have a value. If $t$ is confluent, then the value is unique.
Hence we have the following. 

\begin{lemma}
  If $t:\N$ is a confluent closed term, then $t$ has a unique normal form.
\end{lemma}

By using the weak normalization result, 
two closed terms $t$ and $u$ equivalent with respect to $\sim_\N$, namely $t \sim_\N u$,
have a common numeral as their unique normal form. 

\begin{lemma}\label{lem:uniq_NF}
  Let $t_1$ and $t_2$ be closed terms in $\CTlambda$. 
  Assume $t_1 \sim_\N t_2$. 
  \begin{enumerate}
  \item\label{lem:uniq_NF1}
    There exists $n \in \Num$ such that $t_1 \safeReducesAst n$ and $t_2 \safeReducesAst n$.
    Moreover, for each $i\in \{1,2\}$, if $t_i \safeReducesAst u$ and $u$ is a normal form, then $u = n$. 
  \item\label{lem:uniq_NF2}
    If $t_1 \safeReduces t'_1$ and $t_2 \safeReduces t'_2$, then $t'_1 \sim_\N t'_2$. 
  \end{enumerate}
  
\end{lemma}
\begin{proof}
  We show point~\ref{lem:uniq_NF1}. 
  By Corollary~\ref{cor:WN_typeN}, there are $n_1,n_2 \in \Num$ such that
  $t_1 \safeReducesAst n_1$ and $t_2\safeReducesAst n_2$.
  Then $n_1=n_2$ holds by $t_1\sim_\N t_2$. 
  If $t_i \safeReducesAst u$ and $u$ is a normal form, then $u = n_2$ by $t_1\sim_\N t_2$.

  Next we show point~\ref{lem:uniq_NF2}.
  It is enough to show the statement: $t_1\sim_\N t_2$ and $t_1 \safeReduces t'_1$ implies $t'_1 \sim_\N t_2$,
  because we can prove point~\ref{lem:uniq_NF2} by using the statement twice with the symmetricity of $\sim_\N$. 
  For showing the statement, assume that $t_1\sim_\N t_2$ and $t_1 \safeReduces t'_1$. Our goal is $t'_1 \sim_\N t_2$. 
  To show it, take arbitrary normal forms $v'_1$ and $v_2$ such that $t'_1 \safeReducesAst v'_1$ and $t_2 \safeReducesAst v_2$.
  By point~\ref{lem:uniq_NF1}, there is $n\in\Num$ for $t_1$ and $t_2$ that satisfies the properties as described.
  Hence we have $v'_1 = n = v_2$ by using the property with $t_1 \safeReduces t'_1 \safeReducesAst v'_1$ and $t_2 \safeReducesAst v_2$. 
  Therefore $t'_1 \sim_\N t_2$ is obtained. 
\end{proof}

%For confluent closed term $t$ of $\CTlambda$ of type $\N$, "safe" reduction preserves the value. Indeed,
%if $t \reduces u$ then $u$ is a closed term of $\CTlambda$ of type $\N$, therefore $u$ has a value
%$m$, which is also a value of $t$. By confluence of $t$ we conclude that $n=m$.

%If $t$ is as above, with value $n$, and $t \reduces \Succ (u)$, then $n>0$, 
%$u$ is a confluent closed term $t$ of $\CTlambda$ of type $\N$, and the value of $u$ is $n-1$.
%Indeed,if $u \reduces u'$ and $u'$ is normal, then $t \reduces \Succ (u')$, therefore $\Succ (u')=n$ 
%by confluence of $t$, and we conclude that $n>0$ and $u'=n-1$. Thus, the normal form of $u$ is unique.

We have $t \sim_A t$ if and only if there is no U-counterexample for $t$.
This is shown by the following lemma.
\begin{lemma}\label{lem:uniq_value}
  Let $t$ be a term of $\vec{x}:\vec{D} \vdash t[\vec{x}]:\vec{A}\rightarrow\N$. 
  Let $\vec{d},\vec{a},\vec{d'},\vec{a'}$ be vectors of closed terms.
  Assume that
  $\vec{d}\sim_{\vec{D}}\vec{d'}$ and $\vec{a}\sim_{\vec{A}}\vec{a'}$ implies $t[\vec{d}](\vec{a}) \sim_N t[\vec{d'}](\vec{a'})$. 
  Then $t[\vec{x}]$ is confluent. 
\end{lemma}
\begin{proof}
  Take any $\vec{d},\vec{a},\vec{d'},\vec{a'}$ such that $\vec{d}\sim_{\vec{D}}\vec{d'}$ and $\vec{a}\sim_{\vec{A}}\vec{a'}$.
  Then $t[\vec{d}](\vec{a}) \sim_N t[\vec{d'}](\vec{a'})$ holds by the assumption.
  We have $t[\vec{d}] \sim_{\vec{A}\rightarrow\N} t[\vec{d'}]$ by the definition of $\sim_{\vec{A}\rightarrow\N}$.
  Hence $t[\vec{x}] \sim_{\vec{A}\rightarrow\N} t[\vec{x}]$ holds by the definition. 
\end{proof}

Taking contraposition of this lemma, there exists a U-counterexample for non-confluent $t$. 
Using this, we prove the following theorem.

\begin{theorem}
  Assume $\Pi:\Gamma\vdash t:A$ (hence $t \in \WTyped$). 
  If $t$ is not confluent,
  then $t \not\in \GTC$, that is, $\Pi$ does not satisfy the global trace condition. 
\end{theorem}
\begin{proof}
  Assume that $t$ is not confluent. 
  Let $\Gamma = \vec{x}:\vec{D}$ and $A = \vec{A}\rightarrow\N$. 
  By Proposition \ref{prop:trace_assign}.\ref{prop:trace_assign2} it is enough to prove that
  there are an infinite path $\pi=(e_1, e_2, \ldots)$ of $\Pi$
  and a trace-compatible assignment $\rho$ for $\pi$.
  
  By induction on $i \in \N$, for each $i$, we construct a path $l_i = (e_1,\ldots,e_{i-1})$
  and value assignments $\vec{v_i} = (\vec{d_i},\vec{a_i})$ and $\vec{v'_i} = (\vec{d'_i},\vec{a'_i})$
  for $t_i = t \restr l_i$, which is a subterm of $t$, such that 
  both $(\vec{v_1},\ldots,\vec{v_i})$ and $(\vec{v'_1},\ldots,\vec{v'_i})$ are
  trace-compatible assignments for the node $(t_1, \ldots, t_i)$.
  We have to find $(l_i,\vec{d_i},\vec{a_i},\vec{d'_i},\vec{a'_i})$ such that:
  \begin{enumerate}
  \item[(i)]
    For each $i\ge 1$, $l_i$ is a node of $\Pi$ and
    $\Label(\Pi,l_i) = \vec{x_i}:\vec{D_i}\vdash t_i:\vec{A_i}\rightarrow\N$. 
  \item[(ii)]
    $t_i$ is not confluent whose U-counterexample is $(\vec{d_i},\vec{a_i},\vec{d'_i},\vec{a'_i})$. 
  \item[(iii)]
    $((\vec{d_1},\vec{a_1}),\ldots,(\vec{d_i},\vec{a_i}))$ and
    $((\vec{d'_1},\vec{a'_1}),\ldots,(\vec{d'_i},\vec{a'_i}))$ are trace-compatible assignments for
    $(t_1,\ldots,t_i)$. 
  \end{enumerate}

  For the case $i=1$,
  by Lemma~\ref{lem:uniq_value},
  there exist closed values $\vec{d_1},\vec{d'_1}:\vec{D}$ and $\vec{a_1},\vec{a'_1}:\vec{A}$ such that
  $\vec{d_1}\sim_{\vec{D}}\vec{d'_1}$, $\vec{a_1}\sim_{\vec{A}}\vec{a'_1}$ and 
  $t[\vec{d_1}]\vec{a_1} \not\sim_{\N} t[\vec{d'_1}]\vec{a'_1}$. 
  Then we take $((),\vec{d_1},\vec{a_1},\vec{d'_1},\vec{a'_1})$ for this case.
  Points (i), (ii), and (iii) are immediately shown. 

  Next, assume that $(l_i,\vec{d_i},\vec{a_i},\vec{d'_i},\vec{a'_i})$ is already constructed. 
  Then we define $(l_{i+1},\vec{d_{i+1}},\vec{a_{i+1}},\vec{d'_{i+1}},\vec{a'_{i+1}})$ by the case analysis
  on the last rule for the node $l_i$ in $\Pi$. 

  \begin{enumerate}
  \item
    The case of $\weak$, namely
    $\Label(\Pi,l_i) = \Gamma'\vdash t:\vec{A_i}\rightarrow\N$
    is obtained from $\Gamma\vdash t:\vec{A_i}\rightarrow\N$, where
    $\Gamma = x_1:D_1,\ldots,x_n:D_n$, $\Gamma' = x'_1:D'_1,\ldots,x'_m:D'_m$, and $\Gamma \subseteqsim \Gamma'$
    with an injection $\phi:\{1,\ldots,n\}\to\{1,\ldots,m\}$. 
    We have $x_i = x'_{\phi(i)}$ and $D_i = D'_{\phi(i)}$ for all $i \in \{1,\ldots,n\}$.
    By the induction hypothesis and (ii), we have 
    $t[d_{i,\phi(1)}/x_1,\ldots,d_{i,\phi(n)}/x_n]\vec{a_i} = t[\vec{d_i}/\vec{x'}]\vec{a_i} \not\sim_\N t[\vec{d'_i}/\vec{x'}]\vec{a'_i} = t[d'_{i,\phi(1)}/x_1,\ldots,d'_{i,\phi(n)}/x_n]\vec{a'_i}$, 
    where $\vec{d_i} = d_{i,1}\ldots d_{i,m}$ and $\vec{d'_i} = d'_{i,1}\ldots d'_{i,m}$.
    Then we define $l_{i+1} = l_i \conc (1)$ taking the unique child node of $l_i$ in $\Pi$, and we
    define
    $\vec{d_{i+1}} = (d_{i,\phi(1)}\ldots d_{i,\phi(n)})$, 
    $\vec{d'_{i+1}} = (d'_{i,\phi(1)}\ldots d'_{i,\phi(n)})$, 
    $\vec{a_{i+1}} = \vec{a_i}$, and $\vec{a'_{i+1}} = \vec{a'_i}$. 
    We obtain (i), (ii), and (iii) for $i+1$, as expected.
    We also have (iv) since $((\vec{d_i},\vec{a_i}),(\vec{d_{i+1}},\vec{a_{i+1}}))$
    and $((\vec{d'_i},\vec{a'_i}),(\vec{d'_{i+1}},\vec{a'_{i+1}}))$
    are trace compatible for $(t_i,t_{i+1})$. 

  \item
    The case of $\var$-rule, namely $\Label(\Pi,l_i) = \Gamma\vdash x:D$ for some $x:D \in \Gamma$
    cannot be, because $t_i = x$ is confluent. 

  \item
    The case of $0$-rule, namely $\Label(\Pi,l_i) = \Gamma\vdash 0:\N$, cannot be.
    Indeed, $t_i = 0$ is confluent. 

  \item  
    The case of $\Succ$-rule, namely $\Label(\Pi,l_i) = 
    \Gamma\vdash t_i: \N = \Gamma\vdash \Succ(u): \N$ for some $u$ 
    is obtained from our assumptions on
    $\Gamma\vdash u: \N$. In this case $\vec{a_i}$ is empty, $t_{i+1}=u$, and
    by the induction hypothesis $\Succ(u)[\vec{d_i}]:\N$ is not confluent. 
    Then, by the definition of confluent, $u[\vec{d_i}] =t_{i+1}[\vec{d_i}] $ is not confluent. 
    We define $l_{i+1} = l_i \conc (1)$ taking the unique child of $l_i$ in $\Pi$. 
    We also define $\vec{d_{i+1}} = \vec{d_i}$ and $\vec{a_{i+1}} = ()$.  
    We obtain (i), (ii), (iii), and (iv) for $i+1$, as expected.

  \item
    The case of the $\apnotvar$-rule, namely 
    $\Label(\Pi,l_{i+1}) = \Gamma\vdash f[\vec{x}](u[\vec{x}]): \vec{A}\rightarrow\N$,
    for some $f$ and $u$, 
    is obtained from $\Gamma\vdash f[\vec{x}]: B \rightarrow \vec{A}\rightarrow\N$ 
    and $\Gamma\vdash u[\vec{x}]: B$, where $u$ is not a variable.
    By the induction hypothesis,
    $f[\vec{d_i}](u[\vec{d_i}])\vec{a_i} \not\sim_\N f[\vec{d'_i}](u[\vec{d'_i}])\vec{a'_i}$
    We argue by case on the statement: \emph{$u[\vec{d_i}] \sim_B u[\vec{d'_i}]$}.

    \begin{enumerate}
    \item
      We first consider the subcase that \emph{$u[\vec{d_i}] \sim_B u[\vec{d'_i}]$}.
      We define $b,b':B$ by $b = b' = n\in\Num$ such that $u[\vec{d_i}] \safeReducesAst n$ if $B=\N$,
      by $b = u[\vec{d_i}]$ and $b' = u[\vec{d'_i}]$ otherwise.
      By lemma \ref{lem:total_value}.\ref{lem:total_value1}, $b$ and $b'$ are values.
      We also have $b \sim_B b'$ by the assumption of this case.
      Then define $l_{i+1}=l_i\conc(1)$, the index of $f[\vec{x}]$,
      and define $\vec{d_{i+1}} = \vec{d_i}$, $\vec{d'_{i+1}} = \vec{d'_i}$,
      $\vec{a_{i+1}} = b,\vec{a_i}$, and $\vec{a'_{i+1}} = b',\vec{a'_i}$.
      Then, by the induction hypothesis, $\vec{d_{i+1}} \sim_{\vec{D}} \vec{d'_{i+1}}$ and
      $\vec{a_{i+1}} \sim_{B,\vec{A}} \vec{a'_{i+1}}$. 
      Using Lemma \ref{lem:total_value}.\ref{lem:total_value1} and \ref{lem:total_value}.\ref{lem:total_value2}
      we obtain (i), (ii), (iii) for $i+1$, as expected. 
      We also have (iv) since the connections 
      from $(\vec{d_i},\vec{a_i})$ to $(\vec{d_{i+1}},\vec{a_{i+1}}) = (\vec{d_i},b,\vec{a_i})$
      and
      from $(\vec{d'_i},\vec{a'_i})$ to $(\vec{d'_{i+1}},\vec{a'_{i+1}}) = (\vec{d'_i},b,\vec{a'_i})$
      are trace-compatible: all connected 
      $\N$-argument of $t_i=f(u)[\vec{x}]$ and $t_{i+1}[\vec{x}] = f[\vec{x}]$ are the same,
      because the only fresh arguments of $f[\vec{x}]$ 
      are $b$ and $b'$, and no argument of $f(u)[\vec{x}]$ is connected to them.
    \item
      Next we consider the subcase that \emph{$u[\vec{d_i}] \not\sim_B u[\vec{d'_i}]$}.
      Let $B = \vec{C}\rightarrow\N$. 
      By lemma \ref{lem:uniq_value},
      there are sequences of values $\vec{c}, \vec{c'}:\vec{C}$ such that
      $u[\vec{d_i}]\vec{c} \not\sim_\N u[\vec{d_i}]\vec{c'}$. 
      Define $l_{i+1}= l_i \conc (2)$ taking the child $u[\vec{x}]$,
      and define $\vec{d_{i+1}} = \vec{d_i}$, $\vec{a_{i+1}} = \vec{c}$, 
      $\vec{d'_{i+1}} = \vec{d'_i}$, and $\vec{a'_{i+1}} = \vec{c'}$. 
      We obtain (i), (ii), (iii) for $i+1$, as expected.
      We also have (iv) since the connections from 
      $(\vec{d_i},\vec{a_i})$ to $(\vec{d_{i+1}},\vec{a_{i+1}}) = (\vec{d_i},\vec{c})$
      and
      $(\vec{d'_i},\vec{a'_i})$ to $(\vec{d'_{i+1}},\vec{a'_{i+1}}) = (\vec{d'_i},\vec{c'})$
      are trace compatible: all connected $\N$-argument of $t_{i}=f(u)[\vec{x}]$ and $t_{i+1}=u[\vec{x}]$ are 
      in $\vec{d_i}$ and therefore are the same.
    \end{enumerate}

  \item
    The case of $\apvar$-rule, namely 
    $\Label(\Pi,l_i) 
    = 
    \Gamma \vdash f[\vec{x}](x): \vec{A}\rightarrow\N$ is obtained from
    $\Gamma \vdash f[\vec{x}]: D,\vec{A} \rightarrow \N$,
    where $\Gamma=x_1:D_1,\ldots,x_m:D_m$ and $x_k:D_k = x:D\in\Gamma$. 
    By the induction hypothesis, $f[\vec{d_i}]d_{i,k}\vec{a_i} \not\sim_\N f[\vec{d'_i}]d'_{i,k}\vec{a'_i}$. 
    where $\vec{d_i} = (d_{i,1},\ldots,d_{i,m})$ and $\vec{d'_i} = (d'_{i,1},\ldots,d'_{i,m})$. 
    We define $l_{i+1}=l_i\conc(1)$ as the unique child of $l_i$ in $\Pi$. 
    We also define $\vec{d_{i+1}} = \vec{d_i}$ and $\vec{a_{i+1}} = d_{i,k},\vec{a_i}$
    and define $\vec{d'_{i+1}} = \vec{d'_i}$ and $\vec{a'_{i+1}} = d'_{i,k},\vec{a'_i}$. 
    We obtain (i), (ii), and (iii) for $i+1$, as expected.
    We also have (iv) since the connections from
    $(\vec{d_i},\vec{a_i})$ to $(d_{i+1},\vec{a_{i+1}}) = (\vec{d_i},d_{i,k},\vec{a_i})$
    and
    $(\vec{d'_i},\vec{a'_i})$ to $(d'_{i+1},\vec{a'_{i+1}}) = (\vec{d'_i},d'_{i,k},\vec{a'_i})$
    are trace compatible: all connected $\N$-arguments in $\vec{d_{i}},\vec{a_{i}}$ and 
    $\vec{d_{i+1}},\vec{a_{i+1}}$ are the same. Similarly
    the connections of $\N$-arguments in $\vec{d'_i},\vec{a'_i}$ and $\vec{d'_{i+1}},\vec{a'_{i+1}}$ are the same.
    The only differences are that, if $D_k = \N$, the values $d_{i,k}$ and $d'_{i,k}$ of type $\N$
    for the variable $x_k^\N$ in $f[x_1,\ldots,x_k,\ldots,x_m](x_k)$ are duplicated to the values
    $d_{i,k}$ and $d'_{i,k}$ of the first unnamed argument of $ f[\vec{x}]$, respectively. 

  \item
    The case of $\lambda$-rule, namely
    $\Label(\Pi,l_i) = 
    \Gamma\vdash \lambda x^A.u[\vec{x},x] : A, \vec{A} \rightarrow \N$ is obtained from
    $\Gamma,x:A\vdash t_{i+1}[\vec{x},x^A]:\vec{A}\rightarrow\N$, 
    where $t_{i+1}[\vec{x},x]=u[\vec{x},x]$.
    By the induction hypothesis, $(\lambda x.(u[\vec{d_i},x]))a\vec{a_*} \not\sim_\N (\lambda x.(u[\vec{d'_i},x]))a'\vec{a'_*}$, 
    where $\vec{a_i} = a,\vec{a_*}$ and $\vec{a'_i} = a',\vec{a'_*}$.    
    Then, by Lemma~\ref{lem:uniq_NF}.\ref{lem:uniq_NF2}, $t_{i+1}[\vec{d_i},a]\vec{a_*}\not\sim_\N t_{i+1}[\vec{d'_i},a']\vec{a'_*}$.
    We define $l_{i+1}=l_i\conc(1)$ as the unique child node of $l_i$,
    and define $\vec{d_{i+1}} = \vec{d_i},a$ and $\vec{a_{i+1}} = \vec{a_*}$,
    and also define $\vec{d'_{i+1}} = \vec{d'_i},a'$ and $\vec{a'_{i+1}} = \vec{a'_*}$.
    We obtain (i), (ii), (iii) for $i+1$, as expected.
    We also have (iv) since the connections from 
    $(\vec{d_i},\vec{a_i}) = (\vec{d_i},a,\vec{a_*})$ to $(\vec{d_{i+1}},\vec{a_{i+1}}) = (\vec{d_i},a,\vec{a_*})$
    and
    $(\vec{d'_i},\vec{a'_i}) = (\vec{d'_i},a',\vec{a'_*})$ to $(\vec{d'_{i+1}},\vec{a'_{i+1}}) = (\vec{d'_i},a',\vec{a'_*})$
    are trace-compatible.
    We note that the values $a$ in $(\vec{d_i},\vec{a_i})$ and $a'$ in $(\vec{d'_i},\vec{a'_i})$ are
    the first unnamed arguments of $t_i[\vec{x}]$, and, in $(\vec{d_{i+1}},\vec{a_{i+1}})$ and $(\vec{d_{i+1}},\vec{a_{i+1}})$,
    they are moved to the values of the last variable of type $A$ of $t_{i+1}[\vec{x},x]=u[\vec{x},x]$.

  \item  
    The case of $\cond$-rule, namely
    $\Label(\Pi,l_i) = \Gamma\vdash C\rightarrow \cond(f[\vec{x}],g[\vec{x},x]):\N,\vec{A}\rightarrow\N$
    is obtained from 
    $\Gamma\vdash f[\vec{x}]:\vec{A}\rightarrow\N$
    and
    $\Gamma\vdash g[\vec{x}]:\N,\vec{A}\rightarrow\N$. 
    By the induction hypothesis, $\cond(f[\vec{d_i}],g[\vec{d_i},x])m\vec{a_*} \not\sim_\N \cond(f[\vec{d'_i}],g[\vec{d'_i},x])m\vec{a'_*}$, 
    where $\vec{a_i} = m,\vec{a_*}$ and $\vec{a'_i} = m',\vec{a'_*}$ with $m,m' \in \Num$.
    Since $m,m'$ are normal and $m \sim_\N m'$, we have $m = m'$. 
    We argue by cases on $m$.
    \begin{enumerate}
    \item
      We first consider the \emph{subcase $m=m'=0$}.
      Define $l_{i+1}=l_i\conc(1)$, namely the first child node of $l_i$ whose term is $f[\vec{x}]$. 
      We define $\vec{d_{i+1}} = \vec{d_i}$ and $\vec{a_{i+1}} = \vec{a_*}$,
      and also define $\vec{d'_{i+1}} = \vec{d'_i}$ and $\vec{a'_{i+1}} = \vec{a'_*}$. 
      We obtain (i), (ii), (iii) for $i+1$, as expected. 
      We also have (iv) since the connection from 
      $(\vec{d_i},\vec{a_i}) =(\vec{d_i},0,\vec{a_*})$ to
      $(\vec{d_{i+1}},\vec{a_{i+1}}) = (\vec{d_i},\vec{a_*})$ is trace compatible: 
      each argument of $t_i$ is connected to some equal argument of $t_{i+1}[\vec{x}]=f[\vec{x}]$,
      the first argument of $t_i[\vec{x}]$ disappears but it is connected to no $\N$-argument in $f[\vec{x}]$.
    \item
      Next we consider the \emph{subcase $m=m'=\Succ(m_*)$}. 
      Define $l_{i+1}=l_i\conc(2)$, namely the second child node of $l_i$ whose term is $g[\vec{x}]$. 
      We define $\vec{d_{i+1}} = \vec{d_i},m_*$ and $\vec{a_{i+1}} = \vec{a_*}$,
      and also define $\vec{d'_{i+1}} = \vec{d'_i},m_*$ and $\vec{a'_{i+1}} = \vec{a'_*}$. 
      We obtain (i), (ii), (iii) for $i+1$, as expected.
      We also have (iv) since the connections from 
      $(\vec{d_i},\vec{a_i}) =(\vec{d_i},\Succ(m_*),\vec{a_*})$ to $(\vec{d_{i+1}},\vec{a_{i+1}}) = (\vec{d_i},m_*,\vec{a_*})$
      and
      $(\vec{d'_i},\vec{a'_i}) =(\vec{d'_i},\Succ(m_*),\vec{a'_*})$ to $(\vec{d'_{i+1}},\vec{a'_{i+1}}) = (\vec{d'_i},m_*,\vec{a'_*})$      
      are trace compatible.
      We note that, in both trace-compatible assignments, the first unnamed argument $m = \Succ(m_*)$ of $t_i[\vec{x}]$ 
      which is connected the first unnamed argument $m_*$ of $g[\vec{x}]$, \emph{decreasing by $1$ from $m$}.
      This is indeed as we expected for trace-compatible assignments since this $i$ is a progress point of them. 
    \end{enumerate}
  \end{enumerate}
  
  By the above construction, we have an infinite path $\pi=(e_1,e_2,\ldots)$ in $\Pi$ and
  two trace compatible assignments
  $\rho = ((\vec{d_1},\vec{a_1}),(\vec{d_2},\vec{a_2}),\ldots)$
  and
  $\rho' = ((\vec{d'_1},\vec{a'_1}),(\vec{d'_2},\vec{a'_2}),\ldots)$ for $\pi$,
  as we wished to show. 
  
\end{proof}

As a corollary from the theorem we will conclude: 
if $t$ satisfies the global trace condition, then there is no such $\sigma$, therefore $t \sim t$, 
as we wished to show.

\begin{corollary}\label{cor:UniqueNF_typeN}
  \begin {enumerate}
  \item
    Assume $\Gamma\vdash t:A$. Then $t \in \GTC$ implies that $t$ is confluent. 
  \item
    For any closed $t:\N$, there is numeral $n\in\Num$ such that $t\safeReducesAst n$.
    Moreover, for any normal form $u$ such that $t \safeReducesAst u$, we have $u=n$. 
  \end{enumerate}
\end{corollary}


%We prove that for any term $ (t \sim t)$ with counter-example $\vec{a},\vec{d} \sim_0 \vec{b},\vec{e}$
%and $\neg (t[\vec{a}](\vec{d}) \sim_0  t[\vec{b}](\vec{e}))$ we can find some immediate
%subterm $t'$ and with a counter-example $\vec{a'},\vec{d'} \sim_0 \vec{b'},\vec{e'}$
%and $\neg (t'[\vec{a'}](\vec{d'}) \sim_0  t'[\vec{b'}](\vec{e'}))$, \emph{compatible with trace
%condition}.


%21:09 20/03/2024
 % SUBSUMED IN THE CHURCH-ROSSER SECTION

%\section{Appendix: Infinite reductions}
\label{section-infinite-reductions}
\bfColor{red}{(This section is but a draft)}
For infinite terms we can define a notion of reduction contracting infinitely many redexes at the same time.
In this section investigate this notion, 
we show that for recursive terms it is a recursive operation, and that the result
is well-defined for the set $\GTC$ of terms with the global trace condition. 
We also show that $\GTC$ is closed under one step 
of reduction contracting infinitely many redexes at the same time.

Assume $t \in \LAMBDA$ and $X$ is a set of redexes in $t$. We define an operation $\reduces_X$ 
on each node $t'$ of $t$, and its result $u \in \LAMBDA$ by induction on the distance between
$t'$ and $t$.

If $t \not \in X$ and $t=0,\Succ(u), x, \lambda x^Tu, f(u), \cond(f,g)$, we apply $\reduces_X$
on all immediate subterms $t'$ of $t$. If $t = (\lambda x^T.b)(u), \cond x^\N.(f,g)(0), \cond x^\N.(f,g)(\Succ(u))$,
we contract $t$, respectively, to $b[u/x], f, g[u/x]$, then we apply $\reduces_X$ to the result.

The operation $\reduces_X$ loops if in some path of $t$ we always find redexes in $X$. This is the case of 
$t = (\lambda x^T.t)(x)$ and of $u=\cond x^\N.(0,u)(1)$ and of $X=$ all redexes of $t$, $u$ respectively. 
In both terms, the unique infinite path of the term is made of redexes of $X$, and $\reduces_X$ never
provides a single symbol of the output.

If we want to consider infinite redexes in $\LAMBDA$ we should allow undefined sub-terms inside a term.
This is not the case for the set $\GTC$ of terms with the global trace condition:
we can show that $\GTC$ is closed under $\reduces_X$.

We first check that $\reduces_X$ applied to $\GTC$ produces no undefined sub-term.

\begin{lemma}[No infinite rexed path]
If $t \in \GTC$ (if $t$ has the global trace condition)
then $\reduces_X$ applied to $t$ produces no undefined sub-term.
\end{lemma}

\begin{proof}
We have to prove that there is no path in $t$ only made  of redexes.
Suppose a path in $t$ is only made  of redexes 
$r=(\lambda x^T.b)(u)$ or $r=(\cond x^\N.(f,g))(t)$, where $t=0$ or $t=\Succ(u)$. 
We first check that no trace $\tau$ starting from $r$ can progress in the next sub-term of the path.
Indeed, if $\tau$ progresses, then $r=(\cond x^\N.(f,g))(t)$
and $\tau$ to the type of the variable $x$ bound by $\cond x^\N.(f,g)$. 
The only way to obtain this is that $t=x$ and $r$ is obtained by an $\etaRule$,
and in this case $r$ is no redex, because $x \not = 0, \Succ(u)$.

This implies that in a path $\pi$ only made of redexes any trace $\tau$ 
cannot progress after its first point, and therefore it cannot infinitely progress and the term has
no global trace condition.
\end{proof}


We prove that $\GTC$ is closed under $\reduces_X$.


\begin{lemma}
If $t \in \GTC$ and $t \reduces_X u$ then $u \in \GTC$
\end{lemma}

\begin{proof}
We prove that for any infinite path $\pi$ in $u$ there is some infinite path $\rho$ in $t$ such that
for all traces $\tau$ of $\rho$ there is some trace $\sigma$ of $\pi$ which is obtained by
removing some points of $\tau$ which are either the first point or are not progress points. 
It will follow that if we have an infinitely progressing trace $\tau$ in $\rho$, then we have some infinitely
progressing trace $\sigma$ in $\pi$.

We follow $\pi$ and we apply $\reduces_X$. We find finitely many redexes which we reduce to
some $v[t_n/x_n]\ldots[t_1/x_1]$ with $v$ not redex. A single step in $\rho$ through $v$
corresponds to $n+1$ steps in $\pi$, in such a way that a trace in $\pi$ is restricted to a trace
in $\rho$. The restriction removes no progresses point but at most the first. 

If the path $\rho$ continues in this way forever then we are done.

The other possibility is: $\pi$ could move to some $t_i$ which was an $x_i$ in $t$. In this case we restart
the path in $t$ through $t_i$, while $\rho$ continues in $t_i[t_{i-1}/x_{i-1}]\ldots[t_1/x_1]$.
After finitely many steps $\rho$ can continue through $t_j$ for some $1 \le j < i$, which was an $x_j$
in $t$. This process can continue at most $n$ times. 
Eventually $\rho$ continues forever inside $t_k$ for some $1 \le k \le n$. In this case there is a $1$ to $1$
connection between the trace in $\rho$ and in $t_k$ and the traces in $\pi$ and in $t_k$.
\end{proof}          % DRAFT

%\section{Appendix: Normalization and Fairness}
\label{section-normalization-fairness}
\bfColor{red}{(This section is but a draft)}
\\
\emph{(Here we should insert our notes for the proof of strong normalization for safe reductions)}
\ldots\ldots\ldots\ldots\ldots\ldots\ldots\ldots\ldots\ldots
\\

There are terms requiring infinite reductions to normalize, but there are reductions normalizing in the limit.
We conjecture that we can characterize reductions normalizing in the limit using the notion of "fair" reductions.

\begin{definition}
\begin{enumerate}
\item
We say that a node is in the $k,\cond$-level if it has less than $k$ nodes $\cond$ in its branch.
\item
We say that an infinite reduction sequence is is \emph{fair} for nodes of $k,\cond$-level in the following case:
for all $i\in \N$, all $k>0$, there is some  $j \ge i$ such that at step $j$ either all nodes in the 
$k,\cond$-level are normal, or one of them is reduced at step $j$.
\end{enumerate}
\end{definition}

\emph{Conjecture 1}. For all $k \in \N$, an infinite reduction sequence $\sigma$ 
eventually stop reducing nodes in the $k,\cond$-level.

Conjecture 1 should be proved by adapting the proof of strong normalization for safe reductions to a
termination result on the first $k$-levels for all reductions, and assuming the same result for $k-1$.

\emph{Conjecture 2}. An infinite reduction sequence $\sigma$ 
is normalizing in the limit if and only if for all $k \in \N$,
$k > 0$, $\sigma$ is fair for the task of reducing nodes in the $k,\cond$-level.

Conjecture 2 should follow in few steps from conjecture Conjecture 1.

    %DRAFT

\newpage

\section{Appendix: Equivalence Between Cyclic and non-Cyclic System T} 
\label{section-equivalence-cyclic-non-cyclic-T}
%\bfColor{red}{(This section is but a draft)}
In this section we prove that the two systems $\systemT$ and $\CTlambda$ 
are each interpretable into the other. 
Both interpretations preserve the reduction relation, applications, $0$ and $\Succ$ and contexts.

The formal definition of the system $\systemT$ is as follows:

Types (denoted by $A,B,\ldots$) of $\systemT$ are inductively defined by $\N$ and $A\to B$.

Terms (denoted by $t,u,\ldots$) of $\systemT$ are inductively defined by
$x^A$, $\lambda x^A.t$, $tu$, $0$, $\Succ(t)$, $\Rec(t,u)$, and $\Ifz(t,u)$.

Sequents is $\Gamma\vdash t:A$, where $\Gamma$ is a context $x_1:A_1,\ldots,x_n:A_n$.

Typing rules are defined as follows:
\begin{enumerate}
\item
  The $\var$ rule: $\Gamma\vdash x:A$, where $x:A\in\Gamma$.
\item
  The $\weak$ rule: $\Gamma\vdash t:A$ and $\Gamma\subseteqsim \Gamma'$, where $\Gamma'\vdash t:A$.
\item
  The $\lambda$ rule: $\Gamma,x:A\vdash t:B$ implies $\Gamma \vdash \lambda x.t:A\to B$.
\item
  The $\ap$ rule: $\Gamma\vdash t:A\to B$ and $\Gamma\vdash u:A$ implies $\Gamma \vdash tu:B$.
\item
  The $0$ rule: $\Gamma\vdash 0:\N$.
\item
  The $\Succ$ rule: $\Gamma\vdash t:\N$ implies $\Gamma\vdash \Succ(t):\N$.
\item
  The $\Rec$ rule: $\Gamma\vdash a:A$ and $\Gamma\vdash h:A,\N\to A$ implies $\Gamma\vdash \Rec(a,h):\N \to A$.
\item
  The $\Ifz$ rule: $\Gamma\vdash u:A$ and $\Gamma\vdash v:A$ implies $\Gamma\vdash \Ifz(u,v):\N \to A$.
\end{enumerate}

The basic reduction relation $\reduces_0$ is defined by:
\begin{enumerate}
\item
  $(\lambda x.t)u \reduces_0 t[u/x]$, 
\item
  $\Rec(a,h)0 \reduces_0 a$, 
\item
  $\Rec(a,h)\Succ(u) \reduces_0 h(\Rec(a,h)u))u$, 
\item
  $\Ifz(u,v)0 \reduces_0 u$, 
\item
  $\Ifz(u,v)\Succ(u) \reduces_0 v$, 
\end{enumerate}

Then the reduction relation $\reduces_\systemT$ is the coontext and transitive closure of $\reduces_0$. 

%In both interpretations 
%we neglect the terms of $\CTlambda$ including type variables in their type or in their context,
%because these terms have no corresponding in $\systemT$.\\

We first provide the interpretation $\TtoCT{-}$ from $\systemT$ to $\CTlambda$
that is inductively defined as follows. 
\begin{center}
  $\TtoCT{x^A} = x^A$, 
  \qquad
  $\TtoCT{\lambda x^A.t} = \lambda x^A.\TtoCT{t}$, 
  \qquad
  $\TtoCT{tu} = \TtoCT{t}\TtoCT{u}$, 
  \qquad
  $\TtoCT{0} = 0$, 
  \qquad
  $\TtoCT{\Succ(t)} = \Succ(\TtoCT{t})$, 
  \\
  $\TtoCT{\Ifz(u,v)} = \cond(\TtoCT{u},\lambda z.\TtoCT{v})$, where $z\not\in\FV(v)$, and 
  \\
  $\TtoCT{\Rec(a,h)} = f$, where $f = \cond(\TtoCT{a},\lambda n.\TtoCT{h}(fn)n)$. 
\end{center}

Note that the translated terms are regular.
We can easiliy check that the translation preserves substitution,
namely $\TtoCT{t[u/x]} = \TtoCT{t}[\TtoCT{u}/x]$. 

The interpretation preserves typablity. 

\begin{proposition}\label{prop:TtoCTproof}
$\Gamma \vdash t:A$ holds in $\systemT$ implies $\Gamma\vdash \TtoCT{t}:A$ holds in $\CTlambda$. 
\end{proposition}
\begin{proof}
  It is shown by induction on the derivation of $\Gamma \vdash t:A$.
  We check only the two cases: the last rule is $\Ifz$ or $\Rec$.

  The case of $\Ifz$:
  the sequent $\Gamma\vdash\Ifz(u,v):\N\to A$ is obtained from subproofs of
  $\Gamma\vdash u:A$ and $\Gamma\vdash v:A$. 
  Then, by the induction hypothesis, we have proofs of $\Gamma\vdash \TtoCT{u}:A$ and $\Gamma\vdash \TtoCT{v}:A$
  in $\CTlambda$.
  Hence we have a proof of $\Gamma\vdash\cond(\TtoCT{u},\lambda z.\TtoCT{v}):\N\to A$
  by $\weak$ and $\cond$, where $z$ is a fresh variable. 
  This is a proof of $\Gamma\vdash \TtoCT{\Ifz(u,v)}:\N\to A$ in $\CTlambda$.
  Note that this is an expected proof, since it satisfies the global trace condition and regularity. 

  The case of $t = \Rec(a,h)$: 
  the sequent $\Gamma\vdash\Rec(a,h):\N\to A$ is obtained from subproofs of
  $\Gamma\vdash a:A$ and $\Gamma\vdash h:A,\N\to A$.
  By the definition, $\TtoCT{\Rec(a,h)} = \cond(\TtoCT{a},\lambda n.\TtoCT{h}(\TtoCT{\Rec(a,h)}n)n)$ holds.
  Then we have the following proof in $\CTlambda$. 
  \begin{center}\small
    $\infer[\cond]{
      \Gamma \vdash \TtoCT{\Rec(a,h)}:\redN\to A\quad(\dagger)
    }{
      \infer*[\Pi_a]{
        \Gamma \vdash \TtoCT{a}:A
      }{}
      &
      \infer{
        \Gamma \vdash \lambda n.\TtoCT{h}(\TtoCT{\Rec(a,h)}n)n:\redN\to A
      }{
        \infer[\apvar]{
          \Gamma,n:\redN \vdash \TtoCT{h}(\TtoCT{\Rec(a,h)}n)n: A
        }{
          \infer{
            \Gamma,n:\redN \vdash \TtoCT{h}(\TtoCT{\Rec(a,h)}n): \N\to A
          }{
            \infer{
              \Gamma,n:\N \vdash \TtoCT{h}: A,\N\to A
            }{
              \infer*[\Pi_h]{
                \Gamma \vdash \TtoCT{h}: A,\N\to A
              }{}
            }
            &
            \infer[\apvar]{
              \Gamma,n:\redN \vdash \TtoCT{\Rec(a,h)}n: A
            }{
              \infer{
                \Gamma,n:\N \vdash \TtoCT{\Rec(a,h)}: \redN\to A
              }{
                \Gamma \vdash \TtoCT{\Rec(a,h)}: \redN\to A\quad(\dagger)
              }
            }
          }
        }
      }
    }$
  \end{center}
  This proof satisfies the global trace condition:
  Its infinite path is either
  one that eventually an infinite path of $\Pi_a$,
  one that eventually an infinite path of $\Pi_h$, or 
  one that loops between the upper $(\dagger)$ and the lower $(\dagger)$. 
  Then paths of the former two cases contains a progressing trace by the induction hypothesis, and
  the last case contains a progressing trace (the trace consists of the red $\N$s). 
  Hence we obtain a proof of $\Gamma \vdash \TtoCT{\Rec(a,h)}:\N\to A$ as we wished. 
\end{proof}

For considering the corresponding reduction relation of $\systemT$ and the safe reduction of $\CTlambda$,
we need to introduce a reduction strategy, namely the call-by-name strategy,
that prevents reductions of terms inside $\Rec$ and $\Ifz$ that require
to reduce terms inside $\cond$ of translated $\CTlambda$. 

\begin{definition}
  The call-by-name reduction relation $\reduces_n$ is inductively defined as follows:
  \begin{enumerate}
  \item
    If $t \reduces_0 t'$, then $t \reduces_n u$, 
  \item
    $t \reduces_n t'$ implies $tu \reduces_n t'u$,
  \item
    $t \reduces_n t'$ implies $\Rec(a,h)t \reduces_n \Rec(a,h)t'$, and 
  \item
    $t \reduces_n t'$ implies $\Ifz(u,v)t \reduces_n \Ifz(u,v)t'$. 
  \end{enumerate}
\end{definition}

The call-by-name reduction and the 

\begin{lemma}\label{lem:cbnT}
  Assume that $t$ has a closed term of type $\N$. Then the following claims hold. 
  \begin{enumerate}
  \item\label{lem:cbnT1}
    If $t$ is not a numeral, then $t \reduces_n u$ for some $u$. 
  \item\label{lem:cbnT2}
    $t\reduces n$ if and only-if $t \reduces_n n$.
  \end{enumerate}
\end{lemma}
\begin{proof}
  The claim \ref{lem:cbnT1} is shown by induction on $t$. 
  By the assumption, $t$ has a form of either $\Succ(t')$, $(\lambda x.t_1)t_2\vec{t}$,
  $\Rec(a,h)t'\vec{t}$, or $\Ifz(v_1,v_2)t'\vec{t}$.
  For the first case, there is $u'$ such that $t\reduce_n u'$ by the induction hypothesis. 
  Then, take $u = \Succ(u')$.
  For the second case, take $u = t_1[t_2/x]\vec{t}$.
  For the third case, take $u = a\vec{t}$ if $t'=0$, take $u=h(\Rec(a,h)t'')t''\vec{t}$ if $t'=\Succ(t'')$, and
  take $\Rec(a,h)u'\vec{t}$ otherwise, where $t'\reduces_n u'$ by the induction hypothesis.
  For the last case, take $u = v_1\vec{t}$ if $t'=0$, take $u=v_2\vec{t}$ if $t'=\Succ(t'')$, and
  take $\Ifz(v_1,v_2)u'\vec{t}$ otherwise, where $t'\reduces_n u'$ by the induction hypothesis.

  We show the claim \ref{lem:cbnT2}.
  The right-to-left direction is trivially shown. The left-to-right direction 
  is shown by \ref{lem:cbnT1} and the confluency of $\reduces$: 
  Assume that $t\reduces n$. By the first claim, $t \reduces_n n'$ for some numeral $n'$.
  Then we have $n=n'$ by the confluency. Hence $t\reduces_n n$ holds as we wished. 
\end{proof}

\begin{proposition}\label{prop:TtoCTreduction}
  Let $t$ and $u$ be terms of $\systemT$.  
  \begin{enumerate}
  \item\label{prop:TtoCTreduction1}
    $t\reduces_n u$ implies $\TtoCT{t} \safeReduces \TtoCT{u}$. 
  \item\label{prop:TtoCTreduction2}
    If $t$ is a closed term of type $\N$ and $t\reduces n$, then $\TtoCT{t} \safeReduces n$. 
  \end{enumerate}
\end{proposition}  
\begin{proof}
  The point \ref{prop:TtoCTreduction1} is shown by induction on $\reduces_n$.
  It is enough to check the base cases.

  The case of $(\lambda x.t)u \reduces_n t[u/x]$ is shown by 
  $\TtoCT{(\lambda x.t)u} = (\lambda x.\TtoCT{t})\TtoCT{u} \safeReduces \TtoCT{t}[\TtoCT{u}/x] = \TtoCT{t[u/x]}$. 

  The case of $\Rec(a,h)0 \reduces_n a$ is shown by
  $\TtoCT{\Rec(a,h)0} = \cond(\TtoCT{a},\lambda n.\TtoCT{h}(\TtoCT{\Rec(a,h)}n)n)0 \safeReduces \TtoCT{a}$.

  The case of $\Rec(a,h)\Succ(u) \reduces_n h(\Rec(a,h)u)u$ is shown by
  \begin{align*}
    \TtoCT{\Rec(a,h)\Succ(u)}
    &= \cond(\TtoCT{a},\lambda n.\TtoCT{h}(\TtoCT{\Rec(a,h)}n)n)\Succ(\TtoCT{u})
    \safeReduces (\lambda n.\TtoCT{h}(\TtoCT{\Rec(a,h)}n)n)\TtoCT{u}
    \\
    &\safeReduces \TtoCT{h}(\TtoCT{\Rec(a,h)}\TtoCT{u})\TtoCT{u})
    = \TtoCT{h(\Rec(a,h)u)u}
  \end{align*}
  
  The case of $\Ifz(u,v)0 \reduces_n u$ is shown by
  $\TtoCT{\Ifz(u,v)0} = \cond(\TtoCT{u},\lambda z.\TtoCT{v})0 \safeReduces \TtoCT{u}$.

  The case of $\Ifz(u,v)\Succ(t) \reduces_n u$ is shown by
  $\TtoCT{\Ifz(u,v)\Succ(t)} = \cond(\TtoCT{u},\lambda z.\TtoCT{v})\Succ(t) \safeReduces (\lambda z.\TtoCT{v})t \safeReduces \TtoCT{v}$.

  Hence we have the point \ref{prop:TtoCTreduction1}.
  The point \ref{prop:TtoCTreduction2} is obtained by the point \ref{prop:TtoCTreduction1}
  and Lemma~\ref{lem:cbnT}~\ref{lem:cbnT2}. 
\end{proof}

%For any type $T$ we define a term $\Rec:T,(\N,T \rightarrow T),\N\rightarrow T$ such that
%$\Rec(a,f,0) = a$ and $\Rec(a,f)(n+1) = f(n,\Rec(a,f,n))$, for all numeral $n \in \Num$.
%The definition is $\Rec = \lambda a,f.\rec$ 
%with $\rec = \cond (a,\lambda x^{\N}.f(x,\rec(x))) : \N \rightarrow T$.

%\begin{proposition}
%$\Rec$ is a term of $\CTlambda$.
%\end{proposition}
%% \begin{proof}
%% \begin{enumerate}
%% \item
%%  $\Rec$ is regular by construction.
%% \item 
%% The only infinite path of $\Rec$ loops from $\rec$ to $\rec$ infinitely many times, and it includes
%% an infinitely progressing trace. Here it is: from the first unnamed argument of $\rec$ to 
%% to the first unnamed argument of $\lambda x^{\N}.f(x,\rec(x))$, then to $x^\N$ 
%% in the context of $f(x,\rec(x))$, to $x^\N$ in the context of $\rec(x)$,
%% and eventually to the first unnamed argument of $\rec$ again.
%% \end{enumerate}
%% \end{proof}

%% If we replace each primitive recursion in $\systemT$ with the term $\Rec$ in $\CTlambda$
%% we define an interpretation from $\systemT$ into $\CTlambda$, 
%% preserving reductions, applications, $0$ and $\Succ$ and contexts.
%% \\

\bfColor{red}{The rest of the section if but an early draft}
The opposite interpretation, from $\CTlambda$ to $\systemT$, it has been defined by proof-theory for the
combinatorial version of circular $\systemT$. For $\CTlambda$, instead we will define an algorithm
taking an infinite term in  $\CTlambda$, described as a finite circular tree, 
and returning a term in $\systemT$ with the required properties. 

First, we need a notion of confluence and of extensional equality 
for  functionals of $\CTlambda$ and of $\systemT$.

We define $t \sim_{\beta,\rec} u$ (a syntactical confluence for $\systemT$) 
if and and only if $t, u \in \systemT$ and $t,u$ have the same type
and for some $v \in \systemT$: $t \reduces_{\beta,\rec} v$ and $u \reduces_{\beta,\rec} v$.

We define an equivalence relation (an extensional equality for $\systemT$) 
$\sim_{\systemT}$ on $\systemT$, by induction on the type. 


\begin{definition}[An extensional equality on $\systemT$]
Assume $t,u \in \systemT$.
\begin{enumerate}
\item
If $t,u:\N$ and $t,u$ are closed then we set: 
$(t \sim_{\systemT} u) \Leftrightarrow  (t \sim_{\beta,\rec} u)$.
\item
If $t,u:A,\vec{A}\rightarrow\N$ and $t,u$ are closed then we set: 
$(t \sim_{\systemT} u) \Leftrightarrow  
\forall a \in \systemT. (a:A), (\mbox{$a$ closed}) \Rightarrow (t(a) \sim_{\systemT} u(a))$.
\item
If $t,u:\vec{A}\rightarrow\N$ and $\FV(t), \FV(u) \subseteq \vec{x}$ then we set:
$$
(t \sim_{\systemT} u) 
\Leftrightarrow  
\forall \vec{a} \in \systemT. 
(\vec{a}:\vec{A}), (\mbox{$\vec{a}$ closed})  
\Rightarrow 
(t[\vec{a}/\vec{x}] \sim_{\systemT} u[\vec{a}/\vec{x}])
$$
\end{enumerate}
\end{definition}

Then we can consider $\systemT$ as a structure $(\systemT/\sim_{\systemT}, 0, \Succ, \ap)$
with natural numbers, functionals and extensional equality. 
In the same way we define an equivalence relation on terms of $\CTlambda$ which denote functionals.
We only consider terms whose type and context only include the atomic type $\N$, and no type variables. 


\begin{definition}[An extensional equality on $\CTlambda$]
Assume $t,u \in \CTlambda$.
\begin{enumerate}
\item
If $t,u:\N$ and $t,u$ are closed then we set: 
$(t \sim_{\CTlambda} u) \Leftrightarrow  (t \sim_{\CTlambda} u)$.
\item
If $t,u:A,\vec{A}\rightarrow\N$ and $t,u$ are closed then we set: 
$(t \sim_{\CTlambda} u) \Leftrightarrow  
\forall a \in \CTlambda. (a:A), (\mbox{$a$ closed}) \Rightarrow (t(a) \sim_{\CTlambda} u(a))$.
\item
If $t,u:\vec{A}\rightarrow\N$ and $\FV(t), \FV(u) \subseteq \vec{x}$ then we set:
$(t \sim_{\CTlambda} u) 
\Leftrightarrow  
\forall \vec{a} \in \CTlambda. 
(\vec{a}:\vec{A}), (\mbox{$\vec{a}$ closed})  \Rightarrow (t[\vec{a}/\vec{x}] \sim_{\CTlambda} u[\vec{a}/\vec{x}])$.
\end{enumerate}
\end{definition}

Our goal is now to prove that there is an embedding from the types of $\N$-functionals in 
$(\CTlambda/ \!\! \sim_{\CTlambda}, 0, \Succ, \ap)$ to the entire
$(\systemT/ \!\! \sim_{\systemT}, 0, \Succ, \ap)$.
We ignore types of $\CTlambda$ including type variables, they have no corresponding in $\systemT$.

For each $\N$-functional type $T$ we have to define a map $\phi_T$ from terms of type $T$ of $\CTlambda$
to terms of type $T$ of $\systemT$. We abbreviate $\phi_T$ with $\phi$ and we require:

\begin{enumerate}
\item
If $t \sim_{\CTlambda} u$ \ \ \ \ \ \ \ \ \ \ \ \ \ then $\phi(t) \sim_{\systemT} \phi(u)$

\item
If $t: A \rightarrow B$, $u:A$ \ \ \ then $\phi(t(u)) \sim_{\systemT} \phi(t)(\phi(u))$

\item
$\phi(0) \sim_{\systemT} 0$ and
$\phi(\Succ(t)) \sim_{\systemT} \Succ(\phi(t))$

\end{enumerate}

We will define an embedding from $\CTlambda$ to $\systemT$ extended with $+,\times$-types.
There is an embedding from  $\systemT$ extended with $+,\times$-types to  $\systemT$ with only
$\N, \rightarrow$, therefore it is enough to embed $\CTlambda$ to $\systemT$ extended with $+,\times$-types.

We suppose be fixed a cyclic $\lambda$-term $t \in \CTlambda$, $t : T$,
 and we use several ingredients. First we consider all $\Gamma_i \vdash u_i : U_i$ for $u_i$ sub-term of $t$,
we turn it into a simple type $\Gamma_i \rightarrow U_i$, then we consider a single type
$S = \Sigma_i \Gamma_i \rightarrow U_i$. 

%14:47 11/06/2024

\begin{enumerate}
\item
We suppose be given a map $\trunk_t:\N \rightarrow S = \Sigma_i \Gamma_i \rightarrow U_i$, 
such that $\trunk_t(n)$ is 
the unfolding of $t$ with all subterms $u:U$ in the level number $n$ replaced by a dummy term $0_U$.
We use the type $S$ in order to have a common type and context for all sub-terms $u$.

\item
For each pair of subterms $u,v$ of $t$ and each path $\pi$ from $u$ to $v$. 
we suppose be given a one-to-many trace
relation $R_\pi$ between the indexes of $\N$-arguments of $u$ and of the $\N$-arguments of $v$. 
We close the set of relations $R_\pi$ by composition 
and we obtain a finite set (of exponential size in $t$).
\end{enumerate}

By the global trace condition for each infinite composition of $R_\pi$ there is some infinitely progressing
trace. This means that there is some index $i$ in the domain of $R_\pi$ such that for some $n \in \N$
we have $R^n_\pi(i,i)$ and the variable $i$ progresses.


We consider any assignment $t' \equiv t[\vec{n},\vec{x}](\vec{m},\vec{y})$ of the arguments of $t$.
The idea is to find map $\phi$ such that for all $p \ge \vec{n}, \vec{n}$ we have
$\trunk_{t'}(m)$ stationary for all $m \ge \phi(p)$, therefore $t' = \trunk_{t'}(\phi(p))$.

$\phi_c(p)$ is the map computing the maximum number of nodes for an Erdos tree in $c$
colors with height $\le p$ and branching $\le c$ (there is at most one child per color).
Whenever an Erdos tree has at least a branch of length $cp+1$ and $c$ colors, 
then we have a monotonically colored sequence
of length $cp+1$, therefore at least one homogeneous set of length $p+1$. If we take any composition of 
$\phi_c(p)+1$ times some relations $R_\pi$, there is some homogeneous set of $(p+1)$ elements
decorated with a single $R_\pi$, and for some $i$ some variable starting from some value $\le p$ 
and decreasing $p$ times. 

This means that each branch terminates and the whole computation of $\trunk_{t'}(\phi(p))$ terminates.

For a $c$-color tree, the value of $\phi_c(p)$ is $1+c+c^2+\ldots+c^{p-1} = (c^{p}-1)/(c-1)$.
Suppose $f:\N \rightarrow \N$ is some weakly increasing map. We want to prove that
there is some Erdos tree including some branch including some $f$-homogeneous set: some
homogeneous set with first point $h$ followed by $f(h)$ more points. We
want to define a functional in $\systemT$ such that all Erdos trees with $\ge F(f)$ nodes
include some $f$-homogeneous set.

We call a Ramsey functional any functional $F(\vec{x},c)$ associated to $e$
any functional taking weakly increasing functionals $\vec{x}$,
and returning an upper bound for the size of a $c$-color Erdos tree including some homogeneous set
with first node $l$ with value $\le e(\vec{x},\vec{F},l)$, 
followed by $\le e(\vec{x},\vec{F},l)$ more nodes, in which some variable $i$ decreases by $1$
each $n$ steps ($n$ number of sub-terms of the cyclic term). 

We require that $e$ is any primitive recursive
functional, that is, $e$ is 
defined by simply typed lambda calculus plus recursion on $\Seq(\N)$, and $\vec{F}$ are Ramsey functionals.

We claim that all sub-terms of $t$ have a computation time bounded by some Ramsey functional.
If this is not the case, we define some infinite path with a infinitely progressing trace
associated to some infinitely decreasing numeral, contradiction.

%17:03 12/06/2024

%?????????????????????
%22:31 10/06/2024

\ldots\ldots\ldots

%We require the following property of $\systemT$: we can define terms of $\systemT$ by $n$
%simultaneous lexicographic inductions. If $\vec{n} \in \N^m$ are numeral, we define 
%$\vec{n} \lexicographic{} \vec{m}$ if and only if $\vec{n} \not = \vec{m}$ and for the first $i \in [1,m]$
%such that $n_i \not = m_i$ we have $n_i +1 = m_i$. 
%If $T=\vec{A} \rightarrow \N$ is a functional type we define
%$0_T = \lambda \vec{x}:\vec{A}.0$. If $f:T$ we define $f \restr  \vec{x}$
%as the restriction of $f_i$ to the set of $\vec{y} \lexicographic{} \vec{x}$, 
%extended by the dummy value $0_T$:
%\begin{center}
% $(f \restr  \vec{x})(\vec{y}) = f(\vec{y})$ if $\vec{y} \lexicographic{} \vec{x}$ 
%\ \ \ 
%and 
%\ \ \ 
%$(f \restr  \vec{x})(\vec{y}) = 0_T$ otherwise
%\end{center}

%
%\begin{proposition}[Simultaneous lexicographic induction in $\systemT$]
%Assume $\vec{x}:\N^m$,
%and that we have any equation list in $\systemT$, in the meta-variables $f_1, \ldots, f_n$:
%$$
%f_i(\vec{x}) 
%\ \ \ 
%\sim_{\systemT} 
%\ \ \ 
%F_i(\vec{x}, f_1 \restr  \vec{x}, \ldots, f_n \restr  \vec{x})
%\ \ \ 
% : 
%\ \ \ 
%T_i
%$$
%This  equation list has solutions 
%$
%f_1:\N^m \rightarrow T_1, 
%\ldots, 
%f_n:\N^m \rightarrow T_n
%$ 
%in $\systemT$ and we can compute them. 
%\end{proposition}
%
%
%Assume $t:T$ is a cyclic term, represented as a cyclic tree with node $t_1$, \ldots, $t_n$.
%We translate it to a term of $\systemT$, obtained by solving an equation list whose meta-variables
%are the nodes the cyclic tree, with $\vec{x} = \vec{y},c$,
%and $\vec{y}$ the union of the $\N$-arguments of each node,
%and $c$ a variable used as a counter. The counter 
%$c$ starts from $n$, the number of nodes, and decreases
%of $1$ unit in each recursive call. 
%Within $n$ recursive calls, one or more value of $\vec{y}$ decreases by $1$,
%while all other values stay the same. In this case the varabile $c$ is reset to $n$: $c$ is the only
%variable which can increase during computation.
%
%
%%
%%This is a first draft about how to do it.
%%
%%%%%%%%%%%%%%%
%% % TO BE IMPROVED
%%%%%%%%%%%%%%%
%%
%%\begin{enumerate}
%%
%%\item
%%We first move all nodes to a context with the same number on type $\N$ variables, 
%%by adding dummy variables and dummy arguments.
%%This operation preserves regularity and global trace condition.
%%Now $t_1$, \ldots, $t_n$ all have context $\Gamma$ and type $A$.
%%
%%\item
%%We merge all buds into the same term, defined by some $u$ such that $u(i)=t_i$, for $i=1, \ldots, n$,
%%and $u(i)=$ some dummy term of type $A$ otherwise. We replace each $t_i$ with $u(i)$, 
%%for $i=1, \ldots, n$.
%%This operation preserves regularity and global trace condition.
%%Now we have $n$ buds, all are the same $u$ with context $\Gamma$ and type $\N \rightarrow A$.
%%Each bud $b$ defines a partial bijection between the occurrence of $\N$ in its context and type
%%$\Gamma \vdash \N \rightarrow A$, and the occurrences of $\N$ in the context and type
%%$\Gamma \vdash \N \rightarrow A$ of its companion. 
%%We extend this partial bijection to any total bijection $\tau$, depending on the but $b$.
%%
%%\item
%%We close the partial bijections defined by each bud by composition. The number of partial 
%%bijections can grow in an exponential  way.
%%
%%\item
%%Assume we have $m$ occurrences of $\N$ inside the context and type 
%%$\Gamma \vdash \N \rightarrow A$ of $u$.
%%We fix a permutation $\sigma:\{1, \ldots, m\}$ 
%%and we label them by variables $x_1, \ldots, x_n$ of $\systemT$,
%%with $x_i$ label of the argument with type $\N$ and number $i$.
%%We will define a translation $t^\sigma \in \systemT$ of  $t \in \CTlambda$.
%%
%%\item
%%All traces move from $u$ to any of the occurrences of $u$ inside $u$. 
%%Some traces of some $\N$
%%in $\Gamma \vdash \N \rightarrow A$ disappear, some other are moved to some other $\N$,
%%in an injective way. Two traces never merge.
%%We label each trace in the bud $u$ with the name $x_i$ of the corresponding trace, if any.
%%All those corresponding to no trace are labeled at random using the remaining variable names.
%%
%%At least one trace progresses, otherwise by repeating infinitely many times this step we would get a
%%path with no progressing trace. The same is true for any combination of one or more movements
%%from $u$ to $u$. 
%%
%%%After $m$ movements to any $u$ inside $u$, 
%%%each of the $m$ traces either disappeared or cycles. After $m!$ steps, all
%%%cycles are back to their original point. 
%%%
%%%All traces are now restarted or move from one $\N$ to the same $\N$, with or without progression.
%%
%%\item
%%At least one trace $x_i$ progresses and it is not erased by any other trace. Otherwise we could follow a path
%%in which each progress is erased in some new step, and so there is no infinite progressing trace.
%%We use this trace as the main variable $x_i$ of the recursion. In all steps, either $x_i$ is constant or decreases,
%%and in at least one case it decreases. In all cases in which $x_i$ decrease we use primitive
%%recursion on $x_i$ in $\systemT$, as main variable. 
%%In all other case, $x_i$ is not removed, therefore it stays the same. 
%%We isolate the main variable $x_j$ of the recursion for these steps, it is progressing therefore $j \not = i$.
%%We use primitive recursion on $x_j$: this is the second variable of primitive recursion. 
%%We continue in this way and we define a primitive recursion in $\systemT$, with pairwise distinct 
%%indexes $x_{i_1} = x_i$, $x_{i_2} = x_j$, \ldots, $x_{i_k}$ for some $k \ge 1$. We extend 
%%$x_{i_1}, \ldots, x_{i_k}$ to $x_{i_1}, \ldots, x_{i_n}$ in a random way: we defined in this way a
%%permutation $\sigma$ on $\{1, \ldots, m\}$ by $\sigma(j) = i_j$ for $j \in \{1, \ldots, m\}$
%%We define in this way a closed primitive recursive term 
%%$\lambda \vec{x}.t^\sigma \in \systemT$. Each bud $u$
%%defining a permutation $\tau$ is replaced by $\exch_{\tau}(f)$.
%%The term $\exch_\tau \in \systemT$ applies the permutation $\tau$ to the arguments of $f$,
%%and during the recursive call $f$ is replaced by $\lambda \vec{x}.u^\sigma$.
%%\end{enumerate}
%%
%%We claim that the infinite term $t^\sigma \in \systemT$ 
%%is equivalent to the cyclic recursive term $t \in \CTlambda$ we started from.
 %WORKING DRAFT



%
%\section{appendix}
%
%\begin{verbatim}
%
%To: kmr@is.sci.toho-u.ac.jp (Daisuke Kimura)
%Re: proof of Weak Normalization to an integer for CT-lambda
%Fri, 22 Mar 2024 08:25:57 +0100 
%
%    By the way, I re-checked the weak curry-howard proof, now i think that the proof 
%does not require the property p-->q, a-->b ==> p[a/x]-->q[b/x] and can be completed 
%with the notion of safe reduction.
%but in fact it would be more interesting to prove full church-rosser for Circular T-lambda, 
%as anupam does for his circular T.
%
%    About strong normalization, we can prove it for "safe" reductions, those inside no cond. 
%More in general, we know that we can have infinite reduction sequences, because we can 
%have infinitely many redexes. However, for any infinite reduction sequence sigma, I conjecture 
%we can prove a kind of stabilization of the term. After some reduction step, the term only 
%changes inside some cond nested k times. 
%
%    Namely, I conjecture that
%
%"for any cyclic lambda term t, any infinite reduction sequence (sigma(n)|n in N) with sigma(0)=t, 
%any k in N, there is a n0 in N such that for all n>=n0, the terms sigma(n) and sigma(n0)  
%coincide on all branches with at most k times cond."
%
%    Best, Stefano
%
%\end{verbatim}
%
%\section{The $\Succ$-length of a term}
%We define the safe trunk of a term as the part of the term which we can normalize with safe reductions only,
%and  the $\Succ$-length of a term $t \in \LAMBDA$ the number of $\Succ$ in front of the safe trunk.
%Here is the formal definition.
%
%\begin{definition}[$\Succ$-length]
%Assume $t \in \LAMBDA$
%\begin{enumerate}
%\item
%The safe trunk of $t$ is any expression $u[\cond(f_1,\cdot), \ldots, \cond(f_n,\cdot)]$
%such that  for some $g_1, \ldots, g_n$ we have $v = u[\cond(f_1,g_1), \ldots, \cond(f_n,g_n)]$
%\emph{safe normal} and $t \reduces v$.
%\item
%The $\Succ$-length of $t$ is
%the number of $\Succ$ in front of any safe-normal form of $t$, if any exists. 
%\end{enumerate}
%\end{definition}
%
%
%\begin{Eg}
%The $\Succ$-length is a kind a value we can assign to \emph{all} terms of $\LAMBDA$.  
%We compute the $\Succ$-length for some normal terms of $\LAMBDA$.
%$0$ has $\Succ$-length $0$, $\Succ(x)$ has $\Succ$-length $1$, 
%$\Succ(\Succ(\cond(f,g)(x))$ has $\Succ$-length $2$, while $t = \Succ(t)$ has infinite $\Succ$-length.
%\end{Eg}
%
%The $\Succ$-length exists for total terms.
%From the Church-Rosser property for $\WTyped$
%we deduce that the $\Succ$-length is unique when it exists.
%
%\begin{lemma}[$\Succ$-length of terms  finite for safe reductions]
%\label{lemma-succ-length}
%Assume $t \in \LAMBDA$.
%
%\begin{enumerate}
%\item 
%\label{lemma-succ-length-01}
%The safe-trunk and $\Succ$-length exist and they are unique. 
%
%\item
%\label{lemma-succ-length-02}
%If $t \reduces \Succ(u)$, then the $\Succ$-length of $u$ is $1$ unit 
%smaller than the $\Succ$-length of $t$.
%
%\end{enumerate}
%\end{lemma}
%
%
%
%\begin{proof}
%Assume $t \in \LAMBDA$.
%
%\begin{enumerate}
%\item
%%\label{lemma-succ-length-01}
%Assume that $u[\cond(f_1,\cdot), \ldots, \cond(f_n,\cdot)]$ and
%$u'[\cond(f'_1,\cdot), \ldots, \cond(f'_{n'},\cdot)]$ are safe-trunks for $t$, in order to prove
%that $u=u'$ and $n=n'$. 
%
%Then for some $g_1, \ldots,g_n$ and some $g'_1, \ldots,g'_n$ we have that 
%$v = u[\cond(f_1,g_1), \ldots, \cond(f_n,g_n)]$ and 
%$v' = u'[\cond(f'_1,g'_1), \ldots, \cond(f'_n,g'_{n'})]$ 
%are safe-normal forms of $t$ and all $\cond$-expressions shown are maximal. 
%The decomposition of each safe-normal form $v$ is therefore unique:
%if $v = u"[\cond(f"_1,g"_1), \ldots, \cond(f"_n,g"_{n"})]$ then $u=u"$ and $n=n"$.
%Each reduction from any of the safe-trunks takes place in some $g_1, \ldots,g'_{n'}$. 
%
%By Church-Rosser  (\S \ref{section-church-rosser}) we deduce that $v$ and $v'$ are confluent, therefore
%for some $v"$ we have $v, v' \reduces v"$. Since the reductions on $v, v'$ take place in some $g_1, \ldots,g'_{n'}$, we deduce that $v" = u[\cond(f_1,g"_1), \ldots, \cond(f_n,g"_n)]$
%and $v" = u[\cond(f_1,g'"_1), \ldots, \cond(f_n,g'"_{n'})]$ for some $g"_1, \ldots,g'"_{n'}$.
%From the unicity of the decomposition of $v"$
%with maximal $\cond$-subterms we conclude that $u=u'$ and $n=n'$. 
%\\
%
%From unicity of the safe-trunk we deduce that the $\Succ$-length is unique.
%
%\item
%%\label{lemma-succ-length-02}
%\emph{Assume that $t \reduces \Succ(u)$, in order to prove that 
%the $\Succ$-length of $u$ is $1$ unit smaller than the $\Succ$-length of $t$}.
%By the point \ref{lemma-succ-length-01} above, 
%$u$ has $\Succ$-length some $k \in \N$. Then $u \reduces \Succ^k(v)$
%for some $v$ not successor, 
%therefore $t \reduces \Succ^{k+1}(v)$ and the $\Succ$-lenght of $t$ is $\ge k+1$.
%
%\end{enumerate}
%\end{proof}


\end{document}
