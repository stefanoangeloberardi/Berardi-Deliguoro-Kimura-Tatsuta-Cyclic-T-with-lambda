
% lmcs, 2018.5.2, 2019.9.9, 2020.2.10

%\newtheorem{definition}{Definition}[section]
\newtheorem{fact}[definition]{Fact}
\newtheorem{theorem}[definition]{Theorem}
\newtheorem{corollary}[definition]{Corollary}
\newtheorem{lemma}[definition]{Lemma}
\newtheorem{proposition}[definition]{Proposition}
\newtheorem{remark}[definition]{Remark}
%\newtheorem{claim}[definition]{Claim}
%

\newenvironment{newenumerate}{%
  \begin{enumerate}%
    \renewcommand{\labelenumi}{(\roman{enumi})}%
    }{\end{enumerate}}

\newcommand{\qedblack}{\hfill\blacksquare}
\newcommand{\qed}{\hfill \ensuremath{\Box}}

\newenvironment{proof}{\bigskip \noindent {\em Proof.} \rm }
                     % {\begin{flushright} $\qed$ \end{flushright}}

%comandi per l'esempio
\newcounter{esemconta}
\newcommand{\iniese}
                    {
                                \refstepcounter{esemconta}
                                \medskip
    \noindent {\bf Example \theesemconta}
                        }
                \newcommand{\finese}
                        {
            \smallskip
                        }

%comandi per l'osservazione
\newcounter{remconta}
\newcommand{\inirem}
                        {
                                \refstepcounter{remconta}
                                \bigskip
    \noindent {\bf Remark \theremconta}
                        }
                \newcommand{\finrem}
                        {
            \smallskip
                        }

% nieuwe enumerate: parts
\def\parts{
\ifnum \@enumdepth >3 \@toodeep
\else
   \advance\@enumdepth \@ne
   \edef\@enumctr{enum\romannumeral\the\@enumdepth}
   \list {\csname label\@enumctr\endcsname} {\usecounter{\@enumctr}
    \itemsep 1pt \topsep 2pt \itemindent 0.5em
   \ifnum \@enumdepth < 2 \topsep 4pt plus 1pt \itemsep 5pt plus 1pt \fi
   \parsep 0pt plus 1pt \def\makelabel##1{\hss\llap{##1}}}
\fi}
\let\endparts =\endlist
% einde parts


\newif\ifdraft \draftfalse
\drafttrue % put % for lipics


\long\def\Stefano#1{{{\color{red}{SB: #1}}}}
\long\def\Makoto#1{{{\color{blue}{MT: #1}}}}
\long\def\Daisuke#1{{{\color{green}{DK: #1}}}}

\ifdraft

\documentclass{article}
\usepackage{mystyle}
\A4page
%\usepackage{tikzpicture}
\usepackage{graphicx}

\else

\documentclass{lmcs}
\usepackage{hyperref}
\usepackage{graphicx}

\fi

\usepackage{amssymb,amsmath,graphicx,color}
\usepackage{mymath,proof,latexsym}
\usepackage{xcolor}
\usepackage[pdftex,outline]{contour}


%%%%%%%%%%%%%%%%%%%%%%%%%%%%%%%%%%%%
% WE ADD THE COMMANDS theorem lemma proposition corollary definition % 
%%%%%%%%%%%%%%%%%%%%%%%%%%%%%%%%%%%%

\newtheorem{theorem}{Theorem}[section]
\newtheorem{lemma}[theorem]{Lemma}
\newtheorem{proposition}[theorem]{Proposition}
\newtheorem{corollary}[theorem]{Corollary}
\newtheorem{definition}[theorem]{Definition}

\newenvironment{proof}[1][Proof]{\begin{trivlist}
\item[\hskip \labelsep {\bfseries #1}]}{\end{trivlist}}
\newenvironment{example}[1][Example]{\begin{trivlist}
\item[\hskip \labelsep {\bfseries #1}]}{\end{trivlist}}
\newenvironment{remark}[1][Remark]{\begin{trivlist}
\item[\hskip \labelsep {\bfseries #1}]}{\end{trivlist}}



\begin{document}
\sloppy 
\hbadness=10000
\vbadness=10000

% cut note 2

\def\Expand{{\rm expand}}

% cut elim

\def\RCLKIDomega{{{\bf RCLKID}^\omega}}
\def\Ren{{\rm Ren}}
\def\Terms{{\rm Terms}}
\def\IE{{\rm IE}}
\def\Cut{{\rm Cut}}
\def\Rcut{{\rm Rcut}}

% cmcs

\def\Co{{\rm co}}
\def\Axiom{{\rm Axiom}}
\def\Wk{{\rm Wk}}
\def\Cut{{\rm Cut}}
\def\BS{{\rm BS}}
\def\Bit{{\rm Bit}}
\def\LK{{\rm LK}}
\def\Head{{\rm head}}
\def\Tail{{\rm tail}}

% intuitionistic

\def\Univ{{\rm Univ}}
\def\Left{{\rm Left}}
\def\Right{{\rm Right}}
\def\Set{{\rm set}}
\def\ES{{\rm ES}}
\def\PC{{\rm PC}}
\def\Fin{{\rm Fin}}
\def\CLJIDomega{{{\bf CLJID}^\omega}}
\def\CLJIDHA{{{\bf CLJID}^\omega+{\bf HA}}}
\def\LJIDHA{{{\bf LJID}+{\bf HA}}}
\def\LJID{{\bf LJID}}
\def\OP{{\rm OP}}
\def\Div{{\rm Div}}
\def\RP{{\rm RP}}
\def\MS{{\rm MS}}
\def\ET{{\rm ET}}
\def\LiftedTree{{\rm LiftedTree}}
\def\First{{\rm first}}
\def\Er{{\rm Er}}
\def\ER{{\rm ER}}
\def\Insert{{\rm insert}}
\def\ET{{\rm ET}}
\def\SInd{{\rm SInd}}
\def\Trans{{\rm Trans}}
\def\HA{{\bf HA}}
\def\DM{{\rm DM}}
\def\Ext{{\rm ext}}
\def\KB{{\rm KB}}
\def\DT{{\rm DT}}
\def\DS{{\rm DS}}
\def\Monoseq{{\rm Monoseq}}
\def\Eq{{\rm seq}}
\def\Last{{\rm last}}

% equivalence cyclic proofs

%\def\Uor{\displaystyle \mathop {\widetilde\bigvee}}
\def\Uor{\biguplus}
\def\CLKIDomega{{{\bf CLKID}^\omega}}
\def\CLKIDPA{{{\bf CLKID}^\omega+{\bf PA}}}
\def\LKIDPA{{{\bf LKID}+{\bf PA}}}
\def\PA{{\bf PA}}

% simple equiv note

\def\LKIDN{{\bf LKIDN}}
\def\CLKIDN{{\bf CLKIDN}}

% equiv note

\def\Choose{{\rm Choose}}
\def\Asym{{\rm Asym}}
\def\Peano{{\bf Peano}}
\def\Code#1{{\lceil{#1}\rceil}}
\def\Ineq{{\rm Ineq}}
\def\MonoPath{{\rm MonoPath}}
\def\BoundedPath{{\rm BoundedPath}}
\def\InfPath{{\rm InfPath}}
\def\InfMonoPath{{\rm InfMonoPath}}
\def\HomSeq{{\rm HomSeq}}
\def\InfHomSeq{{\rm InfHomSeq}}

\def\ErdosTree{{\rm ErdosTree}}
\def\KB{{\rm KB}}
\def\MonoList{{\rm MonoList}}
%\def\Seq{{\rm Seq}}
\def\Univ{{\rm univ}}
\def\Infinite{{\rm Infinite}}
\def\S{{\bf S}}
%\def \N {{N}} %\def\N{{\bf N}} %
\def\Ind{{\rm Ind}}
\def\LKID{{\bf LKID}}
\def\PA{{\bf PA}}
\def\LKIDExt{{\bf LKIDExt}}
\def\CLKID{{\bf CLKID^\omega}}

% LICS

\def\Over#1#2{\deduce{#2}{#1}}
\def\List{{\rm List}}
\def\ListX{{\rm ListX}}
\def\ListO{{\rm ListO}}
\def\ListE{{\rm ListE}}
\def\Nl{{\rm nl}}

% CSLID

\def\SLLID{{\rm SLID}}
\def\SLID{{\rm SLID}}
\def\CSLID{{\rm CSLID}}
\def\CSLIDomega{{{\rm CSLID}\omega}}
\def\Eclass{{\rm Eclass}}
\def\Eq{{\rm Eq}}
\def\Deq{{\rm Deq}}
\def\Satom{{\rm Satom}}
\def\Cells{{\rm Cells}}
\def\Roots{{\rm Roots}}
\def\Elim{{\rm Elim}}
% \def\Case#1{{\rm Case-$#1$}}
\def\Case{{\rm Case}}
\def\Unfold{{\rm Unfold}}
\def\Pred{{\rm Pred}}
\def\Start{{\rm Start}}
\def\Unsat{{\rm Unsat}}
\def\Split{{\rm Split}}
\def\Underscore{\underline{\phantom{x}}}
\def\Rootcell{{\rm rootcell}}
\def\Rootshape{{\rm rootshape}}
\def\Jointleaf{{\rm jointleaf}}
\def\Jointleafimplicit{{\rm jointleafimplicit}}
\def\Jointnode{{\rm jointnode}}
\def\Directjoint{{\rm directjoint}}

% cyclicnote2

\def\Range{{\rm Range}}
\def\Xstart{X_{{\rm start}}}
\def\kstart{k_{{\rm start}}}
\def\Leaves{{\rm Leaves}}
\def\PureDist{{\rm PureDist}}
\def\Pure{{\rm Pure}}
\def\Identity{{\rm Identity}}
\def\Subst{{\rm Subst}}

% weak establish

\def\Connected{{\rm Connected}}
\def\Eststablished{{\rm Eststablished}}
\def\Valued{{\rm Valued}}

% monadic new

\def\SLRDbtw{{\rm SLRD}_{btw}}
\def\BTW{{{\rm BTW}}}
\def\Loc{{{\rm Loc}}}
\def\Ne{{\ne}}
\def\Nil{{{\rm nil}}}
\def\eqDef{=_{{\rm def}}}
\def\Inf#1{{\infty_{#1}}}
\def\Stores{{\rm Stores}}
\def\SVars{{{\rm SVars}}}
\def\Val{{{\rm Val}}}
\def\MSO{{{\rm MSO}}}
\def\Sep{{{\rm Sep}}}
\def\Septwo{{{\rm SLMI}}}
\def\SLMI{{{\rm SLMI}}}
\def\Sepinf{{\rm Sep}\infty}
\def\THeaps{{\rm THeaps}}
\def\Root{{\rm Root}}
\def\TG{{\rm TGraph}}

% monadic

\def\Nil{{{\rm nil}}}
\def\eqDef{=_{{\rm def}}}
\def\Inf#1{{\infty_{#1}}}
\def\Stores{{\rm Stores}}
\def\SVars{{{\rm SVars}}}
\def\Val{{{\rm Val}}}
\def\MSO{{{\rm MSO}}}
\def\Sep{{{\rm sep}}}
\def\THeaps{{\rm THeaps}}
\def\Root{{\rm Root}}
\def\TG{{\rm TGraph}}

\def\Tilde{\widetilde}
\def\Bar{\overline}
\def\Lequiv{\Longleftrightarrow}
\def\Lto{\Longrightarrow}
\def\Lfrom{\Longleftarrow}
\def\Norm{{\rm Norm}}
\def\Noshare{{\rm Noshare}}
\def\Roots{{\rm Roots}}
\def\Forest{{\rm Forest}}
\def\Var{{\rm Var}}
\def\Range{{\rm Range}}
\def\Cell{{\rm Cell}}
%\def\Tree{{\rm Tree}}
\def\Switch{{\rm Switch}}
\def\All{{\rm All}}
\def\To{\leadsto}
\def\tree{{\rm tree}}
\def\Paths{{\rm Paths}}
\def\Finite{{\rm Finite}}
\def\Const{{\rm Const}}
\def\Leaf{{\rm Leaf}}
\def\LeastElem{{\rm LeastElem}}
\def\LeastIndex{{\rm LeastIndex}}
\def\WSnS{{{\rm WSnS}}}
\def\Expand{{{\rm Expand}}}

% previous

\long\def\J#1{} % skip
\def\T#1{\hbox{\color{green}{$\clubsuit #1$}}}
\def\W#1{\hbox{\color{Orange}{$\spadesuit #1$}}}

\def\Node{{{\rm Node}}}
\def\LL{{{\rm LL}}}
\def\DSN{{{\rm DSN}}}
\def\DCL{{{\rm DCL}}}
\def\LS{{{\rm LS}}}
\def\Ls{{{\rm ls}}}

\def\FPV{{{\rm FPV}}}
\def\Lfp{{{\rm lfp}}}
\def\IsHeap{{{\rm IsHeap}}}

\def\Equiv{\quad \equiv\quad }
\def\Null{{{\rm null}}}
\def\Emp{{{\rm emp}}}
\def\If{{{\rm if\ }}}
\def\Then{{{\rm \ then\ }}}
\def\Else{{{\rm \ else\ }}}
\def\While{{{\rm while\ }}}
\def\Do{{{\rm \ do\ }}}
\def\Cons{{{\rm cons}}}
\def\Dispose{{{\rm dispose}}}
\def\Vars{{{\rm Vars}}}
\def\Locs{{{\rm Locs}}}
\def\States{{{\rm States}}}
\def\Heaps{{{\rm Heaps}}}
\def\FV{{{\rm FV}}}
\def\True{{{\rm true}}}
\def\False{{{\rm false}}}
\def\Dom{{{\rm Dom}}}
\def\Abort{{{\rm abort}}}
\def\New{{{\rm New}}}
\def\W{{{\rm W}}}
\def\Pair{{{\rm Pair}}}
\def\Lh{{{\rm Lh}}}
\def\lh{{{\rm lh}}}
\def\Elem{{{\rm Elem}}}
\def\EEval{{{\rm EEval}}}
% \def\BEval{{{\rm BEval}}} % not used 
\def\PEval{{{\rm PEval}}}
\def\HEval{{{\rm HEval}}}
\def\EVal{{{\rm Eval}}}
\def\Domain{{{\rm Domain}}}
\def\Exec{{{\rm Exec}}}
\def\Store{{{\rm Store}}}
\def\Heap{{{\rm Heap}}}
\def\Storecode{{{\rm Storecode}}}
\def\Heapcode{{{\rm Heapcode}}}
\def\Lesslh{{{\rm Lesslh}}}
\def\Addseq{{{\rm Addseq}}}
\def\Separate{{{\rm Separate}}}
\def\Result{{{\rm Result}}}
\def\Lookup{{{\rm Lookup}}}
\def\ChangeStore{{{\rm ChangeStore}}}
\def\ChangeHeap{{{\rm ChangeHeap}}}
\def\Wand{\mathbin{\hbox{\hbox{---}$*$}}}
%\def\Eval#1{[\kern -1pt[{#1}]\kern -1pt]}
\def\Eval#1{\llbracket{#1}\rrbracket}
\def\Vec{\overrightarrow}

\def\Tilde{\widetilde}
\def\Break{\hfil\break\hbox{}}

%Macro for the work on Cyclic System T with lambda
%22/03/2024
\newcommand{\ap}                { {\tt ap} }
\newcommand{\cond}            { {\tt cond} }
\newcommand{\reduces}       { \mathbin{\mapsto} }
\newcommand{\CTlambda}    { { {\tt CT} \mbox{-} \lambda} }
\newcommand{\CT}                { {\tt CT} }
%\newcommand{\nameIn}       { {\tt name\mbox{-}in} }
\newcommand{\etaRule}       { {\eta\mbox{-rule}} }
\newcommand{\ins}               { {\tt ins} }
\newcommand{\struct}          { {\tt struct} }
\newcommand{\Iter}              { {\tt It} }
\newcommand{\iter}               { {\tt it} }
\newcommand{\Interval}       { {\tt In} }
\newcommand{\interval}        { {\tt in} }
\newcommand{\Succ}             { {\tt S} }
\newcommand{\nil}                { {\tt nil} }
\newcommand{\cons}             { {\tt cons} }
\newcommand{\outline}  [1]   { {\contour{black}{\textcolor{white}{#1}}}  }
\newcommand{\systemT}       { {\outline{\tt T}} }
\newcommand{\LAMBDA}       { {\outline{$\Lambda$}} }
\newcommand{\Rec}               { {\tt Rec} }
\newcommand{\rec}                { {\tt rec} }
\newcommand{\exch}             { {\tt exch} }
\newcommand{\composed}     { {\mbox{\tiny $\circ$}} }
\newcommand{\WTyped}        { {\tt WTyped} }
\newcommand{\safe}              { {\hbox{\scriptsize \tt safe}} }
\newcommand{\GlobalTraceCondition}  { {\tt GlTrCon} }
\newcommand{\Sum}              { {\tt Sum} }
\newcommand{\safeReduces} {\mathbin{\reduces_\safe}}
\newcommand{\safeReducesAst} {\mathbin{\reduces_\safe^*}}
\newcommand{\var}                { {\tt var} }
\newcommand{\Graph}            { {\tt Graph} }
\newcommand{\bfColor} [2]     {{\bf \color{#1}{#2}}}
\newcommand{\weak}             {{\tt weak}}

%colors from the web site: https://latexcolor.com/
\definecolor{amber}{rgb}{1.0, 0.75, 0.0}
\definecolor{oldgold}{rgb}{0.81, 0.71, 0.23}

\newcommand{\SubTerm}               {{\tt SubT}}
\newcommand{\Tree}                      {{\tt Tree}}
\newcommand{\N}                           {{N}}
\newcommand{\Num}                      {{\tt Num}}
\newcommand{\Nat}                        {{\outline{\tt N}}}
\newcommand{\GTC}                       {{\tt GTC}}
\newcommand{\Reg}                       {{\tt Reg}}
\newcommand{\Type}                      {{\tt Type}}
\newcommand{\conc}                      {{\star}}
\newcommand{\typingRule}            {{\tt r}}
%\newcommand{\apvar}                     {{\ap_{\mbox{\tiny var}}}}
%\newcommand{\apnotvar}               {{\ap_{\neg\mbox{\tiny var}}}}
\newcommand{\apvar}                    {{{\tt ap}_{\mbox{\tiny $v$}}}}
\newcommand{\apnotvar}               {{{\tt ap}_{\mbox{\tiny $\neg v$}}}}
\newcommand{\subseteqsim}         {\mathbin{\raisebox{2pt}{$\subset$}_{\!\!\!\!\!\sim}}}
\newcommand{\supseteqsim}         {\mathbin{\raisebox{2pt}{$\supset$}_{\!\!\!\!\!\sim}}}
\newcommand{\Ctxt}                       {{\tt Ctxt}}
\newcommand{\Seq}                        {{\tt Seq}}
\newcommand{\Stat}                       {{\tt Stat}}
\newcommand{\Rule}                       {{\tt Rule}}
\newcommand{\restr}                      {{\lceil}}
\newcommand{\setT}                       {{\mathcal T}}
\newcommand{\base}                      {{\tt base}}
\newcommand{\inductive}               {{\tt ind}}
\newcommand{\universe}     [1]       {{|#1|}}
\newcommand{\Label}                     {{\tt Lbl}}
\newcommand{\quotationMarks} [1]     {{\textquotedblleft#1\textquotedblright}}
\newcommand{\lexicographic} [1]   {{ <^{#1}_{\mbox{\tiny lex}}} }
\newcommand{\trunk}                     {{\tt trunk}}


\newcommand\Restrict[2]{#1_{#2}}
\newcommand\simIndex[1]{\sim^{#1}_{{\tt idx}}}
\newcommand\mergeCtx[2]{#1\sharp#2}
\newcommand\Ifz{{\tt ifz}}
\newcommand\TtoCT[1]{(#1)^\circ}



%\newcommand{\bfColor}[2]{{\bf \color{#1}{#2}}}
%\definecolor{oldgold}{rgb}{0.81, 0.71, 0.23}
\definecolor{purple}{rgb}{0.8, 0.2, 0.8}

\newcommand{\Rapv}{\hbox{$(\text{ap}_{\text{v}})$}}
\newcommand{\RapNv}{\hbox{$(\text{ap}_{\lnot\text{v}})$}}
\newcommand{\Reta}{\hbox{$(\eta)$}}
\newcommand{\Rap}{\hbox{$(\text{ap})$}}
\newcommand{\Rcond}{\hbox{$(\text{cond})$}}
%\newcommand{\FV}[1]{\text{FV}(#1)}
\newcommand{\Ack}{{\tt Ack}}
\newcommand{\AckA}{{\tt ack1}}
\newcommand{\AckB}{{\tt ack2}}
\newcommand{\ACK}{{\redM Ack}}
\newcommand{\Cond}[2]{\text{cond}(#1,#2)}
\newcommand{\Suc}[1]{{\tt S}(#1)}
%\newcommand{\N}{N}
\newcommand{\goldN}{{\bfColor{oldgold}{N}}}
\newcommand{\redN}{{\bfColor{red}{N}}}
\newcommand{\bluen}{{\bfColor{blue}{n}}}
\newcommand{\blueN}{{\bfColor{blue}{N}}}
\newcommand{\redblueN}{{\bfColor{red}{N}\hspace{-1em}\bfColor{blue}{N}}}
\newcommand{\redM}{{\bfColor{red}{m}}}

\newenvironment{claim}[1][Claim]{\begin{trivlist}
\item[\hskip \labelsep {\bfseries #1}]}{\end{trivlist}}




\newcommand{\Pfin}                       {{\setP_{\mbox{\tiny fin}}}}
\newcommand{\eqMap}                  {{\tt eq}}
\newcommand{\BadSeq}                 {{\tt BadS}}


\newcommand{\labelled}                  { {\mbox{\tiny \tt lab}} }  

\newcommand{\CTlambdaLabelled} {{\CTlambda_\labelled}}



\ifdraft

\title{CT-$\lambda$, a cyclic simply typed $\lambda$-calculus with fixed points}

\author{Stefano Berardi,  Ugo de' Liguoro, Daisuke Kimura, Koji Nakazawa, Makoto Tatsuta}
\date{}

\else

\title[Equivalence]
{
???
}

\author[S. Berardi]{Stefano Berardi}
\address{Universit\`{a} di Torino,
Torino, Italy}
\email{stefano@unito.it}


%% mandatory lists of keywords 
\keywords{
proof theory
inductive definitions
Brotherston-Simpson conjecture
cyclic proofs
}
\fi

\maketitle

\begin{abstract}
A motivation for having a circular syntax for G\"{o}del System $\systemT$ is that terms 
in a circular syntax are much shorter than the equivalent terms written in $\systemT$, 
yet they can be checked mechanically.

Several circular syntax for $\systemT$ have been proposed.
In this paper explore the possibility of defining a circular syntax
directly for $\lambda$-abstraction and simple types, instead of interpreting
 $\lambda$-abstraction through combinators first, then inserting a  circular syntax afterwards.
A circular syntax using binders instead of combinators
could be more familiar for researchers working in the field of Type Theory.

We introduce our circular syntax as a fixed point operator defined by cases through a test on $0$.
We prove the expected results for this circular simply typed $\lambda$-calculus, which we call $\CTlambda$: 
\emph{(i)} every closed term of type $\N$ reduces to some numeral;
\emph{(ii)}  strong normalization for all reductions sequences outside all fixed points; 
\emph{(iii)}  strong normalization for ``fair'' reductions; \emph{(iv)} Church-Rosser property;
and \emph{(v)} the equivalence between circular syntax and ordinary syntax 
for G\"{o}del system $\systemT$. 


\end{abstract}

\iffalse
key words: 
proof theory,
inductive definitions,
Brotherston-Simpson conjecture,
cyclic proofs,
Martin-Lof's system of inductive definitions,
infinite Ramsey theorem
Podelski-Rybalchenko termination theorem
size-change termination theorem
\fi


% Section 1
% !TEX root = main.tex

\section{Introduction} 
We will introduce $\LAMBDA$, a set of infinite $\lambda$-terms with a circular syntax.
The types of $\LAMBDA$ are: the atomic type $\N$, 
possibly type variables $\alpha, \beta, \ldots$, and all types $A \rightarrow B$ for any types $A$, $B$. 
Type variables are only used to provide examples of terms and play a minor role in our paper.

The terms of $\LAMBDA$  are all possibly infinite trees representing expressions defined with 
$0$,$\Succ $,$\ap$ (application), 
variables $x^T$ (with a type superscript $T$),  $\lambda$ (the binder for defining maps), 
and $\cond$, the arithmetic conditional (i.e., the test on zero). 
If we have no term variables and no type variables, 
then the trees in $\LAMBDA$ represent partial functionals on $\N$, 
provided we add reduction rules transforming closed terms of type $\N$ in notations for natural numbers.

%19:42 19/04/2024

In this paper will consider two sets of terms: 
\begin{enumerate}
\item
the set $\CTlambda$ of well-typed terms in
a circular syntax, which are equivalent to the set of terms in G\"{o}del system $\systemT$.  
\item
The set of terms $\GTC$, satisfying a condition called global trace condition, which are possibly non-recursive
terms used to provide semantics for  $\CTlambda$ (and  $\systemT$, of course)
\end{enumerate}
We introduce more sets of terms only used 
as intermediate steps in the definition of the semantic $\GTC$ and the syntax $\CTlambda$.

\begin{enumerate}
\item
 $\WTyped \subseteq \LAMBDA$, the set of well-typed terms, is the
of set terms having a unique type
\item
$\Reg$ is the set of terms of $\LAMBDA$ which are regular trees (i.e., having finitely
many subtrees). They are possibly infinite terms which are finitely presented 
by a finite graph possibly having cycles.
\item 
$\GTC \subseteq \WTyped$ will be defined as the set of well-typed circular 
$\lambda$-terms satisfying the global trace condition (and possibly non-recursive, as we said). %and regular. 
Terms of $\GTC$ denote total functionals. 
\end{enumerate}

We will prove that $\CTlambda$ is a decidable subset of $\Reg$.
$\CTlambda$ is a new variant of the existing circular version of G\"{o}del system $\systemT$. 
Differently from all previous circular versions of $\systemT$, our system $\CTlambda$
uses binders instead of combinators. 
As we anticipated, 
 circular syntax has the advange of writing much shorter terms while preserving decidability of termination.
Besides, by introducing a circular syntax with binders, we hope to provide 
a circular syntax more familiar to researchers working in the field of Type Theory.
\\

We will prove the expected results for the circular syntax $\CTlambda$:
strong normalization for reductions not in the right-hand side of any $\cond$ and Church-Rosser. 
We will prove normalization in the limit if we use reductions which are ``fair'':
fair reductions can reduce \emph{inside} the right-hand side of some $\cond$, but they never forget entirely 
the task of reducing \emph{outside} all such subterms.

Eventually, we will prove that the closed terms $\CTlambda$ (those without free variables)
represent exactly the total computable functionals definable in G\"{o}del system $\systemT$.


%14:57 17/04/2024
%21:08 19/04/2024



\newpage
% Section 2


\section{The set of infinite $\lambda$-terms}
We define the set $\LAMBDA$ of infinite circular terms, the subset $\WTyped$
of well-typed terms and a reduction relation for them. Infinite terms are labeled binary trees, therefore we
have to define binary trees first, and before them the lists on $\{1,2\}$ and more related notions.

\subsection{Lists and Binary Trees}

\begin{definition}[Lists on $\{1,2\}$]
\begin{enumerate}
\item
We denote with $\List(\{1,2\})$ the set of all lists of elements of $\{1,2\}$: $(),(1),(1),(1,1),(1,2),\ldots$. 
We call the elements of $\List(\{1,2\})$ just lists for short.
\item
If $i_1, \ldots, i_n \in \{1,2\}$, we write $(i_1, \ldots, i_n)$
for the list with elements $i_1, \ldots, i_n$. 
\item
We write $\nil$ for the empty list, or $()$. 
\item
If $l=(i_1, \ldots, i_n)$, $m=(j_1, \ldots, j_m)$ and $l, m \in \List(\{1,2\})$ we define
$$
l \conc m = (i_1, \ldots, i_n,\ j_1, \ldots, j_m)\in \List(\{1,2\})
$$
%If $I \subseteq \List(\{1,2\})$
%we write $l \conc I = \{l \conc m | m \in I\}$. 
\item
The prefix order $\le$ on lists
is defined by $l \le m$ if and only if $m = l \conc m'$ for some $m' \in  \List(\{1,2\})$.
\end{enumerate}
\end{definition}

Binary trees are represented by set of lists.

\begin{definition}[Binary trees]
\begin{enumerate}
\item 
A binary tree, just a tree for short, is any set $T\subseteq \List(\{1,2\})$ 
of lists including the empty list and closed by prefix: $\nil \in T$ and for all $l, m \in \List(\{1,2\})$
if $m \in T$ and $l \le m$ then $l \in T$. 
\item
$\nil$ is called the root of $T$, any $l \in T$ is called a node of
$T$ and if $l \in T$, then any $l \conc (i) \in T$ is called a child of $l$ in $T$.
\item
A node $l$ of $T$ is a leaf of $T$ if $l$ is an maximal element of $T$ w.r.t. the prefix order (i.e.,
if $l$ has no children). 
\item
If $T$ is a tree and $l \in T$ we define $T \restr l = \{m \in \List(\{1,2\}) | l \conc m \in T\}$.
$T \restr l$ is a tree and we say that $T \restr l$ is a subtree of $T$ in the node $l$,
and for each $m \in T \restr l$ we say that $l \conc m$ is the corresponding node in $T$.
\item
If $l=(i)$ we say that $T_l$ is an immediate sub-tree of $T$.
\item
A binary tree labeled on a set $L$ is any pair $\setT = (T, \phi)$ with 
$\phi:T \rightarrow L$. We call $\universe{\setT}=T$ the set of nodes of $\setT$.
For all $l \in T$ we call $\Label(\setT,l) = \phi(l)$ the label of $l$ in $\setT$.
\item
We define $\setT \restr l = (T \restr l, \phi_l)$ and $\phi_l(m) = l \conc m$. 
We call any $\setT \restr l$ a labeled subtree of $\setT$
in the node $l$, an immediate sub-tree if $l=(i)$. 
\end{enumerate}
\end{definition}

The labeling of a node $m \in \universe{( \setT \restr l )}$ 
is the same label of the corresponding node $l \conc m$ of $\universe{\setT}$.
 
Let $n = 0, 1, 2$. Assume $\setT_1, \ldots,\setT_n$ are trees labeled on $L$ and $l \in L$.
Then we define $$\setT = l(\setT_1, \ldots, \setT_n)$$ as the unique tree labeled on $L$
with the root labeled $l$ and with:
$\setT \restr (i) = \setT_i$ for all $1 \le i \le n$.


\subsection{Types, Terms and Contexts}
Now we define the types of $\LAMBDA$, then the terms and the contexts of $\LAMBDA$.

\begin{definition}[Types of $\LAMBDA$]
\mbox{}
\begin{enumerate}

\item
The types of $\LAMBDA$ are: the type $\N$ of natural numbers, an infinite list 
$\alpha,\beta,\ldots$ of type variables, and with $A,B$ also  $A \rightarrow B$.
We call them simple types, just \emph{types} for short. 

\item 
$\Type$ is the set of simples types.

\item
We suppose be given a set $\var$, consisting of all pairs $x^T=(x,T)$ 
of a variable name $x$ and a type $T \in \Type$.
\end{enumerate}
\end{definition}

Terms of $\LAMBDA$ are labeled binary trees.


\begin{definition}[Terms of $\LAMBDA$]
The terms of $\LAMBDA$ 
are all binary trees $t$ with set of labels $L=\{x^T, \lambda x^T., \ap, 0, \Succ, \cond\}$
such that:
\begin{enumerate}
\item 
a node $l$ labeled $x^T$ or $0$ is a leaf of $t$ (no children). 
\item
a node $l$ labeled $\lambda x^T.$ or $\Succ$ has a unique child $l \conc (1)$. 
\item
a node labeled $\ap$, $\cond$ has two children $l \conc (1)$, $l \conc (2)$.
\end{enumerate}
We say that $t$ is a sub-term of $u$ if $t$ is a labeled sub-tree of $u$,
an immediate subterm if it is an immediate sub-tree.
\end{definition}
 
Assume  $l \in L = \{x^T, \lambda x^T., \ap, 0, \Succ, \cond\}$.
We  use the operation $l( \setT_1 \ldots \setT_n )$ on labeled trees to define new terms in $\LAMBDA$.
If $t, u \in \LAMBDA$ then $x^T, 0, \lambda x^T.t,\ap(t,u),\Succ(t),\cond(t,u) \in \LAMBDA$ 
are terms with the root labeled $l$, and with immediate subterms among $t$, $u$. 
Conversely, each $v \in \LAMBDA$ is in one of the forms:
$x^T, 0, \lambda x^T.t$, $\ap(t,u)$, $\Succ(t)$, $\cond(t,u)$ for some $t, u \in \LAMBDA$.

Now we define the regular terms $\LAMBDA$.

%11:55 22/04/2024

\begin{definition}[Regular terms of $\LAMBDA$]
\mbox{}
\begin{enumerate}

\item
We write $\SubTerm(t)$ for the (finite or infinite) set of subterms of $t$. 
Subterms are coded by the nodes of $t$, different nodes can code the same subterm. 

\item
$\Reg$ is the set of terms $t \in \LAMBDA$ such that $\SubTerm(t)$ is finite.
We call the terms of $\Reg$ the \emph{regular terms}.

%\item
%We write $\Tree(t)$ for the tree of all chains
%$(t_1, \ldots, t_n)$ with $t_1=t$ and weakly increasing by $\sqsubset_1$: 
% for all $(i+1) \le n$ we have $t_{i+1}=t_i$ or $t_{i+1} \sqsubset_1 t_i$.

\item
As usual, we abbreviate $\ap(t,u)$ with $t(u)$.

\item
When $t = \Succ ^n(0)$ for some natural number $n \in \Nat$
we say that $t$ is a numeral. We write $\Num$ for the set of numeral in $\LAMBDA$.

\item
A variable $x^T$ is free in $t$ if there is some $l \in t$ labeled $x^T$ in $T$, 
and \emph{no $m < l$ labeled $\lambda x^T.$} in $t$. 
$\FV(t)$ is the set of free variables of $t$.
\end{enumerate}
 
\end{definition}

%We use two different names for the operation $\ap(t,u)$: 
%we call it $\ap$ when $u$ is not a variable and $\apvar$ when $u$ is a variable. 

$\Num$ is the representation inside $\LAMBDA$ of the set $\Nat$ of natural numbers.
All numerals are finite trees of $\Lambda$. 
All finite well-typed typed $\lambda$-terms 
we can define with the rules above are finite terms of $\LAMBDA$.
Regular terms can be represented by the subterm relation restricted to $\SubTerm(t)$:
this relation defines a graph with possibly cicles. Even if $\SubTerm(t)$ is finite, $t$ can be an infinite tree. 

We do not assume implicit renaming of bound variables, namely $\alpha$-equivalence
because the typing system (given later) does not have renaming rule as a primitive rule. 

\begin{Eg}
\label{example-regular-infinite}
An example of regular term which is an infinite tree: the term $t = \cond(0,t) \in \LAMBDA$. 
The set $\SubTerm(t)=\{t,0\}$ of subterms  of $t$ is finite, therefore $t$ is a regular term.
However, $t$ is an infinite tree (it includes itself as a subtree). 
The immediate sub-term chains of $t$ are all $(t,t,t,\ldots,t)$ and $(t,t,t,\ldots,0)$.
There is a unique infinite sub-term chain, which is $(t,t,t,\ldots)$. 
\end{Eg}

In order to define the type of an infinite term, we first 
define contexts and sequences for any term of $\LAMBDA$.
A context is a list of type assignments to variables: our variables already have a type superscript,
so in fact a type assignment $x^T:T$ is \emph{redundant} for our variables (not for our terms).
Yet, we add an assignment relation $x^T:T$ for uniformity with the notation $x:T$ 
in use in Type Theory.

%13:32 22/04/2024
%11:17 29/04/2024

\begin{definition}[Contexts of $\LAMBDA$]
\mbox{}
\begin{enumerate}

\item
A  context of $\LAMBDA$ is any finite list $\Gamma = ({x_1}^{A_1}:A_1, \ldots, x_n^{A_n}:A_n)$ 
of \emph{pairwise distinct variables}, each assigned to its type superscript $A_1, \ldots, A_n \in \Type$. 

\item
We denote the empty context with $\nil$. We write $\Ctxt$ for the set of all contexts.

\item
A sequent is the pair of a context $\Gamma$ and a type $A$, which we write as $\Gamma \vdash A$.
We write $\Seq = \Ctxt \times \Type$ for the set of all sequents.

\item 
A typing judgement is the list of a context  $\Gamma$, a term $t$ and a type $A$, 
which we write as $\Gamma \vdash t:A$.
We write $\Stat = \Ctxt \times \LAMBDA \times \Type$ for the set of all typing judgements.

\item
We write $\FV(\Gamma) = \{ {x_1}^{A_1}, \ldots, {x_n}^{A_n} \}$.
We say that $\Gamma$ is a context for $t \in \LAMBDA$ and we write $\Gamma \vdash t$ 
if $\FV(t) \subseteq \FV(\Gamma)$.

\item
If $\Gamma = ({x_1}^{A_1}:A_1, \ldots, x_n^{A_n}:A_n)$ ,
$\Gamma' = (x'_1:A'_1, \ldots, x'_n:A'_{n'})$ are context of $\LAMBDA$, then we
write 
\begin{enumerate}
\item
$\Gamma \ \subseteqsim \ \Gamma'$ \ if for all $(x^A:A)$:  \ 
$(x^A:A) \in \Gamma  \ \Rightarrow  \  (x^A:A)\in\Gamma'$
\item
$\Gamma \sim \Gamma'$  \  if for all $(x^A:A)$:  \ 
$(x^A:A) \in \Gamma  \ \Leftrightarrow  \  (x^A:A)\in\Gamma'$
\end{enumerate}


\item
If $\Gamma$ is a context of $\LAMBDA$, then $\Gamma\setminus\{x^T:T\}$ is the context obtained
by removing $x_i^{A_i}:A_i$ from $\Gamma$ for all $x_i^{A_i}=x^T$. 
If $x \not \in \FV(\Gamma)$ then $\Gamma\setminus\{x^T:T\} = \Gamma$.

\end{enumerate}
\end{definition}

We have $\Gamma \subseteqsim \Gamma'$ if and only if
if there is a (unique) map $\phi:\{1,\ldots,n\} \rightarrow \{1,\ldots,n'\}$
such that $x_{i}=x'_{\phi(i)}$ and $A_{i}=A'_{\phi(i)}$ for all $i \in \{1,\ldots,n\}$.
We have  $\Gamma \sim \Gamma'$ if and only if  $\Gamma \subseteqsim \Gamma'$
and  $\Gamma \supseteqsim \Gamma'$ if and only if $\Gamma$, $\Gamma'$ are permutation
each other if and only if the map $\phi$ above is a bijection. We do not identify two contexts
which are one a permutation of the other, therefore we will need a rule to move from one to the other.


%12:38 17/04/2024
%15:37 17/04/2024

%From the context for a term we can define a context for each subterm of the term.


%21:26 19/04/2024

%\begin{definition}[Inherited Contexts of $\LAMBDA$]
%
%Given any context $\Gamma$, any $t \in \Lambda$ and any subterm chain 
%$\pi = (t_1, \ldots, t_n) \in \Tree(t)$, we define a unique inherited context for $t_n$ in $\pi$.
%The inherited context is obtained by repeatedly adding $x^T:T$ to the context whenever we
%cross a term $t_i = \lambda x^T.u_i$, while simultaneously removing $x^T:T$ from the previous
%context, if it was there.
%%08:00 20/04/024
%
%\begin{enumerate}
%
%\item
%$t$ has inherited context $\Gamma$.
%
%\item
%Any binder on $x$ subtracts the variable $x$ from the context of its \emph{last} argument:
%if $t = \lambda x^T.u, \cond(f,g)$ has inherited context $\Delta$, 
%then $u$ and $g$ have context $\Delta \setminus \{x^T:T\}, x^T$, 
%while $f$ has  inherited context $\Delta$.
%
%\item
%In any other case the context of a term and of the immediate subterm are the same:
%ff $t=\Succ(u), f(a)$ have inherited context $\Delta$,
% then $u,f,a$ have  inherited context $\Delta$.
%\end{enumerate}
%We abbreviate \emph{`` inherited context from $\Gamma$''} with \emph{context}
%when $\Gamma$ is fixed.
%\end{definition}



%The scope of the binder $\lambda x^T.t$ is $t$.
%The scope of the binder $\cond(f,g)$ is $g$ ($f$ is \emph{not} in the scope of $\cond(f,g)$).

We define the term $t[u/x]$ obtained by substituting the infinite term $u$ for the free variable $x$ in $t$ 
avoiding variable captures.
In order to define the capture avoiding substitution
under the assumption of not having implicit $\alpha$-equivalence, 
we implicitly fix a well-order on variables
and define the simultaneous substitution,
written $t[u_1/x_1,\ldots,u_n/x_n]$ or $t[\vec{u}/{\vec{x}}]$,
namely it satisfies the following:
\begin{itemize}
\item
  $x_i[u_1/x_1,\ldots,u_n/x_n] = u_i$ and $y[\vec{u}/\vec{x}] = y$, where $y\not\in\vec{x}$. 
\item
  $(f(a))[\vec{u}/\vec{x}] = f[\vec{u}/\vec{x}](a[\vec{u}/\vec{x}])$.
\item
  $(\lambda z^T.b)[\vec{u}/\vec{x}] =
  \left\{
  \begin{array}{ll}
    \lambda z^T.(b[\Restrict{(\vec{u}/\vec{x})}{\overline{z}}]),
    &
    \text{if $z \not\in \FV(\vec{u})$},
    \\
    \lambda z'^T.(b[\Restrict{(\vec{u}/\vec{x})}{\overline{z}},z'/z]),
    &
    \text{otherwise},
  \end{array}
  \right.$
  \\
  where
  $\Restrict{(\vec{u}/\vec{x})}{\overline{z}}$ is the one obtained by removing $u/z$ (for some $u$) from $\vec{u}/\vec{x}$,
  and 
  $z'$ is the least variable (with respect to the well-order)
  such that $z' \not\in \FV(b,\vec{u})$.
\item
  $0[\vec{u}/\vec{x}] = 0$.
\item
  $\Succ(t)[\vec{u}/\vec{x}] = \Succ(t[\vec{u}/\vec{x}])$.
\item
  $\cond(f,g)[\vec{u}/\vec{x}] = \cond(f[\vec{u}/\vec{x}],g[\vec{u}/\vec{x}])$.
\end{itemize}

Formally we define the substitution as follows.
\begin{definition}[Substitution for terms of $\LAMBDA$]
  A set $\theta=\{u_1/x_1,\ldots,u_n/x_n\}$ is called a {\em substitution}
  if $x_1,\ldots,x_n$ are pairwise distinct variables.
  It is called {\em renaming} if $u_1,\ldots,u_n$ are pairwise distinct variables. 
  We define $\theta(y)$ by $u_i$ if $y=x_i$, and by $y$ otherwise.
  We write $\theta,\theta'$ if $\theta\cup\theta'$ is a substitution.   
  The substitution $\Restrict{\theta}{\overline{x}}$ is defined 
  as the one obtained by removing $u/x$ (for some $u$) from $\theta$. 
  
  Let $t$ be a term of $\LAMBDA$, which is the labeled tree $(T_t,\phi_t)$,
  and $\theta$ be $\{u_1/x_1,\ldots,u_n/x_n\}$, where each $u_i$ is $(T_{u_i},\phi_{u_i})$. 
  Then the term $t[\theta]$ is $(T,\phi)$, where 
  $T = T_t \cup \{l\conc m \mid \text{$\phi_t(l) = x_i \in \FV(t)$ and $m\in T_{u_i} $}\}$.
  For each $l\conc m$ in the latter part, $\phi(l\conc m) = \phi_{u_i}(m)$.  
  For $l\in T_t$, we inductively define $\phi(l)$ and
  $\theta_l$ (such that $\theta_l = \theta',\theta_{ren}$, where $\theta'\subseteq \theta$ and $\theta_{ren}$ is a renaming) as follows. 
  \begin{itemize}
  \item
    $\theta_{\nil} = \theta$.
  \item
    If $\phi_t(l) = y^T \not\in\FV(t)$, then $\phi(l) = \theta_l(y^T)$.
  \item
    If $\phi_t(l) = 0$, then $\phi(l) = 0$. 
  \item    
    If $\phi_t(l) = \Succ$, then $\phi(l) = \Succ$ and $\theta_{l\conc (1)} = \theta_l$. 
  \item    
    If $\phi_t(l) \in \{\ap,\cond\}$, then $\phi(l) = \phi_t(l)$
    and $\theta_{l\conc (1)} = \theta_{l\conc(2)} = \theta_l$.
  \item
    If $\phi_t(l) = \lambda z^T.$ and $z\not\in\FV(\vec{u})$,
    then $\phi(l) = \lambda z^T.$ and $\theta_{l\conc(1)} = \Restrict{(\theta_l)}{\overline{z}}$. 
  \item
    If $\phi_t(l) = \lambda z^T.$ and $z\in\FV(\vec{u})$,
    then $\phi(l) = \lambda z'^T.$ and $\theta_{l\conc(1)} = \Restrict{(\theta_l)}{\overline{z}},z'/z$, where $z'$ is the least variable such that $z' \not\in \FV(t\restr l\conc(1),\vec{u})$.
  \end{itemize}
  We often abbreviate $t[\{u_1/x_1,\ldots,u_n/x_n\}]$ by $t[u_1/x_1,\ldots,u_n/x_n]$.
  For a context $\Gamma = (x_1:A_1,\ldots,x_n:A_n)$ and a renaming $\theta$,
  we write $\Gamma[\theta]$ for $(\theta(x_1):A_1,\ldots,\theta(x_n):A_n)$ if it is a context. 
\end{definition}


\subsection{Typing Rules}
We define typing rules for terms of $\LAMBDA$ and the subset $\WTyped$ of well-typed terms.
We consider a term well-typed if typing exists and it is unique for the term and for all its subterms.

The typing rules are the usual ones but for %the conditional binder $\cond$, and for 
the \emph{typing rule $\ap$ for an application} $t(u)$, which we split in two sub-rules, $\apvar$
and $\apnotvar$, according if $u$ is a variable or is not a variable.
%$\apvar$ corresponds to an $\eta$-expansion and it introduces a global variable name $x^T$
%for the first argument of $t$. 
%As we said, 
We need to insert this extra information $t$ in typing because it is important for checking termination
of the computation of $t(u)$, as we will explain later.

Remark that we introduced a unique \emph{term notation}
for application: we write $\ap(t,u)$ no matter if $u$ is a variable or not, but we we split 
the typing rule for $\ap$ into two sub-cases, $\apvar$ and $\apnotvar$.

%We have a single structural rule  $\struct_f$, which can be used for:
% weakening, variable permutation and variable renaming. 
We have a a single structural rule $\weak$ for extending a context $\Gamma$ to a context 
$\Gamma' \supseteqsim \Gamma$. When $\Gamma' \sim \Gamma$ (when 
$\Gamma'$, $\Gamma$ are permutation each other), 
the rule $\weak$ can be used for variable permutation.
Variable renaming for a term $t[\vec{x}]$ can be obtained by writing 
$(\lambda \vec{x}.t[\vec{x}])(\vec{x'})$. 
Therefore we do not assume having a primitive rule for renaming. 
The lack of a renaming rule is a 
\emph{difference with the circular syntax for inductive \underline{proofs}}.

The reason of not having implicit $\alpha$-conversion mentioned in the previous section 
comes from this assumption. 
For example, if we identify the $\alpha$-equivalent terms $\lambda x^A.x$ and $\lambda z^A.z$,
the left derivation is a correct proof, but the right one is not ($(z:A,z:A)$ is not a context).
That is, the proof structure does not closed under identity of the $\alpha$-equivalence. 
\begin{center}
  $\infer{
    z:A \vdash \lambda x^A.x:A\to A
  }{
    \infer{
      z:A, x:A \vdash x:A
    }{}
  }$
  \qquad
  $\infer{
    z:A \vdash \lambda z^A.z:A\to A
  }{
    \infer{
      z:A, z:A \vdash z:A
    }{}
  }$  
\end{center}
The provability of sequents is closed under the variable renaming and the $\alpha$-equivalence.
We will discuss this point later. 

%11:56 29/04/2024

\begin{definition}[Typing rules of $\LAMBDA$]
Assume $\Gamma = {x_1}^{A_1}:A_1, \ldots, x_n^{A_n}:A_1$ is a context. 
Let $p=0,1,2$.
%$\Delta = y_1:B_1, \ldots, y_n:B_m$ are sequents of length $n$, $m$ respectively. Suppose
%$f:\{1, \ldots, n\} \rightarrow \{1, \ldots, m\}$ is any injection, compatible with types
%in $\Gamma$, $\Delta$: we assume $A_i = B_{f(i)}$ for all $1 \le i \le n$.

\begin{enumerate}
%\item
%$\struct_f$-rule.
%If $t: \Gamma \vdash T$ then $t[ y_{f(1)}/x_1, \ldots,  y_{f(p)}/x_p]:\Delta \vdash T$
\item
A rule is a list of $p+1$ typing judgements: 
$\Gamma \vdash t_1:A_1, \ldots, \Gamma \vdash t_p:A_p, \Gamma \vdash t : A$.
The first $p$ sequents are the premises of the rule (at most two in our case), 
the last sequent is the conclusion of the rule.
\item
We read the rule above: \emph{``if $\Gamma \vdash t_1:A_1, \ldots, \Gamma \vdash t_p:A_p$
then $\Gamma \vdash t : A$"}.
\end{enumerate}

We list the typing rules of $\LAMBDA$: 
the rules $\apvar$ and $\apnotvar$ are restricted.

\begin{enumerate}

\item
$\weak$-rule (Weakening + Exchange).
If $\Gamma \vdash t:T$ and $\Gamma \subseteqsim \Gamma'$
then $\Gamma' \vdash t : T$

\item
$\var$-rule.
If $x^A \in \Gamma$ then $\Gamma \vdash x^A:A$.



\item
$\lambda$-rule.
If $\Gamma, x^A:A \vdash b: B$
then $ \Gamma \vdash \lambda x^A.b :A \rightarrow B$.

\item
$\apvar$-rule.
If $\Gamma \vdash f: A \rightarrow B$ then $\Gamma \vdash f(x^A) :  B$,
provided  $(x^A:A)\in  \Gamma$.

\item
$\apnotvar$-rule.
If $\Gamma \vdash f:A \rightarrow B$ and $\Gamma \vdash a:A$
then $\Gamma \vdash f(a) : B$, provided $a$ is \emph{not} a variable 

\item
$0$-rule.
$\Gamma \vdash 0: \N$

\item
$\Succ$-rule.
If $\Gamma \vdash t:\N$ then $\Gamma \vdash \Succ (t):\N$.

\item
$\cond$-rule.
If $\Gamma \vdash  f :T$ and  $\Gamma \vdash g : \N \rightarrow T$ 
then $\Gamma \vdash \cond(f,g) : \N \rightarrow T$.
\end{enumerate}
We abbreviate $\nil \vdash  t:A$ ($t:A$ in the empty context) with $\vdash t:A$. 
We write the set of rules of $\LAMBDA$ as
\[
\Rule = 
\{r \in \Seq \cup \Seq^2 \cup \Seq^3 | r \mbox{ instance of some rule of }\LAMBDA\}
\]
\end{definition}

A rule is uniquely determined from its conclusion, provided we know whether the rule is a weakening or not.
Two rules are equal if they have the same assumptions and the same conclusion.

\begin{proposition}[Rules and subterms]
\label{proposition-rules-subterms}
Assume $r, r' \in \Rule$ have conclusion $\Gamma \vdash t:A$, $\Gamma' \vdash t':A'$
respectively.
\begin{enumerate}
%%%%%%%%%%%%%%%%%%%%%%%%%%%%%%%%
% NO THIS IS WRONG ASSUMPTIONS OF weak ARE NOT UNIQUE. 
% Anyway we do not need this point.
%%%%%%%%%%%%%%%%%%%%%%%%%%%%%%%%
%\item
%If $r, r'$ are weakening rules and $\Gamma \vdash t:A = \Gamma' \vdash t':A'$ then $r = r'$.
%%%%%%%%%%%%%%%%%%%%%%%%%%%%%%%%
\item
If $r, r'$ are non-weakening rules and $\Gamma \vdash t:A = \Gamma' \vdash' t':A'$ then $r = r'$.
\item
If $r$ is not a weakening and the premises of $r$ are some 
$\Gamma_1 \vdash t_1:A_1, \ldots, \Gamma_p \vdash t_p:A_p$,
then the list of immediate subterms of $t$ is exactly $t_1, \ldots, t_p$.
\end{enumerate}
\end{proposition}

From the typing rules we define the proofs that a term is well-typed. 
Proofs are binary trees on the set $\Rule$, each node is associated to a typing judgement which is
a consequence of the typing judgements associated to the children node under some rule $r \in \Rule$
of $\LAMBDA$.

%12:55 29/04/2024

The next definition is the formal definition of proof.
We mention that we will introduce a restriction we call \emph{Almost-left-finite} on proof trees. 
The restriction Almost-left-finite is used to prevent trivial proofs. 
For instance, we want to forbid a proof cyclically switching the variables $x$ and $y$ in the context of a term,
by an infinite nesting of $\weak$-rule (Weakening + Exchange  rule):
 \[
  \infer[\weak]{
    (x^T:T, y^U:U) \prove t : A
  }{
    \infer[\weak]{
    (y^U:U,  x^T:T) \prove t : A
  }{\infer*{}{}}
  }
  \]
Another example: we want to forbid an infinite left nesting of applications, as in a proof of 
$t_{n}:A^n \rightarrow A$ where $t_{n} = t_{n+1}(t_0)$:
 \[
  \infer[\apnotvar]{
   \prove t_n : A^n \rightarrow A
  }{
    \infer[]{
   \prove t_{n+1} : A^{n+1} \rightarrow A
  }{\ldots}
   &
   \infer[]{
   \prove t_{0} : A
  }{\ldots}
  }
  \]
These proof trees are meaningless and should be ruled out. The Almost-Left-Finite condition will ask
that all leftmost branches in any sub-proof are finite, say, that no infinite leftmost branch made of
$\weak$-rules or of $\apnotvar$ exists. 
The idea is that when the main operation in a programming language is application, 
then the leftmost branch uniquely  determines the type of the term and therefore it should be finite.
%12:54 03/06/2024


\begin{definition}[Well-typed term of $\LAMBDA$]
Assume $\Pi=(T,\phi)$ is a binary tree labeled on $\Rule$.
Assume $\Gamma$ is a context of $t$ (i.e, $\FV(t) \subseteq \Gamma$) and $A \in \Type$ 

Then we write $\Pi: \Gamma \vdash t:A$, and we say that $\Pi$ is a proof of $\Gamma \vdash t:A$ if:

\begin{enumerate}
\item 
  $\phi(\nil) = \Gamma \vdash t:A$.
\item
  Assume that
  \begin{enumerate}
  \item
    $l \in T$ is labeled with $\phi(l) = \Delta \vdash u: B$.
  \item
    The children of $l$ in $T$ are labeled 
$
\phi(l \conc (1)) = \Delta_1 \vdash u_1: B_1, 
\ldots, 
\phi(l \conc (p)) = \Delta_p \vdash u_p: B_p
$
  \end{enumerate}
  Then for some rule $r \in \Rule$ of $\LAMBDA$ we have
  \[
  \infer[r]{
    \Delta \prove u : B
  }{
    \Delta_1  \prove u_1:B_1
    &
    \cdots
    &
    \Delta_p  \prove u_p:B_p
  }
  \]
\item
A path $\pi$ in a proof tree $\Pi$ is \emph{almost-leftmost} if there are only finitely many rules with two
 premises in $\pi$ such that $\pi$ does \emph{not} choose the leftmost premise.
\item
A proof $\Pi$ is Almost-left-finite if all almost-lefmost paths $\pi$ in $\Pi$ are finite
\end{enumerate}

\end{definition}

We can check that a proof is almost-left-finite if and only if all leftmost paths of all sub-proofs are finite.

In the case of a regular proof, we can decide in polynomial time in the number of sub-proofs whether a
proof is almost-left-finite. Here is the algorithm. Suppose a possibly infinite proof has $n$ subproofs.
For each sub-proof we follow the leftmost path, after $n$ steps either we reach some rule without
premises (either a $\var$-rule or a $0$-rule), or we loop. Then:
\begin{enumerate}
\item
If for all sub-proofs we stop then the proof is almost-left-finite
\item
If for some sub-proof we loop, then there is infinite leftmost path and the proof is \emph{not} almost-left-finite.
\end{enumerate}
We estimate the computation time to $O(n)$ for each of the $n$ sub-proofs, 
and the total computation time to $O(n^2)$.
\\

Eventually we define the well-typed terms of $\LAMBDA$ using almost-left-finite proofs.

\begin{definition}[Well-typed term of $\LAMBDA$]
\mbox{}
\begin{enumerate}
\item
$\Gamma \vdash t:A$ is true if and only if $\Pi:\Gamma \vdash t:A$ for some $\Pi$.
\item
$t \in \LAMBDA$ is well-typed if and only if 
$\Pi:\Gamma \vdash t:A$ for some \emph{almost-left-finite} proof $\Pi$,
for some $A$ and some context $\Gamma \supseteqsim \FV(t)$.
\item
$\WTyped$ is the set of well-typed $t \in \LAMBDA$.
\end{enumerate}
%A proof $\Pi:\Gamma \vdash  t :T$ is canonical if all $\weak$-rules
%\begin{enumerate}
%\item
% either follow some $\lambda$-rule for a term $\lambda x^T.b$, and they introduce $x^T$ in the context,
%\item
%or follow some $\cond$-rule for a term $\cond(f,g)$, and and they introduce $x^\N$ in the context,
%\end{enumerate}
\end{definition}

Well-typed terms are closed by substitution.

We prove that the type assigned to a term by an Almost-Left-Finite proof is \emph{unique}: 
if two Almost-Left-Finite proofs assign two types $T$, $T'$ to the same possibly infinite term $t$,
then $T = T'$.

\begin{proposition}[Uniqueness of almost-left-finite type]
If $t \in \LAMBDA$, $\Pi:\Gamma \vdash t:A$ and  $\Pi':\Gamma' \vdash t:A'$ are two almost-left-finite
proofs then $A = A'$.
\end{proposition}

%08:56 19/06/2024
\begin{proof}
By induction on the sum of the length of the leftmost branch in $\Pi$ and $\Pi'$. These lengths are finite
because $\Pi$ and $\Pi'$ are assumed to be almost-left-finite. Let $r$ be the last rule of $\Pi$ and $r'$ be the
last rule of $\Pi'$.
\begin{enumerate}
\item
Assume $r = \weak$. Then $\Pi$ is obtained from a proof $\Pi_1:\Gamma_1 \vdash t:A$. 
By induction hypothesis on $\Pi_1$ and $\Pi'$ we conclude that $A = A'$.

\item
Assume $r \not = \weak$ and $r' = \weak$. 
Then $\Pi'$ is obtained from a proof $\Pi'_1:\Gamma_1 \vdash t:A$. 
By induction hypothesis on $\Pi$ and $\Pi'_1$ we conclude that $A = A'$.

\item
Assume  $r \not = \weak$ and $r' \not = \weak$. Then both $r$ and $r'$ are the typing rule for
the outermost constructor of $t$ and for some
$h=0,1,2$ the proof $\Pi$ has premises $\Pi_1, \ldots, \Pi_h$
and the proof $\Pi'$ has premises $\Pi'_1, \ldots, \Pi'_h$ (for the same $h$). 
We distinguish one case for each constructor.

\begin{enumerate}
\item
Assume $t = x^T$. Then $r=r'=\var$-rule and $A = A' = T$.

\item
Assume $t =0^\N$. Then $r=r'=0$-rule and $A = A' = \N$.

\item
Assume $t =\Succ(u)$. Then $r=r'=\Succ$-rule  and $\Pi_1:\Gamma \vdash u: \N$
and $\Pi_2:\Gamma' \vdash u:\N$ and $A = A' = \N$.

\item
Assume $t = f(u)$. Then $r=r'=\ap$-rule and $\Pi_1:\Gamma \vdash f:U \rightarrow A$
and $\Pi_2:\Gamma' \vdash f:U' \rightarrow A'$. By induction hypothesis on $\Pi_1$ and
$\Pi'_1$ we deduce that $(U \rightarrow A) = (U' \rightarrow A')$, in particular that $A = A'$.

\item
Assume $t = \lambda x^T.b$. Then $r=r'=\lambda$-rule and 
$\Pi_1:\Gamma \vdash b:B$
and $\Pi_2:\Gamma' \vdash b:B'$. By induction hypothesis on $\Pi_1$ and
$\Pi'_1$ we deduce that $B=B'$, in particular that $A = T \rightarrow B = T \rightarrow B' = A'$.

\item
Assume $t = \cond(f,g)$. Then $r=r'=\cond$-rule and $\Pi_1:\Gamma \vdash f:A$
and $\Pi_2:\Gamma' \vdash f:A'$. By induction hypothesis on $\Pi_1$ and
$\Pi'_1$ we deduce that $A = A'$.

\end{enumerate}
\end{enumerate}
\end{proof}


We provide some examples of well-typed and not well-typed terms.
%23:30 23/04/2024
Some term in $\LAMBDA$ has no type, like the application $0(0)$ of the non-function $0$. 

\begin{Eg}
We provide a term in $\LAMBDA$ having more than one type.
Let $t=u(0)$ and $u=\cond(t,u)$. $t$ has an infinite lefmost branch $t,u,t,u, \ldots$, therefore
$t$ is not almost-left-finite and it has no \emph{almost-left-finite typing proof}.
Unicity fails and we can prove $\vdash t:A$ for all types $A \in \Type$. 
The subterms of $t$ are $\{t, u, 0\}$ and a proof $\Pi: \emptyset \vdash t:A$ is
\[
\infer[\apnotvar]{
  \prove t : A
}{
  \infer[\cond]{
    \prove u:\N \rightarrow A
  }{
    \infer*{
      \prove t : A
    }{}
    &
    \quad
    &
    \infer*{
      \prove u : \N \rightarrow A
    }{} 
  }
  &
  \quad
  &
  \infer[0]{
    \prove 0:\N
  }{}
}
\]
Formally, the typing proof $\Pi=(T,\phi):\vdash t:A$ 
is defined by $\universe{\Pi}=\universe{t}$ and:
\begin{enumerate}
\item
$\Label(\Pi,l)=(\vdash A)$ \ \ \ \ \ \ \ for all $l \in \universe{t}$ such that  $t \restr l = t$,
\item
$\Label(\Pi,l) = (\vdash \N \rightarrow A)$  for all $l \in \universe{t}$ such that $t \restr l = u$,
\item
 $\Label(\Pi,l)=(\vdash\N)$ \ \ \ \ \ \ \  for all $l \in \universe{t}$ such that $t \restr l = 0$. 
\end{enumerate}
%09:06 24/04/2024
\end{Eg}


A term with at least two types has the leftmost branch infinite, as it is the case for $t$ above.
We proved that if \emph{all} subterms of a term have the leftmost branch finite, then the term 
has at most one type.


\emph{Claim (existence of a polynomial-time typing algorithm)}.
We claim without proof that if a term $t$ is regular then we can decide if an almost-left-finite proof 
$\Pi:\Gamma \vdash A$ exists. If an almost-left-finite proof $\Pi$ exists 
then at least one of almost-left-finite proofs is regular and we can compute it. 
We first check whether the term is almost-left-finite in time $O(n^2)$. If it is not we reject the term.
Otherwise we start the standard recursive typing algorithm, until a type has been inferred consistently
for each of the finitely many subterms, or some type inconsistency has been found.
In the first case we accept the term and we return the time, in the second case we reject the term
and we return no type.
The computation requires quadratic time in the size of the graph representing the term.


%10:26 20/04/2024
%13:49 29/04/2024


\subsection{Reduction Rules}
Our goal is to provide a set of well-formed term for $\LAMBDA$ and interpret them as partial functionals.
Some terms, those satisfying the global trace condition (to be introduced later) will be total functionals.
Our first step is to provide reduction rules for $\LAMBDA$.

%18:14 27/03/2024

\begin{definition}[reduction rules for $\LAMBDA$]
\mbox{}
\begin{enumerate}

\item
$\reduces_\beta$: $(\lambda x^A.b)(a) \reduces_\beta b[a/x]$

\item 
$\reduces_\cond$: $\cond(f,g)(0) \reduces_\cond f$ and
$\cond(f,g)(\Succ (t)) \reduces_\cond g(t)$.

\item
$\reduces$ is the contraction of one $\beta$ or $\cond$ redex: 
the context closure of $\reduces_\beta$ and $\reduces_\cond$.

\item
$\reduces^*$ is the transitive closure of $\reduces$ (finite sequence of reductions).


\item
The $\cond$-depth of a node $l=(i_1, \ldots, i_n) \in t$ 
is the number of $1 \le h < n$ such that $(i_1, \ldots, i_h)$ has label $\cond$
and $i_{h+1} = 2$
(the number of $\cond$ such that $l$ occurs in the right-hand-side of such $\cond$).

\item
The $n$-safe level of $t$ is the set of nodes of $t$ of depth $\le n$.

\item
We say that $t \nsafeReduces{n} u$, or that $t$ $n$-safely reduces to $u$,  
if we reduce $t$ to $u$ by contracting a single redex in the $n$-safe level of $t$ 
(i.e. of $\cond$-depth $\le n$).
%We call \emph{unsafe} a reduction inside any $\cond(f,g)$.

\item
A term is $n$-safe-normal if all its redexes (if any) have $\cond$-depth $>n$.

We abbreviate \quotationMarks{$0$-safe normal} with just \quotationMarks{safe normal}.
\end{enumerate}
\end{definition}

%09:19 24/04/2024
%14:00 29/04/2024

\begin{Eg}
Let $u = \cond. (0, (\lambda z.u)(z) )$, where we omitted the type superscript
of $z$ because it is irrelevant. Then $u$ is $0$-safe-normal (safe normal for short), because
all redexes in $u$ are of the form  $(\lambda z.u)(z)$ and are in the right-hand-side of a $\cond$. 
However $u$ has infinitely many nodes which are $\beta$-redexes. 
Indeed, the tree form of $u$ has the following branch:
\begin{center}
  $u$, 
  \quad
  $(\lambda z.u)(z)$, 
  \quad
  $\lambda z.u$, 
 \quad
  $u$, 
 \quad $\ldots$
\end{center}
This branch is cyclic, infinite,
and it includes infinitely many $\beta$-redexes, all equal to the same term $(\lambda z.u)(z)$.
\end{Eg}

Through branches crossing the right-hand-side
of some $\cond$ we will-express fixed-point equations.
Reductions on fixed-point equations can easily loop.
However, in order to produre the output of a computation we only need 
to reduce closed terms of type $\N$ to a numeral. We will prove that for this taks we only
need $0$-safe reductions, those in the the right-hand-side
of \emph{no} $\cond$, and that $0$-safe reductions strongly normalize. 

\begin{Eg}
This is an example of safe $\cond$-reductions from a term $v(n)$ to a normal form. 
Assume $n \in \Num$ is any numeral and $v = \cond(1, v)$. There are only finitely many reductions
from $v(n)$, and they are all $0$-safe. $v(n)$ $\cond$-reduces to $v(n-1)$, 
then we loop: $v(n-1)$ $\cond$-reduces to $v(n-2)$ and so forth.
After $n$ $\cond$-reductions we get $v(0)$. With one last $\cond$-reduction we get $1$ and we stop. 
Thus, the term $v(n)$ strongly normalizes to $1$ in $(n+1)$-steps, in fact $v(n)$ has a unique reduction path,
having length $(n+1)$ and terminating in $1$.
\end{Eg}

There are terms without a normal form, but we can get as close as we want to a normal form
In fact, we will prove that for all $n$, all terms have some $n$-safe normal form, that is, 
with no redexes of $\cond$-depth $\le n$. 
The $n$-safe normal form of a term is unique up $n$-safe equality: any two $n$-safe normal forms have the same $n$-safe level, though they may be different in some level of $\cond$-depth $>n$.

We will also prove that terms have a limit normal form, 
obtained by taking the limit of the $n$-safe normal forms of the term, and that this limit normal form is unique.



\newpage
% Section 3

\section{The trace of the infinite $\lambda$-terms}
%19:34 27/03/2024
We define a notion of trace for possibly infinite $\lambda$-terms, 
describing how an input of type $\N$ is used when computing the output.
The first step toward a trace is defining a notion of \emph{connection} between arguments
of type $\N$ in the proof that $t$ is well-typed. 
To this aim, we need the notions of \emph{list of argument
 types} and of \emph{index of atomic types} for a term.

We sketch the notion of connection through an example.
Assume 
\[
  {x_1}^{A_1}:A_1,{x_2}^{A_2}:A_2 \vdash t[x_1,x_2]:B_3 \rightarrow \N
\]
Then the list of argument types of $t$
is $A_1, A_2, B_3$. $A_1,A_2$ are arguments with names $x_1, x_2$, while $B_3$ is an unnamed
argument (it will be denoted by some bound variable in $t$). 
The index of an $\N$-argument of $t$ is any $j \in \{1,2,3\}$ such that $A_j=\N$
or $B_j=\N$ respectively: in this case all integers $1,2,3$ are indexes of some $\N$-argument,
in general we can have argument with type different from $\N$.

Remark that for an open term $ t[x_1,x_2]$ we list among the ``argument types'' also the
types $A_1, A_2$ of the free variables. We motivate this terminology:
in a sense, $t$ is an abbreviation of the closed term $t' = \lambda  
{x_1}^{A_1},{x_2}^{A_2}.t: (  A_1,A_2,B_3 \rightarrow \N )$, and the argument types of $t'$ are
$A_1, A_2, B_3$ and they include $A_1, A_2$. 

Below is the formal definition of argument types for a term.


\begin{definition}[List of argument types of a term]
Assume that $\vec{A} = A_1, \ldots, A_n$, $\vec{B}=B_{n+1}, \ldots, B_{n+m}$, 
$\Gamma = \{\vec{x}:\vec{A}\}$,
and $\Gamma \vdash t: \vec{B} \rightarrow \N$.

\begin{enumerate}
\item
The \emph{list of argument types} of $t$ is $\vec{C} = \vec{A},\vec{B}$. 

\item
$A_1, \ldots, A_n$ are the \emph{named arguments}, with names $x_1, \ldots, x_n$.

\item
$B_{n+1}, \ldots, B_{n+m}$ are the \emph{unnamed arguments}.

\item
An \emph{index of an $\N$-argument} 
of $t$ is any $j \in \{1, \ldots, n+m\}$ such that $C_j = \N$.

\end{enumerate}
\end{definition}

We now define the connection between $\N$-arguments of subterms of $t$
in a proof $\Pi: t:\Gamma \vdash A$ of  $\LAMBDA$. The connection describes
how an input  moves through the infinite unfolding of the term.
The definition of  atom connection for a syntax including the binder $\lambda$ 
is the main contribution of this paper. 
%Two $\N$-argument are in
%connection if and only if they receive the same global input: local input are ignored.
%In many cases two corresponding argument types have the same index, but if we insert or remove
%free variables or arguments the index may change.
Before providing the general definition, we discuss the notion of connection through examples. 
We draw in the same color two $\N$-argument which are in connection. 

\begin{Eg}\label{eg:0}\rm
An example of  atom connection for some instance of the $\weak$-rule.
\[
\infer[(\weak)]{
  {x_1} : \bfColor{red}{\N},{x_2} : \bfColor{blue}{\N}, x_3:\bfColor{oldgold}{\N}
  \prove t : \bfColor{orange}{\N} \rightarrow \N
}{
  {x_1} : \bfColor{red}{\N},{x_2} : \bfColor{blue}{\N} 
  \prove t : \bfColor{orange}{\N} \rightarrow \N
}
\]
\end{Eg}
Remark that the type $\N$ of the variable $x_3$, colored in \bfColor{oldgold}{old gold} and 
introduced by weakening, is in connection with no type in the rule $\weak$.

%20:13 15/04/2024
\begin{Eg}\label{eg:1}%\rm
An example of  atom connection for some instance of the $\apnotvar$-rule.
We assume that $a$ is \emph{not} a variable.
\[
\infer[(\apnotvar)]{
  x_1 : \bfColor{red}{\N},{x_2} : \bfColor{blue}{\N}
  \prove f(a) : \bfColor{orange}{\N} \rightarrow \N
}{
  x_1 : \bfColor{red}{\N},{x_2} : \bfColor{blue}{\N}
  \prove f : \bfColor{oldgold}{\N}, \bfColor{orange}{\N} \rightarrow \N
  &
  x_1 : \bfColor{red}{\N}, x_2: \bfColor{blue}{\N} \prove a : \N
}
\]
\end{Eg}
Remark that the first unnamed argument of $f$ (colored in \bfColor{oldgold}{old gold}) 
is in connection with no argument in the rule $\apnotvar$.
The reason is that in the term $f(a)$,
the first argument of $f$ receives a value from the value $a$ which is local to the term $f(a)$,
does not receive a value from outside the term.
However, the first argument of $f$ can be in connection with some argument higher in the typing proof. 
%13:21 15/04/2024

We postpone examples with the rule $\apvar$.

\begin{Eg}\label{eg:2}%\rm
An example of  atom connection for the rule $\cond$.
\[
\infer[(\cond)]{
  {x_1} : \bfColor{red}{\N},{x_2} : \bfColor{blue}{\N}
  \prove \cond(f,g) : \bfColor{oldgold}{\N} \rightarrow \N
}{
  x_1 : \bfColor{red}{\N},{x_2} : \bfColor{blue}{\N} \prove f : \N
  &
  x_1:\bfColor{red}{\N}, x_2:\bfColor{blue}{\N} \prove g:\bfColor{oldgold}{\N} \rightarrow \N
}
\]
\end{Eg}
Remark that the first unnamed argument of $\cond(f,g)$ (colored in \bfColor{oldgold}{old gold}) 
is in connection the type of the first unnamed one of $g$ in the second premise,
but it has no connection with the first premise. 


\subsection{A formal definition of connection}
The definition of connection requires to define an injection 
\[
\ins:\N,\N \rightarrow\N
\]
We set $\ins(p,x)=x+1$ if $x \ge p+1$, and $\ins(p,x)=x$ if $x\le p$.
The role of $\ins$ is inserting one fresh index $p+1$ for a new argument type higher in the proof
connected to no argument in the conclusion.

\begin{definition}[Connection in a proof of  $\LAMBDA$]
Assume $\vec{A} = A_1, \ldots, A_n$, $\vec{B}=B_1, \ldots, B_m$, $\Gamma = \vec{x}:\vec{A}$,
and $\Gamma \vdash t:\vec{B} \rightarrow \N$.

For each $\N$-argument $k$ in $t$, for $t'=t$ or $t'$  immediate subterm of $t$ 
each $\N$-argument $k'$ in $t'$ we define the relation: ``$k',t'$ the successor of $k,t$". We require:
\begin{enumerate}
\item
if $t$ is obtained by a rule $\weak$ from $t'=t$, described by a map $\phi:\Gamma \rightarrow \Delta$,  
then $k = \phi(k')$.
\item
if $t=f(a)$ is obtained by a rule $\apnotvar$ (i.e., $a$ is not a variable) and $t'=f$ 
then $k' = \ins(n,k)$. If $t'=a$ then $k'=k$ and $k' \le n$.
\item
if $t=f(x_i^{A_i})$ is obtained by a rule $\apvar$ and $t'=f$ 
then $k' = \ins(n,k)$or $k'=n+1$ and $k=i$.
\item
if $t=\cond(f,g)$ and $t'=f$ 
then $k' = \ins(n,k)$. If $t'=g$ then $k'=k$.
\end{enumerate}
We require $k = k'$ in all other cases, 
which are: $\Succ $, $\lambda$. 
\\
No connection is defined for the rules $\var$, $0$.
\end{definition}

Below we include some examples. 
We draw in the same color two $\N$-argument which are in connection. 
In the case of the rule $\apvar$, an argument can be connected to two arguments above it.


\begin{Eg}\label{eg:3}%\rm
Atom connection for $\apvar$.
We assume that $x$ is a variable.
\[
\infer[(\apvar)]{
  x_1 : \bfColor{red}{\N},{x_2} : \bfColor{blue}{\N}, x_3  : \bfColor{oldgold}{\N}
  \prove f(x_3) : \bfColor{orange}{\N} \rightarrow \N
}{
  x_1 : \bfColor{red}{\N},{x_2} : \bfColor{blue}{\N}, x_3  : \bfColor{oldgold}{\N}
  \prove f : \bfColor{oldgold}{\N}, \bfColor{orange}{\N} \rightarrow \N
}
\]
\end{Eg}

Remark that the last variable in the conclusion of the rule
is in connection with the first unnamed argument of $f$ (colored in \bfColor{oldgold}{old gold}) 
in the premise, and with the variable with the same name in the premise: there are
two connections.

\begin{Eg}\label{eg:4}%\rm
Atom connection for  $\lambda$-rule.
We assume that $x$ is  a variable.
\[
\infer[(\ap)]{
  x_1: \bfColor{red}{\N}, x_2: \bfColor{blue}{\N}
  \prove \lambda x.b : \bfColor{oldgold}{\N} \rightarrow \N
}{
  x_1: \bfColor{red}{\N}, x_2: \bfColor{blue}{\N}, x:\bfColor{oldgold}{\N} \prove b : \N
}
\]
\end{Eg}
Remark that the first unnamed argument of $\lambda x.b$ (colored in \bfColor{oldgold}{old gold}) 
in the conclusion is in connection with the last variable type in the premise of the rule.
\\

Summing up, when
we move up in a $\apvar$-rule, the type of last free variable corresponds to the type of the first
unnamend argument.
When  we move up in a $\lambda$-rule it is the other way round. 

The connection on $\N$-arguments in a proof $\Pi:\Gamma\vdash t:A$ defines a graph $\Graph(\Pi)$ 
whose nodes are all pairs $(k,u)$, with $u$ subterm of $t$ and $k$ index of some $\N$-argument of  $u$.
$\Graph(\Pi)$ is finite when $t$ is regular.
A trace represents the movement of an input information through the infinite unfolding of a tree.
Formally, we define a trace as a (finite or infinite) path in the graph $\Graph(\Pi)$.

\begin{definition}[Trace for well-typed terms in $\LAMBDA$]
Assume $\Pi$ is any typing proof of $t \in \WTyped$.
\begin{enumerate}
\item
A path of $\Pi$ is any finite or infinite branch $\pi =(i_0, \ldots, i_n, \ldots)$ of $\Pi$.

\item
Assume $\pi =(t_0, \ldots, t_n, \ldots)$ is a path of $\Pi$, finite or infinite. 
A finite or infinite \emph{trace} $\tau$ of $\pi$ in $\Pi$ is a list 
$\tau =( (k_m,t_m), \ldots, (k_n,t_n), \ldots)$ such that for all $i=m,\ldots, n,\ldots$:
\begin{enumerate}
\item
$k_i$ is an index of some $\N$-argument of $t_i$
\item
if $i+1$ is an index of $\tau$ then $(k_{i+1},t_{i+1})$ is connected with $(k_i, t_i)$ in $\Pi$.
\end{enumerate}

\end{enumerate}
\end{definition}

The starting point of a trace could be different from the starting point of the path to which the trace belongs.

%20:50 15/04/2024
%10:35 24/04/2024
%12:28 30/04/2024


\newpage
% Section 4

\section{The circular system $\CTlambda$}
We introduce  a condition call \emph{global trace condition} for all proofs $\Pi$ that the term $t$ is well-typed,
then the subset $\GTC$ of the set $\WTyped$ of well-typed terms of $\LAMBDA$ 
satisfying the global trace condition. $\GTC$ is a set of total recursive functionals. We use $\GTC$
as a semantics for the set $\CTlambda$ of circular $\lambda$-terms we will define.

$\CTlambda$ will be the regular terms of $\CTlambda$. 
$\CTlambda$ is a decidable set.
For the terms of $\CTlambda$ we will prove
strong normalization, church-rosser for the \quotationMarks{safe} part of a term, 
and the fact that every closed normal term of type
$\N$ is a numeral. 
As a consequence, all terms $\CTlambda$ will be interpreted as total functionals. 
\\

From the $\N$-argument connection we now define the global trace condition and the set of 
terms of $\GTC$  and of $\CTlambda$.



\begin{definition}[Global trace condition and terms of $\CTlambda$]
\label{definition-global-trace-condition}
\mbox{}
 \linebreak 
Assume $\tau =( (k_m,t_m), \ldots, (k_n,t_n), \ldots)$ 
is a trace of a path $\pi = (t_1, \ldots, t_n, \ldots)$ of $t \in \WTyped$. 
Assume $i=m,\ldots, n$.
\begin{enumerate}
\item
$\tau$ is progressing in $i$ if $t_i=\cond(f,g)$ for some $f$, $g$,
and $k_i$ is the index of the first \emph{unnamed} argument the $\cond$-rule, 
otherwise $\tau$ is not progressing in $i$.

\item
$t$ satisfies the global trace condition if for some typing proof $\Pi$,
of $t$, for all infinite paths $\pi$ of $t$ in $\Pi$,
there is some infinitely progressing path $\tau$ in $\pi$ and $\Pi$.
$\GTC$ is the set of well-typed terms $t \in \LAMBDA$ satisfying the global trace condition.

\item
$\CTlambda = \GTC \cap \Reg$: the cyclic $\lambda$-terms of 
$\CTlambda$ are all well-typed terms which are regular trees (having finitely many subtrees), 
and which satisfy the global trace condition.

\end{enumerate}
\end{definition}

The notion of global trace condition is defined through a universal quantification on proof 
$\Pi:\Gamma \vdash t:A$, and proofs $\Pi$ are possibly infinite objects and they form 
a non-countable sets.
Therefore we would expect that the global trace condition is not decidable. 
Surprisingly, global trace is decidable in polynomial space in $t$. The reason is that is some proof satisfies
the global trace condition then all proofs do, and for regular terms there is a typing proof
which is a finite graph and for which the global trace condition is decidable.

We believe that the global trace condition is decidable quite fast
for realistic $\lambda$-terms. Another nice feature of the global trace condition is that for 
$t \in \GTC$ we require that there is some proof $\Pi:\Gamma \prove t:A$ whose
infinite paths all have some infinite progressing trace. This proof is almost-left-finite, because
all leftmost paths from a sub-proof have no progress point and therefore are finite. 
We conclude that $t \in \GTC$ implies that $t \in \WTyped$: we can assign a unique type to $t$
with a \quotationMarks{meaningful proof}, that is, an almost-left-finite finite proof.

We include now some examples of cyclic $\lambda$-terms. We recall that they have to satisfy
$\GTC$, therefore they are almost-left-finite, and by definition they are regular.

%18:05 03/06/2024

%\section{Examples of terms of $\CTlambda$}

\subsection{The sum map}

A first example of term of  $\CTlambda$. 
We provide an infinite regular term $\Sum$ computing the sum on $\N$.
In this example, the type superscript $\N$ of variables $x^\N$, $z^\N$ is omitted.

\begin{Eg}
\label{example-sum}
We set $\Sum = \lambda x.\cond(x,\lambda z.\Succ(\Sum(x)(z)))$.
$\Sum$ is a regular term because it has finitely many subterms: 
\begin{center}
  $\Sum$,
  \quad
  $\cond(x,\lambda z.\Succ(\Sum(x)(z)))$,
  \quad
  $x$,
  \quad
  $\lambda z.\Succ(\Sum(x)(z))$,
 \quad
  $\Succ(\Sum(x)(z))$,
  \quad
  $\Sum(x)(z)$,
  \quad
  $\Sum(x)$,
  \quad
   z
\end{center}
$\Sum$ is well-typed by the following derivation, in which we have a back edge from the 
$\bfColor{oldgold}{\dagger}$ above to the $\bfColor{oldgold}{\dagger}$ below.
\[
\infer[\lambda]{
  \vdash \Sum:\N, \bfColor{oldgold}{\N} \rightarrow \N 
   \quad (\bfColor{oldgold}{\dagger})
}{
  \infer[\cond]{
    x : \N \vdash 
    \cond(x,\lambda z.\Succ(\Sum(x)(z))): \bfColor{oldgold}{\N} \rightarrow \N
     \quad (\bfColor{oldgold}{ \spadesuit})
  }{
    \infer[\var]{
      x : \N \vdash x : \N
    }{}
    &
    \infer[\lambda]{
      x:\N \vdash \lambda z.\Succ(\Sum(x)(z)): \bfColor{oldgold}{\N} \rightarrow \N  
      %\quad (\bfColor{oldgold}{ \spadesuit})
    }{
      \infer[\Succ]{
        x:\N, z : \bfColor{oldgold}{\N} 
        \vdash \Succ(\Sum(x)(z)): \N  
        % \quad (\bfColor{oldgold}{ \spadesuit})
      }{
        \infer[\apvar]{
          x:\N, z : \bfColor{oldgold}{\N} 
          \vdash \Sum(x)(z): \N
        }{
          \infer[\apvar]{
            x:\N,  z : \bfColor{oldgold}{\N}
            \vdash \Sum(x): \bfColor{oldgold}{\N} \rightarrow \N
          }{
            \infer[\weak]{
              x:\N,  z : \bfColor{oldgold}{\N}
              \vdash \Sum: \N, \bfColor{oldgold}{\N} \rightarrow  \N
            }{
              \infer*{\vdash \Sum: \N, \bfColor{oldgold}{\N} \rightarrow \N 
                \quad (\bfColor{oldgold}{\dagger})}{}
            }
          }
        }
      }
    }
  }
}
\]
\end{Eg}

We colored in \bfColor{oldgold}{old gold} one sequence of atoms $\bfColor{oldgold}{\N}$:
this is one trace of the proof.
The colored trace is the unique infinite trace, it is cyclic and it is infinitely progressing.
We marked $\bfColor{oldgold}{ \spadesuit}$ the only progress point of the only infinite trace.
The progress point is repeated infinitely many times.


%
%\begin{tikzpicture}
%  %\draw [help lines] (-3,-1) grid (9,7);
%  \coordinate (a) at (0,0) node at (a) {A};
%  \coordinate (c) at (0,5) node at (c) {C};
%  \draw (0,0) -- (0:2cm);
%  \draw (0,0) -- (30:3cm);
%  \draw (0,5) -- +(0:2cm);
%\end{tikzpicture}


%10:43 16/04/2024

\subsection{The Iterator}

\begin{Eg}
We define a term $\Iter$ of  $\CTlambda$ computing the iteration of maps on $\N$.
The term $\Iter$ is a normal term of type $(\N \rightarrow \N), \N,\N \rightarrow \N$ such that
$\Iter(f,a,n)=f^n(a)$ for all numeral $n \in \Num$. 
We have to solve the equations:

\begin{enumerate}
\item
$\Iter(f,a,0) \sim a$ 
\item
$\Iter(f,a,\Succ (t)) \sim f(\Iter(f,a,t))$
\end{enumerate}

where $f$, $a$ abbreviate $f^{\N\rightarrow\N}$ and $a^\N$.
We solve them with $\Iter = \lambda f, a.\iter$
where 
$$
\iter = \cond (a, \lambda x.f(\iter(x))):\N \rightarrow \N
$$ 
is a term in the context $\Gamma = (f:\N \rightarrow \N, a:\N)$.
\end{Eg}

\begin{proposition}
$\Iter \in \Reg \cap \WTyped \cap \GTC = \CTlambda$.
\end{proposition}

\begin{proof}
The term $\Iter$ is well-typed and regular by definition. 
We check the global trace condition. 
\\
We follow the unique infinite trace $\tau$ of the last unnamed argument $\N$ of $\Iter$ 
in the unique infinite path $\pi$ of $\Iter$. 
The trace $\tau$ moves from  $\Iter = \lambda f. \lambda a.\iter$
 to the first unnamed argument of the sub-term $\lambda a.\iter$, 
then to $a:\N$ in the context of $\iter = \cond (a, \lambda x.f(\iter(x)))$.
Then the infinite path $\pi$ and moves in this order to:
 $\lambda x.f(\iter(x))$, $f(\iter(x))$, $ \iter(x)$, $\iter$
In the meanwhile, $\tau$ progresses from $\cond$ to the second argument $\lambda x.f(\iter(x))$
of $\cond$, then moves to $x$ in the context of $f(\iter(x))$,
then to $x$ in the context $\iter(x)$, and eventually to $x$ in the context of $\iter$.
%10:30 24/03/24
In this moment the infinite path $\pi$ loops from $\iter$ to $\iter$. At each loop the trace $\tau$ 
progresses once. We conclude that $\tau$ infinitely progresses.
\end{proof}

The proof above includes a type inference from the term $\iter$ to itself.
We draw the type inference in the picture below. 
We have a back edge from the 
$\bfColor{oldgold}{\dagger}$ on the top to the $\bfColor{oldgold}{\dagger}$ in the bottom,
and we marked $\bfColor{oldgold}{ \spadesuit}$ the only progress point, which is cyclically repeated
in $\tau$.
We abbreviate $\Gamma = (f:\N \rightarrow \N, a:\N)$.
\[
%\infer[\lambda]{ %opening: \infer[\lambda]
%  \vdash \Iter:(\N \rightarrow \N), \N, \bfColor{oldgold}{\N} \rightarrow \N
% }{
%  \infer[\lambda]{ %opening: \infer[\lambda]
%  f:\N \rightarrow \N
%  \vdash \lambda a.\iter:\N, \bfColor{oldgold}{\N} \rightarrow \N
%  }
{
    \infer[\cond]{ %opening: \infer[\cond]
      \Gamma 
      \vdash \iter: \bfColor{oldgold}{\N} \rightarrow \N 
        \quad (\bfColor{oldgold}{ \spadesuit}, \bfColor{oldgold}{\dagger})
     }{ 
         \infer[\var]{
       \Gamma 
      \vdash a:\N}{}
     &
        {\ \ \ \ \ \ }
        {\infer[\lambda] %opening: \infer[\lambda]
         {
         \Gamma
          \vdash \lambda x.f(\iter(x)):\bfColor{oldgold}{\N} \rightarrow \N
         }{
         \infer[\ap]{ %opening: \infer[\ap]
           \Gamma, x:\bfColor{oldgold}{\N}
          \vdash f(\iter(x)):\N
           }{
          \infer[\var]{
       \Gamma, x:\N 
      \vdash f:\N \rightarrow \N}{}
           {\ \ \ \ \ \ \ \ \ \ \ \ }
           {\infer[\apvar] %opening: \infer[\apvar]
            {\Gamma, x:\bfColor{oldgold}{\N}
        \vdash \iter(x): \N 
             }{
          \infer[\weak]{\Gamma, x:\N
                                 \vdash \iter: \bfColor{oldgold}{\N} \rightarrow \N}
                                {\infer*{\Gamma
                                 \vdash \iter: \bfColor{oldgold}{\N} \rightarrow \N
                                  \ \ \ (\bfColor{oldgold}{ \dagger})}{}
             }
           }
          }
        }%closing: \infer[\apvar]
      }%closing: \infer[\ap]
    }%closing: \infer[\lambda]
   }%closing: \infer[\cond]
 }%closing: \infer[\lambda]
%}%closing: \infer[\lambda]
\]




\subsection{The Interval Map}
A third example. We simulate lists with two variables $\nil:\alpha$ and 
$\cons:\N,\alpha \rightarrow \alpha$. We recursively define a notation for lists by $[]=\nil$,
$a @ l=\cons(a,l)$ and $[a,\vec{a}] = a @ [\vec{a}]$. We add no elimination rules for lists, though,
only the variables $\nil$ and $\cons$. Elimination rules are not required in our example.

\begin{Eg}
We will define a term $\Interval$ with one argument $f:\N \rightarrow \N$ and three argument
$a,x,y:\N$ (we skip all type superscripts), such that 
\[
\Interval(f,a,n,m) \  \sim \ [f^n(a), f^{n+1}(a), \ldots, f^{n+m}(a)] \  : \ \alpha
\]
for all numeral $n,m \in \Num$. 
We have to solve the recursive equations 
\begin{center}
  $\Interval(f,a,n,0) \sim [f^n(a)]$
  \quad
  and
  \quad
  $\Interval(f,a,n,\Succ (m))  \sim f^n(a) @ \Interval(f,a,\Succ(n),m)$.
\end{center}
Assume $\iter = \cond (a, \lambda x.f(\iter(x))):\N \rightarrow \N$ is the term
in the context $(f:\N\rightarrow\N, a:\N)$ defined 
in the previous sub-section: in particular, $\iter(n) \sim f^n(a)$ for all $n \in \Num$.
We solve the recursive equation for $\Interval$ with $\Interval = \lambda f,a.\interval$,
where 
\[
\interval:\N,\N \rightarrow \alpha
\]
is a term in the context $(f:\N\rightarrow\N, a:\N)$ defined by 
\[
\interval 
\ \ \ = \ \ \ 
\lambda x.\cond (\base,  \lambda y.\inductive),
\]
where the base case and the inductive case are
\[
\base 
\ \ \ =\ \ \  
[\iter(x)]
\ \ \ =\ \ \ 
\cons(\iter(x),\nil)
\]
\[
\inductive 
\ \ \ = \ \ \ 
\iter(x) @ \interval(\Succ(x))(y)
\ \ \ = \ \ \  
\cons(\ \iter(x), \ \interval(\Succ(x))(y) \ )
\]
\end{Eg}

\begin{proposition}
$\Interval \in \Reg \cap \WTyped \cap \GTC = \CTlambda$.
\end{proposition}

\begin{proof}
The term $\Interval$ is well-typed and regular by definition. We check the global trace condition.
Any infinite path $\pi$ either moves to $\iter$, for which we already checked the global trace condition,
or cyclically moves from $\interval$ to $\interval$.
We follow the unique infinite 
trace $\tau$ of the last unnamed argument $\N$ of $\Interval$ inside this infinite path.
The trace $\tau$ moves to the last unnamed argument $\N$ of  
$\interval:\N,\N \rightarrow \N$, then to the last unnamed argument of
$\cond (\ [\iter(x)],  \  \lambda y.\iter(x) @ (\lambda x.\interval)(\Succ(x))(y) \ )$.
In this step $\tau$ progresses, and moves to 
the first unnamed argument of $\lambda y.\iter(x) @ (\lambda x.\interval)(\Succ(x))(y) \ )$,
then to $y:\N$ in the context of $\iter(x) @ (\lambda x.\interval)(\Succ(x))(y) \ )$.
After one $\apvar$ rule, the trace $\tau$ reaches the unique unnamed argument of 
$(\lambda x.\interval)(\Succ(x))$, then the last unnamed argument $\tau$ of $\interval$. 
From $\interval$ the trace $\tau$ loops. Each time $\tau$ moves from $\interval$ to $\interval$
then $\tau$ progresses once. We conclude that $\tau$ infinitely progresses.
\end{proof}
%14:27 24/04/2024

The proof above includes a type inference from the term $\interval$ to itself.
We draw the type inference in the figure \ref{figure-term-interval}. 
In this figure, we include a back edge from the 
$\bfColor{oldgold}{\dagger}$ on the top to the $\bfColor{oldgold}{\dagger}$ in the bottom,
and we marked $\bfColor{oldgold}{ \spadesuit}$ the only progress point, which is cyclically repeated
in $\tau$.
We abbreviate 
$\Sigma = (\nil:\alpha, \cons:\N,\alpha \rightarrow \N)$
and
$\Gamma = \Sigma,(f:\N \rightarrow \N, a:\N)$.

%PROOF SNAPSHOT
\begin{figure}
\label{figure-term-interval}

%\includegraphics[scale=0.6]{type-inference-term-interval.PNG}

\begin{center}
%% LATEX SOURCE CODE THEN A SNAPSHOT 
%% THEN THE SNAPSHOT HAS BEEN MODIFIED IN WORD 
%% THEN A LAST SNAPSHOT
  
\[
%\infer[\lambda]
% {\Sigma\vdash \Interval:(\N \rightarrow \N),\N,\N,\bfColor{oldgold}{\N}\rightarrow\alpha}
% {\infer[\lambda]
%   {\Sigma,f:\N\rightarrow\N \vdash \lambda a.\interval:\N,\N,\bfColor{oldgold}{\N}   
%      \rightarrow\alpha}
   {\infer[\lambda]  
     {\Gamma \vdash \interval:\N,\bfColor{oldgold}{\N}\rightarrow\alpha 
       \ \ \ (\bfColor{oldgold}{\dagger}) }
       {\infer[\cond]{\Gamma, x:\N 
	\vdash 
	\cond (\base,  \inductive )
	:\bfColor{oldgold}{\N}\rightarrow\alpha \ \ \ (\bfColor{oldgold}{ \spadesuit}) } 
       {\infer[\apvar]{\Gamma, x:\N 
	           \vdash 
	           \base:\alpha}
             {\infer[]{\Gamma, x:\N 
	           \vdash 
	           \cons(\iter(x)):\alpha \rightarrow\N}{\ldots}}
            {\ \ \ \ \ \ }  &
          \infer[\lambda]{\Gamma, x:\N 
	           \vdash 
	           \lambda y.\inductive : \bfColor{oldgold}{\N}\rightarrow\alpha}
             {\infer[\apnotvar]{\Gamma, x:\N, y:\bfColor{oldgold}{\N} \vdash
               \inductive : \alpha}
            {\infer[]
                     {\Gamma, x:\N, y:\N 
                          \vdash \cons(\iter(x)):\alpha\rightarrow\N}{\ldots}
               {\ \ \ \ \ \ }
                     {\infer[\apvar]{\Gamma, x:\N, y:\bfColor{oldgold}{\N} 
                          \vdash \interval(\Succ(x))(y):\alpha}
                         {\infer[\apnotvar]{\Gamma, x:\N, y:\N 
                          \vdash \interval(\Succ(x)):\bfColor{oldgold}{\N} \rightarrow\alpha}
                              {\infer[\weak]{\Gamma, x:\N, y:\N 
                          \vdash \interval:\N,\bfColor{oldgold}{\N} \rightarrow\alpha}
                                {\infer[]{\Gamma 
                          \vdash \interval:\N,\bfColor{oldgold}{\N} \rightarrow\alpha
                           \ \ \ (\bfColor{oldgold}{\dagger})}
              {\ldots}}}}}
             }
           }
         }
       }  
    }
%   }
\]

\mbox{Figure \ref{figure-term-interval}}
\end{center}

\end{figure}

If we carefully examine the term $\Interval$, we can guess several results of this paper.
We have infinitely many nested $\beta$-reduction $(\lambda x. \ldots)(\Succ (x))$.
We can remove all of them in a single step. Inside the $\beta$-redex number $k$ we obtain a sub-term
$\iter[\Succ (x)/x]\ldots[\Succ (x)/x]$ (substitution repeated $k$ times).
The result is $\iter[\Succ ^k(x)/x] $.
The nested substitution produce infinitely many pairwise different sub-terms 
$\iter(\Succ ^k(x))$ for all $k \in \N$.
We need infinitely many steps to normalize all $\iter(\Succ ^k(x))$ to $f^k(I)$, 
even if we allow to reduce all $\beta,\cond$-redexes at the same time.
Also the normal form is not regular: it contains all terms $f^k(\iter(x))$ for $k \in \N$, hence
infinitely many pairwise different terms. 
%These infinite sub-terms are of a particulary simple form, though. 
%They are obtained by the repeating $k$ times the assignment $z:=f(z)$, then applying $z:=I$ once
%to the result.

In conclusion, 
$\Interval$ is some term of $\CTlambda$ which can be safely normalized, but which 
cannot be fully normalized in finite time, not even if we allow
infinite parallel reductions without any "safety" restriction. 
The normal form is produced \emph{only in the limit}
and it is \emph{not regular}. If we allow to reduce infinitely many nested existing
$\beta$-redexes in one step, also
the intermediate steps of the infinite reduction of $\Interval$ are not regular.


%16:32 30/04/2024
%22:30 03/06/2024



\newpage
% Section 5

\section{An example of \quotationMarks{interactive use} of $\CTlambda$}
Our thesis is that we can have two very similar definitions $f_1$, $f_2$ 
of the same map $F:\N,\N \rightarrow \N$
by two regular well-typed terms of $\LAMBDA$. However, $f_1$ does not satisfies the Global
Trace Condition, therefore $f_1 \not \in \CTlambda$ is \emph{not} automatically recognized as total, 
while $f_2$ satisfies the Global Trace condition and is automatically recognized as total.
\\

Why does it happen? The global trace condition for $\CTlambda$ follows the trace of some
type-$\N$variables, for instance it can follow the trace of the input $x^\N$ for $g(x)$, but it cannot
follow the trace of a non-variable $u$ used as input for $g(u)$. This means that for the
global trace algorithm can follow the trace of the local name $x$ of $u$ in $(\lambda x^\N.g(x))(u)$,
while cannot follow the trace of $u$ itself in $g(u)$. If $x$ is infinitely decreasing the global trace algorithm
can deduce termination in the first case, cannot deduce termination in the second case.

In a sense, the $\GTC$-algorithm is \emph{interactive} for $\CTlambda$. When we believe that 
$u:\N$ is infinitely decreasing in every infinite computation, and that this fact is crucial to insure termination,
we should to assign to $u$ a local name $x^\N$. In this way the $\GTC$-algorithm can follow the trace of 
$x^\N$ in any computation. This way of using the global trace algorithm 
is similar to the way of using the Size-Change-Termination algorithm in Neil Jones.


\subsection{The Ackermann Function}
The Ackermann function is a good example if we want to test the global trace 
condition in a case in which it is difficult to prove the convergence of a map. We check that the
global trace condition can prove that a definition of Ackermann function in $\LAMBDA$ 
is convergent, provided we use local names to denote the sub-expressions  ofour definition 
which are infinitely progressing toward $0$.
\\

The Ackermann function $\Ack:\N,\N\rightarrow \N$ is a map whose convergence can be
 proved in$\PA$, but only if we use induction over formulas with at least two unbounded quantifiers. 
For a similar reason, $\Ack$ can be defined in system $\systemT$, but only if we use primitive recursion over
terms whose type has degree $ \ge 2$. $\Ack$ grows faster than all primitive recursive maps.


Here is a fixed point definition of $\Ack$. We will express it in $\CTlambda$ using a nested conditional.
\begin{align*}
  \Ack(0,n) &= n+1
  \\
  \Ack(m+1,0) &= \Ack(m,1)
  \\
  \Ack(m+1,n+1) &= \Ack(m,\Ack(m+1,n))
\end{align*}

We can prove that $\Ack$ is convergent by induction on the lexicographic order on $(m,n)$. 
In the next subsections we try to prove convergence of $\Ack$ using $\GTC$.


\subsection{A Term Representation of Ackermann \emph{without} the Global Trace Condition}

We give a first representation $\AckA \in \LAMBDA$ of the Ackermann function. 
In this article, we sometimes abbreviate $f(x)(y)$ with $f(x,y)$, in this way we also improve readability.

\begin{definition}[$\AckA$]
  We define $\AckA \in \LAMBDA$ by the following infinite regular term .
  \[
  \AckA = \lambda \redM\bluen.\Cond{\Suc{\bluen}}{\lambda m'.\Cond{\AckA(m',1)}{\lambda n'.\AckA(m',\AckA(\Suc{m'},n'))}(\bluen)}(\redM)
  \]
\end{definition}

We check that $\AckA$ almost-left-finite. 
Any infinite path in $\AckA$ must move infinitely many times from $\AckA$ to $\AckA$, and whenever we 
do so we move to the right-hand-side of a $\cond$. Therefore any infinite path in $\AckA$ is 
not almost-leftmost, by taking the contrapositive we prove that all almost-leftmost paths in $\AckA$
are finite. $\AckA$ has (unique) type $\N,\N\rightarrow \N$.
\\

\noindent{\bf Checking the fixed point equation for $\AckA$}
\begin{align*}
  \AckA(0,n)
  &\mapsto
  \Cond{\Suc{n}}{\lambda m'.\Cond{\AckA(m',1)}{\lambda n'.\AckA(m',\AckA(\Suc{m'},n'))}(n)}(0)
  \\
  &\mapsto
  \Suc{n}
\end{align*}
%%%%%%%
\begin{align*}
  \AckA(\Suc{m},0)
  &\mapsto
  \Cond{\Suc{0}}{\lambda m'.\Cond{\AckA(m',1)}{\lambda n'.\AckA(m',\AckA(\Suc{m'},n'))}(0)}(\Suc{m})
  \\
  &\mapsto
  (\lambda m'.\Cond{\AckA(m',1)}{\lambda n'.\AckA(m',\AckA(\Suc{m'},n'))}(0))(m)
  \\
  &\mapsto
  \Cond{\AckA(m,1)}{\lambda n'.\AckA(m',\AckA(\Suc{m'},n'))}(0)
  \\
  &\mapsto
  \AckA(m,1)
\end{align*}
%%%%%%%
\begin{align*}
  \AckA(\underline{\Suc{m}},\Suc{n})
  &\mapsto
  \Cond{\Suc{\Suc{n}}}{\lambda m'.\Cond{\AckA(m',1)}{\lambda n'.\AckA(m',\AckA(\Suc{m'},n'))}(\Suc{n})}(\underline{\Suc{m}})
  \\
  &\mapsto
  (\lambda m'.\Cond{\AckA(m',1)}{\lambda n'.\AckA(m',\AckA(\Suc{m'},n'))}(\Suc{n}))(\underline{m})
  \\
  &\mapsto
  \Cond{\AckA(m,1)}{\lambda n'.\AckA(m,\AckA(\Suc{\underline{m}},n'))}(\Suc{n})
  \\
  &\mapsto
  (\lambda n'.\AckA(m,\AckA(\Suc{\underline{m}},n')))(n)
  \\
  &\mapsto
  \AckA(m,\AckA(\Suc{\underline{m}},n))
\end{align*}


The global trace algorithm will fail to prove that $\AckA$ is terminating. The reason is in the last case. 
The term $\underline{\Suc{m}}$ of the first line and $\Suc{\underline{m}}$ of the last line are slightly 
different: $\underline{\Suc{m}}$ is decomposed into $\underline{m}$ by the $\text{cond}$-reduction, 
then $\Suc{\underline{m}}$ is constructed by substituting $m'$ of $\Suc{m'}$ by $\underline{m}$. 
When we loop from the rightmost occurrence of $\AckA$ 
in the picture above back to the root of $\AckA$, the input $\Suc{m}$ of $\AckA$ is sent without any change to the first argument of $\AckA$. The globale trace algorithm fails to notice this: 
in order to notice it, the two terms $\Suc{m}$ should have the same local name $\redM$.
As a consequence, the trace of $\Suc{m}$ is cut in this point and some infinitely progressing trace
disappears.



\subsection{$\AckA:\N\to\N\to\N$ is not in $\GTC$}

\begin{claim}
  $\AckA:\N\to\N\to\N$ is well-typed, but $\AckA \not \in \GTC$.
\end{claim}

In the following, weakening is implicitly applied. 

\begin{center}

{\scriptsize
  \hspace{-3cm}
  $\infer{
    \vdash \AckA:\redN\to\goldN\to\N\ (\dagger)
  }{
    \infer[\Rapv]{
      m:\redN, n:\goldN \vdash \Cond{\Suc{n}}{\lambda m'.\Cond{\AckA(m',1)}{\lambda n'.\AckA(m',\AckA(\Suc{m'},n'))}(n)}(m): \N
    }{
      \infer[\Rcond]{
        m:\redN, n:\goldN \vdash \Cond{\Suc{n}}{\lambda m'.\Cond{\AckA(m',1)}{\lambda n'.\AckA(m',\AckA(\Suc{m'},n'))}(n)}: \redN\to\N
      }{
        \infer{
          n:\goldN \vdash \Suc{n}: \N
        }{
          \infer{
            n:\goldN \vdash n: \N
          }{}
        }
        &
        \infer{
          n:\goldN \vdash \lambda m'.\Cond{\AckA(m',1)}{\lambda n'.\AckA(m',\AckA(\Suc{m'},n'))}(n): \redN\to\N
        }{
          \infer[\Rapv]{
            n:\goldN, m':\redN \vdash \Cond{\AckA(m',1)}{\lambda n'.\AckA(m',\AckA(\Suc{m'},n'))}(n): \N
          }{
            \infer[\Rcond]{
              n:\goldN, m':\redN \vdash \Cond{\AckA(m',1)}{\lambda n'.\AckA(m',\AckA(\Suc{m'},n'))}: \goldN\to\N
            }{
              \infer{
                m':\redN \vdash \AckA(m',1):\N
              }{
                \infer[\Rapv]{
                  m':\redN \vdash \AckA(m'):\N\to\N
                }{
                  \infer{
                    m':\redN \vdash \AckA:\redN\to\N\to\N
                  }{
                    \deduce{
                      \vdash \AckA:\redN\to\N\to\N
                    }{(\dagger1)}
                  }
                }
                &
                \infer{
                  \vdash 1:\N
                }{}
              }
              &
              \infer{
                m':\redN \vdash \lambda n'.\AckA(m',\AckA(\Suc{m'},n')): \goldN\to\N
              }{
                \infer{
                  m':\redN, n':\goldN \vdash \AckA(m',\AckA(\Suc{m'},n')): \N
                }{
                  \infer[\Rapv]{
                    m':\redN \vdash \AckA(m'): \N\to\N
                  }{
                    \infer{
                      m':\redN \vdash \AckA: \redN\to\N\to\N
                    }{
                      \deduce{
                        \vdash \AckA: \redN\to\N\to\N
                      }{(\dagger2)}
                    }
                  }
                  &
                  \infer[\Rapv]{
                    m':\redN, n':\goldN \vdash \AckA(\Suc{m'},n'): \N
                  }{
                    \infer[\RapNv]{
                      m':\redN, n':\goldN \vdash \AckA(\Suc{m'}): \goldN\to\N
                    }{
                      \deduce{
                        \vdash \AckA:\N\to\goldN\to\N
                      }{
                        (\dagger3)
                      }
                      &
                      \infer{
                        m':\redN \vdash \Suc{m'}:\N
                      }{
                        \infer{
                          m':\redN \vdash m':\N
                        }{}
                      }
                    }
                  }
                }
              }
            }
          }
        }
      }
    }
  }$  
}

\end{center}

The problem is with the rightmost rule $\RapNv$ in the picture above. 
The same value $\Suc{m'} = m$ is send to the first argument of $\AckA$, but this fact 
is not reported in the trace.
This cuts the trace chasing of $\redN$ (red color). A similar situation arises with the leftmost rule
$\RapNv$  in the picture above: the trace chasing of $\goldN$ (gold color) is cut.

As a consequence, the infinite path $(\dagger)\rightsquigarrow(\dagger1)\rightsquigarrow(\dagger)
\rightsquigarrow(\dagger3)\rightsquigarrow(\dagger)\rightsquigarrow(\dagger1)
\rightsquigarrow(\dagger)\rightsquigarrow(\dagger3)\rightsquigarrow\cdots$ 
does not contains any infinite trace, and with more reason any infinitely progressing trace. 



\subsection{Second term representation: $\AckB$}

We give the second representation $\AckB$ of Ackermann function. In this case we inform
the global trace algorithm that the local term $\Suc{m'}$ is equal to $m$ in any computation
by writing $m$ in the place of $\Suc{m'}$.

\begin{definition}[$\AckB$]\redM
  We define $\AckB$ by the following infinite regular term of $\LAMBDA$.
  \[
  \AckB = \lambda \redM\bluen.
\Cond{\Suc{\bluen}}{\lambda m'.\Cond{\AckB(m',1)}{\lambda n'.\AckB(m',\AckB(\redM,n'))}(\bluen)}(\redM)
  \]
\end{definition}

Again, the term $\AckB$ is almost-left-finite and has type $\N, \N \rightarrow \N$.
\\

\noindent{\bf Checking the fixed point equation for $\AckB$}

We check only the last case. 
\begin{align*}
  \AckB(\underline{\Suc{m}},\Suc{n})
  &\mapsto
  \Cond{\Suc{\Suc{n}}}{\lambda m'.\Cond{\AckB(m',1)}{\lambda n'.\AckB(m',\AckB(\underline{\Suc{m}},n'))}(\Suc{n})}(\underline{\Suc{m}})
  \\
  &\mapsto
  (\lambda m'.\Cond{\AckB(m',1)}{\lambda n'.\AckB(m',\AckB(\underline{\Suc{m}},n'))}(\Suc{n}))(\underline{m})
  \\
  &\mapsto
  \Cond{\AckB(m,1)}{\lambda n'.\AckB(m,\AckB(\underline{\Suc{m}},n'))}(\Suc{n})
  \\
  &\mapsto
  (\lambda n'.\AckB(m,\AckB(\underline{\Suc{m}},n')))(n)
  \\
  &\mapsto
  \AckB(m,\AckB(\underline{\Suc{m}},n))
\end{align*}

The point of $\AckB$ is that the global trace algorithm by checking the
traces discover that $\underline{\Suc{m}}$ 
at the first line and the one at the last line are exactly the same, thanks to the fact that
they have the same local name $m$.




\subsection{$\AckB:\N\to\N\to\N$ is in $\GTC$}

\begin{proposition}
  $\AckB:\N\to\N\to\N$ has a proof $\Pi$ that satisfies $\GTC$.
\end{proposition}

\begin{proof}
Below is a proof $\Pi$ of $\AckB:\N\to\N\to\N$. 
In the following, weakening is implicitly applied. 


\begin{flushright}

{\scriptsize
  \hspace{-3cm}
  $\infer{
    \vdash \AckB:\redblueN\to\goldN\to\N\ (\dagger)
  }{
    \infer[\Rapv]{
      m:\redblueN, n:\goldN \vdash \Cond{\Suc{n}}{\lambda m'.\Cond{\AckB(m',1)}{\lambda n'.\AckB(m',\AckB(m,n'))}(n)}(m): \N
    }{
      \infer[\Rcond]{
        m:\redN, n:\goldN \vdash \Cond{\Suc{n}}{\lambda m'.\Cond{\AckB(m',1)}{\lambda n'.\AckB(m',\AckB(m,n'))}(n)}: \blueN\to\N
      }{
        \infer{
          n:\goldN \vdash \Suc{n}: \N
        }{
          \infer{
            n:\goldN \vdash n: \N
          }{}
        }
        &
        \infer{
          m:\redN, n:\goldN \vdash \lambda m'.\Cond{\AckB(m',1)}{\lambda n'.\AckB(m',\AckB(m,n'))}(n): \blueN\to\N
        }{
          \infer[\Rapv]{
            m:\redN, n:\goldN, m':\blueN \vdash \Cond{\AckB(m',1)}{\lambda n'.\AckB(m',\AckB(m,n'))}(n): \N
          }{
            \infer[\Rcond]{
              m:\redN, n:\goldN, m':\blueN \vdash \Cond{\AckB(m',1)}{\lambda n'.\AckB(m',\AckB(m,n'))}: \goldN\to\N
            }{
              \infer{
                m':\blueN \vdash \AckB(m',1): \N
              }{
                \infer[\Rapv]{
                  m':\blueN \vdash \AckB(m'): \N\to\N
                }{
                  \infer{
                    m':\blueN \vdash \AckB: \blueN\to\N\to\N
                  }{
                    \deduce{
                      \vdash \AckB: \blueN\to\N\to\N
                    }{(\dagger1)}
                  }
                }
                &
                \infer{
                  \vdash 1:\N
                }{}
              }
              &
              \infer{
                m:\redN, m':\blueN \vdash \lambda n'.\AckB(m',\AckB(m,n')): \goldN\to\N
              }{
                \infer{
                  m:\redN, m':\blueN, n':\goldN \vdash \AckB(m',\AckB(m,n')): \N
                }{
                  \infer[\Rapv]{
                    m':\blueN \vdash \AckB(m'): \N\to\N
                  }{
                    \infer{
                      m':\blueN \vdash \AckB: \blueN\to\N\to\N
                      }{
                      \deduce{
                        \vdash \AckB: \blueN\to\N\to\N
                      }{(\dagger2)}
                    }
                  }
                  &
                  \infer[\Rapv]{
                    m:\redN, n':\goldN \vdash \AckB(m,n'): \N
                  }{
                    \infer{
                      m:\redN, n':\goldN \vdash \AckB(m): \goldN\to\N
                    }{
                      \infer[\Rapv]{
                        m:\redN \vdash \AckB(m): \goldN\to\N
                      }{
                        \infer{
                          m:\redN \vdash \AckB: \redN\to\goldN\to\N
                        }{
                          \deduce{
                            \vdash \AckB: \redN\to\goldN\to\N
                          }{(\dagger3)}
                        }
                      }
                    }
                  }
                }
              }
            }
          }
        }
      }
    }
  }$
}

\end{flushright}

\vspace{1cm}

We check that this proof $\Pi$ satisfies the global trace condition.
We first note that:

\begin{itemize}
\item
  The path $(\dagger)\rightsquigarrow(\dagger1)$ contains a progressing blue trace $\tau_1 = (\redblueN,\redblueN,\blueN,\ldots,\blueN)$. The gold trace is cut in $(\dagger1)$.
\item
  The path $(\dagger)\rightsquigarrow(\dagger2)$ contains a progressing blue trace $\tau_2 = (\redblueN,\redblueN,\blueN,\ldots,\blueN)$. The gold trace is cut in $(\dagger2)$.
\item
  The path $(\dagger)\rightsquigarrow(\dagger3)$ contains a progressing gold trace 
$\tau'_3 = (\goldN,\ldots,\goldN)$ and a non-progressing red trace 
$\tau_3 = (\redblueN,\redblueN,\redN,\ldots,\redN)$. 
\end{itemize}

Each trace $\tau_i$, blue or red, can be composed with each trace $\tau_j$. 

The first steps of each $\tau_i$ are in common and we write them with the blue and red colors superposed.
Then the blue and the red traces fork.

Take an infinite path $\pi$ from this proof $\Pi$ 
in order to prove that it includes some infinitely progressing trace.

\begin{enumerate}
\item
If $\pi$ passes through $(\dagger1)$ or $(\dagger2)$ infinitely many times,
we take its infinitely progressing trace by combining $\tau_1$, $\tau_2$, and $\tau_3$,
according if we pass through$(\dagger1)$ or $(\dagger2)$ or $(\dagger3)$. 

\item
If $\pi$ passes through $(\dagger1)$ and $(\dagger2)$ only finitely many times,
namely it eventually becomes a loop of $(\dagger)$ and $(\dagger3)$,
from this point on we take its infinitely progressing trace by repeating $\tau'_3$ for infinitely many times.
\end{enumerate}

\end{proof}



  
%\end{document}
%
%\infer{}{}
%
%\lambda mn.\Cond{\Suc{n}}{\lambda m'.\Cond{\Ack(m')(1)}{\lambda n'.\Ack(m')(\Ack(\Suc{m'})(n'))}(n)}(m)


\newpage
% Section 6

\newcommand{\xx}{\boldsymbol{x}}

\section{Subject Reduction for Well-Typed Infinite Lambda Terms}
\label{section-subject-reduction}

We show the subject reduction for well-typed terms of $\LAMBDA$,
and also show the global trace condition is preserved by reductions. 
We first introduce some auxiliary notations for proofs of them.

Let $X$ be a set of variables.
We write $\Gamma_X$ be $\{x^T:T \in \Gamma \mid x^T \in X \}$.
Let $\Gamma$ and $\Delta$ be contexts of $\LAMBDA$.
We say that $\Gamma$ and $\Delta$ are {\em consistent} if 
$\Gamma_{\FV(\Gamma)\cap\FV(\Delta)} \sim \Delta_{\FV(\Gamma)\cap\FV(\Delta)}$. 
The merged context $\mergeCtx{\Gamma}{\Delta}$
is defined by $\Gamma\conc\Delta_{\FV(\Delta)\setminus\FV(\Gamma)}$
if $\Gamma$ and $\Delta$ are consistent, and is undefined otherwise. 

Let $\theta$ be a renaming, and 
$S_1 = \Gamma_1\vdash t_1:\vec{B_1}\rightarrow N$
and $S_2 = \Gamma_2[\theta]\vdash t_2[\theta]:\vec{B_2}\rightarrow N$ be sequents. 
Let $k_1$ and $k_2$ be indexes of $N$-arguments of $t_1$ and $t_2$, respectively. 
Then we write $(k_1,S_1) \simIndex{\theta} (k_2,S_2)$ (or $(k_1,t_1) \simIndex{\theta} (k_2,t_2)$ for short)
if $k_1$ and $k_2$ are respectively indexes of unnamed arguments at the same position
in $\vec{B_1}$ and $\vec{B_2}$, or they are respectively those of named arguments for some $y$ and $\theta(y)$. 
Note that the index equivalent to $k_1$ is unique (if it exists), namely 
$(k_1,t_1) \simIndex{\theta} (k_2,t_2)$ and $(k_1,t_1) \simIndex{\theta} (k'_2,t_2)$ implies $k_2=k'_2$.
We write $(k_1,t_1) \simIndex{} (k_2,t_2)$ if $\theta$ is the identity renaming. 

Due to our assumption of not implicit identifying $\alpha$-equivalent terms,
proofs of this section requires some delicate treatment of variables. 
We consider the following restricted $\lambda$ rule (called $\lambda'$) as follows:
\begin{itemize}
\item
  $\lambda'$-rule.
  If $\Gamma, x^A:A \vdash b: B$ and $\FV(\Gamma) = \FV(\lambda x.b)$, 
  then $ \Gamma \vdash \lambda x^A.b :A \rightarrow B$.
\end{itemize}

The next lemma says that if $\Gamma\vdash t:A$ is provable,
then $t:A$ can be shown in a restricted context $\Gamma_{\FV(t)}$
even if the rule $\lambda$ is restricted to $\lambda'$. 

\begin{lemma}\label{lem:thinning}
  Assume $\Pi:\Gamma\vdash t:A$.
  Then $\Pi':\Gamma_{\FV(t)}\vdash t:A$ for some $\Pi'$ that have $\lambda'$ and not have $\lambda$.
  Moreover if the global trace condition holds for $\Pi$, then it also holds for $\Pi'$. 
\end{lemma}
\begin{proof}
  First we call the following admissible rule $\lambda'\weak$:
  \begin{itemize}
  \item
    $\lambda'\weak$-rule.
    If $\Gamma, x^A:A \vdash b: B$, $\FV(\Gamma) = \FV(\lambda x.b)$ and $\Gamma\subseteqsim \Gamma'$, 
    then $ \Gamma' \vdash \lambda x^A.b :A \rightarrow B$.
  \end{itemize}
  We write $\Rule'$ as the set of rule instances obtained by removing
  those of $\lambda$ from $\Rule$ and adding those of $\lambda'\weak$. 
  Note that if we have a proof of a sequent with $\lambda'\weak$,
  then we also have a proof of the same sequent not with $\lambda'\weak$
  by a proof transformation that splits each $\lambda'\weak$ by $\lambda'$ and $\weak$. 
  Also note that this proof transformation preserves the global trace condition.

  Let $\Pi$ be $(T,\phi)$.
  For each $l \in T$, we write $\Gamma_l\vdash t_l:A_l$
  for the conclusion of $\phi(l) \in \Rule$. 
  To show the lemma, from a given proof $\Pi$, 
  it is enough to construct a proof $\Pi'$ of $\Restrict{\Gamma}{\FV(t)}\vdash t:A$ with $\Rule'$.
  
  We define the set of nodes of $\Pi'$ is $T$, which is the same one of $\Pi$. 
  For each $l\in T$, we define $\phi'(l)$ and $\Gamma'_l$ that satisfies
  the following requirements:
  \begin{itemize}
  \item[(a)]
    $\Gamma'_l\vdash t_l:A_l$ is the conclusion of $\phi'(l) \in \Rule'$,
  \item[(b)]
    $\Restrict{(\Gamma_l)}{\FV(t_l)} \subseteqsim \Gamma'_l \subseteqsim \Gamma_l$
    and $\Gamma'_{\nil} = \Restrict{\Gamma}{\FV(t)}$, 
  \item[(c)]
    if $\tilde{k},t_{l\conc(i)}$ is the successor of $k,t_l$ in $\Pi$
    and $k$ is an index of some unname argument
    or a named argument of some name $z\in\FV(t_l)$, 
    then there are $\tilde{k'}$ and $k'$ such that
    $\tilde{k'},t_{l\conc(i)}$ is the successor of $k',t_l$ in $\Pi'$, 
    $(k,t_l) \simIndex{} (k',t_l)$,
    and $(\tilde{k},t_{l\conc(i)}) \simIndex{} (\tilde{k'},t_{l\conc(i)})$.
    Moreover, if $k$ is an index of a progressing argument, then so is $k'$.
  \end{itemize}
  Note that the proof is done if we construct $\Pi'$ that satisfies these requirements. 
  
  We define $\Gamma'_{\nil} = \Restrict{\Gamma}{\FV(t)}$.
  Then it satisfies (b) since $t_\nil = t$. 
  Next, assuming the induction hypothesis that $\Gamma'_l$ which satisfies (b)
  is already defined, 
  we define $\phi'(l)$ and $\Gamma_{l\conc(i)}$, 
  for each $i$ such that $l\conc(i)\in T$, 
  that satisfies (a), (b), and (c). 
  It is done by the case analysis of $\phi(l)$.

  The case of $\phi(l) = \Gamma_l\vdash x:A_l$,
  which is an instance of the rule $\var$ with $x:A_l\in\Gamma_l$. 
  Then define $\phi'(l) = \Gamma'_l\vdash x:A_l$. This is an instance of $\var$
  because $x:A_l \in \Restrict{(\Gamma_l)}{\FV(x)} \subseteqsim \Gamma'_l$ by (b).

  The case of $\phi(l) = (\Gamma_{l\conc(1)}\vdash t_l:A_l,\Gamma_{l}\vdash t_l:A_l)$,
  which is an instance of the rule $\weak$ with
  $t_{l\conc(1)} = t_l$, $A_{l\conc(1)} = A_{l}$, and
  $\Gamma_{l\conc(1)}\subseteqsim \Gamma_l$.
  Let $\psi$ be the unique map that determines $\Gamma_{l\conc(1)}\subseteqsim \Gamma_l$.
  Then define $\Gamma'_{l\conc(1)}$ such that
  $\Gamma'_{l\conc(1)}\subseteqsim \Gamma'_l$ determined by
  the induced map from $\psi$ restricting the range to $\FV(\Gamma'_l)$.
  Note that $\FV(\Gamma'_{l\conc(1)}) = \FV(\Gamma'_l)\cap\FV(\Gamma_{l\conc(1)})$. 
  Then it satisfies (b) since $\FV(t_l)\subseteq \FV(\Gamma'_l) \cap \FV(\Gamma_{l\conc(1)}) = \FV(\Gamma'_{l\conc(1)}) \subseteq \FV(\Gamma_{l\conc(1)})$ by (b) for $\Gamma'_l$.
  Define $\phi'(l) = (\Gamma'_{l\conc(1)}\vdash t_l:A_l,\Gamma'_{l}\vdash t_l:A_l)$
  as an instance of the rule $\weak$. 
  Hence the requirement (a) holds.
  We can also show (c): if $k$ is an index in $\Gamma_l\vdash t_l:A_l$ for a name
  $z\in\FV(t_{l\conc(1)}) = \FV(t_l)$, then we can take an index $k'$
  in $\Gamma'_l\vdash t_l:A_l$ for $z$ by $\FV(t_l)\subseteq\FV(\Gamma'_l)$ by (b). 
  An index $\tilde{k'}$ for the name $z$ can be taken
  from $\Gamma'_{l\conc(1)}\vdash t_l:A_l$
  since $z \in \FV(t_l) \subseteq \FV(\Gamma'_{l\conc(1)})$. 
  
  The case of $\phi(l) = (\Gamma_l,z:C\vdash b:\vec{B}\to N, \Gamma_l\vdash\lambda z^C.b:C,\vec{B}\to N)$
  that is an instance of the rule $\lambda$. 
  Define $\Gamma'_{l\conc(1)} = \Restrict{(\Gamma_l)}{\FV(\lambda z.b)},z:C$ and 
  $\phi'(l) = (\Restrict{(\Gamma_l)}{\FV(\lambda z.b)},z:C\vdash b:\vec{B}\to N, \Gamma'_l\vdash \lambda z.b:C,\vec{B}\to N)$. 
  This is an instance of the rule $\lambda'\weak$, since $\Restrict{(\Gamma_l)}{\FV(\lambda z.b)}\subseteqsim \Gamma'_l$ by the induction hypothesis.
  Trivially we have (a). 
  We also have (b) by $\Restrict{(\Gamma_l,z:C)}{\FV(b)} \subseteqsim (\Restrict{(\Gamma_l)}{\FV(\lambda z.b)},z:C) \subseteqsim (\Gamma_l,z:C)$.   
  The requirement (c) holds: 
  if $k$ is an index of some named argument $y\in\FV(\lambda z.b)$ in $\Gamma_l$,
  then $k'$ can be taken as the index of $y$ in $\Gamma'_l$ by (b) for $\Gamma_l$.
  If $k$ is an index of some unnamed argument in $C,\vec{B}$,
  then $k'$ can be taken as the index of some unnamed argument. 
  In both cases, their successors $\tilde{k}$ and $\tilde{k'}$ are uniquely
  determined by $k$ and $k'$, respectively. 
  
  The case that $\phi(l)$ is an instance of the rule $\apvar$
  whose conclusion is $\Gamma_l\vdash f(x^B):\vec{C}\to N$ with $x:B\in \Gamma_l$.  
  Define $\Gamma'_{l\conc(1)} = \Gamma'_l$.
  This satisfies (b) by the induction hypothesis. 
  Then define
  $\phi'(l) = (\Gamma'_{l\conc(1)}\vdash f:B,\vec{C}\to N, \Gamma'_l\vdash f(x^B):\vec{C}\to N)$
  as an instance of the rule $\apvar$. This satisfies (a). 
  In order to show (c), 
  take an index $k$ for $\Gamma_l\vdash f(x^B):\vec{C}\to N$, which is the one for
  a named argument $y \in \FV(f(x^B))$ in $\Gamma_l$
  or for some unnamed argument of $\vec{C}$. 
  For the latter case, the indexes $\tilde{k}$, $k'$ and $\tilde{k'}$ can be taken
  as those of the unnamed argument in $\vec{C}$ at the same position as $k$.
  For the former case, we take $k'$ as the index for $y$ in $\Gamma'_l$.
  In order to take $\tilde{k'}$,
  we further have two subcases according to $\tilde{k}$: 
  The first one is $\tilde{k}$ is the index for the same named argument as $k$, 
  and the second one is $\tilde{k}$ is the index for the unnamed argument, namely $B=N$.
  In both subcases, we can take the equivalent $\tilde{k'}$ to $\tilde{k}$ as we wished.
    
  The case that $\phi(l)$ is an instance of the rule $\apnotvar$
  whose conclusion is $\Gamma_l\vdash f(b^B):A_l$. 
  Define $\Gamma'_{l\conc(1)} = \Gamma'_{l\conc(2)} = \Gamma'_l$.
  This satisfies (b) by the induction hypothesis.
  Then define $\phi'(l) = (S_1, S_2, \Gamma'_l\vdash f(b^B):\vec{C}\to N)$,
  where $S_1$ is $\Gamma'_{l}\vdash f:B,\vec{C}\to N$
  and $S_2$ is $\Gamma'_{l}\vdash b:B$, as an instance of $\apnotvar$.
  This satisfies (a).
  In order to show (c), 
  take an index $k$ for $\Gamma_l\vdash f(x^B):\vec{C}\to N$, which is the one for
  a named argument $y \in \FV(f(x^B))$ in $\Gamma_l$
  or for some unnamed argument of $\vec{C}$.
  For the former case, the successor $\tilde{k}$ of $k$ is uniquely taken.
  By $\FV(f(b)) \subseteq \FV(\Gamma'_l)$, the index $k'$ equivalent to $k$ is uniquely taken. 
  Then the successor $\tilde{k'}$ of $k'$ is also taken as we wished. 
  For the latter case, the successor $\tilde{k}$ of $k$ is uniquely taken
  as the one for unnamed argument at the same position as $k$. 
  Then $k'$ equivalent to $k$ is uniquely taken, and 
  its successor $\tilde{k'}$ is also taken as we wished. 

  The case that $\phi(l)$ is an instance of the rule $0$
  whose conclusion is $\Gamma_l\vdash 0:N$. 
  Define $\phi'(l) = \Gamma'_l\vdash 0:N$ as an instance of the rule $0$.
  This satisfies the requirements (a), (b), and (c). 

  The case that $\phi(l)$ is an instance of the rule $\Succ$
  whose conclusion is $\Gamma_l\vdash \Succ(t_{l\conc(1)}):N$ with $A_l=N$.
  Define $\Gamma'_{l\conc(1)} = \Gamma'_l$, and 
  $\phi'(l) = (\Gamma'_l\vdash t_{l\conc(1)}:N, \Gamma'_l\vdash \Succ(t_{l\conc(1)}):N)$
  as an instance of $\Succ$. They satisfy (a) and (b). 
  The requirement (c) is also satisfied:
  Take an index $k$ for $\Gamma_l\vdash \Succ(t_{l\conc(1)}): N$,
  which is the one for a named argument $y \in \FV(\Succ(t_{l\conc(1)})$ in $\Gamma_l$.
  Then the successor $\tilde{k}$ of $k$ is uniquely taken.  
  By $\FV(t_{l\conc(1)}) \subseteq \FV(\Gamma'_l)$, the index $k'$ equivalent to $k$ is uniquely taken. 
  Then the successor $\tilde{k'}$ of $k'$ is also taken as we wished. 
  
  The case that $\phi(l)$ is an instance of the rule $\cond$
  whose conclusion is $\Gamma_l\vdash \cond(f,g):N,\vec{C}\to N$. 
  Define $\Gamma'_{l\conc(1)} = \Gamma'_{l\conc(2)} = \Gamma'_l$, and 
  $\phi'(l) = (S_1,S_2, \Gamma'_l\vdash \cond(f,g):N,\vec{C}\to N)$,
  where $S_1$ is $\Gamma'_l\vdash f:\vec{C}\to N$ and $S_2$ is $\Gamma'_l \vdash g:N,\vec{C}\to N$, 
  as an instance of $\cond$. They satisfy (a) and (b). 
  To show (c), take an index $k$ for $\Gamma_l\vdash \cond(f,g): N,\vec{C}\to N$ as required. 
  We need to consider three cases:
  $k$ is the index for a named argument $y \in \FV(\cond(f,g))$ in $\Gamma_l$,
  is the one for an unnamed argument in $\vec{C}$, or
  is the one for an unnamed argument in $N$ (the first one of $N,\vec{C}\to N$). 
  For the first and second cases, we can take $\tilde{k}$, $k'$, and $\tilde{k'}$
  similar to the other cases.
  For the last case, the successor $\tilde{k}$ of $k$ should be taken
  as the index of the unnamed $N$-argument of $\Gamma'_l \vdash g:N,\vec{C}\to N$
  at the same position as $k$. 
  Then $k'$ and $\tilde{k'}$ are taken as those equivalent to $k$ and $\tilde{k}$, respectively.
  Note that this $k$ is an index of a progressing argument $N$, and so is $k'$. 
\end{proof}



\begin{lemma}[Substitution lemma]
  Assume that $\Pi_u: \Delta \vdash u:A$ and $\Pi_t:\Gamma,x:A \vdash t:B$ hold,
  and $\Delta$ and $\Gamma$ are consistent. 
  Then there exists $\Pi^\circ$ such that $\Pi^\circ:\mergeCtx{\Delta}{\Gamma} \vdash t[u/x]:B$. 
  Moreover, if both $\Pi_u$ and $\Pi_t$ satisfy the global trace condition,
  then $\Pi^\circ$ also satisfies it. 
\end{lemma}
\begin{proof}
  Assume that $\Pi_u: \Delta \vdash u:A$ and $\Pi_t:\Gamma,x:A \vdash t:B$ hold,
  and $\Delta$ and $\Gamma$ are consistent. 
  By Lemma~\ref{lem:thinning}, without loss of generality,
  we can assume $\FV(\Delta) = \FV(u)$ and $\Pi_t$ is a proof with the restricted $\lambda$ rule (the $\lambda'$ rule). 
  Let $\Pi_u=(T_u,\phi_u)$ and $\Pi_t=(T_t,\phi_t)$.
  For each $l\in T_t$, we write $\Gamma_l\vdash t_l:C_l$
  for the conclusion of $\phi_t(l)$.
  We use $\xx$ to mark occurences of $x$ in $\Pi_t$ that connects
  with the explicit $\xx$ of $\Gamma,\xx:A\vdash t:B$. 
  In the following, we construct a proof $\Pi^\circ = (T^\circ,\phi^\circ)$ of
  $\mergeCtx{\Delta}{\Gamma}\vdash t[u/x]:B$ such that 
  \begin{itemize}
  \item[(a1)]
    $T^\circ = T_t \cup T_{\var} \cup T_{\apvar}$, where
    $T_{\var} = \{l\conc(1)\conc l' \mid \text{$l\in T_t$, $\phi(l)=\var$ of $\xx$, and $l'\in T_u$} \}$
    and
    $T_{\apvar} = \{l\conc(2)\conc l' \mid \text{$u$ is not a variable, $l\in T_t$, $\phi(l)=\apvar$ of $f(\xx)$ for some $f$, and $l'\in T_u$}\}$.
    The explicit $1$ of $l\conc(1)\conc l' \in T_\var$ is called the switching point of $l\conc(1)\conc l'$.
    The explicit $2$ of $l\conc(2)\conc l' \in T_\apvar$ is also called the switching point of $l\conc(2)\conc l'$.     \item[(a2)]
    For each $l\conc(i)\conc l' \in T_\var \cup T_\apvar$ with switching point $i$,
    we have $\phi^\circ(l\conc(i)\conc l') = \phi_u(l')$. 
  \end{itemize}
  Moreover, for all $l \in T_t$,
  the rule instance $\phi^\circ(l)$ satisfies the following requirements
  with an auxiliary function $\sigma^\circ:T_t \to \Seq$ and a substitution $\theta_l$: 
  \begin{itemize}
  \item[(b1)]
    The sequent $\sigma^\circ(l)$ has the form $\Gamma^\circ_l\vdash t_l[\theta_l]:C_l$.
    The substitution $\theta_l$ is $\{u/\xx\}\cup\theta_{{\rm ren}}$
    if $\xx:A \in \Gamma_l$, and is $\theta_{{\rm ren}}$ otherwise, 
    where $\theta_{{\rm ren}}$ is some renaming.
    $\Gamma^\circ_l \sim \Delta_l\sharp\Restrict{(\Gamma_l)}{\overline{\xx}}[\theta_l]$ holds,
    where $\Delta_l$ is $\Delta$ if $\xx:A\in\Gamma_l$, and is $\emptyset$ otherwise. 
  \item[(b2)]
    $\sigma^\circ(l)$ is the conclusion of $\phi^\circ(l)$,
    and $\sigma^\circ(l\conc(i))$ is the $i$-th assumption of $\phi^\circ(l)$ if $l\conc(i) \in T_t$.
  \item[(b3)]
    Assume $\tilde{k},t_{l\conc(i)}$ is a successor of $k,t_l$ and $k$ is not a named index for $\xx$. 
    Then there exist $\tilde{k^\circ}$ and $k^\circ$ such that
    $\tilde{k^\circ},t_{l\conc(i)}[\theta_{l\conc(i)}]$ is a successor of $k^\circ,t_l[\theta_l]$ and 
    $(k,t_l) \simIndex{\Restrict{(\theta_l)}{\overline{\xx}}} (k^\circ,t_l[\theta_l])$.
    If $k$ is an index of a progressing argument, then so is $k^\circ$.    
  \end{itemize}
  
  Note that if we have $\Pi^\circ$ that satisfies the requirements, its possibly infinite path
  is a path of $\Pi_t$ or a path of $\Pi_u$ or a path $l\conc(i)\conc l'$,
  where $l$ is a path of $T_t$, $i$ is a switching point, and $l'$ is a path of $T_u$.
  Hence $\Pi^\circ$ is Almost-left-finite, since $\Pi_t$ and $\Pi_u$ are Almost-left-finite.
  In addition, if $\Pi_t$ and $\Pi_u$ satisfies the global trace condition,
  so is $\Pi^\circ$ by (b3). 
  Therefore, to complete the proof, it is enough to construct $\Pi^\circ$. 

  First, for each $l\conc(j)\conc l'\in T_\var\cup T_\apvar$ with a switching point $j$, 
  we define $\phi^\circ(l\conc(j)\conc l') = \phi_u(l')$. 

  In the following, for each $l\in T_t$, we inductively define $\Gamma^\circ_l$, $\phi^\circ(l)$, and $\sigma(l)$
  that satisty (b1), (b2), and (b3).
  The case of $l=\nil$, define $\Gamma^\circ_\nil = \Delta\sharp\Gamma$ and $\theta_\nil=\{u/\xx\}$,
  and define $\sigma^\circ(\nil)$ as $\mergeCtx{\Delta}{\Gamma}\vdash t[u/\xx]:A$
  using the consistency of $\Delta$ and $\Gamma$ by the assumption. 
  Hence we have (b1) since
  $\Gamma^\circ_\nil = \mergeCtx{\Delta}{\Gamma} \sim \mergeCtx{\Delta}{\Restrict{(\Gamma,\xx:A)}{\overline{\xx}}[\theta_\nil]}$. 

  For each $l\in T_t$,
  with the induction hypothesis that $\sigma^\circ(l)$ and $\theta_l$ that satisfy (b1) are already defined, 
  we define $\phi^\circ(l)$, and define $\theta_{l\conc(i)}$ and $\sigma^\circ(l\conc(i))$ when $l\conc(i)\in T_t$, 
  such that they satisfy (b1), (b2), and (b3). 
  We perform this by the case analysis of $\phi(l)$.

  The case $\phi(l)= (\Gamma_l\vdash \xx:A)$ that is an instance of $\var$.
  We can define $\phi^\circ(l) = (\Delta\vdash u:A, \Gamma^\circ_l\vdash u:A)$ as $\weak$,
  since $\Delta\subseteqsim \mergeCtx{\Delta}{\Restrict{(\Gamma_l)}{\overline{\xx}}[\theta_l]} \sim \Gamma^\circ_l$
  holds by (b1). 
  As $l\conc(i)\not\in T_t$, (b1), (b2), and (b3) trivially hold.
  
  The case $\phi(l) = (\Gamma_l\vdash y:B)$ that is an instance of $\var$ with $y \neq \xx$. 
  We can define $\phi^\circ(l) = (\Gamma^\circ_l\vdash \theta_l(y):B)$ as $\var$,
  since $\theta_l(y):B \in \Restrict{(\Gamma_l)}{\overline{\xx}}[\theta_l] \subseteqsim \mergeCtx{\Delta_l}{\Restrict{(\Gamma_l)}{\overline{\xx}}[\theta_l]} \sim \Gamma^\circ_l$ holds by (b1).
  As $l\conc(i)\not\in T_t$, (b1), (b2), and (b3) trivially hold.

  The case $\phi(l) = (S_1,S_2,\Gamma_l\vdash f(b):C)$ that is an instance of $\apnotvar$,
  where $S_1 = \Gamma_l\vdash f:B\to C$, $S_2 = \Gamma_l\vdash b:B$, and $b$ is not a variable. 
  Define $\phi^\circ(l) = (S'_1, S'_2,\Gamma^\circ_l\vdash f[\theta_l](b[\theta_l]):C)$ as $\apnotvar$, 
  where $S'_1 = \Gamma^\circ_l\vdash f[\theta_l]:B\to C$ and $S'_2 = \Gamma^\circ_l\vdash b[\theta_l]:B$,
  since $b[\theta_l]$ is not a variable.
  Also define $\sigma^\circ(l\conc(1)) = S'_1$, $\sigma^\circ(l\conc(2)) = S'_2$,
  and $\theta_{l\conc(1)} = \theta_{l\conc(2)} = \theta_l$. 
  Then (b1) and (b2) trivially hold.
  To check (b3),
  assume that $\tilde{k},t$ is a successor of $k,f(b)$, where $t$ is $f$ or $b$,
  and $k$ is not a named index for $\xx$.
  Then (i) both $k$ and $\tilde{k}$ are named indexes for some variable $y^D\in\Dom(\Gamma_l)$,
  (ii) both $k$ and $\tilde{k}$ are unnamed indexes in $C$. 
  For the case (ii), take $k'$ and $\tilde{k'}$ as the same indexes as $k$ and $\tilde{k}$, respectively.
  For the case (i), we have
  $\theta_l(y):D \in \Restrict{(\Gamma_l)}{\overline{\xx}}[\theta_l] \subseteqsim \Gamma^\circ_l \subseteqsim \Gamma^\circ_{l\conc(1)}$ 
  by $y \neq \xx$. Then take $k'$ and $\tilde{k'}$ as named indexes for $\theta_l(y)$.
  In each case, $k'$ and $\tilde{k'}$ satisfy (b3) as expected. 
  
  The case $\phi(l) = (S_1,\Gamma_l\vdash f(\xx):C)$ that is an instance of $\apvar$, 
  where $S_1 = \Gamma_l \vdash f:A\to C$ and $\xx:A\in\Gamma_l$. 
  We need to consider the two subcases whether $u$ is a variable or not.
  \begin{itemize}
  \item
    If $u$ is a variable $y^B$, 
    we can define $\phi^\circ(l) = (S'_1, \Gamma^\circ_l\vdash f[\theta_l](y):C)$,
    where $S'_1 = \Gamma^\circ_l\vdash f[\theta_l]:A\to C$ as $\apvar$,
    since
    $y:B \in \Delta \subseteqsim \mergeCtx{\Delta}{\Restrict{(\Gamma_l)}{\overline{\xx}}[\theta_l]} \sim \Gamma^\circ_l$
    holds by (b1). 
    Also define $\sigma^\circ(l\conc(1)) = S'_1$ and $\theta_{l\conc(1)} = \theta_l$. 
    Then (b1) and (b2) trivially hold. (b3) is checked in a similar way to the case $\apnotvar$.
  \item
    If $u$ is not a variable, 
    define $\phi^\circ(l) = (S'_1, S'_2, \Gamma^\circ_l\vdash f[\theta_l](u):C)$,
    where
    $S'_1 = \Gamma^\circ_l\vdash f[\theta_l]:A\to C$ and
    $S'_2 = \Gamma^\circ_l\vdash u:A$ as $\apnotvar$. 
    Also define $\sigma^\circ(l\conc(1)) = S'_1$, $\sigma^\circ(l\conc(2)) = S'_2$,
    and $\theta_{l\conc(1)} = \theta_{l\conc(2)} = \theta_l$. 
    Then (b1) and (b2) trivially hold. (b3) is checked in a similar way to the case of $\apnotvar$. 
  \end{itemize}

  The case $\phi(l) = (S_1,\Gamma_l\vdash f(y):C)$
  that is an instance of $\apvar$, where $S_1 = \Gamma_l \vdash f:B \to C$, $y:B\in\Gamma_l$ and $y\neq \xx$. 
  We can define $\phi^\circ(l) = (S'_1, \Gamma^\circ_l\vdash f[\theta_l](\theta_l(y)):C)$ as $\apvar$,
  where $S'_1 = \Gamma^\circ_l\vdash f[\theta_l]:B\to C$, 
  since
  $\theta_l(y):B \in \Restrict{(\Gamma_l)}{\overline{\xx}}[\theta_l] \subseteqsim \mergeCtx{\Delta_l}{\Restrict{(\Gamma_l)}{\overline{\xx}}[\theta_l]} \sim \Gamma^\circ_l$
  holds by (b1). 
  Also define $\sigma^\circ(l\conc(1)) = S'_1$ and $\theta_{l\conc(1)} = \theta_l$. 
  Then (b1) and (b2) trivially hold. (b3) is checked in a similar way to the case $\apnotvar$.
  
  The case $\phi(l) = (\Gamma_l, z:C\vdash b:B,\Gamma_l\vdash \lambda z.b:C\to B)$
  that is an instance of $\lambda'$. 
  By the assumption, we have $\theta_l$
  and $\sigma^\circ(l)=\Gamma^\circ\vdash (\lambda z.b)[\theta_l]:C\to B$ that satisfy (b1).
  Let $\theta_l=\{\vec{u}/\vec{x}\}$. 
  We consider two subcases.
  \begin{itemize}
  \item
    The first subcase is when $z\not\in\FV(\vec{u})$.
    We have $(\lambda z.b)[\theta_l] = \lambda z.(b[\Restrict{(\theta_l)}{\overline{z}}])$.
    Then define $\phi^\circ(l) = (\Gamma^\circ_l, z:C\vdash b[\Restrict{(\theta_l)}{\overline{z}}]:B,\Gamma^\circ_l\vdash \lambda z.(b[\Restrict{(\theta_l)}{\overline{z}}]):C\to B)$ as $\lambda$. 
    Note that $\Gamma^\circ_l$ and $z:C$ are consistent
    because $z:C \not\in \mergeCtx{\Delta_l}{\Restrict{(\Gamma_l)}{\overline{\xx}}[\theta_l]} \sim \Gamma^\circ_l$
    holds by (b1), $z\not\in \FV(\Gamma_l)$ and $z \not\in\FV(\vec{u}) \supseteq \FV(\Delta_l)$
    (recall that $\Delta_l$ is $\Delta$ when $\xx:A\in\Gamma_l$, and is $\emptyset$ otherwise).
    Define $\sigma^\circ(l\conc(1)) = \Gamma^\circ_l,z:C\vdash b[\Restrict{(\theta_l)}{\overline{z}}]:B$
    and $\theta_{l\conc(1)} = \Restrict{(\theta_l)}{\overline{z}}$.
    We have (b2) by the definition. 
    We also have (b1) by $(\Gamma^\circ_l,z:C) \sim (\mergeCtx{\Delta_l}{\Restrict{(\Gamma_l)}{\overline{\xx}}}[\theta_l],z:C) = \mergeCtx{\Delta_l}{\Restrict{(\Gamma_l,z:C)}{\overline{\xx}}}[\theta_{l\conc(1)}]$.
    To check (b3),
    assume that $\tilde{k},b$ is a successor of $k,\lambda z.b$ and $k$ is not a named index for $\xx$.
    Then (i) both $k$ and $\tilde{k}$ are named indexes for some variable $y^D\in\Dom(\Gamma_l)$,
    (ii) both $k$ and $\tilde{k}$ are unnamed indexes (not for $C$), or
    (iii) $k$ is an unnamed indexes for $C$ and $\tilde{k}$ is a named index for $z$.
    The cases (i) and (ii) can be checked in a similar way to the case of $\apnotvar$.
    The case (iii) is checked by taking $k'$ and $\tilde{k'}$ as the same indexes as $k$ and $\tilde{k}$,
    respectively.
  \item
    The second subcase is when $z\in\FV(\vec{u})$.
    We have $(\lambda z.b)[\theta_l] = \lambda z'.(b[\Restrict{(\theta_l)}{\overline{z}},z'/z])$, 
    where $z' \not\in\FV(b,\vec{u})$. 
    Then define $\phi^\circ(l) = (\Gamma^\circ_l, z':C\vdash b[\Restrict{(\theta_l)}{\overline{z}},z'/z]:B,\Gamma^\circ_l\vdash \lambda z'.(b[\Restrict{(\theta_l)}{\overline{z}},z'/z]):C\to B)$ as $\lambda$. 
    We need to check that $\Gamma^\circ_l$ and $z':C$ are consistent. 
    It is shown by
    $z':C \not\in \mergeCtx{\Delta_l}{\Restrict{(\Gamma_l)}{\overline{\xx}}[\theta_l]} \sim \Gamma^\circ_l$
    using (b1), 
    $z'\not\in \FV(\lambda z.b) = \FV(\Gamma_l)$ and $z' \not\in\FV(\vec{u}) \supseteq \FV(\Delta_l)$. 
    Define $\sigma^\circ(l\conc(1)) = \Gamma^\circ_l,z':C\vdash b[\Restrict{(\theta_l)}{\overline{z}},z'/z]:B$
    and $\theta_{l\conc(1)} = \Restrict{(\theta_l)}{\overline{z}}\cup\{z'/z\}$.
    We have (b2) by the definition. 
    We also have (b1) by $(\Gamma^\circ_l,z':C) \sim (\mergeCtx{\Delta_l}{\Restrict{(\Gamma_l)}{\overline{\xx}}}[\theta_l],z':C) = \mergeCtx{\Delta_l}{\Restrict{(\Gamma_l,z:C)}{\overline{\xx}}}[\theta_{l\conc(1)}]$.
    Checking (b3) is done in a similar way to that of the first subcase.
  \end{itemize}

  The case $\phi(l) = (S_1,S_2,\Gamma_l\vdash \cond(f,g):\N\to C)$, where
  $S_1 = \Gamma_l\vdash f:C$, $S_2 = \Gamma_l\vdash g:\N\to C$, as an instance of $\cond$. 
  Define $\phi^\circ(l) = (S'_1,S'_2,\Gamma^\circ_l\vdash \cond(f[\theta_l],g[\theta_l]):\N\to C)$ as $\cond$,
  where $S'_1 = \Gamma^\circ_l \vdash f[\theta_l]:C$ and $S'_2 = \Gamma^\circ_l \vdash g[\theta_l]:\N\to C$.
  Also define $\sigma^\circ(l\conc(1)) = S'_1$, $\sigma^\circ(l\conc(2))  = S'_2$,
  and $\theta_{l\conc(1)} = \theta_{l\conc(2)} = \theta_l$. 
  Then (b1) and (b2) trivially hold.
  (b3) is checked in a similar way to the case of $\apnotvar$.

  The case $\phi(l) = (\Gamma_l\vdash 0:\N)$, as an instance of $0$. 
  Define $\phi^\circ(l) = (\Gamma^\circ_l\vdash 0:\N)$ as $0$. 
  Since $l\conc(i)\not\in T_t$, (b1), (b2), and (b3) trivially hold.
  
  The case $\phi(l) = (\Gamma_l\vdash t:\N, \Gamma_l\vdash \Succ(t):\N)$, as an instance of $\Succ$. 
  Define
  $\phi^\circ(l) = (\Gamma^\circ_l\vdash t[\theta_l]:\N, \Gamma^\circ_l\vdash \Succ(t[\theta_l]):\N)$ as $\Succ$. 
  Also define $\sigma^\circ(l\conc(1)) = \Gamma^\circ_l\vdash t:\N$
  and $\theta_{l\conc(1)} = \theta_l$.
  Then (b1) and (b2) trivially hold.
  (b3) is checked in a similar way to the case of $\apnotvar$.

  Therefore our construction of $\Pi^\circ$ is completed.
\end{proof}

\begin{lemma}\label{lem:inversion}
  \begin{enumerate}
  \item\label{lem:inversion1}
    If $\Pi:\Gamma\vdash f(a):A$ and $\Pi$ satisfies the global trace condition, then
    there exist $\Pi_1$, $\Pi_2$ and $B$ such that
    $\Pi_1:\Gamma\vdash f:B\to A$, $\Pi_2:\Gamma\vdash a:B$,
    and both $\Pi_1$ and $\Pi_2$ satisfy the global trace condition. 
  \item\label{lem:inversion2}
    If $\Pi:\Gamma\vdash \lambda x^T.b:A$, where $x\not\in\FV(\Gamma)$,
    and $\Pi$ satisfies the global trace condition, then
    there exist $\Pi_1$ and $B$ such that
    $\Pi_1:\Gamma,x:T\vdash b:B$ and $A = T\to B$,
    and $\Pi_1$ satisfies the global trace condition. 
  \item\label{lem:inversion3}
    If $\Pi:\Gamma\vdash \cond(f,g):A$ and $\Pi$ satisfies the global trace condition, then
    there exist $\Pi_1$, $\Pi_2$ and $B$ such that
    $\Pi_1:\Gamma \vdash f:B$, $\Pi_2:\Gamma \vdash g:N\to B$, $A = N\to B$,
    and both $\Pi_1$ and $\Pi_2$ satisfy the global trace condition. 
  \item\label{lem:inversion4}
    If $\Pi:\Gamma\vdash \Succ(t):A$ and $\Succ(t) \in \GTC$, then    
    there exists $\Pi_1$ such that $\Pi_1:\Gamma \vdash t:N$, $A=N$, 
    and $\Pi_1$ satisfies the global trace condition. 
  \end{enumerate}
\end{lemma}
\begin{proof}
  We first claim that if $t\in\GTC$ and $\Pi:\Gamma\vdash t:A$ with $\Pi=(T,\phi)$, 
  then there exists $l\in T$ such that $\phi(l) \neq \weak$ and $\phi(m) = \weak$ for all $m < l$.
  Because, if not, the only infinite path in $\Pi$ is the consective use of $\weak$,
  which does not contain progressing trace, this contradicts with $t\in\GTC$.

  We show the point \ref{lem:inversion1}.
  Assume that $\Pi:\Gamma\vdash f(a):A$ with $\Pi=(T,\phi)$ and $f(a) \in \GTC$.
  By the claim, take $l\in T$ such that $\phi(l) \neq \weak$ and $\phi(m) = \weak$ for all $m < l$.
  Let $\Gamma' \vdash f(a):A$ be the conclusion of $\phi(l)$. Then $\Gamma'\subseteqsim \Gamma$ holds
  by the transitivity of $\subseteqsim$.
  Now $\phi(l)$ is $\apvar$ or $\apnotvar$.
  In the former case $\phi(l) = \apvar$, we have $\Pi\restr l\conc(1):\Gamma'\vdash f:B\to A$ for some $B$,
  and $a = x^B \in \Gamma'$. Hence we have a proof $\Pi_1:\Gamma\vdash f:B\to A$ by $\weak$. 
  We also obtain $\Pi_2:\Gamma\vdash a:B$ by $a = x^B \in \Gamma'$ and $\weak$. 
  In the latter case $\phi(l) = \apnotvar$,
  we have $\Pi\restr l\conc(1):\Gamma'\vdash f:B\to A$ and $\Pi\restr l\conc(2):\Gamma'\vdash a:B$ for some $B$. 
  Hence we have a proof $\Pi_1:\Gamma\vdash f:B\to A$ and $\Pi_2:\Gamma\vdash a:B$ by $\weak$.
  In both cases, if $t\in\GTC$, we have $f\in\GTC$ and $a\in\GTC$ by the construction of $\Pi_1$ and $\Pi_2$. 
  
  The points \ref{lem:inversion2}, \ref{lem:inversion3}, and \ref{lem:inversion4} are shown similarly. 
\end{proof}


\begin{theorem}[Subject reduction]
  Assume that $\Pi_t:\Gamma\vdash t:A$, $\Pi_t$ satisfies the global trace condition, and $t\reduces u$.
  Then there exists $\Pi_u$ such that $\Pi_u:\Gamma\vdash u:A$ and $\Pi_u$ satisfies the global trace condition. 
\end{theorem}
\begin{proof}
  By the definition of $t\reduces u$, there is a context $\hat{t}[-]$ such that
  $t=\hat{t}[t_0]$, $u=\hat{t}[u_0]$, and $t_0\reduces_\Box u_0$, where $\Box\in\{\beta,\cond\}$. 
  Since $t_0$ is a subterm of $t$, there is $l$ such that $\Pi_t\restr l: \Gamma_0\vdash t_0:A_0$.
  Note that $\Pi_t\restr l$ satisfies the global trace condition,
  since it is a subtree of $\Pi_t$, which satisfies the global trace condition. 
  Then, if we have $\Pi'_u:\Gamma_0\vdash u_0:A_0$ that satisfies the global trace condition,
  the tree obtained from $\Pi_t$ by replacing the subtree $\Pi_t\restr l$ by $\Pi'_u$ is also
  a proof of $\Gamma\vdash \hat{t}[u_0]:A$ that satisfies the global trace condition. 
  Hence it is enough to show the following:
  \begin{itemize}
  \item[(a)]
    If $\Pi:\Gamma \vdash (\lambda x^A.b)(a): B$ and it satisfies the global trace condition,
    then $\Pi':\Gamma \vdash b[a/x]: B$ for some $\Pi'$ that satisfies the global trace condition.
  \item[(b)]
    If $\Pi:\Gamma \vdash \cond(f,g)(0): B$ and it satisfies the global trace condition,
    then $\Pi':\Gamma \vdash f: B$ for some $\Pi'$ that satisfies the global trace condition. 
  \item[(c)]
    If $\Pi:\Gamma \vdash \cond(f,g)(\Succ(t)): B$ and it satisfies the global trace condition,
    then $\Pi':\Gamma \vdash g(t): B$ for some $\Pi'$ that satisfies the global trace condition. 
  \end{itemize}
  (b) is shown immediately by Lemma~\ref{lem:inversion}~\ref{lem:inversion3}.
  We show (c).
  Assume $\Pi:\Gamma \vdash \cond(f,g)(\Succ(t)): B$.
  Then by \ref{lem:inversion1}, \ref{lem:inversion3}, and \ref{lem:inversion4} of Lemma~\ref{lem:inversion},
  we have
  $\Pi_1:\Gamma \vdash g:N\to B$ and $\Pi_2:\Gamma \vdash t:N$,
  where $\Pi_1$ and $\Pi_2$ satisfy the global trace condition. 
  Hence, by applying $\apnotvar$ or $\apvar$ to $\Pi_1$ and $\Pi_2$, 
  we have a proof $\Pi':\Gamma\vdash g(t):B$ that satisfies the global trace condition. 
  In order to show (a), assume $\Pi:\Gamma \vdash (\lambda x^A.b)(a): B$.
  Then by Lemma~\ref{lem:inversion}~\ref{lem:inversion1},
  we have $\Pi_1:\Gamma \vdash \lambda x^A.b:A\to B$ and $\Pi_2:\Gamma \vdash a:A$,
  where $\Pi_1$ and $\Pi_2$ satisty the global trace condition. 
  Then, by Lemma~\ref{lem:thinning},
  we have $\Pi'_1:\Restrict{\Gamma}{\FV(\lambda x.b)} \vdash \lambda x^A.b:A\to B$,
  where $\Pi'$ satisfies the global trace condition. 
  By Lemma~\ref{lem:inversion}~\ref{lem:inversion2},
  we have a proof $\Pi''_1:\Restrict{\Gamma}{\FV(\lambda x.b)},x:A \vdash b:B$
  that satisfies the global trace condition. 
  Hence, by the substitution lemma, we have a proof $\Pi':\Gamma\vdash b[a/x]:B$
  that satisfies the global trace condition, as we wished. 
\end{proof}

Using the subject reduction theorem, variable renaming is shown to be admissible in our system. 

\begin{proposition}\label{prop:renaming}
  \begin{enumerate}
  \item\label{prop:renaming1}
    Let $\theta$ be a renaming.
    If $\Pi:\Gamma\vdash t:A$ and $\Gamma[\theta]$ is a context, 
    then $\Pi':\Gamma[\theta]\vdash t[\theta]:A$ for some $\Pi'$.
    Moreover, if $\Pi$ satisfies the global trace condition, so is $\Pi'$.
  \item\label{prop:renaming2}
    If $\Pi:\Gamma\vdash t:A$ and $t'$ is an $\alpha$-equivalent term of $t$,
    then $\Pi':\Gamma\vdash t':A$ for some $\Pi'$.
    Moreover, if $\Pi$ satisfies the global trace condition, so is $\Pi'$.
  \end{itemize}
\end{proposition}
\begin{proof}
  We show the point \ref{prop:renaming1}. 
  It is enough to show the claim for a single renaming $\theta = \{y'/y\}$.
  Then, by the assumption, we have a proof $\Pi_1$ of $\Gamma[y'/y] \vdash (\lambda y^B.t)y':A$
  such that $\Pi_1$ satisfies the global trace condition if $\Pi$ satisfies it.  
  Hence we have $\Pi': \Gamma[y'/y] \vdash t[y'/y]:A$ by the subject reduction theorem
  such that $\Pi'$ satisfies the global trace condition if $\Pi$ satisfies it.

  Next we show the point \ref{prop:renaming2}.
  It is enough to show when $t = \lambda x.b$ and $t' = \lambda x'.(b[x'/x])$, where $x'\not\in\FV(\lambda x.b)$. 
  Let $\Pi$ be a proof of $\Gamma\vdash \lambda x.b:A\to B$.
  Then we have $\Pi_1: \Restrict{\Gamma}{\FV(\lambda x.b)} \vdash \lambda x.b:A\to B$
  by Lemma~\ref{lem:thinning}, and
  also have $\Pi_2: \Restrict{\Gamma}{\FV(\lambda x.b)},x':B \vdash b[x'/x]:B$
  by the subject reduction theorem. 
  Hence we have $\Pi': \Gamma \vdash \lambda x'.b[x'/x]:A\to B$ by the rules $\lambda$ and $\weak$. 
  Note that $\Pi'$ satisfies the global trace condition if $\Pi$ satisfies it. 
\end{proof}



\newpage
% Section 7
\section{Weak Church-Rosser for safe reductions}
\label{section-n-safe-church-rosser}

In this section we prove a result for the $n$-safe part of a term, which we call 
\quotationMarks{\emph{unicity of the $n$-safe part of the safe normal form up to $\nequal{n}$}}. By this we mean:
for all $t, u, v \in \GTC$, if $t \nsafeReduces{n} u$ and $t \nsafeReduces{n} v$ and $u$, $v$ are $n$-safe-normal 
then $u \nequal{n} v$ holds. 

Our first idea (wrong) is to prove a full Church-Rosser property for $\LAMBDA$: 
for all $t,u,v \in \LAMBDA$, if $t \reduces u$ and $t \reduces v$ then for some $w \in \LAMBDA$
we have $u \reduces w$ and $v \reduces w$. This property is false: for some $t \in \LAMBDA$, finding a 
common reduction of $u$, $v$ takes infinitely many steps. This even in the case $t \in \CTlambda$,
as the next example shows.

\begin{Eg}[Failure of Church-Rosser for $\CTlambda$]
  Let $b = \cond(x^{\N},b):\N \rightarrow \N$ a normal form
  and $t = (\lambda x^{\N}.b)(r):\N \rightarrow \N$, 
  where $r = (\lambda x^{\N}.x^{\N})(3)$ is some redex. 

  We have $b[r/x](n) \reduces r \reduces 3$ for all numerals $n$, 
  therefore $t$ and $\lambda \_.3$ are extensionally equal, however $t \not \reduces \lambda \_.3$. 
  We have $t \in \CTlambda$. Indeed, 
  \begin{enumerate}
  \item
    $t$ is regular by construction.
  \item
    We have $t \in \GTC$, because the unique infinite path of $t$ is 
    $t, \lambda x^{\N}.b, b, b, b, \ldots$, and the
    unique unnamed argument of $b:\N \rightarrow \N$ in the path progresses infinitely many times.
  \end{enumerate}

  Now consider the reductions: $t \reduces b[r/x^\N]$ and $t \reduces  (\lambda x^{\N}.b)(3)$.
  We expect $b[3/x^\N]$ as common normal form. But we have $b[r/x^\N] = \cond(r,b[r/x^\N]$,
that is, we have replicated the redex $r$ infinitely many times in $b[r/x^\N]$. Therefore to reduce 
$b[r/x^\N]$ to $b[3/x^\N]$ takes infinitely many steps, and for \emph{no finite reduction we have}
$b[r/x^\N] \reduces b[3/x^\N]$. 

We proved that Church-Rosser is false for $\CTlambda$.
\end{Eg}

In the following we will work in the $n$-safe level of $t\in \GTC$. 
The global trace condition guarentees the finiteness of the $n$-safe level for each $n$. 
\begin{lemma}
  For any $n\ge 0$, the set of the $n$-safe level of $t\in \GTC$ is finite.
\end{lemma}
\begin{proof}
  Let $t\in \GTC$. We show the finiteness of any $n$-level of $t$ by proof of contradiction.
  Assume that the finiteness fails. 
  Then take the least number $n$ such that the $n$-safe level of $t$ is infinite.
  
  The case of $n=0$.
  Since the binary tree obtained from $t$ restricting to the $0$-safe level of $t$ contains infinite nodes,
  we can take an infinite path $t,t_1,t_2,\ldots$ of the tree by using K\"{o}nig's lemma.
  Then this infinite path does not contain a progressing trace since any $t_i$
  cannot be the right subterm of $\cond$. This contradicts $t\in \GTC$.
  
  The case of $n>0$. By the leastness of $n$, for any $n'<n$, all $n'$-level of $t$ are finite.
  We can also take an infinite path from 
  the binary tree that is a restriction of $t$ with nodes until the $n$-safe level of $t$. 
  The path has the form $\vec{t_0},\vec{t_1},\ldots,\vec{t_n}$,
  where each $\vec{t_i}$ is a sequence of nodes of the $i$-safe level,
  and only $\vec{t_n}$ is infinite.
  Then this infinite path does not contain a progressing trace as in the case of $n=0$. 
  This also contradicts $t\in \GTC$.
  
  Hence we have the finiteness of the $n$-safe level of $t\in\GTC$, as we wished. 
\end{proof}

We write $\Lv{n}{t}$ for the number of the $n$-safe level of $t\in\GTC$. 

We define a $n$-safe level context $\Lctx{n}$
with holes (written $\cdot$) that will be filled with terms in $\LAMBDA$,
and whose positions are at the $n$-safe level. 
\begin{definition}[Safe level context]
  We define $\Lctx{-1} = \cdot$. 
  For $n \ge 0$, the $n$-safe level context, written $\Lctx{n}$, is inductively defined as
  follows:
  \[
  \Lctx{n} ::= x^T \mid 0 \mid \Lctx{n}\Lctx{n} \mid \lambda x^T.\Lctx{n}
  \mid \Suc{\Lctx{n}} \mid \Cond{\Lctx{n}}{\Lctx{n-1}}. 
  \]
\end{definition}
Multiple holes may appear in a context and each holes are distinguished.
The resulting term obtained by filling $k$-holes in $\Lctx{n}$
with terms $t_1,\ldots,t_k$ is written $\Lctx{n}[t_1,\ldots,t_k]$. 
Note that filling holes with terms is not substitution,
but just putting terms at the positions of holes, namely, for example,
the result of filling the unique hole in $\lambda x.\Cond{0}{\cdot}$ with $x$
is $\lambda x.\Cond{0}{x}$. 

The finiteness of the $n$-safe level of $t\in\GTC$ enables 
to split $t$ by a $n$-safe level context and terms that apper at level $>n$
as stated in the next lemma. 

\begin{lemma}\label{lem:split_context}
  \begin{enumerate}
  \item\label{lem:split_context1}
    For any $\Lctx{n}$, there uniquely exists
    $(\Lctx{0},\Lctx{n-1}^{1},\ldots,\Lctx{n-1}^{k})$ such that
    $\Lctx{0}$ has $k$-holes and
    $\Lctx{n} = \Lctx{0}[\Lctx{n-1}^1,\ldots,\Lctx{n-1}^k]$, namely $\Lctx{n}[\vec{t_1},\ldots,\vec{t_k}] = \Lctx{0}[\Lctx{n-1}^1[\vec{t_1}],\ldots,\Lctx{n-1}^k[\vec{t_k}]]$ holds for any $\vec{t_1},\ldots,\vec{t_k}$. 
  \item\label{lem:split_context2}
    Let $t\in\GTC$ and $n\ge 0$.
    There uniquely exists $(\Lctx{n},t_1,\ldots,t_k)$
    such that $\Lctx{n}$ has $k$-holes, $t = \Lctx{n}[t_1,\ldots,t_k]$,
    and all $t_1,\ldots,t_k$ appear at the $(n+1)$-safe level of $t$.
  \end{enumerate}
\end{lemma}
\begin{proof}
  The point~\ref{lem:split_context1} is shown by induction on the construction of $\Lctx{n}$.
  If $\Lctx{n}$ is $x^T$ or $0$, then define $\Lctx{0}$ by the same one as $\Lctx{n}$. 
  For the case of $\Lctx{n} = \lambda x^T.\Lctx{n}'$, 
  we have a unique $(\Lctx{0}',\overrightarrow{\Lctx{n-1}'})$ such that 
  $\Lctx{n}' = \Lctx{0}'[\overrightarrow{\Lctx{n-1}'}]$.
  Then $(\lambda x^T.\Lctx{0}',\overrightarrow{\Lctx{n-1}'})$
  satisfies the expected condition for $\Lctx{n}$ of this case. 
  The cases of $\Lctx{n} = \Lctx{n}'\Lctx{n}''$ and $\Lctx{n} = \Suc{\Lctx{n}'}$
  are shown similarly by using the induction hypothesis.
  We show the case $\Lctx{n} = \Cond{\Lctx{n}'}{\Lctx{n-1}''}$. 
  By the induction hypothesis, there exists a unique
  $(\Lctx{0}',\overrightarrow{\Lctx{n-1}'})$ such that 
  $\Lctx{n}' = \Lctx{0}'[\overrightarrow{\Lctx{n-1}'}]$.
  Then $(\Cond{\Lctx{0}'}{\cdot}, \overrightarrow{\Lctx{n-1}'}, \Lctx{n-1}'')$
  satisfies the expected condition for $\Lctx{n}$ of this case. 
  
  The point~\ref{lem:split_context2} is shown
  by induction on $n$ using \ref{lem:split_context1}.

  We first show the case of $n=0$ by induction on the number of the $0$-safe level of $t$.
  If $t = x^T$ or $t = 0$, it is shown by taking $\Lctx{0}$ as $t$.
  If $t = \lambda x^T.t'$, by the induction hypothesis,
  there uniquely exists $(\Lctx{0}',\vec{t'})$ that satisfies the condition for $t'$. 
  Then $(\lambda x^T.\Lctx{0}',\vec{t'})$ satisfies
  the expected condition for $\lambda x^T.t'$. 
  If $t =t't''$ or $t = \Suc{t'}$, it is also shown by the induction hypothesis. 
  If $t = \Cond{t'}{f}$, by the induction hypothesis, 
  there uniquely exists $(\Lctx{0}',\vec{t'})$ that satisfies the condition for $t'$. 
  Then $(\Cond{\Lctx{0}'}{\cdot},\vec{t'},f)$ satisfies
  the expected condition for $\Cond{t'}{f}$. 

  Then we show the case of $n>0$.
  By the result of the case of $n=0$, there uniquely exists $(\Lctx{0},t_1,\ldots,t_k)$
  that satisfies $t = \Lctx{0}[t_1,\ldots,t_k]$ and
  each $t_i$ appears at the $1$-safe level of $t$.
  For each $i$, by applying the induction hypothesis to $t_i$,
  we have unique $(\Lctx{0}^i,\vec{t'_i})$
  satisfies $t_i = \Lctx{0}^i[\vec{t'_i}]$ and $\vec{t'_i}$ appear
  at the $(n-1)$-safe level of $t_i$. 
  Then $(\Lctx{0}[\Lctx{n-1}^1,\ldots,\Lctx{n-1}^k],\vec{t'_1},\ldots,\vec{t'_k})$
  satisfies the expected condition for $t$.
  Its uniqueness is obtained by using the uniqueness of $\Lctx{0}$ and $\Lctx{n-1}^i$,
  and the point \ref{lem:split_context1}. 
\end{proof}

We define the notion \quotationMarks{$n$-safe equality} that intuitively means
two terms are approximately equal excepting for deeper safe levels more than $n$. 

\begin{definition}[$n$-safe equality]
  Let $\Gamma\vdash t_1:A$ and $\Gamma\vdash t_2:A$. 
  The terms $t_1$ and $t_2$ are \quotationMarks{$n$-safe equal} (written $\Gamma \vdash t_1 \nequal{n} t_2 : A$)
  is defined by after possibly renaming the bound variables of $t_1$ and $t_2$, 
  they have the forms $\Lctx{n}[\vec{u}_1]$ and $\Lctx{n}[\vec{u}_2]$
  with some terms $\vec{u}_1$ and $\vec{u}_2$.
  We sometimes write $t_1 \nequal{n} t_2$ instead of $\Gamma \vdash t_1\nequal{n} t_2:A$
  when $\Gamma$ and $A$ are clear from the context. 
\end{definition}

The $n$-safe equality is necessary to fill \quotationMarks{gaps}
after $n$-safe reductions from the same term.
For example, let $t$ be $(\lambda x^{\N\to\N}.\Cond{0}{x})(IS)$,
where $I = \lambda z^{\N\to\N}.z$ and $S = \lambda n^\N.\Suc{n}$.
For the two $0$-safe reducuctions 
$t \nsafeReducesAst{0} \Cond{0}{S}$ and $t \nsafeReducesAst{0} \Cond{0}{IS}$, 
we have $\Cond{0}{IS} \nequal{0} \Cond{0}{S}$
since $\Cond{0}{IS} \nsafeReduces{0} \Cond{0}{S}$ does not hold. 

Note that $\nequal{n}$ is an equivalence relation. Moreover it is closed under substitution. 
\begin{lemma}\label{lem:nequal_subst}
  If $\Gamma \vdash u_1 \nequal{n} u_2:A$ and $\Gamma, x^A:A \vdash t_1 \nequal{n} t_2:B$, then
  $\Gamma \vdash t_1[u_1/x^A] \nequal{n} t_2[u_2/x^A] :B$. 
\end{lemma}
\begin{proof}
  By the substitution lemma, we have $\Gamma \vdash t_i[u_i/x^A]: B$ for $i\in\{1,2\}$.
  Thus we need to show the equality part. 
  It is easily checked that $t_1[u_1/x^A] \nequal{-1} t_2[u_2/x^A]$ by the definition of the safe equality.
  For $n\ge 0$, by $t_1\nequal{n} t_2$, there exists $(\Lctx{n},\vec{u_1},\vec{u_2})$ such that
  $t_1 = \Lctx{n}[\vec{v_1}]$ and $t_2 = \Lctx{n}[\vec{v_2}]$.
  We show $\Lctx{n}[\vec{v_1}][u_1/x^A] \nequal{n} \Lctx{n}[\vec{v_2}][u_2/x^A]$ by induction on $\Lctx{n}$. 

  The case of $\Lctx{n}=x^A$ is shown by $x^A[u_1/x^A] = u_1 \nequal{n} u_2 = x^A[u_2/x^A]$.
  The cases of $\Lctx{n}=0$ and $\Lctx{n}=z^C \neq x^A$
  are shown by $t_1[u_1/x^A] = t_1 \nequal{n} t_2 = t_2[u_2/x^A]$.
  The case of $\Lctx{n}=\lambda z^C.\Lctx{n}'$ with $z^C\not\in\FV(\vecu_1,u_2)$ by renaming
  is shown by
  $(\lambda z^C.\Lctx{n}'[\vec{v_1}])[u_1/x^A] = \lambda z^C.(\Lctx{n}'[\vec{v_1}][u_1/x^A]) \nequal{n} \lambda z^C.(\Lctx{n}'[\vec{v_2}][u_2/x^A]) = (\lambda z^C.\Lctx{n}'[\vec{v_2}])[u_2/x^A]$ by the induction hypothesis.
  The cases of $\Lctx{n}=\Lctx{n}'\Lctx{n}''$ and $\Lctx{n}=\Suc{\Lctx{n}'}$
  are also shown by the induction hypothesis.
  We show the case of $\Lctx{n}=\Cond{\Lctx{n}'}{\Lctx{n-1}'}$.
  If $n=0$, by induction hypothesis, we have $\Lctx{0}[\vec{v_1}][u_1/x^A] = \Cond{\Lctx{0}'[\vec{v_1}][u_1/x^A]}{\Lctx{-1}'[\vec{v_1}][u_1/x^A]} \nequal{0} \Cond{\Lctx{0}'[\vec{v_2}][u_2/x^A]}{\Lctx{-1}'[\vec{v_2}][u_2/x^A]} = \Lctx{0}[\vec{v_2}][u_2/x^A]$ since we do not care the right-hand side of $\cond$ at the $0$-safe level. 
  If $n>0$, by induction hypothesis, we have $\Lctx{n}[\vec{v_1}][u_1/x^A] = \Cond{\Lctx{n}'[\vec{v_1}][u_1/x^A]}{\Lctx{n-1}'[\vec{v_1}][u_1/x^A]} \nequal{n} \Cond{\Lctx{n}'[\vec{v_2}][u_2/x^A]}{\Lctx{n-1}'[\vec{v_2}][u_2/x^A]} = \Lctx{n}[\vec{v_2}][u_2/x^A]$.
  Hence we obtain the result as we wished. 
\end{proof}


The main theorem of this section is a weak form of Church-Rosser
of the $n$-safe reductions that holds up to the $n$-equalities. 

\begin{theorem}[Weak Church-Rosser of $n$-safe reduction modulo $n$-safe equality]
  Let $t\in \GTC$.
  If $t_1 \nsafeReducesAstL{n} t \nsafeReducesAst{n} t_2$, 
  then there exist $t'_1$ and $t'_2$ such that
  $t_1 \nsafeReducesAst{n} t'_1 \nequal{n} t'_2 \nsafeReducesAstL{n} t_2$. 
\end{theorem}

This theorem will be proved by a variant of the parallel reduction technique. 

\begin{definition}[Safe parallel reduction]
  Let $n\ge -1$.
  We define the $n$-safe parallel reduction relation $\nsafePReduces{n}$ on the $\GTC$ terms.
  The $\nsafePReduces{-1}$ is defined by $t\nsafePReduces{-1} t$ for all $t\in\GTC$. 
  For $n\ge 0$, the relation $\nsafePReduces{n}$ is inductively defined as follows: 
  \begin{itemize}
  \item[(id)]
    $t \nsafePReduces{n} t$.
  \item[$(\Succ)$]
    If $t \nsafePReduces{n} t'$, then $\Suc{t} \nsafePReduces{n} \Suc{t'}$.
  \item[$(\lambda)$]
    If $t \nsafePReduces{n} t'$, then $\lambda x^T.t \nsafePReduces{n} \lambda x^T.t'$.
  \item[(ap)]
    If $f \nsafePReduces{n} f'$ and $a \nsafePReduces{n} a'$,
    then $f(a) \nsafePReduces{n} f'(a')$.
  \item[$(\cond)$]
    If $a \nsafePReduces{n} a'$ and $f \nsafePReduces{n-1} f'$,
    then $\Cond{a}{f} \nsafePReduces{n} \Cond{a'}{f'}$.
  \item[$(\beta)$]
    If $t \nsafePReduces{n} t'$ and $u \nsafePReduces{n} u'$,
    then $(\lambda x^T.t)u \nsafePReduces{n} t'[u'/x]$.
  \item[$(\cond\,0)$]
    If $a \nsafePReduces{n} a'$, 
    then $\Cond{a}{f}(0) \nsafePReduces{n} a'$.
  \item[$(\cond\,\Succ)$]
    If $f \nsafePReduces{n-1} f'$ and $u \nsafePReduces{n} u'$, 
    then $\Cond{a}{f}(\Suc{u}) \nsafePReduces{n} f'(u')$.
  \end{itemize}
\end{definition}

The relation $\nsafePReduces{-1}$ is a special case to simplify
the rules $(\cond)$ and $(\cond\,\Succ)$.
For $n\ge 0$, if $t$ is reduced to $u$ by the $n$-safe parallel reduction,
redexes that appear in the $n$-safe level of $t$ can be reduced. 
For example, $\Cond{I0}{\Cond{I0}{IS}} \nsafePReduces{n} \Cond{0}{\Cond{0}{S}}$
holds if $n=2$, but does not hold if $n=1$. 
The case (id) cannot be replaced by $x^T \nsafePReduces{n} x^T$ and $0 \nsafePReduces{n} 0$ expecting (id) can be derived from them, but it is not the case in our setting
because we are working on the infinite terms. 

The relation $\nsafePReduces{n}$ satisfies the following basic properties. 
\begin{lemma}\label{lem:parallel_basic}
  Let $t, u\in \GTC$. The following properties hold. 
  \begin{enumerate}
  \item\label{lem:parallel_basic1}
    $n < n'$ and $t \nsafePReduces{n} t'$ implies $t \nsafePReduces{n'} t'$ (monotonicity). 
  \item\label{lem:parallel_basic2}
    If $t \nsafeReduces{n} t'$, then $t \nsafePReduces{n} t'$. 
  \item\label{lem:parallel_basic3}
    If $t \nsafePReduces{n} t'$, then $t \nsafeReducesAst{n} t'$.
  \item\label{lem:parallel_basic4}
    If $t \nsafePReduces{n} t'$ and $u \nsafePReduces{n} u'$, 
    then $t[u/x^A] \nsafePReduces{n} t'[u'/x^A]$.
  \end{enumerate}
\end{lemma}
\begin{proof}
  The point~\ref{lem:parallel_basic1} is shown by induction on $\nsafePReduces{n}$.
  
  We show the point~\ref{lem:parallel_basic2}.
  For $t$ and $t'$ such as $t\nsafeReduces{n} t'$
  that reduces a redex $t_R$ occuring in $t$, 
  define $r(t,t')$ as the length of the path from $t$ to $t_R$.
  The proof is done by induction of $r(t,t')$.
  The case of $r(t,t') = 0$, namely $t = t_R$,
  we have $t \nsafePReduces{n} t'$ by the rule $(\beta)$, $(\cond\,0)$,
  or $(\cond\,\Succ)$. 
  The case of $r(t,t') > 0$ is shown by the case analysis of the form of $t$.
  The case of $t=\Suc{t_1}$ and $t_R$ appears in $t_1$.
  The reduction is $\Suc{t_1} \nsafeReduces{n} \Suc{t'_1} = t'$ with $t_1 \nsafeReduces{n} t'_1$. 
  Then we have $t_1 \nsafePReduces{n} t'_1$ by the induction hypothesis with $r(t_1,t'_1) < r(t,t')$.
  Hence $\Suc{t_1} \nsafePReduces{n} \Suc{t'_1}$ by $(\Succ)$. 
  The cases of $t=t_1t_2$ and $t=\lambda x^A.t_1$ are also shown by the induction hypothesis. 
  The case of $t=\Cond{a}{f}$ and $t_R$ appears in $a$ is shown by the induction hypothesis.
  The case of $t=\Cond{a}{f}$ and $t_R$ appears in $f$.
  The reduction is $\Cond{a}{f} \nsafeReduces{n} \Cond{a}{f'} = t'$ with $f \nsafeReduces{n-1} f'$. 
  Then we have $f \nsafePReduces{n-1} f'$ by the induction hypothesis with $r(f,f') < r(t,t')$.
  Hence $\Cond{a}{f} \nsafePReduces{n} \Cond{a}{f'}$ by ($\cond$).

  The point~\ref{lem:parallel_basic3} is shown by induction on $\nsafePReduces{n}$. 

  To prove the point~\ref{lem:parallel_basic4}, we first show that
  $u \nsafePReduces{n} u'$ implies $t[u/x^A] \nsafePReduces{n} t[u'/x^A]$ by induction on $\Lv{n}{t}$.
  The case $t=x^A$ is shown by the assumption $u \nsafePReduces{n} u'$.
  The cases $t=y^B\neq x^A$ and $t=0$ are shown by $(id)$. 
  The case $t=y^B\neq x^A$ is shown by $(id)$.   
  The case $t=\lambda y^B.t'$ with $y^B\not\in\FV(u,u')$ by renaming is shown by
  $(\lambda y^B.t')[u/x^A] = \lambda y^B.(t'[u/x^A]) \nsafePRuduces{n} \lambda y^B.(t'[u'/x^A]) = (\lambda y^B.t')[u'/x^A]$ by the induction hypothesis and $(\lambda)$.
  The cases $t=t_1t_2$ and $t=\Suc{t'}$ are also shown by the induction hypothesis.
  We show the case $t=\Cond{a}{f}$ and $n>0$. 
  Since $\Lv{n-1}{a}, \Lv{n}{f} < \Lv{n}{\Cond{a}{f}}$, 
  we have $a[u/x^A] \nsafePReduces{n} a[u'/x^A]$ and $f[u/x^A] \nsafePReduces{n-1} f[u'/x^A]$
  by the induction hypothesis. Hence $\Cond{a}{f}[u/x^A] \nsafePReduces{n} \Cond{a}{f}[u'/x^A]$ by $(\cond)$. 
  The case $t=\Cond{a}{f}$ and $n=0$ is shown similarly. 

  Then we show the the point~\ref{lem:parallel_basic4} by induction on $\nsafePReduces{n}$.
  The case of $(id)$ is already shown. 
  The cases of $(\Succ)$, $(\lambda)$, $(ap)$, $(\cond)$, $(\cond\,0)$, and $(\cond\,\Succ)$
  are shown by the induction hypothesis.
  We show the case $(\beta)$ with $t=(\lambda y^B.t_1)t_2$, where $y^B\not\in\FV(u,u')$ by renaming,
  $t'=t'_1[t'_2/y^B]$, $t_1\nsafePReduces{n} t'_1$, and $t_2\nsafePReduces{n} t'_2$. 
  By the induction hytpothesis, $t_1[u/x^A] \nsafePReduces{n} t'_1[u'/x^A]$ and $t_2[u/x^A]\nsafePReduces{n} t'_2[u'/x^A]$ hold.
  Then, by Lemma~\ref{lem:subst_ex},
  $((\lambda y^B.t_1)t_2)[u/x^A] = (\lambda y^B.(t_1[u/x^A]))t_2[u/x^A] \nsafePReduces{n} t_1[u/x^A][t_2[u/x^A]/y^B] = t_1[t_2/y^B][u/x^A]$ as we wished. 
\end{proof}

The next lemma says that the $n$-equality simulates 
\begin{lemma}\label{lem:parallel_eq}
  Let $t_1, t_2 \in \GTC$ and $n\ge 0$.
  If $t_1 \nequal{n} t_2 \nsafePReduces{n} t_2'$,
  then there exists $t_1'\in \GTC$ such that $t_1 \nsafePReduces{n} t_1' \nequal{n} t_2'$.
\end{lemma}
\begin{proof}
  The proof is done be induction on $t_2 \nsafePReduces{n} t_2'$.
  
  The case of $(id)$, namely $t_2=t_2'$, is shown by taking $t_1$ as $t_1'$.
  
  The case of $(\Succ)$, namely $t_2=\Suc{u_2} \nsafePReduces{n} \Suc{u_2'}=t_2'$ that is obtained
  from $u_2\nsafePReduces{n} u_2'$. 
  There is $u_1$ such that $t_1 = \Suc{u_1}$ and $u_1\nequal{n} u_2$ by $t_1 \nequal{n} \Suc{u_2}$ with $n\ge 0$.
  Then, by the induction hypothesis,
  there exists $u_1'\in\GTC$ such that $u_1 \nsafePReduces{n} u_1' \nequal{n} u_2'$.  
  This case is shown by taking $\Suc{u_1'}$ as $t_1'$,
  since $\Suc{u_1} \nsafePReduces{n} \Suc{u_1'} \nequal{n} \Suc{u_2'}$.
  
  The cases of $(\lambda)$, $(ap)$, and $(\cond)$ are shown similarly by using the induction hypothesis.
  
  The case of $(\cond\,0)$, namely $t_2 = \Cond{a_2}{f_2}(0) \nsafePReduces{n} a_2'$ that is
  obtained from $a_2 \nsafePReduces{n} a_2'$.
  There are $a_1$, $f_1$ such that $t_1 = \Cond{a_1}{f_1}(0)$ and $a_1 \nequal{n} a_2$
  by $t_1 \nequal{n} \Cond{a_2}{f_2}(0)$ with $n\ge 0$.
  Then, by the induction hypothesis,
  there exists $a_1'\in\GTC$ such that $a_1 \nsafePReduces{n} a_1' \nequal{n} a_2'$.  
  This case is shown by taking $a_1'$ as $t_1'$,
  since $\Cond{a_1}{f_1}(0) \nsafePReduces{n} a_1' \nequal{n} a_2'$.  

  The case of $(\cond\,\Succ)$ is also shown by the induction hypothesis in a similar way to the case $(\cond\,0)$.
  
  The case of $(\beta)$, namely $t_2 = (\lambda x^A.u_2)v_2 \nsafePReduces{n} u_2'[v_2'/x^A]$ that is 
  obtained from $u_2 \nsafePReduces{n} u_2'$ and $v_2 \nsafePReduces{n} v_2'$.
  There are $u_1$, $v_1$ such that $t_1 = (\lambda x^A.u_1)v_1$, $u_1 \nequal{n} u_2$, and $v_1 \nequal{n} v_2$
  by $t_1 \nequal{n} (\lambda x^A.u_2)v_2$ with $n\ge 0$.
  Then, by the induction hypothesis,
  there exist $u_1',v_1'\in\GTC$ such that $u_1 \nsafePReduces{n} u_1' \nequal{n} u_2'$ and
  $v_1 \nsafePReduces{n} v_1' \nequal{n} v_2'$.
  Hence this case is shown by taking $u_1'[v_1'/x^A]$ as $t_1'$,
  since $(\lambda x^A.u_1)v_1 \nsafePReduces{n} u_1'[v_1'/x^A] \nequal{n} u_2'[v_2'/x^A]$ by Lemma~\ref{lem:nequal_subst}.
  
  Therefore we obtain the expected result. 
\end{proof}

The $n$-safe parallel reduction satisfies the following \quotationMarks{pentagon shape} property. 
\begin{lemma}[Pentagon property]\label{lem:parallel_pentagon}
  Let $t \in \GTC$.
  If $t_1 \nsafePReducesL{n} t \nsafePReduces{n} t_2$,
  then there exists $t_1',t_2'\in \GTC$ such that $t_1 \nsafePReduces{n} t_1' \nequal{n} t_2' \nsafePReducesL{n} t_2$.
\end{lemma}
\begin{proof}
  The lemma is shown by induction on $t \nsafePReduces{n} t_1$. 

  The case of $(id)$, namely $t=t_1$, is shown by taking $t_2$ as both $t_1'$ and $t_2'$. 
  
  The case of $(\Succ)$, namely $t=\Suc{u} \nsafePReduces{n} \Suc{u_1}=t_1$ that is obtained
  from $u \nsafePReduces{n} u_1$.
  Then $t=\Suc{u} \nsafePReduces{n} t_2$ is $(id)$, namely $t_2 = t$, or 
  $(\Succ)$, namely $t_2=\Suc{u_2}$ with $u \nsafePReduces{n} u_2$.
  The subcase of $(id)$ is immediately shown by taking $t_1$ as both $t_1'$ and $t_2'$. 
  We show the subcase of $(\Succ)$.
  By the induction hypothesis, there are $u_1', u_2' \in \GTC$ such that
  $u_1 \nsafePReduces{n} u_1' \nequal{n} u_2' \nsafePReducesL{n} u_2$.
  Hence this case is shown by taking $(\Suc{u_1'},\Suc{u_2'})$ as $(t_1',t_2')$, 
  since $t_1 = \Suc{u_1} \nsafePReduces{n} \Suc{u_1'} \nequal{n} \Suc{u_2'} \nsafePReducesL{n} \Suc{u_2} = t_2$.
  
  The cases of $(\lambda)$ and $(\cond)$ are shown similarly by using the induction hypothesis.  

  The case of $(\cond\,0)$, namely $t = \Cond{a}{f}(0) \nsafePReduces{n} t_1$ that is
  obtained from $a \nsafePReduces{n} t_1$.
  Then $t \nsafePReduces{n} t_2$ is
  (i) $t=\Cond{a}{f}(0) \nsafePReduces{n} \Cond{a_2}{f_2}(0) = t_2$
  with $a \nsafePReduces{n} a_2$ and $f \nsafePReduces{n} f_2$, which is $(id)$ or $(ap)$, or
  (ii) $t=\Cond{a}{f}(0) \nsafePReduces{n} t_2$ with $a \nsafePReduces{n} t_2$, which is $(\cond\,0)$. 
  We first show the subcase (i).
  By the induction hypothesis, there are $a_1', a_2' \in \GTC$ such that
  $t_1 \nsafePReduces{n} a_1' \nequal{n} a_2' \nsafePReducesL{n} a_2$.
  Then the subcase (i) is shown by taking $(a_1',a_2')$ as $(t_1',t_2')$. 
  Next we show the subcase (ii). 
  By the induction hypothesis, there are $a_1', a_2' \in \GTC$ such that
  $t_1 \nsafePReduces{n} a_1' \nequal{n} a_2' \nsafePReducesL{n} t_2$.
  Then the subcase (ii) is shown by taking $(a_1',a_2')$ as $(t_1',t_2')$. 

  The case of $(\cond\,\Succ)$ is also shown by the induction hypothesis in a similar way to the case $(\cond\,0)$.

  The case of $(\beta)$, namely $t = (\lambda x^A.u)v \nsafePReduces{n} u_1[v_1/x^A] = t_1$ that is 
  obtained from $u \nsafePReduces{n} u_1$ and $v \nsafePReduces{n} v_1$.
  Then $t \nsafePReduces{n} t_2$ is: 
  (i) $t=(\lambda x^A.u)v \nsafePReduces{n} (\lambda x^A.u_2)v_2 = t_2$
  with $u \nsafePReduces{n} u_2$ and $v \nsafePReduces{n} v_2$ that is $(id)$ or $(ap)$, or
  (ii) $t=(\lambda x^A.u)v \nsafePReduces{n} u_2[v_2/x^A]=t_2$
  with $u \nsafePReduces{n} u_2$ and $v \nsafePReduces{n} v_2$ that is $(\lambda)$.
  The first subcase (i) is shown by the induction hypothesis.
  We show the second subcase (ii).
  By the induction hypothesis, there are $u_1', u_2',v_1',v_2' \in \GTC$ such that  
  $u_1 \nsafePReduces{n} u_1' \nequal{n} u_2' \nsafePReducesL{n} u_2$ and
  $v_1 \nsafePReduces{n} v_1' \nequal{n} v_2' \nsafePReducesL{n} v_2$.
  Then we have $u_1[v_1/x^A] \nsafePReduces{n} u_1'[v_1'/x^A] \nequal{n} u_2'[v_2'/x^A] \nsafePReducesL{n} u_2[v_2/x^A]$ by Lemma~\ref{lem:parallel_basic}.\ref{lem:parallel_basic4} and Lemma~\ref{lem:nequal_subst}. 
  Then the subcase (ii) is shown by taking $(u_1'[v_1'/x^A],u_2'[v_2'/x^A])$ as $(t_1',t_2')$. 
  
  Finally we check the case of $(ap)$.
  The reduction $t \nsafePReduces{n} t_2$ is $(id)$, $(ap)$, $(\cond\,0)$, $(\cond\,\Succ)$, or $(\beta)$.
  The subcases $(id)$ and $(ap)$ are shown by the induction hypothesis.
  The other subcases are already checked in the above cases. 
\end{proof}

For $k\ge 0$, we write $t \nsafePReducesMany{n}{k} t'$ if and only if
there is a $k$-step safe parallel reduction sequence
$t=t_0 \nsafePReduces{n} t_1 \nsafePReduces{n} \cdots \nsafePReduces{n} t_k = t'$. 
The pentagon property also holds for the $k$-step safe parallel reduction. 

\begin{lemma}[Many step pentagon property]\label{lem:parallel_pentagon_many}
  Let $t \in \GTC$ and $n,k,k_1,k_2\ge 0$. 
  \begin{enumerate}
  \item\label{lem:parallel_pentagon_many1}
    If $t_1 \nsafePReducesL{n} t \nsafePReducesMany{n}{k} t_2$,
    then there exist $u_1',u_2'\in \GTC$
    such that $t_1 \nsafePReducesMany{n}{k} t_1' \nequal{n} t_2' \nsafePReducesL{n} t_2$.
  \item\label{lem:parallel_pentagon_many2}
    If $t_1 \nsafePReducesManyL{n}{k_1} t \nsafePReducesMany{n}{k_2} t_2$,
    then there exist $t_1',t_2'\in \GTC$
    such that $t_1 \nsafePReducesMany{n}{k_2} t_1' \nequal{n} t_2' \nsafePReducesManyL{n}{k_1} t_2$.
  \end{enumerate}
\end{lemma}
\begin{proof}
  The point~\ref{lem:parallel_pentagon_many1} is shown by induction on $k$.
  The base case $k=0$ is immediately shown by taking $t_1$ as both $t_1'$ and $t_2'$.
  We show the case $k>0$.
  Let $t \nsafePReducesMany{n}{k-1} t_* \nsafePReduces{n} t_2$.
  By the induction hypothesis, there are $t_1'$ and $t_*'$ such that 
  $t_1 \nsafePReducesMany{n}{k-1} t_1' \nequal{n} t_*' \nsafePReducesL{n} t_*$.
  Then, by $t_*' \nsafePReducesL{n} t_* \nsafePReduces{n} t_2$ and the pentagon property,
  we have $t_*' \nsafePReduces{n} t_*'' \nequal{n} t_2'' \nsafePReducesL{n} t_2$ for some $t_*''$ and $t_2''$. 
  Hence, by $t_1' \nequal{n} t_*' \nsafePReduces{n} t_*''$ and Lemma~\ref{lem:parallel_eq},
  we have $t_1' \nsafePReduces{n} t_1'' \nequal{n} t_*''$ for some $t_1''$.
  Therefore, using the transitivity of $\nequal{n}$,
  we obtain $t_1 \nsafePReducesMany{n}{k} t_1'' \nequal{n} t_2'' \nsafePReducesL{n} t_2$ as we wished. 
  
  The point~\ref{lem:parallel_pentagon_many2} is shown by induction on $k_1$.  
  The base case $k_1=0$ is immediately shown by taking $t_2$ as both $t_1'$ and $t_2'$.
  To show the inductive case $k>0$, let $t \nsafePReducesMany{n}{k_1-1} t_* \nsafePReduces{n} t_1$.
  Then, by the induction hypothesis,
  $t_* \nsafePReducesMany{n}{k_2} t_*' \nequal{n} t_2' \nsafePReducesManyL{n}{k_1-1} t_2$ holds
  for some $t_*'$ and $t_2'$. 
  By $t_1 \nsafePReducesL{n} t_* \nsafePReducesMany{n}{k_2} t_*'$ and the point~\ref{lem:parallel_pentagon_many1},
  we have $t_1 \nsafePReducesMany{n}{k_2} t_1'' \nequal{n} t_*'' \nsafePReducesL{n} t_*'$
  for some $t_1''$ and $t_*''$. 
  Then, by $t_2' \nequal{n} t_*' \nsafePReduces{n} t_*''$ and Lemma~\ref{lem:parallel_eq},
  $t_2' \nsafePReduces{n} t_2'' \nequal{n} t_*''$ for some $t_2''$.
  Hence, using the transitivity of $\nequal{n}$, we have
  $t_1 \nsafePReducesMany{n}{k_2} t_1'' \nequal{n} t_2'' \nsafePReducesManyL{n}{k_1} t_2$ as we wished.
\end{proof}

\noindent\textbf{Proof of the weak Church-Rosser of $n$-safe reduction.}

Assume that $t\in \GTC$ and $t_1 \nsafeReducesAstL{n} t \nsafeReducesAst{n} t_2$. 
By Lemma~\ref{lem:parallel_basic}.\ref{lem:parallel_basic2},
we have $t_1 \nsafePReducesManyL{n}{k_1} t \nsafePReducesMany{n}{k_2} t_2$ for some $k_1$ and $k_2$.
Then there exist $t_1', t_2' \in \GTC$ such that
$t_1 \nsafePReducesMany{n}{k_2} t_1' \nequal{n} t_2' \nsafePReducesManyL{n}{k_1} t_2$.
By Lemma~\ref{lem:parallel_basic}.\ref{lem:parallel_basic3},
$t_1 \nsafeReducesAst{n} t'_1 \nequal{n} t'_2 \nsafeReducesAstL{n} t_2$ holds as we wished.




\newpage
% Section 8
\section{Terms with the Global Trace Condition are Finite for Safe Reductions}

\label{section-finite-safe-reductions}
In this section we prove that for all $n \in \N$, every infinite reduction sequence $\pi$ 
from some term of $\GTC$ includes
only finitely many \quotationMarks{safe} reduction steps.    
In particular,
this implies that in $\GTC$ the reduction sequences made of only \quotationMarks{$n$-safe} reductions 
are finite. That is, this implies strong normalization for \quotationMarks{$n$-safe} reduction steps:
no matter how we reduce within the safe level of a term, eventually we obtain some safe normal form.
\\

We introduce the property of being \quotationMarks{finite for $n$-safe reductions}.

\begin{definition}
\label{definition-finite-n-safe-reduction}
Assume $t \in \CTlambda$. We say that $t$ is \emph{finite for $n$-safe reductions} if all infinite
reduction paths include only finitely many $n$-safe reductions.
\end{definition}

In the following, we explicitly write $t[x_1,\ldots,x_n]$,
when each free variable in a term $t$ is some $x_i$, 
and, under this notation, we also write $t[a_1,\ldots,a_n]$ instead of $t[a_1/x_1,\ldots,a_n/x_n]$. 

It is enough to consider the case $n=0$. 
We abbreviate \quotationMarks{$0$-safe} with \quotationMarks{safe}.
We have to prove that all terms in $\GTC$ are finite for safe reductions, for all $n \in \N$.
For, we define total well-typed terms by induction on the type, as in Tait's normalization proof.

%We recall that $u \in \LAMBDA$ is \emph{finite for safe reduction} 
%if and only if all infinite reduction sequences from $t$ include only finitely many "safe" reduction steps
%(Def. \ref{definition-safe-trunk}).

\begin{definition}[Total well-typed terms of $\LAMBDA$]
\label{definition-total-term}
Let $t \in \WTyped$ and $t : T$.
We define \quotationMarks{$t$ is total of type $T$} by induction on $T$

\begin{enumerate}
\item
Let $T$ be any atomic type: $T = \N$ or $=\alpha$ for some type variable $\alpha$.
Then $t$ is total of type $T$ if and only if $t$ is finite for safe reductions, for all $n \in \N$.

\item
Let $T$ be any arrow type $A \rightarrow B$.
Then $f$ is total of type $T$ if and only if for all total $a$ of type $A$ we have $f(a)$ total of type $B$.
\end{enumerate}

A \emph{total assignment} $[\vec{x}/\vec{v}]$ 
is any assignment of total terms to variables of the same type.

A term $t$ is \emph{total by substitution} if and only if:
$t[\vec{x}/\vec{v}]$ is total for all total assignments $[\vec{x}/\vec{v}]$
\end{definition}

By definition, any total term is well-typed, in particular it has exactly one type. 

We define a well-founded relation predecessor relation on terms finite for safe reductions and of type $\N$.

\begin{definition}[The $\Succ$-order]
Assume $t, u \in \WTyped$ and $t, u :\N$ (possibly open terms).
Then:
$$
(u \prec t) \Leftrightarrow (t \reduces^* \Succ(u))
$$
\end{definition}


\begin{Eg}
If $t = \Succ^2(x)$ then $x \prec \Succ(x) \prec t$. 
If $t = \Succ(t)$ then $t \prec t$ and $\prec$ is \emph{not} well-founded on $t$:
\end{Eg}

We did not prove Church-Rosser yet, 
therefore we ignore whether all decreasing sequences of $\prec$ have the same length. 

However, we can prove that if $t$ is a term finite for safe reductions, 
then any $\prec$-decreasing sequence from $t$ terminates.


\begin{lemma}[The $\prec$-order]
\label{lemma-prec-order}
Assume $t \in \WTyped$ has type $\N$, and $t$ is finite for safe reductions .

\begin{enumerate}
\item
\label{lemma-prec-order-01}
There is no infinite sequence 
$\sigma: t = t_0 \reduces \Succ(t_1) \reduces \Succ^2(t_2) \reduces \ldots$

\item
\label{lemma-prec-order-02}
$\prec$ is well-founded on total terms of type $\N$.
\end{enumerate}
\end{lemma}


\begin{proof}
\begin{enumerate}
\item
%\label{lemma-prec-order-01}
$t$ is finite for safe-reductions, therefore
$\sigma$ only has finitely many safe-reductions. 
Thus, from some $k\in\N$ there are no more safe reductions from
$\Succ^k(t_k)$. This implies that for some $\cond$-free term 
$u$ and some terms $f_1, g_1, \ldots, f_m, g_m$ we have
$t_n = u[\cond(f_1,g_1), \ldots, \cond(f_m,g_m)]$ and all reductions from $t_k$ on are inside
some $g_1, \ldots, g_m$. This means for all $h \in \N$, $h \ge k$ we  have
$\Succ^{h}(t_{h}) =  u[\cond(f_1,g'_1), \ldots, \cond(f_n,g'_m)]$ for some 
$g'_1, \ldots, g'_m$. This implies that first $h$ symbols of $u$ are $\Succ$.
This is a contradiction when $h$ is larger than the number of symbols in $u$.

\item
%\label{lemma-prec-order-02}
Assume for contradiction that there is some infinite sequence
$\ldots t_n \prec \ldots \prec t_2 \prec t_1 \prec t_0$
from some $t_0:\N$ total. By definition, $t_0$ is finite for safe reductions.
Then there is some infinite sequence 
$\sigma: t = u_0 \reduces \Succ(u_1) \reduces \Succ^2(u_2) \reduces \ldots$,
contradicting point \ref{lemma-prec-order-01} above.
\end{enumerate}
\end{proof}


We will prove that if $t$ is not total, then we can assign total terms to  the sub-terms of $t$
in an infinite path of a proof $\Pi : \Gamma \vdash t: A$ in a way compatible with traces.


\begin{definition}[Trace-compatible Assignment]
\label{definition-trace-compatible}
Assume $\pi  = (\Gamma_1 \vdash t_1:A_1, \ldots, \Gamma_n \vdash t_n:A_n, \ldots)$ 
is any finite or infinite branch of a typing proof $\Pi$
and $\vec{v} = (\vec{v_1}, \ldots, \vec{v_n}, \ldots)$ 
is any sequence of total assignments, one on each $t_i$. 
$\vec{v}$ is \emph{trace-compatible} in an index $i$ of $\pi$  
if and only if it satisfies the following condition:

  for all $j$  index of an $\N$-argument of $t_i$, 
  all $k$ index of an $\N$-argument of $t_{i+1}$, 
  if $j$ is connected to $k$ then:
 \begin{enumerate}
 \item
 if $j$ progresses to $k$ then $v_j \prec v_k$ 
 \item
 if $j$ does not progress to $k$ then $v_j = v_k$.
 \end{enumerate}
$\vec{v}$ is \emph{trace-compatible} if it is trace-compatible in all $i$.
\end{definition}

Now we will prove that if an infinite branch $\pi$ of a proof tree $\Pi:\Gamma \vdash t:A$ 
has a trace-compatible assignment made of total terms, 
then all traces $\sigma$ of $\pi$ progress only finitely many times and the term $t$ is not in $\GTC$
($t$ does not satisfy the global trace condition).



\begin{proposition}[Trace assignment]
\label{prop:trace_assign-finiteness}
Assume $\Pi:\Gamma \vdash t:A$ and there is some infinite path $\pi$ of $\Pi$ for which we have
some \emph{total} trace-compatible assignment  $\rho$ to $\pi$. 
\begin{enumerate}
\item
\label{prop:trace_assign-finiteness1}
Any trace $\sigma$ in $\pi$ progresses only finitely many times.
\item
\label{prop:trace_assign-finiteness2}
$t \not \in \GTC$.
\end{enumerate}
\end{proposition}



\begin{proof}
\begin{enumerate}
\item
%\label{prop:trace_assign-finiteness1}
By definition of trace-compatible assignment, if at step $i \in \N$ the trace $\sigma$ progresses, 
then $\sigma(i+1)\prec \sigma(i)$, and if $\sigma$ does not progress, 
then $\sigma(i+1) = \sigma(i)$
The assignment is made of total terms, therefore
 $\prec$ is well-founded by lemma \ref{lemma-prec-order}.\ref{lemma-prec-order-02}.
Thus, any  trace in $\pi$ progresses only finitely many times, as we wished to show.

\item
%\label{prop:trace_assign-finiteness2}
By point \ref{prop:trace_assign-finiteness1} above, no trace $\sigma$ 
from any argument in any term of the branch $\pi$ of $\Tree(t)$ progresses infinitely many times.
We assumed that $\pi$ is an infinite path in $\Pi$.
By definition of $\GTC$, we conclude that $t \not \in \GTC$. 
\end{enumerate}
\end{proof}

%16:46 04/09/2024

We now continue with our Tait's-style proof of Strong Normalization.
We check that total terms are closed by reductions, by application and by variables.
Any total term is finite for safe reductions.



\begin{lemma}\label{lem:total_value-finiteness}
Assume $t,u,f,a \in \WTyped$, $n \in \N$ and $A,B,T$ are types.

  \begin{enumerate}
  \item
\label{lem:total_value-finiteness1}
    Let $t:A$ and $t \reduces^* u$.
    If $t$ is total, then $u$ is total.

  \item
\label{lem:total_value-finiteness2}
    If $f:A \rightarrow B$ and $a:A$ are total terms, then $f(a)$ is total.

  \item
\label{lem:total_value-finiteness2bis}
    $x^T:T$ is total.

 \item
\label{lem:total_value-finiteness2ter}
  If $t$ is total then $t$ is  finite for safe reductions.

  \item
\label{lem:total_value-finiteness3}
    Let $T$ be any atomic type, and $t[\vec{x}]:\vec{A}\rightarrow T$ be a term
    whose all free variables are $\vec{x}:\vec{B}$.

    If for all total $\vec{u}:\vec{B}$, $\vec{a}:\vec{A}$ the term 
    $t[\vec{u}]\vec{a}:\N$ is \emph{finite for safe reductions}, then
    the term $t[\vec{x}]$ is \emph{total by substitution}.
  \end{enumerate}

\end{lemma}




\begin{proof}
\begin{enumerate}

\item
%\label{lem:total_value-finiteness1}
  We show \emph{point \ref{lem:total_value-finiteness1}}  by induction on $A$. 
  We assume that $t:A$ and $t \reduces ^*u$.
    and $t$ is total, in order to prove that $u$ is total.
 By the subject reduction property, $u$ has type $A$.
\begin{enumerate}
\item
  We show the \emph{base case}, namely when $A =\N,\alpha$ is an atomic type.
  By the assumption, $t$ is total.
  By definition of $t$ total for $T$ atomic, all infinite 
  reductions from $t$ only include finitely many safe reductions, for all $n \in \N$.
  In particular, all infinite reductions $\sigma: t \reduces \ldots \reduces 
  u \reduces u_1 \reduces u_2 \ldots$ 
  passing through $u$ only include finitely many safe reductions. We conclude that
  all infinite reductions 
  $\sigma': u \reduces u_1 \reduces u_2 \ldots$  from $u$
  only include finitely many safe reductions. From $u:A =\N,\alpha$ we conclude that  $u$ is total.
\item
  We show the \emph{induction case}, namely when $A = (A_1\rightarrow A_2)$.
  Take any arbitrary total term $a:A_1$ in order to prove that $u(a):A_2$ is total. 
  Then we have $t(a) \reduces^* u(a)$ and 
  $t(a):A_2$ is total by the assumption that $t$ is total.
  Hence $u(a)$ is total by $t(a) \reduces^* u(a)$ and the induction hypothesis on $A_2$. 
  We conclude that $u:A_1\rightarrow A_2$ is total. 
\end{enumerate}

  \item
%\label{lem:total_value-finiteness2}
If $f:A \rightarrow B$, $a:A$ are total  terms, then $f(a)$  is total by definition of total.

\item
%\label{lem:total_value-finiteness3}
We prove that $x^T:T$ is total. 
We actually prove a little more, 
that for all total $\vec{a}:\vec{A}$, if $T = \vec{A} \rightarrow U$ then $x(\vec{a}):U$
is total. The thesis follows if we take $\vec{a} = \nil$. We argue by induction on $U$. 

\begin{enumerate}
\item
Assume $U$ is atomic. Then every reduction on $x(\vec{a}):U$
takes place in $\vec{a}$. By definition of total
for an atomic type we have to prove that in all infinite reduction sequences from $x(\vec{a}):U$ 
there are only finitely many safe reduction. All reductions on $x(\vec{a})$ take place on $\vec{a}$,
and since each $a_i$ in $\vec{a}$ is total only finitely many safe reductions are possible, as we wished.
\item
Assume $U = A_1 \to A_2$. By definition of total
for an arrow type we have to prove that for all total $a$ we have  $x(\vec{a},a):A_2$ total.
This follows by induction hypothesis on $A_2$.
\end{enumerate}

\item
%\label{lem:total_value-finiteness4}
\emph{We assume that  $t:U$ is total in order to prove that $t$ is finite for safe reductions},
for all $n \in \N$.
We argue by induction on $U$.
\begin{enumerate}
\item
If $U$ is atomic then the thesis is true by definition of total.
\item
Suppose $U = (A_1 \rightarrow A_2)$. By point \ref{lem:total_value-finiteness3} above,
$x^{A_1}:A_1$ is total, therefore $t(x):A_2$ is total and by induction hypothesis on $A_2$
any infinite reduction sequence from $t(x)$ only includes finitely many safe reductions. 
Any infinite reduction sequence 
$\sigma: t = t_0 \reduces_1 t_1 \reduces_1 t_2 \reduces_1 \ldots$  from $t$ 
can be raised to an infinite reduction sequence 
$\tau: t = t_0(x) \reduces_1 t_1(x) \reduces_1 t_2(x) \reduces_1 \ldots$ from $t(x)$
while preserving the fact that a reduction is safe, because $t$ occurs in no $\cond$ in $t(x)$.
We conclude that $\sigma$ only includes finitely many safe reductions. 
\end{enumerate}

\item  
%\label{lem:total_value-finiteness5}
We show that $t[\vec{u}]$ is total by induction on the number $|\vec{A}|$ of
elements of $\vec{A}$.
\begin{enumerate}
\item
  The \emph{base case} $|\vec{A}| = 0$ is immediately shown by the assumption.
\item
  We show the \emph{induction case}. Let $\vec{A} = A_0,\vec{A'}$.
  Take arbitrary values $\vec{u}:\vec{B}$, $\vec{a'}:\vec{A'}$, and $a_0:A_0$. 
  By the assumption, we have that $t[\vec{u}]a_0\vec{a'}:\N$ is total  for all 
  vector of total terms $\vec{a'}$. 
  Then $t[\vec{u}]a_0:\vec{A'}\rightarrow\N$ is total for all total $a_0:A_0$
  by the induction hypothesis on $\vec{A'}\rightarrow \N$.
  By definition of total we deduce that $t[\vec{u}] : A_0,\vec{A'}\rightarrow\N$ is total.
\end{enumerate}
We conclude that $t[\vec{x}]$ is total by substitution, as we wished to show.

\end{enumerate}
\end{proof}

%10:28 19/04/2024
%17:32 24/04/2024

Let $\Pi=(T,\phi)$ be a proof of $\Gamma\vdash t:A$ and $l_n$ be a node of $\Pi$, that is, a list  
of integers $l_n=(e_1, \ldots, e_{n-1}) \in \universe{\Pi}$ for some $n \in \N$.
We write $\Gamma_{n}\vdash t_{n}:A_{n} = \Label(\Pi,l_n)$ for the sequent
 labelling the node $l_n$. We want to prove that all terms of $\GTC$ are total by substitution.
If we consider the substitution of a variable with itself 
(a variable is a total term by \ref{lem:total_value-finiteness}.\ref{lem:total_value-finiteness3}),
we will deduce that all all terms of $\GTC$ are total, 
hence finite for safe reductions, hence strongly normalizing for safe reductions.

One problem in the proof of the theorem 
is that reduction does not commute with the second argument of substitution. 
That is, if $a \reduces^* a'$ we cannot deduce that $v[a] \reduces^* v[a']$. 
The reason is that we could have infinitely many free occurrences of $a$ in $v[a]$, and it could take
an infinite number of steps to reduce each of these $a$ to $a'$.

However, we can prove a weaker property, expressed by the following Lemma.


\begin{lemma}[Safe-infinite reductions and Substitution]
 \label{lemma-safe-infinite-substitution}
 Assume $a \reduces^* a'$ and there is some safe-infinite reduction $\sigma$ from $v[a']$.
 Then there is some safe-infinite reduction from $v[a]$.
\end{lemma}

\begin{proof}
We prove that if $v[a'] \reduces w$ then for some $z$ we have $w=z[a']$ and $v[a] \reduces^+ z[a]$,
and if $v[a'] \reduces z[a']$ is a safe then the last reduction in $v[a] \reduces^+ z[a]$ is safe.
The proof is by cases on the reduction $v[a'] \reduces w$.  We argue by cases.

\begin{enumerate}

\item
Assume the reduction $v[a'] \reduces w$ is on a redex of the form $r[a']$, with both the left
and right hand-side of $r$ not included in any $a'$ obtained by substitution. 
In this case we have $v[a'] = e[r[a'],a']
\reduces e[s'[a],a']$, and
$r[a'] = \cond(f[a'],g[a'])(0) \reduces f[a'] = s[a']$ or 
$r[a'] = \cond(f[a'],g[a'])(S(t[a])) \reduces g[a'](t[a']) = s[a']$
or $r[a'] = (\lambda x.t[a',x])(u[a']) \reduces t[a',u[a']] = s[a']$. 
We define $z[a'] = e[s'[a],a']$. 
In all three sub-cases we have $r[a] \reduces s[a]$ and therefore $v[a] \reduces e[r[a],a] = z[a]$
if $v[a'] \reduces z[a']$ is safe then $v[a] \reduces z[a]$ is safe.

\item
Assume the reduction $v[a'] \reduces w$ is on a redex of the form $r$, with either the left
or right hand-side of $r$ not included in some $a'$. In this case either $r$ is included in some $a'$ 
obtained by substitution, or the left or right hand side of $r$ is equal to some $a'$ obtained by substitution.
We cannot have both the left and right hand side equal to $a$, because the two sides of any redex 
have a different type.

In this case there is a single $a'$ whose value matters, that is, $v[a] = v_0[a][a]$, with the first $a$ denoting
the single occurrence of $a$ we are speaking about. We choose $v'[.] = v_0[a'][.]$, then we have
$v'[a']=v_0[a'][a']=v[a']$ by definition, and $v[a] = v_0[a][a] \reduces^* v_0[a'][a] = v'[a]$ 
because $a \reduces^* a'$ and there is a single occurrence of $a$ in $v_0[a][.]$. 

The reduction on $v'[a']$ has both the left
and right hand-side of $r$ not included in any $a'$ obtained by substitution, because this unique
$a'$ is not part of $v'[.]$. We conclude from the previous case applied to $v'[a]$, $v'[a']$.

\end{enumerate}
Then by induction on $n' \in \N$ we prove that if $v[a'] \reduces^{n'} w$ ($n$ steps) 
then for some $z$, some $n \ge n'$ we have $w=z[a']$ and $v[a] \reduces^{n} z[a]$,
with the number of safe reductions in  $v[a'] \reduces^{n'} z[a']$ less or equal than the number
of safe reductions in $v[a] \reduces^{n} z[a]$.

\end{proof}

The Lemma above will be enough to prove our Main Theorem.

%08:13 05/09/2024

\begin{theorem}[Main Theorem]
\label{theorem-main-finite-safe-reduction}
  Assume $\Pi:\Gamma\vdash t : A$ (hence $t\in \WTyped$).
  If $t$ is \emph{not} total by substitution, then $t \not \in \GTC$, i.e.:
  there is some infinite path $\pi = (e_1, e_2, \ldots)$ of $\Pi$ with no infinite progressing trace. 
\end{theorem}

%19:01 16/04/2024
%18:11 30/04/2024
%13:39 04/06/2024


\begin{proof}
  Assume that $t$ is not total by substitution. 
  Let $\vec{x}:\vec{D}$ and $\vec{A}\rightarrow U$ for some \emph{atomic} $U$ 
 be $\Gamma$ and $A$, respectively.
  By Proposition \ref{prop:trace_assign-finiteness}.\ref{prop:trace_assign-finiteness2} it is enough to prove that
  $\Pi$ has some infinite branch $\pi=(e_1, e_2, \ldots)$ 
  and some \emph{total} trace-compatible assignment $\rho$ for $\pi$.
  By case \ref{lem:total_value-finiteness3} of Lemma~\ref{lem:total_value-finiteness},
  there exist total terms $\vec{a}:\vec{A}$ and $\vec{d}:\vec{D}$ for which there is
  an infinite reduction $\sigma$ from $t[\vec{d}]\vec{a}$ having infinitely many safe reductions.
  By induction on $i \in \Nat$, for each $i$, we construct a path 
  $l_i = (e_1,\ldots,e_{i-1}) \in \universe{\Pi}$
  and a total assignment $\vec{v_i} = (\vec{d_i},\vec{a_i})$ such that
  $(\vec{v_1},\ldots,\vec{v_i})$ is a total trace-compatible assignment for the node $l_i$. 

%  We have to find some $(l_i,\vec{d_i},\vec{a_i})$ such that:
%  \begin{itemize}
%  \item[(i)]
%%15:38 19/04/2024
%   $l_i = (e_1, \ldots, e_{i-1}) \in \universe{\Pi}$ 
%     %and $\Label(\Pi,l_i) = \vec{x_{i}}:\vec{D_{i}}\vdash t_{i} : \vec{A_{i}}\rightarrow\N$; 
%%  \item[(ii)]
%%    $t_{i}$ is not total by substitution;
%  \item[(ii)]
%    $\vec{d_{i}}:\vec{D_i}$ and $\vec{a_i}:\vec{A_i}$ are total terms
%    such that $t_{i}[\vec{d_i}]\vec{a_i}$ is not total;
%  \item[(iii)]
%    the total assignment 
%    $\vec{d_1},\vec{a_1}$, \ldots, $\vec{d_i},\vec{a_i}$) is trace-compatible for $l_i$.
%    %\Daisuke{mynote:write this more clearly}
%  \end{itemize}
  
 We recall that \quotationMarks{trace compatible in $i$} means: 
 if $i-1$ is a progress point, namely if $t_{i-1}=\cond(f,g)$ and $e_i=2$ 
    and $t_i=g$,
    then $\vec{a_i} = a',\vec{a'}$ 
    and $a'$, the first unnamed argument of $\cond(f,g):\N \rightarrow A$, reduces to $\Succ(a'')$
     while $\vec{d_i} = \vec{d_i},a''$, 
     the last named argument of $g$ is $1$ unit smaller.
  In all other cases two corresponding arguments are equal.

  We first define $(l_1,\vec{d_1},\vec{a_1})$ for the root node $t$ of $\Pi$.
  In this case $l_1 = \nil$ 
   and $(\vec{d},\vec{a})$ are total terms such that $t[\vec{d}](\vec{a})$ is not total.
  Trace compatibility is vacuous because the branch $l_1$ does not contain two nodes.
  %Points (i), (ii), (iii), (iv) are immediate.

  Next, assume that $(l_i,\vec{d_i},\vec{a_i})$ is already constructed.
  Then we define $(l_{i+1},\vec{d_{i+1}},\vec{a_{i+1}})$ by the case analysis on
  the last rule for the node $l_i$ in $\Pi$. 

%%%%%%%%%%%%%%%%%%%%%%%%%%%%%%%%
% APPARENTLY THE CASE $\struct(f)$ CAN BE SIMPLIFIED TO WEAKENING-Stefano
%%%%%%%%%%%%%%%%%%%%%%%%%%%%%%%%



\begin{enumerate}

\item
%WEAK
  The case of $\weak$, namely
  $\Label(\Pi,l_i) = \Gamma'\vdash t_{i+1}:\vec{A_i}\rightarrow\N$
  is obtained from the induction hypothesis for
  $\Gamma\vdash t_{i}:\vec{A_i}\rightarrow\N$. We have:

\begin{enumerate}
\item
 $t_i = t_{i+1}$. 
\item
  $\Gamma = x_1:D_1,\ldots,x_n:D_n$, $\Gamma' = x'_1:D'_1,\ldots,x'_m:D'_m$, and $\Gamma \subseteqsim \Gamma'$
  with an injection $\phi:\{1,\ldots,n\}\to\{1,\ldots,m\}$ between contexts. 
\item
  $x_i = x'_{\phi(i)}$ and $D_i = D'_{\phi(i)}$ for all $i \in \{1,\ldots,n\}$.
\end{enumerate}

  By the induction hypothesis, 
  $t_i[d_{i,\phi(1)}/x_1,\ldots,d_{i,\phi(n)}/x_n]\vec{a_i} 
   = t_i[d_{i,1}/x'_1,\ldots,d_{i,m}/x'_m]\vec{a_i}$ is not total,
  where $\vec{d_i} = d_{i,1}\ldots d_{i,m}$.
  Then we define $l_{i+1} = l_i\conc(1)$ taking the unique child node of $l_i$ in $\Pi$, and we
  also define $\vec{d_{i+1}}$ and $\vec{a_{i+1}}$ by $d_{i,\phi(1)}\ldots d_{i,\phi(n)}$
  and $\vec{a_i}$, respectively. This is an assignment with total terms.
  $((\vec{d_i},\vec{a_i}),(\vec{d_{i+1}},\vec{a_{i+1}}))$
  is trace compatible for $(t_i,t_{i+1})$: if two arguments are connected then they are assigned
  to the same term. 

\item
%VAR
  The case of $\var$-rule, namely $\Label(\Pi,l_i) = \Gamma\vdash x:D$ 
  for some $x_{i,k}:D_{i,k} = x:D \in \Gamma$
  cannot be, because $t_i [\vec{d_i}/\vec{x_i}] = d_{i,k}$ is total  by assumption on $\vec{d}$.
  
\item
%0-RULE
  The case of $0$-rule, namely $\Label(\Pi,l_i) = \Gamma\vdash 0:\N$, 
  cannot be. Indeed, $t_i = 0$ is total because $0$ is a numeral.

%12:58 06/06/2024

\item 
%SUCC 
  The case of $\Succ$-rule, 
namely $\Label(\Pi,l_i) = \Gamma\vdash t_i: \N = \Gamma\vdash \Succ(u): \N$
  for some $u$ is obtained from our assumptions on
  $\Gamma\vdash u: \N$. In this case $\vec{a_i}$ is empty, $t_{i+1}=u$, and
  by the induction hypothesis $\Succ(u)[\vec{d_i}]:\N$ has an infinite reduction with
  infinitely many safe reductions
  $\Succ(u) \reduces  \Succ(u_1) \reduces \Succ(u_2) \reduces \ldots$,
  all taking place on $u$.
  Then, by the definition of total, $u[\vec{d_i}] =t_{i+1}[\vec{d_i}] $ is not total, because the
 $u \reduces_1  u_1 \reduces_1 u_2 \reduces_1 \ldots$ is an  an infinite reduction with
  infinitely many safe reductions.

  We define $e_{i}=1$, 
  the index of the unique child node of $l_{i+1}=l_i\conc(1)$ in the $\Succ$-rule, and
  we also define $\vec{d_{i+1}} = \vec{d_i}$ and $\vec{a_{i+1}} = ()$. 
  This is an assignment with total terms and 
  trace compatible for $(t_i,t_{i+1})$: if two arguments are connected then they are assigned
  to the same term. 

%13:05 06/06/2024


\item
  The case of the $\apnotvar$-rule, namely 
  $\Label(\Pi,l_i) = \Gamma\vdash t_i: \vec{A}\rightarrow\N$, 
  with $t_i = f[\vec{x}](u[\vec{x}])$ for some $f$ and $u$.
  The premises of the $\apnotvar$-rule
   are $\Gamma\vdash f[\vec{x}]: B \rightarrow \vec{A}\rightarrow\N$ 
  and $\Gamma\vdash u[\vec{x}]: B$, where $u$ is \underline{not} a variable.
  By the induction hypothesis, there is some infine reduction from
  $t_i[\vec{d_i}]\vec{a_i} = f[\vec{d_i}](u[\vec{d_i}])\vec{a_i}:\N$ with infinitely many safe-reductions.
  We argue by cases on the statement: \emph{$u[\vec{d_i}]:B$ is total}.


%11:04 05/06/2024



\begin{enumerate}
\item
  We first consider the subcase: \emph{$u[\vec{d_i}]:B$ is total}.
  We define $b = u[\vec{d_i}]$, then $l_{i+1}=l_i\conc(1)$, taking the first premise of the rule,
  and we define $\vec{d_{i+1}} = \vec{d_i}$ and $\vec{a_{i+1}} = b,\vec{a_i}$. 
  This is an assignment with total values, and providing an infinite reduction with infinitely many safe
  reductions, as expected. 
  The connection from 
  $(\vec{d_i},\vec{a_i})$ to $(\vec{d_{i+1}},\vec{a_{i+1}}) = (\vec{d_i},b,\vec{a_i})$ is
  trace-compatible: all connected 
  $\N$-argument of $t_{i}=f(u)[\vec{x}]$ and $t_{i+1}[\vec{x}] = f[\vec{x}]$ are the same,
  because the only fresh argument of $f[\vec{x}]$ 
  is $b$ and no argument of $f(u)[\vec{x}]$ is connected to it.


\item
  Next we consider the subcase that \emph{$u[\vec{d_i}]:B$ is not total}.
  Let $B = \vec{C}\rightarrow\N$.
  By lemma \ref{lem:total_value-finiteness}.\ref{lem:total_value-finiteness3}
  there is a sequence of values $\vec{c}:\vec{C}$ and an infinite reductions from 
  $u[\vec{d_i}]\vec{c}:\N$ with infinitely many safe reductions.
  Define $l_{i+1}= l_i \conc (2)$ taking the second premise of the rule,
  and define $\vec{d_{i+1}} = \vec{d_i}$ and $\vec{a_{i+1}} = \vec{c}$. 
  This is an assignment with total terms providing an infinite reductions with 
  infinitely many safe reductions, as expected.
  The connection from 
  $(\vec{d_i},\vec{a_i})$ to $(\vec{d_{i+1}},\vec{a_{i+1}}) = (\vec{d_i},\vec{c})$ is
  trace compatible: all connected $\N$-argument of $t_{i} = f(u)[\vec{x}]$ and $t_{i+1}=u[\vec{x}]$ are 
  in $\vec{d_i}$ and therefore are the same, and no unnamed arguments in $\vec{a_i}$
  and $\vec{a_{i+1}}$ are connected each other.
 \end{enumerate}



\item
  The case of $\apvar$-rule, namely 
  $\Label(\Pi,l_i) 
  = 
  \Gamma \vdash f[\vec{x}](x): \vec{A}\rightarrow\N$ is obtained from
  $\Gamma \vdash f[\vec{x}]: D,\vec{A} \rightarrow \N$,
  where $\Gamma=x_1:D_1,\ldots,x_m:D_m$ and $x:D\in\Gamma$, therefore
  $x:D = x_k:D_k$ for some $k \in [1,m]$. 
  By the induction hypothesis, there is an infinite reduction from $f[\vec{d_i}]d_{i,k}\vec{a_i}:\N$
  with infinitely many safe reductions,
  where $\vec{d_i} = (d_{i,1},\ldots,d_{i,m})$. 
  Define $l_{i+1}=l_i\conc(1)$ as the unique child of $l_i$ in $\Pi$, and
  also define $\vec{d_{i+1}} = \vec{d_i}$ and $\vec{a_{i+1}} = d_{i,k},\vec{a_i}$. 
  This is an assignment with total terms providing an infinite reductions with 
  infinitely many safe reductions, as expected.
  The connection from
  $(\vec{d_i},\vec{a_i})$ to
  $(\vec{d_{i+1}},\vec{a_{i+1}}) = (\vec{d_i},d_{i,k},\vec{a_i})$
  is trace compatible: all connected $\N$-arguments in $\vec{d_i},\vec{a_i}$ and 
  $\vec{d_{i+1}},\vec{a_{i+1}}$ are the same.  
  The only difference between the two assignments
  is that, if $D_k = \N$, the value $d_{i,k}$ of type $\N$ for the variable $x_k$
  in $t_i[\vec{x}]=f[x_1,\ldots,x_k,\ldots,x_m](x_k)$ is duplicated to the term $d_{i,k}$ 
  assigned to the first unnamed argument of $ f[\vec{x}]$. 

%08:40 05/09/2024

\item
  The case of $\lambda$-rule, namely when a sequent
  $\Label(\Pi,l_{i+1}) = 
    \Gamma\vdash t_i : A, \vec{A} \rightarrow \N$ with $t_i = \lambda x^A.u[\vec{x},x]$
  is obtained from
  $\Gamma,x:A\vdash t_{i+1}[\vec{x},x^A]:\vec{A}\rightarrow\N$, 
  where $t_{i+1}[\vec{x},x]=u[\vec{x},x]$.
  By the induction hypothesis we have an infinite reduction sequence $\sigma$ from
  $t_i[\vec{d_i}]\vec{a_i} = (\lambda x.(u[\vec{d_i},x]))\vec{a_i}:\N$ with infinitely many safe
  reductions,
  where $\vec{a_i} = a,\vec{b}$. 
  Define $l_{i+1}=l_i\conc(1)$ as the unique child of $l_i$ in $\Pi$,
  and $\vec{d_{i+1}} = a,\vec{d_i}$  and $\vec{a_{i+1}} = \vec{b}$. 
  This is an assignment with total terms.
    The connection from 
  $(\vec{d_i},\vec{a_i})$ to $(\vec{d_{i+1}},\vec{a_{i+1}}) = (\vec{d_i},a, \ \vec{b})$ is
  trace compatible: all connected $\N$-argument of 
  $t_{i}=\lambda x^A.u[\vec{x},x]$ and $t_{i+1}=u[\vec{x},x]$ are 
  the same, except for the first unnamend argument $a$ of $\vec{a_i}$ which is moved to
  the last named argument of $u[\vec{x},x]$, with name $x$.
  
  We have to prove that there is some infinite reduction from 
  $t_{i+1}[\vec{d_{i+1}}](\vec{a_{i+1}})$ with infinitely
  many safe reductions. We argue by case.

\begin{enumerate}
\item
 Suppose all reductions in the infinite reduction $\sigma$ 
  are inside $\lambda x.(u[\vec{d_i},x])$ or $\vec{a_i}$.
 Then there are finitely many safe reductions in $\vec{a_i}$ because $\vec{a_i}$ is total.
 Thus, there are infinitely many safe reductions on the part of $\sigma$
  taking place on $\lambda x.(u[\vec{d_i},x])$.
  We conclude that there is an infinite reduction from $u[\vec{d_i},x]$ with infinitely many safe reduction.
  This is true for $u[\vec{d_i},a]$, too, because reductions and safe reductions are closed by substitution
  of $x$ with $a$.

%09:39 05/09/2024
%non è vero che a-->a' implica v[a] ---> v[a'] ma è vero che se da v[a'] ci sono infinite safe-reductions
% allora da v[a] ci sono infinite safe-reductions
\item
 Suppose there is some reduction in $\sigma$ contracting the first $\beta$-redex.
 Then  $(\lambda x.(u[\vec{d_i},x]))a\vec{b}$ reduces first to some
 $ (\lambda x.v[x])a'\vec{b'}$, then to $v[a']\vec{b'}$, with: 
 $u[\vec{d_i},x] \reduces^* v[x]$ and $a\reduces^* a'$ and $\vec{b} \reduces^* \vec{b'}$.
 Then the reduction sequence $\sigma$ continues 
  with some infinite reduction $\sigma'$ from $v[a']\vec{b'}$, including infinitely many safe reductions. 
 From $u[\vec{d_i},x] \reduces^* v[x]$ and $\vec{b} \reduces^* \vec{b'}$ we deduce that 
$$
        		t_{i+1}[\vec{d_i},a]\vec{b}  
\ \ \ 		=
\ \ \     u[\vec{d_i},a]\vec{b} 
\ \ \    \reduces^*
\ \ \    v[a]\vec{b}
\ \ \    \reduces^*
 \ \ \   v[a]\vec{b'}
$$ 
 We proved that there is a reduction $\sigma'$ including infinitely many safe reductions from  $v[a']\vec{b'}$.
 From $a \reduces^* a'$ and Lemma \ref{lemma-safe-infinite-substitution} 
 we deduce that there is a reduction including infinitely many safe reductions from  $v[a]\vec{b'}$.
 We conclude that the same holds from  $t_{i+1}[\vec{d_i},a]\vec{b}$.
\end{enumerate}

%13:55 06/06/2024


\item
  The case of $\cond$-rule, namely a sequent
  $\Label(\Pi,l_i) = \Gamma\vdash t_i:\N,\vec{A}\rightarrow\N$
  having $t_i[\vec{x}] = \cond(f[\vec{x}],g[\vec{x}])$,
  and obtained from 
  $\Gamma\vdash f[\vec{x}]:\vec{A}\rightarrow\N$
  and
  $\Gamma\vdash g[\vec{x}]:\N,\vec{A}\rightarrow\N$. 
  By the induction hypothesis, there is some infinite reduction sequence $\sigma$ from
  $t_i[\vec{d_i}]a \vec{b} = \cond(f[\vec{d_i}],g[\vec{d_i}])a\vec{b}:\N$
  with infinitely many safe reductions,
  where $\vec{a_i} = a,\vec{b}$. 
  We argue by case.

\begin{enumerate}
\item
  \emph{Suppose the reduction sequence $\sigma$ never contracts the leftmost $\cond$.}

  $a$, $\vec{b}$ are total, only finitely many reduction on them are possible.
  Then infinitely many safe reductions take place on $f[\vec{d_i}]$ or $g[\vec{d_i}]$. 
  All these safe reduction are in $g[\vec{d_i}]$, because inside $g[\vec{d_i}$ 
  a reduction is in the right-hand-side of some $\cond$, therefore it is not safe.
  The safe-infinite reduction from $g[\vec{d_i}]$ can be raised to an infinite reduction from 
  $g[\vec{d_i}](\vec{b})$, with the same safe reductions, therefore with infinite safe reduction.
  In this case we take $g[\vec{x}],\vec{d_i},\vec{b}$ as next step of our path in $\Pi$:
  we choose $l_{i+1} = l_i \conc (1)$ and $\vec{d_{i+1}} = \vec{d_i}$,
  $\vec{a_{i+1}} = \vec{b}$.  This is an assignment with total terms.

  We have trace-compatibility because: 
  each argument of $t_i$ is connected to some equal argument of $t_{i+1}[\vec{x}]=f[\vec{x}]$,
  the first argument of $t_i[\vec{x}]$ disappears but it is connected to no $\N$-argument in $f[\vec{x}]$.

\item
  \emph{Suppose the reduction sequence contracts the leftmost $\cond$-redex at some step.}

  Then $t_i[\vec{d_i}]a \vec{b} \reduces \cond(u',v')a"\vec{b'}$, with $a"=0, \Succ(a')$,
  Then  in the first sub-case $\cond(u',v')a" \reduces u' \vec{b'}$, in the second sub-case
  $\cond(u,v)a" \reduces v' a' \vec{b'}$, 
  with $u \reduces^* u'$ and $v \reduces^* v'$ and $\vec{b} \reduces^* \vec{b'}$.
  After this $\cond$-reduction we have a reduction sequence with infinitely many safe reductions.
  We prove our thesis by sub-cases on $a"$.
   
%19:37 05/09/2024

\begin{enumerate}
\item
  \emph{Suppose $a" = 0$}.
   In this case we take $g[\vec{x}],\vec{d_i},\vec{b}$ as next step of our path in $\Pi$:
  we choose $l_{i+1} = l_i \conc (1)$ and $\vec{d_{i+1}} = \vec{d_i}$,
  $\vec{a_{i+1}} = \vec{b}$. This is an assignment with total terms.
  $f[\vec{d_{i+1}}](\vec{b})$ reduces to $u \vec{b'}$, then we have infinitely many safe reductions.

  We have trace-compatibility because: 
    each argument of $t_i$ is connected to some equal argument of $t_{i+1}[\vec{x}]=f[\vec{x}]$,
    the first argument of $t_i[\vec{x}]$ disappears but it is connected to no $\N$-argument in $f[\vec{x}]$.
%12:53 05/06/2024

 \item
  \emph{Suppose $a" = \Succ(a')$}. 
  We choose $l_{i+1} = l_i \conc (2)$ and $\vec{d_{i+1}} = \vec{d_i}$,
  $\vec{a_{i+1}} = a',\vec{b}$. This is an assignment with total terms: $a'$ is total because
  all reduction sequences from $a'$ can be raised from a reduction sequences from $a'' = \Succ(a')$
  with the same safe-reduction. From $a \reduces^* a''$ we deduce that
  there are finitely many safa-reductions from $a''$, therefore finitely many from $a'$.
 
  We have trace-compatibility because: 
  each argument of $t_i[\vec{x}]$ is connected to some equal argument of 
    $t_{i+1}[\vec{x}]=g[\vec{x}]$,
    but for the first unnamed argument $a$ of $t_i[\vec{x}]$ 
    which is connected the first unnamed argument of $g[\vec{x}]$.
    This is fine because in the second premise of a $\cond$ 
    the trace progresses and we have $a' \prec a$, because $a \reduces  a" = \Succ(a')$.

  \end{enumerate}
 \end{enumerate}
\end{enumerate}

By the above construction, we have an infinite path $\pi = (e_1,e_2,\ldots)$ in $\universe{\Pi}$
and a trace-compatible assignment, as we wished to show.

%  Since $\Pi$ satisfies the global trace condition, $\vec{e}$ contains a progressing trace
%  $(k_{a},k_{a+1},\ldots)$, where, for each $a\le i$, $k_i$ is an $\N$-argument of $t_i$, 
%  and $k_{i+1},t_{i+1}$ is the successor of $k_i,t_i$. 
%  Let $n_i$ be an numeral in $\vec{d_i},\vec{a_i}$ at index $k_i$ for each $a\le i$.
%  Then the sequence $(n_a,n_{a+1},\ldots$ decreases at each progressing point.
%  This means that it decreases infinitely many times
%  since $(k_{a},k_{a+1},\ldots)$ has infinitely many progressing point.
%  Finally we have a contradiction. 
  
\end{proof}

%14:09 06/06/2024

From this theorem we derive the strong normalization result for safe reductions on terms of $\GTC$
(on all terms which satisfy the global trace condition).

\begin{corollary}\label{cor:SN_GTC}
  \begin {enumerate}
  \item
   Assume  $\Gamma\vdash t:A$. Then $t \in \GTC$ implies that $t$ is total and all reduction sequences
   from $t$ include only finitely many $0$-safe reductions.
  
\item
   Assume  $\Gamma\vdash t:A$. Then $t \in \GTC$ implies that $t$ strongly $0$-safe-normalizes
   \end{enumerate}
\end{corollary}

There are closed normal term of $\N$, even regular terms, which are not numeral.
An example is $t = \cond(0, \lambda x^N.x^N)(t)$. $t$ has a unique type $\N$. $t$ is normal,
because $t$ is not $0$ nor a successor, therefore $t$ cannot be reduced, however $t$ is not a number.

Remark that $t$ is regular: the subterms of $t$
subterms are $t$ itself and$\cond(0, \lambda x^N.x^N)$, $0$, $\lambda x^N.x^N$.
However, $t$ is not in $\GTC$. Indeed, $t$ has an infinite path $t,t,t,t,\ldots$, never crossing a $\cond$,
and therefore never progressing.

We check that instead safe-normalization on closed terms of $\GTC$ of type $\N$ produce a numeral. 
As an intermediate step in this proof we need the notion of \qutationMarks{sound} term.

\begin{definition}[Sound terms]
A term $t \in \CTlambda$ is sound if at least one of the following conditions holds.
\begin{enumerate}
\item
\item
\item
\item
\end{enumerate}
\end{definition}

\begin{proposition}[Closed Safe-Normal terms of Type $\N$]
If $t \in \GTC$, $t$ is closed and $0$-safe-normal then $t \in \Num$ ($t$ is a numeral)
\end{proposition}

\begin

\newpage


\newpage
% Section 9


\newpage

\section{Infinite Church-Rosser for Infinite Lambda Terms}
\label{section-safe-church-rosser}

%15:12 16/04/2024
In this section prove a result for the safe part of a term, which we call 
\quotationMarks{\emph{unicity of the safe part of the safe normal form}}. By this we mean:
for all $t,u,v \in \LAMBDA$, if $t \reduces u$ and $t \reduces v$ and $u$, $v$ are safe-normal 
then $u$, $v$ are equal outside the right-hand side of all $\cond$-sub-terms.
We prove more results of a similar form.

%%%%%%%%%%%%%%%%%%%%%%%%%%%
%\ldots\ldots\ldots\ldots\ldots\ldots\ldots\ldots\ldots\ldots
%\\
%\bfColor{red}{(Here we should fill this part by adapting the proof of Church Rosser for the type $\N$- Stefano)}
%\ldots\ldots\ldots\ldots\ldots\ldots\ldots\ldots\ldots\ldots
%%15:46 24/04/2024
%%%%%%%%%%%%%%%%%%%%%%%%%%%

Our first idea (wrong) is to prove a full Church-Rosser property for $\LAMBDA$: 
for all $t,u,v \in \LAMBDA$, if $t \reduces u$ and $t \reduces v$ then for some $w \in \LAMBDA$
we have $u \reduces w$ and $v \reduces w$. This property is false: for some $t \in \LAMBDA$, finding a 
common reduction of $u$, $v$ takes infinitely many steps. This even in the case $t \in \CTlambda$,
as the next example shows.

\begin{Eg}[Failure of Church-Rosser for $\CTlambda$]
Let $b = \cond(x^{\N},b):\N \rightarrow \N$ a normal form
and $t = (\lambda x^{\N}.b)(r):\N \rightarrow \N$, 
where $r = \id(3)$ is some redex. 

We have 
$$b[r/x](n) \reduces b[r/x](n-1) \reduces \ldots \reduces b[r/x](0)\reduces r \reduces 3$$ 
for all numerals $n \in \Num$, 
therefore $b$ and $\lambda \_.3$ define the same map, constantly equal to $3$, 
however $b \not \reduces \lambda \_.3$.

We have $t \in \CTlambda$. Indeed, 
\begin{enumerate}
\item
$t$ is regular by construction.
\item
We have $t \in \GTC$, because the unique infinite path of $t$ is 
$t, \lambda x^{\N}.b, b, b, b, \ldots$, and the
unique unnamed argument of $b:\N \rightarrow \N$ in the path progresses infinitely many times.
\end{enumerate}

With a $\beta$-redudction on $t$ itself we obtain $t \reduces b[r/x^\N]$.
With a $\beta$ reduction $r \reduces 3$ we obtain $t \reduces  (\lambda x^{\N}.b)(3)$.
We expect $b[3/x^\N]$ as common normal form. But we have $b[r/x^\N] = \cond(r,b[r/x^\N])$,
that is, we have replicated the redex $r$ infinitely many times in $b[r/x^\N]$. Therefore to reduce 
$b[r/x^\N]$ to $b[3/x^\N]$ takes infinitely many $\beta$ reductions of the form $r \reduces 3$, 
and for \emph{no finite reduction we have} $b[r/x^\N] \reduces b[3/x^\N]$. 
$b[r/x^\N]$ is a term without normal form.

We proved that Church-Rosser is false for $\CTlambda$.
\end{Eg}

In order to recover Church-Rosser we use Barendregt's trick 
(page 282, Theorem 10.1.11) 
of labeling constructor in $\lambda$-calculus.
Apparently, the other Church-Rosser proof in the same book (page 61, Lemma 3.2.6)
is not suitable for 

We define $\CTlambdaLabelled$ as a version of $\CTlambda$ with $3$ applications: 
$\ap_0, \ap_1, \ap_2$. We assume the same reduction rules for all $\ap_i$, $i =0,1,2$.
$\ap_i(\lambda x.t,u) \reduces t[u/x]$, $\ap_i(\cond(f,g)(0)) \reduces f$ and
$\ap_i(\cond(f,g)(S(t))) \reduces \ap_0(f,t)$. We use the subscript $i\not = 0$ as a label to trace what
happens to a subterm $\ap_i(t,u)$ during a reduction. When we reduce a redex $\ap_i(\ldots)$,
the label $i$ on the first $\ap$-symbol of a redex disappears. 
If $r = \ap_i(\ldots)$ we say that $r$ is an $i$-application, an $i$-redex if $r$ is a redex.

Assume $X \subseteq \{0,1,2\}$ is a decidable set. 
We define a map  $\phi_X(t)$ reducing all $i$-redexes in $t$ for all $i \in X$. 
$\phi_X(t)$ is the limit of a family $\phi_{n,X}(t)$ for $n \rightarrow \infty$, defined by induction on $n$.
We set $\phi_{0,X}(t)=t$. If $n >0$ then $\phi_{n,X}(x^T)=x^T$, $\phi_{n,X}(0)=0$,
$\phi_{n,X}(\Succ(t)) = \Succ(\phi_{n,X}(t))$. If $r = \ap_i(t,u)$ then $\phi_{n,X}(r) =
\ap_i(\phi_{n-1,X}(t),\phi_{n-1,X}(u))$ if $i \not \in X$ or $r$ not $i$-redex.
If $i \in X$ and $r$ is a redex whose contraction is $s$ then $\phi_{n,X}(r) = \phi_{n-1,X}(s)$.
If $r = \cond(f,g)$ then $\phi_{n,X}(r) = \cond(\phi_{n-1,X}(f), \phi_{n-1,X}(g))$.

We define $\phi_X = \lim_{n \rightarrow \infty}(\phi_{n,X})$.
Both $\phi_{n,X}$ and $\phi_X$ are continuous.

We prove that if all $i$-applications of $t$ are redexes for all $i \in X \cup Y$, and $n \ge m$
then $\phi_{n,X}($\phi_{m,Y}(t)) = \phi_{n,X \cup Y}(t)$. Then proof is by induction on $m$.

By taking the limit we conclude that $\phi_{X}(\phi_{Y}(t)) = \phi_{X \cup Y}(t)$.
This implies Church-Rosser: if all $i$-applications of $t$ are redexes for all $i \in X \cup Y$,
$\phi_{X}(\phi_{Y}(t)) = \phi_{Y}(\phi_{X}(t))$, for all $t \in \CTlambdaLabelled$.

In particular, if $t \reduces^* u$ and $t \reduces^* v$, then $u$, $v$ have some limit form in common.

%%%%%%%%%%%%%%%%%%%%%%%%%%%%%%%%%%%%%%%%
%16:09 09/09/2024 this old version of Church-Rosser for infinite terms is commented
%%%%%%%%%%%%%%%%%%%%%%%%%%%%%%%%%%%%%%%%
%
%
%In order to recover Church-Rosser we have to consider a more general notion of reduction $\reduces_X$, 
%which allow to reduce \emph{infinitely many redexes in one step}: 
%all those in a \emph{decidable} set $X$ of redexes of $t$, and possibly all current redexes
%This reduction can generate new redexes of course. 
%
%We will prove that $\reduces_X$ is confluent: namely, we will prove that 
%if $t \reduces_X u$ and $t \reduces_Y v$, 
%then for some $w, Z, T$ we will have $u \reduces_Z w$ and $v  \reduces_T w$.
%We call this property Infinite Church-Rosser.
%Infinite Church-Rosser will imply the unicity of the safe part of the safe normal form.
%
%Our first problem is that infinite reductions can easily loop, therefore Infinite Church-Rosser is stated for the
%set $\LAMBDA_\bot$ of terms with possibly \emph{undefinite} subterms. 
%This is not an obstacle for the goals of this paper, as we will see.
%
%We formally state Infinite Church-Rosser as follows. 
%\quotationMarks{\emph{For all $t,u,v \in \LAMBDA_\bot$, all decidable sets $X$, $Y$ of redexes of $t$, 
%if $t \reduces_X u$ and $t \reduces_Y v$, 
%then for some $w \in \LAMBDA_\bot$, some decidable set $Z$, $T$ of redexes of $u$, $v$
%we have $u \reduces_Z w$ and $v  \reduces_T w$ and $t \reduces_{X \cup Y} w$}}.
%
%%08:00 10/06/2024
%
%We represent a decidable set $X$ of positions of redexes, and in fact any decidable set of sub-terms by a map 
%$\phi_X:\universe{t} \rightarrow \{\True,\False\}$ 
%such that $l \in X$ if and only if $\phi_X(l) =\True$. 
%Our first step is to precise how $\phi_X$ changes when redexes in $X$ are moved or duplicated.
%
%
%\begin{definition}[Substitution, subterms and labels]
%\label{definition-substitution-label}
%Suppose $t, u \in \LAMBDA_\bot$
%and $X$ is a decidable set of redexes of $t$ and $Y$ a decidable set of redexes of $u$.
%\begin{enumerate}
%\item
%$Z = [Y/x]$ is a a decidable set of redexes of $t[u/x]$ defined as:
%for all $l \in \universe{t}$, $m \universe{u}$:
%if $l$ is a free occurrence of $x$ we set $\phi_Z(l \conc m) = \phi_Y(m)$, 
%otherwise $\phi_Z(l \conc m) = \False$.
%
%\item
%$X[Y/x] = X \cup [Y/x]$.
%
%\item
%If $n=0,1,2$ and $t = c(t_1)\ldots(t_n)$ and $X$ a decidable set of redexes of $t$,
%then for all $1 \le i \le n$ we define a set $X_i$ of redexes in $t_i$ by $\phi_{X_i}(l) = \phi_X((i) \conc l)$.
% 
%\item
%If $n=0,1,2$ and $t = c(t_1)\ldots(t_n)$ 
%and for all $1 \le i \le n$ $X_i$ is a decidable set of redexes of $t_i$,
%then we define a set $X$ of redexes in $t$ by $\phi_X(\nil)=\False$
%and $\phi_X((i) \conc l) = \phi_{X_i}(l)$. 
%\end{enumerate}
%\end{definition}
%
%
%
%We define the \emph{unique} $u \in \LAMBDA_\bot$ such that 
%$t \reduces_X u$ as the limit of a map $\rho(t,X,n)$ for $n \rightarrow \infty$. 
%$\rho(t,X,n)$ starts with the undefined value $\bot$, then either holds the value $\bot$ forever,
%or at some step the root of the term $\rho(t,X,n)$ becomes some constructor of $\LAMBDA$ and never
%changes again. At each step, if $t$ itself is a redex in the set $X$ then we reduce it, 
%if $t$ is not a redex in $X$ then we move to the subterms of $t$. 
%In both case we update $X$ accordingly to some set of labels $X'$.
%We update any other set $Z$ of labels in $t$ in the same way to some $Z'$.
%We introduce a map $\sigma(t,X,n,Z)$ computing the  value for a set $Z$ of redexes after $n$ steps.
%In particular we have $X' = \sigma(t,X,1,X)$.
%
%
%\begin{definition}[Infinite reductions]
%\label{definition-infinite-reduction}
%Assume $t \in \LAMBDA_\bot$ and $X,Z$ are decidable sets of redexes of $t$.
%Let $X_1, Y_1, \ldots$ as in Def. \ref{definition-substitution-label}.
%
%We set $\rho(t,X,0)=\bot$. If  $\phi_X(t) = \True$, then we set:
%
%\begin{enumerate}
%\item
%If $t = (\lambda x^T.b)(a)$, 
%then 
%\begin{enumerate}
%\item
%$\rho(t,X,n+1) \ \ \ \ = \rho(b[a/x],X',n)$
%\item
%$\sigma(t,X,n+1,Z) = \sigma(b[a/x],X,n, \ Z')$
%\end{enumerate}
%where $Z' = (Z_1)_1[Z_2/x]$.
%
%\item
%If $t = \cond(f,g)(0)$, 
%then 
%\begin{enumerate}
%\item
%$\rho(t,X,n+1) \ \ \ \ = \rho(f,X',n)$
%\item
%$\sigma(t,X,n+1,Z) = \sigma(f,X,n, Z')$
%\end{enumerate}
%where  $Z' = (Z_1)_1$.
%
%\item
%If $t = \cond(f,g)(\Succ(u))$, 
%then 
%\begin{enumerate}
%\item
%$\rho(t,X,n+1)  \ \ \ \ = \rho(g(u),X',n)$ 
%\item
%$\sigma(t,X,n+1,Z) =  \sigma(g(u),X,n, Z')$
%\end{enumerate}
%where $Z' = \ap((Z_1)_2, (Z_2)_1)$.
%
%\end{enumerate}
%
%Assume  $\phi_X(t) = \False$ 
%and $t=c(t_1)\ldots(t_h)$ for some $h=0,1,2$ some $t_1, \ldots, t_h \in \LAMBDA_\bot$.
%Then we set
%\begin{enumerate}
%\item
%$\rho(t,X,n+1)  \ \ \ \ = c(\rho(t_1,X_1,n)\ldots(\rho(t_h,X_h,n)) $
%\item
%$\sigma(t,X,n+1,Z) = c(  \sigma(t_1,X_1,n,Z_1)\ldots \sigma(t_h,X_h,n,Z_h)  )$
% and $Z' = c(Z_1)\ldots(Z_h) = Z$.
%\end{enumerate}
%We define $\rho(t,X) = \lim_{n \rightarrow \infty} \rho(t,X,n)$ and 
%$\sigma(t,X,Z) = \lim_{n \rightarrow \infty} \sigma(t,X,n,Z)$
%and $t \reduces_X \rho(t,X)$.
%\end{definition}
%
%\begin{proposition}[Reduction and union]
%\label{lemma-reduction-union}
%Suppose $t, u \in \LAMBDA_\bot$
%and $X$ are decidable sest of redexes of $t$ and $Y$ are decidable sets of redexes of $u$
%and $l \in \universe{t}$. Let $X', Y', \ldots$ as in Def. \ref{definition-substitution-label}.
%\begin{enumerate}
%\item
%$[Z \cup T / x] = [Z / x] \cup [T / x] $
%\item
%$(X \cup Y)[Z \cup T / x] = X[Z / x] \cup Y[T / x] $
%\item
%$(X \cup Y)_l = X_l \cup Y_l$
%\item
%$c(X_1 \cup Y_1) \ldots (X_n \cup Y_n) = c(X_1) \ldots (X_n) \cup c(Y_1) \ldots (Y_n)$
%\item
%$(X \cup Y)' = X' \cup Y'$
%%\item
%%$\sigma(t,Z,n,X \cup Y) =\sigma(t,Z,n,X) \cup \sigma(t,Z,n,Y)$ for all $n \in \N$
%\end{enumerate}
%\end{proposition}
%
%We will prove that 
%for all $t,u,v \in \LAMBDA_\bot$, all decidable sets $X$, $Y$ of redexes of $t$ we have
%$\rho(\rho(t,X),Z)  = \rho(t,X \cup Y)$ for $Z = \sigma(t,X,Y)$.
%
%We order $\LAMBDA_\bot$ with $t \le u$ if and only if $\universe{t} \subseteq \universe{u}$
%and for all $l \in \universe{t}$ either $l$ has label $\bot$ in $\universe{t}$ or $l$ has the
%same label in $\universe{t}$ and $\universe{u}$.
%
%We prove $\rho(\rho(t,X),Z)  \le \rho(t,X \cup Y)$ (left-to-right implication)
%and $\rho(t,X \cup Y) \le \rho(\rho(t,X),Z)$ (right-to-left implication).
%
%
%\begin{lemma}[Infinite Church-Rosser (left-to-right)]
%\label{lemma-infinite-church-rosser-left}
%For all $t,u,v \in \LAMBDA_\bot$, all decidable sets $X$, $Y$ of redexes of $t$:
%$\rho(\rho(t,X),Z)  \le \rho(t,X \cup Y)$ for $Z = \sigma(t,X,Y)$.
%\end{lemma}
%
%
%\begin{proof}
%We have to prove that $\rho(\rho(t,X),Z)  = \rho(t,X \cup Y)$ for $Z = \sigma(t,X,Y)$.
%
%Let us abbreviate $Y_{(n)} = \sigma(t,X,n,Y)$ 
%and $Y_{(\omega)} = \lim_{n \rightarrow \infty} Y_{(n)} = Z$.
%then $\rho(\rho(t,X),Z) = \rho(\rho(t,X),Y_{(\omega)} )) $
% is the limit of $\rho(\rho(t,X,n),Y_{(n)},m)$ for $n,m \rightarrow \infty$.
%
%We prove that for all $n,m \in \N$
%there is some $p \in \N$ such that  $\rho(\rho(t,X,n),Y_{(n)},m)  \le \rho(t,X \cup Y,p)$,
%and conversely that for all $p \in \N$ there are $n,m\in\N$ such that 
%$\rho(t,X \cup Y,p) \le \rho(\rho(t,X,n),Y_{(n)},m)$. It will follow that if a node
%is defined in  $\rho(\rho(t,X),Z)$ then it is defined in $\rho(t,X \cup Y)$ and with the same 
%constructor, and conversely.
%
%
%$\rho(\rho(t,X,n),Y_{(n)},m)=\bot$ we are done, suppose 
%$\rho(\rho(t,X,n),Y_{(n)},m) > \bot$.
%If  $\phi_X(t) = \True$, then we have:
%
%\begin{enumerate}
%\item
%If $t = (\lambda x^T.b)(a)$, 
%then 
%$\rho(t,X,n) = \rho(b[a/x],X',n-1)$,
%and by induction hypothesis $\rho(\rho(t,X,n),Y_{(n)},m) = \rho(\rho(b[a/x],X',n-1),Y'_{(n-1)},m) 
%\le \rho(b[a/x], X' \cup Y', p) =  \rho(b[a/x],(X \cup Y)', p) 
%= \rho( (\lambda x^T.b)(a), X \cup Y, p + 1)$.
%
%\item
%If $t = \cond(f,g)(0)$, 
%then 
%$\rho(t,X,n+1) = \rho(f,X',n)$,
%and by induction hypothesis $\rho(\rho(t,X,n),Y_{(n)},m) = \rho(\rho(f,X',n-1),Y'_{(n-1)},m) 
%\le \rho(f, X' \cup Y, p) =  \rho(f, X' \cup Y', p) = \rho(f,(X \cup Y)', p) 
%= \rho( t, X \cup Y, p + 1)$.
%
%\item
%If $t = \cond(f,g)(\Succ(u))$, 
%then 
%$\rho(t,X,n+1)= \rho(g(u),X',n)$ 
%and by induction hypothesis $\rho(\rho(t,X,n),Y_{(n)},m) = \rho(\rho(f,X',n-1),Y'_{(n-1)},m) 
%\le \rho(f, X' \cup Y', p) =  \rho(f,(X \cup Y)', p) 
%= \rho( t, X \cup Y, p + 1)$.
%\end{enumerate}
%
%Assume  $\phi_X(t) = \False$. Then we have
%$\rho(t,X,n)  = c(\rho(t_1,X_1,n-1)\ldots(\rho(t_h,X_h,n-1)) $
%and $X = c(X_1) \ldots c(X_h)$.
%
%Suppose $\phi_Y(t)=\True$.
%
%\begin{enumerate}
%\item
%If $t = (\lambda x^T.b)(a)$, $\rho(t,X,n)  = (\lambda x^T.b')(a')$
%then 
%$\rho( (\lambda x^T.b')(a'),Y_{(n)},m) = \rho(b'[a'/x],Y'_{(n-1)},n-1)$,
%and by induction hypothesis $\rho(b'[a'/x],Y'_{(n-1)},n-1) 
%\le \rho(b'[a'/x], X \cup Y', p) =  \rho(b[a/x], X' \cup Y', p) = \rho(b[a/x],(X \cup Y)', p) 
%= \rho( (\lambda x^T.b)(a), X \cup Y, p + 1)$.
%
%\item
%If $t = \cond(f,g)(0)$, $\rho(t,X,n)  =\cond(f',g')(0)$
%then 
%$\rho(t,X,n) = \rho(f,X',n-1)$,
%and by induction hypothesis $\rho(\rho(t,X,n),Y_{(n)},m) = \rho(\rho(f,X',n-1),Y'_{(n-1)},m) 
%\le \rho(f, X' \cup Y, p) =  \rho(f, X' \cup Y', p) = \rho(f,(X \cup Y)', p) 
%= \rho( t, X \cup Y, p + 1)$.
%
%\item
%If $t = \cond(f,g)(\Succ(u))$, $\rho(t,X,n)  =\cond(f',g')(\Succ(u'))$
%then $\rho(t,X,n)= \rho(g(u),X',n-1)$ 
%and by induction hypothesis $\rho(\rho(t,X,n),Y_{(n)},m) = \rho(\rho(g(u),X',n-1),Y'_{(n-1)},m) 
%\le \rho(g(u), X' \cup Y', p) =  \rho(g(u),(X \cup Y)', p) 
%= \rho(g(u), X \cup Y, p + 1)$.
%\end{enumerate}
%
%
%Assume  $\phi_X(t) = \False$ and $\phi_Y(t)=\False$ and $t=c(t_1)\ldots(t_h)$ for some $h=0,1,2$.
%Then $\rho(\rho(t,X,n),Y_{(n)},m) = 
%c(\rho(\rho(t_1,X_1,n-1),(Y_{(n)})_1,m-1), \ldots, \rho(\rho(t_h,X_h,n-1),(Y_{(n)})_h,m-1)
%= c(\rho(\rho(t_1,X_1,n),(Y_1)_{(n-1)},m), \ldots, \rho(\rho(t_h,X_h,n),(Y_h)_{(n-1)},m) 
% \le
%c(\rho(t_1,X_1 \cup Y_1,p_1), \ldots, \rho(t_h,X_h \cup Y_h,p_h) =
%c(\rho(t_1,(X \cup Y)_1,p_1), \ldots, \rho(t_h,(X \cup Y)_h,p_h)  \le
%c(\rho(t_1,(X \cup Y)_1,p), \ldots, \rho(t_h,(X \cup Y)_h,p) =
%\rho(t, X \cup Y,p)$ for $p = \max(p_1, \ldots, p_h)$.
%
%\end{proof}
%
%The proof of the opposite implication is similar.
%
%\begin{lemma}[Infinite Church-Rosser (right-to-left)]
%\label{lemma-infinite-church-rosser-right}
%For all $t,u,v \in \LAMBDA_\bot$, all decidable sets $X$, $Y$ of redexes of $t$:
%$\rho(t,X \cup Y) \le \rho(\rho(t,X),Z)$ for $Z = \sigma(t,X,Y)$.
%\end{lemma}
%
%\begin{proof}
%\ldots\ldots\ldots
%\end{proof}
%
%
%\begin{theorem}[Infinite Church-Rosser]
%\label{theorem-infinite-church-rosser}
%For all $t,u,v \in \LAMBDA_\bot$, all decidable sets $X$, $Y$ of redexes of $t$:
%
%\begin{enumerate}
%\item
%if $t \reduces_X u$ and $t \reduces_{X \cup Y} w$ and $Z = \sigma(t,X,Y)$ then 
%$u \reduces_{Z} w$.
%
%
%\item
%if $t \reduces_X u$ and $u \reduces_Y v$, 
%then for some $w \in \LAMBDA_\bot$, for some decidable sets
%$Z$, $T$ of redexes of $u, v$
%we have $u \reduces_Z w$ and $v  \reduces_T w$.
%\end{enumerate}
%
%\end{theorem}
%
%\begin{proof}
%\begin{enumerate}
%\item
%Assume if $t \reduces_X u$ and $t \reduces_{X \cup Y} w$ and $Z = \sigma(t,X,Y)$,
%in order to prove $u \reduces_{Z} w$.
%Then $u  = \rho(t,X)$ and $w = \rho(t,X \cup Y)$ and we have to prove  $\rho(u,Z)  = w$. 
%This follows from $\rho(\rho(t,X),Z)  = \rho(t,X \cup Y)$ (lemmas
%\label{lemma-infinite-church-rosser-left} and \label{lemma-infinite-church-rosser-right}).
%
%
%%13:21 12/06/2024
%
%\item
%Assume $t \reduces_X u$ and $u \reduces_Y v$. There is some $w \in \LAMBDA_\bot$
%such that $t \reduces_{X \cup Y} w$. From the previous point  we have
%$u \reduces_{Z} w$ for $Z = \sigma(t,X,Y)$ and $v \reduces_{T} w$ for $T = \sigma(t,Y,X)$
%
%\end{enumerate}
%\end{proof}
%
%
%%%%%%%%%%%%%%%%%%%%%%%%%%%%%%%%%%%%%%%%
% end old version of Church-Rosser 
%%%%%%%%%%%%%%%%%%%%%%%%%%%%%%%%%%%%%%%%

We define the safe trunk of a term as the part of the term which we can normalize with safe reductions only.
In the rest of this section we will
prove that Infinite Church-Rosser implies that if the safe trunk exists then it is unique. 

Infinite Church-Rosser and Safe strong Normalization together imply that after finitely many steps
all safe reductions reach the same safe trunk.

\begin{definition}[Safe Trunk of a term]
\label{definition-safe-trunk}
Assume $t \in \LAMBDA$.
\begin{enumerate}
\item
The safe trunk of $t$ is any expression $u[\cond(f_1,\cdot), \ldots, \cond(f_n,\cdot)]$
such that  for some $g_1, \ldots, g_n$ we have $v = u[\cond(f_1,g_1), \ldots, \cond(f_n,g_n)]$
\emph{safe normal} and $t \reduces v$.
\item
$t$ is \emph{finite for safe reduction} if and only if all infinite reduction sequences from $t$ 
include only finitely many \quotationMarks{safe} reduction steps.  
\end{enumerate}
\end{definition}


\begin{lemma}[Safe Trunk of a term]
\label{lemma-safe-trunk}
Assume $t$ is finite for safe reductions.
If the  safe-trunk of $t$ exists then it is unique. 
\end{lemma}


\begin{proof}
Assume $t$ is finite for safe reductions in order to prove that the safe-trunk of $t$ is unique.

Assume that $u[\cond(f_1,\cdot), \ldots, \cond(f_n,\cdot)]$ and
$u'[\cond(f'_1,\cdot), \ldots, \cond(f'_{n'},\cdot)]$ are safe-trunks for $t$, in order to prove
that $u=u'$ and $n=n'$. 

Then for some $g_1, \ldots,g_n$ and some $g'_1, \ldots,g'_n$ we have that 
$v = u[\cond(f_1,g_1), \ldots, \cond(f_n,g_n)]$ and 
$v' = u'[\cond(f'_1,g'_1), \ldots, \cond(f'_n,g'_{n'})]$ 
are safe-normal forms of $t$ and all $\cond$-expressions shown are maximal. 
The decomposition of each safe-normal form $v$ is therefore unique:
if $v = u"[\cond(f"_1,g"_1), \ldots, \cond(f"_n,g"_{n"})]$ then $u=u"$ and $n=n"$
and $f_1=f"_1$, \ldots, $f_n=f"_n$.
Each reduction from $v$, $v'$ takes place in some $g_1, \ldots,g'_{n'}$. 

By Infinite Church-Rosser  
(theorem \ref{theorem-infinite-church-rosser}) we deduce that $v$ and $v'$ are confluent
outside the right-hand-side of $\cond$-expressions, therefore
for some $v"$ we have $v \reduces_Z v"$ and $v' \reduces_T v'"$. 
Since the reductions on $v, v'$ take place in some $g_1, \ldots,g'_{n'}$, 
we deduce that $v" = u[\cond(f_1,g"_1), \ldots, \cond(f_n,g"_n)]$
and $v'" = u[\cond(f'_1,g'"_1), \ldots, \cond(f'_n,g'"_{n'})]$ for some $g"_1, \ldots,g'"_{n'}$.
From the unicity of the decomposition of $v"$
with maximal $\cond$-subterms we conclude that $u=u'$ and $n=n'$
and and $f_1=f'_1$, \ldots, $f_n=f'_n$, as wished.

\end{proof}
 %DRAFT


\begin{thebibliography}{99}

\bibitem{2021-Anupam-Das}
A Circular Version of G\"{o}del's T and its abstraction complexity.
Anupam Das. arXiv:2012.14421v2 [cs.LO] 16 Jan 2021.

\end{thebibliography}





%\input{section-weak-normalization}        %SUBSUMED IN THE SN SECTION

%\input{section-uniqueness-normal-form} % SUBSUMED IN THE CHURCH-ROSSER SECTION

%\input{section-infinite-reductions}          % DRAFT

%\input{section-normalization-fairness}    %DRAFT

\newpage

\section{Appendix: Equivalence Between Cyclic and non-Cyclic System T} 
\label{section-equivalence-cyclic-non-cyclic-T}
%\bfColor{red}{(This section is but a draft)}
In this section we prove that the two systems $\systemT$ and $\CTlambda$ 
are each interpretable into the other. 
Both interpretations preserve the reduction relation, applications, $0$ and $\Succ$ and contexts.

The formal definition of the system $\systemT$ is as follows:

Types (denoted by $A,B,\ldots$) of $\systemT$ are inductively defined by $\N$ and $A\to B$.

Terms (denoted by $t,u,\ldots$) of $\systemT$ are inductively defined by
$x^A$, $\lambda x^A.t$, $tu$, $0$, $\Succ(t)$, $\Rec(t,u)$, and $\Ifz(t,u)$.

Sequents is $\Gamma\vdash t:A$, where $\Gamma$ is a context $x_1:A_1,\ldots,x_n:A_n$.

Typing rules are defined as follows:
\begin{enumerate}
\item
  The $\var$ rule: $\Gamma\vdash x:A$, where $x:A\in\Gamma$.
\item
  The $\weak$ rule: $\Gamma\vdash t:A$ and $\Gamma\subseteqsim \Gamma'$, where $\Gamma'\vdash t:A$.
\item
  The $\lambda$ rule: $\Gamma,x:A\vdash t:B$ implies $\Gamma \vdash \lambda x.t:A\to B$.
\item
  The $\ap$ rule: $\Gamma\vdash t:A\to B$ and $\Gamma\vdash u:A$ implies $\Gamma \vdash tu:B$.
\item
  The $0$ rule: $\Gamma\vdash 0:\N$.
\item
  The $\Succ$ rule: $\Gamma\vdash t:\N$ implies $\Gamma\vdash \Succ(t):\N$.
\item
  The $\Rec$ rule: $\Gamma\vdash a:A$ and $\Gamma\vdash h:A,\N\to A$ implies $\Gamma\vdash \Rec(a,h):\N \to A$.
\item
  The $\Ifz$ rule: $\Gamma\vdash u:A$ and $\Gamma\vdash v:A$ implies $\Gamma\vdash \Ifz(u,v):\N \to A$.
\end{enumerate}

The basic reduction relation $\reduces_0$ is defined by:
\begin{enumerate}
\item
  $(\lambda x.t)u \reduces_0 t[u/x]$, 
\item
  $\Rec(a,h)0 \reduces_0 a$, 
\item
  $\Rec(a,h)\Succ(u) \reduces_0 h(\Rec(a,h)u))u$, 
\item
  $\Ifz(u,v)0 \reduces_0 u$, 
\item
  $\Ifz(u,v)\Succ(u) \reduces_0 v$, 
\end{enumerate}

Then the reduction relation $\reduces_\systemT$ is the coontext and transitive closure of $\reduces_0$. 

%In both interpretations 
%we neglect the terms of $\CTlambda$ including type variables in their type or in their context,
%because these terms have no corresponding in $\systemT$.\\

We first provide the interpretation $\TtoCT{-}$ from $\systemT$ to $\CTlambda$
that is inductively defined as follows. 
\begin{center}
  $\TtoCT{x^A} = x^A$, 
  \qquad
  $\TtoCT{\lambda x^A.t} = \lambda x^A.\TtoCT{t}$, 
  \qquad
  $\TtoCT{tu} = \TtoCT{t}\TtoCT{u}$, 
  \qquad
  $\TtoCT{0} = 0$, 
  \qquad
  $\TtoCT{\Succ(t)} = \Succ(\TtoCT{t})$, 
  \\
  $\TtoCT{\Ifz(u,v)} = \cond(\TtoCT{u},\lambda z.\TtoCT{v})$, where $z\not\in\FV(v)$, and 
  \\
  $\TtoCT{\Rec(a,h)} = f$, where $f = \cond(\TtoCT{a},\lambda n.\TtoCT{h}(fn)n)$. 
\end{center}

Note that the translated terms are regular.
We can easiliy check that the translation preserves substitution,
namely $\TtoCT{t[u/x]} = \TtoCT{t}[\TtoCT{u}/x]$. 

The interpretation preserves typablity. 

\begin{proposition}\label{prop:TtoCTproof}
$\Gamma \vdash t:A$ holds in $\systemT$ implies $\Gamma\vdash \TtoCT{t}:A$ holds in $\CTlambda$. 
\end{proposition}
\begin{proof}
  It is shown by induction on the derivation of $\Gamma \vdash t:A$.
  We check only the two cases: the last rule is $\Ifz$ or $\Rec$.

  The case of $\Ifz$:
  the sequent $\Gamma\vdash\Ifz(u,v):\N\to A$ is obtained from subproofs of
  $\Gamma\vdash u:A$ and $\Gamma\vdash v:A$. 
  Then, by the induction hypothesis, we have proofs of $\Gamma\vdash \TtoCT{u}:A$ and $\Gamma\vdash \TtoCT{v}:A$
  in $\CTlambda$.
  Hence we have a proof of $\Gamma\vdash\cond(\TtoCT{u},\lambda z.\TtoCT{v}):\N\to A$
  by $\weak$ and $\cond$, where $z$ is a fresh variable. 
  This is a proof of $\Gamma\vdash \TtoCT{\Ifz(u,v)}:\N\to A$ in $\CTlambda$.
  Note that this is an expected proof, since it satisfies the global trace condition and regularity. 

  The case of $t = \Rec(a,h)$: 
  the sequent $\Gamma\vdash\Rec(a,h):\N\to A$ is obtained from subproofs of
  $\Gamma\vdash a:A$ and $\Gamma\vdash h:A,\N\to A$.
  By the definition, $\TtoCT{\Rec(a,h)} = \cond(\TtoCT{a},\lambda n.\TtoCT{h}(\TtoCT{\Rec(a,h)}n)n)$ holds.
  Then we have the following proof in $\CTlambda$. 
  \begin{center}\small
    $\infer[\cond]{
      \Gamma \vdash \TtoCT{\Rec(a,h)}:\redN\to A\quad(\dagger)
    }{
      \infer*[\Pi_a]{
        \Gamma \vdash \TtoCT{a}:A
      }{}
      &
      \infer{
        \Gamma \vdash \lambda n.\TtoCT{h}(\TtoCT{\Rec(a,h)}n)n:\redN\to A
      }{
        \infer[\apvar]{
          \Gamma,n:\redN \vdash \TtoCT{h}(\TtoCT{\Rec(a,h)}n)n: A
        }{
          \infer{
            \Gamma,n:\redN \vdash \TtoCT{h}(\TtoCT{\Rec(a,h)}n): \N\to A
          }{
            \infer{
              \Gamma,n:\N \vdash \TtoCT{h}: A,\N\to A
            }{
              \infer*[\Pi_h]{
                \Gamma \vdash \TtoCT{h}: A,\N\to A
              }{}
            }
            &
            \infer[\apvar]{
              \Gamma,n:\redN \vdash \TtoCT{\Rec(a,h)}n: A
            }{
              \infer{
                \Gamma,n:\N \vdash \TtoCT{\Rec(a,h)}: \redN\to A
              }{
                \Gamma \vdash \TtoCT{\Rec(a,h)}: \redN\to A\quad(\dagger)
              }
            }
          }
        }
      }
    }$
  \end{center}
  This proof satisfies the global trace condition:
  Its infinite path is either
  one that eventually an infinite path of $\Pi_a$,
  one that eventually an infinite path of $\Pi_h$, or 
  one that loops between the upper $(\dagger)$ and the lower $(\dagger)$. 
  Then paths of the former two cases contains a progressing trace by the induction hypothesis, and
  the last case contains a progressing trace (the trace consists of the red $\N$s). 
  Hence we obtain a proof of $\Gamma \vdash \TtoCT{\Rec(a,h)}:\N\to A$ as we wished. 
\end{proof}

For considering the corresponding reduction relation of $\systemT$ and the safe reduction of $\CTlambda$,
we need to introduce a reduction strategy, namely the call-by-name strategy,
that prevents reductions of terms inside $\Rec$ and $\Ifz$ that require
to reduce terms inside $\cond$ of translated $\CTlambda$. 

\begin{definition}
  The call-by-name reduction relation $\reduces_n$ is inductively defined as follows:
  \begin{enumerate}
  \item
    If $t \reduces_0 t'$, then $t \reduces_n u$, 
  \item
    $t \reduces_n t'$ implies $tu \reduces_n t'u$,
  \item
    $t \reduces_n t'$ implies $\Rec(a,h)t \reduces_n \Rec(a,h)t'$, and 
  \item
    $t \reduces_n t'$ implies $\Ifz(u,v)t \reduces_n \Ifz(u,v)t'$. 
  \end{enumerate}
\end{definition}

The call-by-name reduction and the 

\begin{lemma}\label{lem:cbnT}
  Assume that $t$ has a closed term of type $\N$. Then the following claims hold. 
  \begin{enumerate}
  \item\label{lem:cbnT1}
    If $t$ is not a numeral, then $t \reduces_n u$ for some $u$. 
  \item\label{lem:cbnT2}
    $t\reduces n$ if and only-if $t \reduces_n n$.
  \end{enumerate}
\end{lemma}
\begin{proof}
  The claim \ref{lem:cbnT1} is shown by induction on $t$. 
  By the assumption, $t$ has a form of either $\Succ(t')$, $(\lambda x.t_1)t_2\vec{t}$,
  $\Rec(a,h)t'\vec{t}$, or $\Ifz(v_1,v_2)t'\vec{t}$.
  For the first case, there is $u'$ such that $t\reduce_n u'$ by the induction hypothesis. 
  Then, take $u = \Succ(u')$.
  For the second case, take $u = t_1[t_2/x]\vec{t}$.
  For the third case, take $u = a\vec{t}$ if $t'=0$, take $u=h(\Rec(a,h)t'')t''\vec{t}$ if $t'=\Succ(t'')$, and
  take $\Rec(a,h)u'\vec{t}$ otherwise, where $t'\reduces_n u'$ by the induction hypothesis.
  For the last case, take $u = v_1\vec{t}$ if $t'=0$, take $u=v_2\vec{t}$ if $t'=\Succ(t'')$, and
  take $\Ifz(v_1,v_2)u'\vec{t}$ otherwise, where $t'\reduces_n u'$ by the induction hypothesis.

  We show the claim \ref{lem:cbnT2}.
  The right-to-left direction is trivially shown. The left-to-right direction 
  is shown by \ref{lem:cbnT1} and the confluency of $\reduces$: 
  Assume that $t\reduces n$. By the first claim, $t \reduces_n n'$ for some numeral $n'$.
  Then we have $n=n'$ by the confluency. Hence $t\reduces_n n$ holds as we wished. 
\end{proof}

\begin{proposition}\label{prop:TtoCTreduction}
  Let $t$ and $u$ be terms of $\systemT$.  
  \begin{enumerate}
  \item\label{prop:TtoCTreduction1}
    $t\reduces_n u$ implies $\TtoCT{t} \safeReduces \TtoCT{u}$. 
  \item\label{prop:TtoCTreduction2}
    If $t$ is a closed term of type $\N$ and $t\reduces n$, then $\TtoCT{t} \safeReduces n$. 
  \end{enumerate}
\end{proposition}  
\begin{proof}
  The point \ref{prop:TtoCTreduction1} is shown by induction on $\reduces_n$.
  It is enough to check the base cases.

  The case of $(\lambda x.t)u \reduces_n t[u/x]$ is shown by 
  $\TtoCT{(\lambda x.t)u} = (\lambda x.\TtoCT{t})\TtoCT{u} \safeReduces \TtoCT{t}[\TtoCT{u}/x] = \TtoCT{t[u/x]}$. 

  The case of $\Rec(a,h)0 \reduces_n a$ is shown by
  $\TtoCT{\Rec(a,h)0} = \cond(\TtoCT{a},\lambda n.\TtoCT{h}(\TtoCT{\Rec(a,h)}n)n)0 \safeReduces \TtoCT{a}$.

  The case of $\Rec(a,h)\Succ(u) \reduces_n h(\Rec(a,h)u)u$ is shown by
  \begin{align*}
    \TtoCT{\Rec(a,h)\Succ(u)}
    &= \cond(\TtoCT{a},\lambda n.\TtoCT{h}(\TtoCT{\Rec(a,h)}n)n)\Succ(\TtoCT{u})
    \safeReduces (\lambda n.\TtoCT{h}(\TtoCT{\Rec(a,h)}n)n)\TtoCT{u}
    \\
    &\safeReduces \TtoCT{h}(\TtoCT{\Rec(a,h)}\TtoCT{u})\TtoCT{u})
    = \TtoCT{h(\Rec(a,h)u)u}
  \end{align*}
  
  The case of $\Ifz(u,v)0 \reduces_n u$ is shown by
  $\TtoCT{\Ifz(u,v)0} = \cond(\TtoCT{u},\lambda z.\TtoCT{v})0 \safeReduces \TtoCT{u}$.

  The case of $\Ifz(u,v)\Succ(t) \reduces_n u$ is shown by
  $\TtoCT{\Ifz(u,v)\Succ(t)} = \cond(\TtoCT{u},\lambda z.\TtoCT{v})\Succ(t) \safeReduces (\lambda z.\TtoCT{v})t \safeReduces \TtoCT{v}$.

  Hence we have the point \ref{prop:TtoCTreduction1}.
  The point \ref{prop:TtoCTreduction2} is obtained by the point \ref{prop:TtoCTreduction1}
  and Lemma~\ref{lem:cbnT}~\ref{lem:cbnT2}. 
\end{proof}

%For any type $T$ we define a term $\Rec:T,(\N,T \rightarrow T),\N\rightarrow T$ such that
%$\Rec(a,f,0) = a$ and $\Rec(a,f)(n+1) = f(n,\Rec(a,f,n))$, for all numeral $n \in \Num$.
%The definition is $\Rec = \lambda a,f.\rec$ 
%with $\rec = \cond (a,\lambda x^{\N}.f(x,\rec(x))) : \N \rightarrow T$.

%\begin{proposition}
%$\Rec$ is a term of $\CTlambda$.
%\end{proposition}
%% \begin{proof}
%% \begin{enumerate}
%% \item
%%  $\Rec$ is regular by construction.
%% \item 
%% The only infinite path of $\Rec$ loops from $\rec$ to $\rec$ infinitely many times, and it includes
%% an infinitely progressing trace. Here it is: from the first unnamed argument of $\rec$ to 
%% to the first unnamed argument of $\lambda x^{\N}.f(x,\rec(x))$, then to $x^\N$ 
%% in the context of $f(x,\rec(x))$, to $x^\N$ in the context of $\rec(x)$,
%% and eventually to the first unnamed argument of $\rec$ again.
%% \end{enumerate}
%% \end{proof}

%% If we replace each primitive recursion in $\systemT$ with the term $\Rec$ in $\CTlambda$
%% we define an interpretation from $\systemT$ into $\CTlambda$, 
%% preserving reductions, applications, $0$ and $\Succ$ and contexts.
%% \\

\bfColor{red}{The rest of the section if but an early draft}
The opposite interpretation, from $\CTlambda$ to $\systemT$, it has been defined by proof-theory for the
combinatorial version of circular $\systemT$. For $\CTlambda$, instead we will define an algorithm
taking an infinite term in  $\CTlambda$, described as a finite circular tree, 
and returning a term in $\systemT$ with the required properties. 

First, we need a notion of confluence and of extensional equality 
for  functionals of $\CTlambda$ and of $\systemT$.

We define $t \sim_{\beta,\rec} u$ (a syntactical confluence for $\systemT$) 
if and and only if $t, u \in \systemT$ and $t,u$ have the same type
and for some $v \in \systemT$: $t \reduces_{\beta,\rec} v$ and $u \reduces_{\beta,\rec} v$.

We define an equivalence relation (an extensional equality for $\systemT$) 
$\sim_{\systemT}$ on $\systemT$, by induction on the type. 


\begin{definition}[An extensional equality on $\systemT$]
Assume $t,u \in \systemT$.
\begin{enumerate}
\item
If $t,u:\N$ and $t,u$ are closed then we set: 
$(t \sim_{\systemT} u) \Leftrightarrow  (t \sim_{\beta,\rec} u)$.
\item
If $t,u:A,\vec{A}\rightarrow\N$ and $t,u$ are closed then we set: 
$(t \sim_{\systemT} u) \Leftrightarrow  
\forall a \in \systemT. (a:A), (\mbox{$a$ closed}) \Rightarrow (t(a) \sim_{\systemT} u(a))$.
\item
If $t,u:\vec{A}\rightarrow\N$ and $\FV(t), \FV(u) \subseteq \vec{x}$ then we set:
$$
(t \sim_{\systemT} u) 
\Leftrightarrow  
\forall \vec{a} \in \systemT. 
(\vec{a}:\vec{A}), (\mbox{$\vec{a}$ closed})  
\Rightarrow 
(t[\vec{a}/\vec{x}] \sim_{\systemT} u[\vec{a}/\vec{x}])
$$
\end{enumerate}
\end{definition}

Then we can consider $\systemT$ as a structure $(\systemT/\sim_{\systemT}, 0, \Succ, \ap)$
with natural numbers, functionals and extensional equality. 
In the same way we define an equivalence relation on terms of $\CTlambda$ which denote functionals.
We only consider terms whose type and context only include the atomic type $\N$, and no type variables. 


\begin{definition}[An extensional equality on $\CTlambda$]
Assume $t,u \in \CTlambda$.
\begin{enumerate}
\item
If $t,u:\N$ and $t,u$ are closed then we set: 
$(t \sim_{\CTlambda} u) \Leftrightarrow  (t \sim_{\CTlambda} u)$.
\item
If $t,u:A,\vec{A}\rightarrow\N$ and $t,u$ are closed then we set: 
$(t \sim_{\CTlambda} u) \Leftrightarrow  
\forall a \in \CTlambda. (a:A), (\mbox{$a$ closed}) \Rightarrow (t(a) \sim_{\CTlambda} u(a))$.
\item
If $t,u:\vec{A}\rightarrow\N$ and $\FV(t), \FV(u) \subseteq \vec{x}$ then we set:
$(t \sim_{\CTlambda} u) 
\Leftrightarrow  
\forall \vec{a} \in \CTlambda. 
(\vec{a}:\vec{A}), (\mbox{$\vec{a}$ closed})  \Rightarrow (t[\vec{a}/\vec{x}] \sim_{\CTlambda} u[\vec{a}/\vec{x}])$.
\end{enumerate}
\end{definition}

Our goal is now to prove that there is an embedding from the types of $\N$-functionals in 
$(\CTlambda/ \!\! \sim_{\CTlambda}, 0, \Succ, \ap)$ to the entire
$(\systemT/ \!\! \sim_{\systemT}, 0, \Succ, \ap)$.
We ignore types of $\CTlambda$ including type variables, they have no corresponding in $\systemT$.

For each $\N$-functional type $T$ we have to define a map $\phi_T$ from terms of type $T$ of $\CTlambda$
to terms of type $T$ of $\systemT$. We abbreviate $\phi_T$ with $\phi$ and we require:

\begin{enumerate}
\item
If $t \sim_{\CTlambda} u$ \ \ \ \ \ \ \ \ \ \ \ \ \ then $\phi(t) \sim_{\systemT} \phi(u)$

\item
If $t: A \rightarrow B$, $u:A$ \ \ \ then $\phi(t(u)) \sim_{\systemT} \phi(t)(\phi(u))$

\item
$\phi(0) \sim_{\systemT} 0$ and
$\phi(\Succ(t)) \sim_{\systemT} \Succ(\phi(t))$

\end{enumerate}

We will define an embedding from $\CTlambda$ to $\systemT$ extended with $+,\times$-types.
There is an embedding from  $\systemT$ extended with $+,\times$-types to  $\systemT$ with only
$\N, \rightarrow$, therefore it is enough to embed $\CTlambda$ to $\systemT$ extended with $+,\times$-types.

We suppose be fixed a cyclic $\lambda$-term $t \in \CTlambda$, $t : T$,
 and we use several ingredients. First we consider all $\Gamma_i \vdash u_i : U_i$ for $u_i$ sub-term of $t$,
we turn it into a simple type $\Gamma_i \rightarrow U_i$, then we consider a single type
$S = \Sigma_i \Gamma_i \rightarrow U_i$. 

%14:47 11/06/2024

\begin{enumerate}
\item
We suppose be given a map $\trunk_t:\N \rightarrow S = \Sigma_i \Gamma_i \rightarrow U_i$, 
such that $\trunk_t(n)$ is 
the unfolding of $t$ with all subterms $u:U$ in the level number $n$ replaced by a dummy term $0_U$.
We use the type $S$ in order to have a common type and context for all sub-terms $u$.

\item
For each pair of subterms $u,v$ of $t$ and each path $\pi$ from $u$ to $v$. 
we suppose be given a one-to-many trace
relation $R_\pi$ between the indexes of $\N$-arguments of $u$ and of the $\N$-arguments of $v$. 
We close the set of relations $R_\pi$ by composition 
and we obtain a finite set (of exponential size in $t$).
\end{enumerate}

By the global trace condition for each infinite composition of $R_\pi$ there is some infinitely progressing
trace. This means that there is some index $i$ in the domain of $R_\pi$ such that for some $n \in \N$
we have $R^n_\pi(i,i)$ and the variable $i$ progresses.


We consider any assignment $t' \equiv t[\vec{n},\vec{x}](\vec{m},\vec{y})$ of the arguments of $t$.
The idea is to find map $\phi$ such that for all $p \ge \vec{n}, \vec{n}$ we have
$\trunk_{t'}(m)$ stationary for all $m \ge \phi(p)$, therefore $t' = \trunk_{t'}(\phi(p))$.

$\phi_c(p)$ is the map computing the maximum number of nodes for an Erdos tree in $c$
colors with height $\le p$ and branching $\le c$ (there is at most one child per color).
Whenever an Erdos tree has at least a branch of length $cp+1$ and $c$ colors, 
then we have a monotonically colored sequence
of length $cp+1$, therefore at least one homogeneous set of length $p+1$. If we take any composition of 
$\phi_c(p)+1$ times some relations $R_\pi$, there is some homogeneous set of $(p+1)$ elements
decorated with a single $R_\pi$, and for some $i$ some variable starting from some value $\le p$ 
and decreasing $p$ times. 

This means that each branch terminates and the whole computation of $\trunk_{t'}(\phi(p))$ terminates.

For a $c$-color tree, the value of $\phi_c(p)$ is $1+c+c^2+\ldots+c^{p-1} = (c^{p}-1)/(c-1)$.
Suppose $f:\N \rightarrow \N$ is some weakly increasing map. We want to prove that
there is some Erdos tree including some branch including some $f$-homogeneous set: some
homogeneous set with first point $h$ followed by $f(h)$ more points. We
want to define a functional in $\systemT$ such that all Erdos trees with $\ge F(f)$ nodes
include some $f$-homogeneous set.

We call a Ramsey functional any functional $F(\vec{x},c)$ associated to $e$
any functional taking weakly increasing functionals $\vec{x}$,
and returning an upper bound for the size of a $c$-color Erdos tree including some homogeneous set
with first node $l$ with value $\le e(\vec{x},\vec{F},l)$, 
followed by $\le e(\vec{x},\vec{F},l)$ more nodes, in which some variable $i$ decreases by $1$
each $n$ steps ($n$ number of sub-terms of the cyclic term). 

We require that $e$ is any primitive recursive
functional, that is, $e$ is 
defined by simply typed lambda calculus plus recursion on $\Seq(\N)$, and $\vec{F}$ are Ramsey functionals.

We claim that all sub-terms of $t$ have a computation time bounded by some Ramsey functional.
If this is not the case, we define some infinite path with a infinitely progressing trace
associated to some infinitely decreasing numeral, contradiction.

%17:03 12/06/2024

%?????????????????????
%22:31 10/06/2024

\ldots\ldots\ldots

%We require the following property of $\systemT$: we can define terms of $\systemT$ by $n$
%simultaneous lexicographic inductions. If $\vec{n} \in \N^m$ are numeral, we define 
%$\vec{n} \lexicographic{} \vec{m}$ if and only if $\vec{n} \not = \vec{m}$ and for the first $i \in [1,m]$
%such that $n_i \not = m_i$ we have $n_i +1 = m_i$. 
%If $T=\vec{A} \rightarrow \N$ is a functional type we define
%$0_T = \lambda \vec{x}:\vec{A}.0$. If $f:T$ we define $f \restr  \vec{x}$
%as the restriction of $f_i$ to the set of $\vec{y} \lexicographic{} \vec{x}$, 
%extended by the dummy value $0_T$:
%\begin{center}
% $(f \restr  \vec{x})(\vec{y}) = f(\vec{y})$ if $\vec{y} \lexicographic{} \vec{x}$ 
%\ \ \ 
%and 
%\ \ \ 
%$(f \restr  \vec{x})(\vec{y}) = 0_T$ otherwise
%\end{center}

%
%\begin{proposition}[Simultaneous lexicographic induction in $\systemT$]
%Assume $\vec{x}:\N^m$,
%and that we have any equation list in $\systemT$, in the meta-variables $f_1, \ldots, f_n$:
%$$
%f_i(\vec{x}) 
%\ \ \ 
%\sim_{\systemT} 
%\ \ \ 
%F_i(\vec{x}, f_1 \restr  \vec{x}, \ldots, f_n \restr  \vec{x})
%\ \ \ 
% : 
%\ \ \ 
%T_i
%$$
%This  equation list has solutions 
%$
%f_1:\N^m \rightarrow T_1, 
%\ldots, 
%f_n:\N^m \rightarrow T_n
%$ 
%in $\systemT$ and we can compute them. 
%\end{proposition}
%
%
%Assume $t:T$ is a cyclic term, represented as a cyclic tree with node $t_1$, \ldots, $t_n$.
%We translate it to a term of $\systemT$, obtained by solving an equation list whose meta-variables
%are the nodes the cyclic tree, with $\vec{x} = \vec{y},c$,
%and $\vec{y}$ the union of the $\N$-arguments of each node,
%and $c$ a variable used as a counter. The counter 
%$c$ starts from $n$, the number of nodes, and decreases
%of $1$ unit in each recursive call. 
%Within $n$ recursive calls, one or more value of $\vec{y}$ decreases by $1$,
%while all other values stay the same. In this case the varabile $c$ is reset to $n$: $c$ is the only
%variable which can increase during computation.
%
%
%%
%%This is a first draft about how to do it.
%%
%%%%%%%%%%%%%%%
%% % TO BE IMPROVED
%%%%%%%%%%%%%%%
%%
%%\begin{enumerate}
%%
%%\item
%%We first move all nodes to a context with the same number on type $\N$ variables, 
%%by adding dummy variables and dummy arguments.
%%This operation preserves regularity and global trace condition.
%%Now $t_1$, \ldots, $t_n$ all have context $\Gamma$ and type $A$.
%%
%%\item
%%We merge all buds into the same term, defined by some $u$ such that $u(i)=t_i$, for $i=1, \ldots, n$,
%%and $u(i)=$ some dummy term of type $A$ otherwise. We replace each $t_i$ with $u(i)$, 
%%for $i=1, \ldots, n$.
%%This operation preserves regularity and global trace condition.
%%Now we have $n$ buds, all are the same $u$ with context $\Gamma$ and type $\N \rightarrow A$.
%%Each bud $b$ defines a partial bijection between the occurrence of $\N$ in its context and type
%%$\Gamma \vdash \N \rightarrow A$, and the occurrences of $\N$ in the context and type
%%$\Gamma \vdash \N \rightarrow A$ of its companion. 
%%We extend this partial bijection to any total bijection $\tau$, depending on the but $b$.
%%
%%\item
%%We close the partial bijections defined by each bud by composition. The number of partial 
%%bijections can grow in an exponential  way.
%%
%%\item
%%Assume we have $m$ occurrences of $\N$ inside the context and type 
%%$\Gamma \vdash \N \rightarrow A$ of $u$.
%%We fix a permutation $\sigma:\{1, \ldots, m\}$ 
%%and we label them by variables $x_1, \ldots, x_n$ of $\systemT$,
%%with $x_i$ label of the argument with type $\N$ and number $i$.
%%We will define a translation $t^\sigma \in \systemT$ of  $t \in \CTlambda$.
%%
%%\item
%%All traces move from $u$ to any of the occurrences of $u$ inside $u$. 
%%Some traces of some $\N$
%%in $\Gamma \vdash \N \rightarrow A$ disappear, some other are moved to some other $\N$,
%%in an injective way. Two traces never merge.
%%We label each trace in the bud $u$ with the name $x_i$ of the corresponding trace, if any.
%%All those corresponding to no trace are labeled at random using the remaining variable names.
%%
%%At least one trace progresses, otherwise by repeating infinitely many times this step we would get a
%%path with no progressing trace. The same is true for any combination of one or more movements
%%from $u$ to $u$. 
%%
%%%After $m$ movements to any $u$ inside $u$, 
%%%each of the $m$ traces either disappeared or cycles. After $m!$ steps, all
%%%cycles are back to their original point. 
%%%
%%%All traces are now restarted or move from one $\N$ to the same $\N$, with or without progression.
%%
%%\item
%%At least one trace $x_i$ progresses and it is not erased by any other trace. Otherwise we could follow a path
%%in which each progress is erased in some new step, and so there is no infinite progressing trace.
%%We use this trace as the main variable $x_i$ of the recursion. In all steps, either $x_i$ is constant or decreases,
%%and in at least one case it decreases. In all cases in which $x_i$ decrease we use primitive
%%recursion on $x_i$ in $\systemT$, as main variable. 
%%In all other case, $x_i$ is not removed, therefore it stays the same. 
%%We isolate the main variable $x_j$ of the recursion for these steps, it is progressing therefore $j \not = i$.
%%We use primitive recursion on $x_j$: this is the second variable of primitive recursion. 
%%We continue in this way and we define a primitive recursion in $\systemT$, with pairwise distinct 
%%indexes $x_{i_1} = x_i$, $x_{i_2} = x_j$, \ldots, $x_{i_k}$ for some $k \ge 1$. We extend 
%%$x_{i_1}, \ldots, x_{i_k}$ to $x_{i_1}, \ldots, x_{i_n}$ in a random way: we defined in this way a
%%permutation $\sigma$ on $\{1, \ldots, m\}$ by $\sigma(j) = i_j$ for $j \in \{1, \ldots, m\}$
%%We define in this way a closed primitive recursive term 
%%$\lambda \vec{x}.t^\sigma \in \systemT$. Each bud $u$
%%defining a permutation $\tau$ is replaced by $\exch_{\tau}(f)$.
%%The term $\exch_\tau \in \systemT$ applies the permutation $\tau$ to the arguments of $f$,
%%and during the recursive call $f$ is replaced by $\lambda \vec{x}.u^\sigma$.
%%\end{enumerate}
%%
%%We claim that the infinite term $t^\sigma \in \systemT$ 
%%is equivalent to the cyclic recursive term $t \in \CTlambda$ we started from.
 %WORKING DRAFT



%
%\section{appendix}
%
%\begin{verbatim}
%
%To: kmr@is.sci.toho-u.ac.jp (Daisuke Kimura)
%Re: proof of Weak Normalization to an integer for CT-lambda
%Fri, 22 Mar 2024 08:25:57 +0100 
%
%    By the way, I re-checked the weak curry-howard proof, now i think that the proof 
%does not require the property p-->q, a-->b ==> p[a/x]-->q[b/x] and can be completed 
%with the notion of safe reduction.
%but in fact it would be more interesting to prove full church-rosser for Circular T-lambda, 
%as anupam does for his circular T.
%
%    About strong normalization, we can prove it for "safe" reductions, those inside no cond. 
%More in general, we know that we can have infinite reduction sequences, because we can 
%have infinitely many redexes. However, for any infinite reduction sequence sigma, I conjecture 
%we can prove a kind of stabilization of the term. After some reduction step, the term only 
%changes inside some cond nested k times. 
%
%    Namely, I conjecture that
%
%"for any cyclic lambda term t, any infinite reduction sequence (sigma(n)|n in N) with sigma(0)=t, 
%any k in N, there is a n0 in N such that for all n>=n0, the terms sigma(n) and sigma(n0)  
%coincide on all branches with at most k times cond."
%
%    Best, Stefano
%
%\end{verbatim}
%
%\section{The $\Succ$-length of a term}
%We define the safe trunk of a term as the part of the term which we can normalize with safe reductions only,
%and  the $\Succ$-length of a term $t \in \LAMBDA$ the number of $\Succ$ in front of the safe trunk.
%Here is the formal definition.
%
%\begin{definition}[$\Succ$-length]
%Assume $t \in \LAMBDA$
%\begin{enumerate}
%\item
%The safe trunk of $t$ is any expression $u[\cond(f_1,\cdot), \ldots, \cond(f_n,\cdot)]$
%such that  for some $g_1, \ldots, g_n$ we have $v = u[\cond(f_1,g_1), \ldots, \cond(f_n,g_n)]$
%\emph{safe normal} and $t \reduces v$.
%\item
%The $\Succ$-length of $t$ is
%the number of $\Succ$ in front of any safe-normal form of $t$, if any exists. 
%\end{enumerate}
%\end{definition}
%
%
%\begin{Eg}
%The $\Succ$-length is a kind a value we can assign to \emph{all} terms of $\LAMBDA$.  
%We compute the $\Succ$-length for some normal terms of $\LAMBDA$.
%$0$ has $\Succ$-length $0$, $\Succ(x)$ has $\Succ$-length $1$, 
%$\Succ(\Succ(\cond(f,g)(x))$ has $\Succ$-length $2$, while $t = \Succ(t)$ has infinite $\Succ$-length.
%\end{Eg}
%
%The $\Succ$-length exists for total terms.
%From the Church-Rosser property for $\WTyped$
%we deduce that the $\Succ$-length is unique when it exists.
%
%\begin{lemma}[$\Succ$-length of terms  finite for safe reductions]
%\label{lemma-succ-length}
%Assume $t \in \LAMBDA$.
%
%\begin{enumerate}
%\item 
%\label{lemma-succ-length-01}
%The safe-trunk and $\Succ$-length exist and they are unique. 
%
%\item
%\label{lemma-succ-length-02}
%If $t \reduces \Succ(u)$, then the $\Succ$-length of $u$ is $1$ unit 
%smaller than the $\Succ$-length of $t$.
%
%\end{enumerate}
%\end{lemma}
%
%
%
%\begin{proof}
%Assume $t \in \LAMBDA$.
%
%\begin{enumerate}
%\item
%%\label{lemma-succ-length-01}
%Assume that $u[\cond(f_1,\cdot), \ldots, \cond(f_n,\cdot)]$ and
%$u'[\cond(f'_1,\cdot), \ldots, \cond(f'_{n'},\cdot)]$ are safe-trunks for $t$, in order to prove
%that $u=u'$ and $n=n'$. 
%
%Then for some $g_1, \ldots,g_n$ and some $g'_1, \ldots,g'_n$ we have that 
%$v = u[\cond(f_1,g_1), \ldots, \cond(f_n,g_n)]$ and 
%$v' = u'[\cond(f'_1,g'_1), \ldots, \cond(f'_n,g'_{n'})]$ 
%are safe-normal forms of $t$ and all $\cond$-expressions shown are maximal. 
%The decomposition of each safe-normal form $v$ is therefore unique:
%if $v = u"[\cond(f"_1,g"_1), \ldots, \cond(f"_n,g"_{n"})]$ then $u=u"$ and $n=n"$.
%Each reduction from any of the safe-trunks takes place in some $g_1, \ldots,g'_{n'}$. 
%
%By Church-Rosser  (\S \ref{section-church-rosser}) we deduce that $v$ and $v'$ are confluent, therefore
%for some $v"$ we have $v, v' \reduces v"$. Since the reductions on $v, v'$ take place in some $g_1, \ldots,g'_{n'}$, we deduce that $v" = u[\cond(f_1,g"_1), \ldots, \cond(f_n,g"_n)]$
%and $v" = u[\cond(f_1,g'"_1), \ldots, \cond(f_n,g'"_{n'})]$ for some $g"_1, \ldots,g'"_{n'}$.
%From the unicity of the decomposition of $v"$
%with maximal $\cond$-subterms we conclude that $u=u'$ and $n=n'$. 
%\\
%
%From unicity of the safe-trunk we deduce that the $\Succ$-length is unique.
%
%\item
%%\label{lemma-succ-length-02}
%\emph{Assume that $t \reduces \Succ(u)$, in order to prove that 
%the $\Succ$-length of $u$ is $1$ unit smaller than the $\Succ$-length of $t$}.
%By the point \ref{lemma-succ-length-01} above, 
%$u$ has $\Succ$-length some $k \in \N$. Then $u \reduces \Succ^k(v)$
%for some $v$ not successor, 
%therefore $t \reduces \Succ^{k+1}(v)$ and the $\Succ$-lenght of $t$ is $\ge k+1$.
%
%\end{enumerate}
%\end{proof}


\end{document}
