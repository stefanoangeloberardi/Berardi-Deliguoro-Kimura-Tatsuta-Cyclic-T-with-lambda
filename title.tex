\ifdraft

\title{$\CTlambda$, a cyclic simply typed $\lambda$-calculus with fixed points}

\author{Stefano Berardi,  Ugo de' Liguoro, Daisuke Kimura, Koji Nakazawa, Makoto Tatsuta}
\date{}

\else

\title[Equivalence]
{
???
}

\author[S. Berardi]{Stefano Berardi}
\address{Universit\`{a} di Torino,
Torino, Italy}
\email{stefano@unito.it}


%% mandatory lists of keywords 
\keywords{
proof theory
inductive definitions
Brotherston-Simpson conjecture
cyclic proofs
}
\fi

\maketitle

\begin{abstract}
A motivation for having a circular syntax for G\"{o}del System $\systemT$ is that terms 
in a circular syntax are much shorter than the equivalent terms written in $\systemT$, 
yet they can be checked mechanically.

Several circular syntax for $\systemT$ (that is, infinite regular terms with the global trace condition)
have been proposed, like $\CT$ by Anupam (\cite{2021-Anupam-Das}, \S 4.5, page 20).
In this paper we explore the possibility of defining a circular syntax 
directly for $\lambda$-abstraction and simple types, which we call $\CTlambda$, instead of interpreting
 $\lambda$-abstraction through combinators first, then inserting a  global trace condition afterwards.
A circular syntax using binders instead of combinators
could be more familiar for researchers working in the field of Type Theory.
Again for reason of simplicity of use, we drop the restriction of having finitely many variables in infinite terms, 
and finitely many redexes, a restriction considered in previous works on the same topic, 
and we explore the consequences of ths choice.
\\

We introduce our circular syntax as a fixed point operator defined by cases through a test on $0$.
We prove for $\CTlambda$ some of the results proved by Anupam for his combinatorial circular syntax:
\emph{(i)}  every closed term of type $\N$ reduces to some numeral;
\emph{(ii)}  strong normalization for all reductions sequences never reducing inside a fixed point operator;
\emph{(iii)} the equivalence between the set of terms of G\"{o}del system $\systemT$
and the set of terms of $\CTlambda$. 
\\

A consequence of the
 fact that we allow infinitely many variables for our infinite terms is that some other results proved by Anupam,
namely Strong Normalization for all reductions and Church Rosser 
(\cite{2021-Anupam-Das}, Theorem 51, page 35), fail. 
However, we recover for $\CTlambda$ the same properties in the limit:
\emph{(iv)}  existence of a limit for all infinite reductions;
\emph{(v)}  strong normalization in the limit for all fair reductions;
\emph{(vi)} Church-Rosser in the limit.

\end{abstract}

\iffalse
key words: 
proof theory,
inductive definitions,
Brotherston-Simpson conjecture,
cyclic proofs,
Martin-Lof's system of inductive definitions,
infinite Ramsey theorem
Podelski-Rybalchenko termination theorem
size-change termination theorem
\fi
