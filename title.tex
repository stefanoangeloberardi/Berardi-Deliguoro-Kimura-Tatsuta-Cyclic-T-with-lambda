\ifdraft

\title{CT-$\lambda$, a cyclic simply typed $\lambda$-calculus with fixed points}

\author{Stefano Berardi,  Ugo de' Liguoro, Daisuke Kimura, Koji Nakazawa, Makoto Tatsuta}
\date{}

\else

\title[Equivalence]
{
???
}

\author[S. Berardi]{Stefano Berardi}
\address{Universit\`{a} di Torino,
Torino, Italy}
\email{stefano@unito.it}


%% mandatory lists of keywords 
\keywords{
proof theory
inductive definitions
Brotherston-Simpson conjecture
cyclic proofs
}
\fi

\maketitle

\begin{abstract}
A motivation for having a circular syntax for G\"{o}del System $\systemT$ is that terms 
in a circular syntax are much shorter than the equivalent terms written in $\systemT$, 
yet they can be checked mechanically.

Several circular syntax for $\systemT$ have been proposed.
In this paper explore the possibility of defining a circular syntax
directly for $\lambda$-abstraction and simple types, instead of interpreting
 $\lambda$-abstraction through combinators first, then inserting a  circular syntax afterwards.
A circular syntax using binders instead of combinators
could be more familiar for researchers working in the field of Type Theory.

We introduce our circular syntax as a fixed point operator defined by cases through a test on $0$.
We prove the expected results for this circular simply typed $\lambda$-calculus, which we call $\CTlambda$: 
\emph{(i)} every closed term of type $\N$ reduces to some numeral;
\emph{(ii)}  strong normalization for all reductions sequences outside all fixed points; 
\emph{(iii)}  strong normalization for ``fair'' reductions; \emph{(iv)} Church-Rosser property;
and \emph{(v)} the equivalence between circular syntax and ordinary syntax 
for G\"{o}del system $\systemT$. 


\end{abstract}

\iffalse
key words: 
proof theory,
inductive definitions,
Brotherston-Simpson conjecture,
cyclic proofs,
Martin-Lof's system of inductive definitions,
infinite Ramsey theorem
Podelski-Rybalchenko termination theorem
size-change termination theorem
\fi
