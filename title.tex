\ifdraft

\title{CT-$\lambda$, a cyclic simply typed $\lambda$-calculus with fixed points}

\author{Stefano Berardi }
\date{}

\else

\title[Equivalence]
{
???
}

\author[S. Berardi]{Stefano Berardi}
\address{Universit\`{a} di Torino,
Torino, Italy}
\email{stefano@unito.it}


%% mandatory lists of keywords 
\keywords{
proof theory
inductive definitions
Brotherston-Simpson conjecture
cyclic proofs
}
\fi

\maketitle

\begin{abstract}
We explore the possibility of having a circular syntax
directly for $\lambda$-abstraction, instead of interpreting
 $\lambda$-abstraction through combinators and circular syntax afterwards.
We introduce circular syntax as a case fixed point operator.
Our motivation for using binders instead of combinators is that this way of introducing circular syntax 
could be more familiar for researchers in the field of Type Theory.

We prove the usual results for the new syntax: 
every closed term of type $\N$ reduces to some numeral,
and we have strong normalization for reductions outside all fixed points, 
strong normalization for fair reductions, Church-Rosser property
and the equivalence with G\"{o}del system $\systemT$. 
Terms with cyclic syntaxt are much shorter than the equivalent terms of $\systemT$. 
\end{abstract}

\iffalse
key words: 
proof theory,
inductive definitions,
Brotherston-Simpson conjecture,
cyclic proofs,
Martin-Lof's system of inductive definitions,
infinite Ramsey theorem
Podelski-Rybalchenko termination theorem
size-change termination theorem
\fi