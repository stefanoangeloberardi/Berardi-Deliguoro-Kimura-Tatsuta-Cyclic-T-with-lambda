
\section{Infinite Church-Rosser for Infinite Lambda Terms}
\label{section-safe-church-rosser}

%15:12 16/04/2024
In this section prove a result for the safe part of a term, which we call 
\quotationMarks{\emph{unicity of the safe part of the safe normal form}}. By this we mean 
if $t,u,v \in \LAMBDA$ and $t \reduces u$ and $t \reduces v$ and $u, v$ are safe-normal 
then $u$, $v$ are equal outside the right-hand side of all $\cond$-sub-terms. 

%%%%%%%%%%%%%%%%%%%%%%%%%%%
%\ldots\ldots\ldots\ldots\ldots\ldots\ldots\ldots\ldots\ldots
%\\
%\bfColor{red}{(Here we should fill this part by adapting the proof of Church Rosser for the type $\N$- Stefano)}
%\ldots\ldots\ldots\ldots\ldots\ldots\ldots\ldots\ldots\ldots
%%15:46 24/04/2024
%%%%%%%%%%%%%%%%%%%%%%%%%%%

Our first idea is to prove a full Church-Rosser property for $\LAMBDA$: 
if $t,u,v \in \LAMBDA$ and $t \reduces u$ and $t \reduces v$ then for some $w \in \LAMBDA$
we have $u \reduces w$ and $v \reduces w$. However this property is false: for some $t$, finding a 
common reduction of $u,v$ takes infinitely many steps, even for terms of $\CTlambda$,
as the next example shows.

\begin{Eg}
Let $b = \cond(x^{\N},b):\N \rightarrow \N$ 
and $t = (\lambda x^{\N}.b)(r)$, with $r = (\lambda x^{\N}.x^{\N})(3)$ some redex. 

We first remark that  $t \in \CTlambda$. Indeed, 
\begin{enumerate}
\item
$t$ is regular by construction.
\item
We have $t \in \GTC$, because the unique infinite path of $t$ is 
$t, \lambda x^{\N}.b, b, b, b, \ldots$, and the
unique unnamed argument of $b:\N \rightarrow \N$ in the path progresses infinitely many times.
\end{enumerate}

Now consider the reductions: $t \reduces b[r/x^\N]$ and $t \reduces  (\lambda x^{\N}.b)(3)$.
We expect $b[3/x^\N]$ as common reduction. But we have $b[r/x^\N] = \cond(r,b[r/x^\N]$,
that is, we have replicated the redex $r$ infinitely many times in $b[r/x^\N]$. Therefore to reduce 
$b[r/x^\N]$ to $b[3/x^\N]$ takes infinitely many steps, and for \emph{no finite reduction we have}
$b[r/x^\N] \reduces b[3/x^\N]$.
\end{Eg}

In order to recover Church-Rosser we have to consider a more general notion of reduction $\reduces_X$, 
which allow to reduce \emph{infinitely many redexes in one step}: 
all those in a \emph{decidable} set $X$ of redexes of $t$. Then we
will prove that $\reduces_X$ is confluent: namely, we will prove that 
if $t \reduces_X u$ and $t \reduces_Y v$, 
then for some $w, Z, T$ we will have $u \reduces_Z w$ and $v  \reduces_T w$.
We call this property Infinite Church-Rosser.
Infinite Church-Rosser implies unicity of the safe part of the safe normal form.

Our first problem is that infinite reductions can easily loop, therefore Infinite Church-Rosser is stated for the
set $\LAMBDA_\bot$ of terms with possibly \emph{undefinite} subterms. 
This is not an obstacle as we will see.
We formally state Infinite Church-Rosser as follows. 
\quotationMarks{\emph{For all $t,u,v \in \LAMBDA_\bot$, all decidable sets $X$ of redexes of $t$, 
if $t \reduces_X u$ and $t \reduces_Y v$, 
then for some $w \in \LAMBDA_\bot$, some decidable set $Z, T$ of redexes of $u, v$
we have $u \reduces_Z w$ and $v  \reduces_T w$ and $t \reduces{X \cup Y} w$}}.

%08:00 10/06/2024

We represent a decidable set $X$ of positions of redexes by a map 
$\phi_X:\universe{t} \rightarrow \{\True,\False\}$ 
such that $l \in X$ if and only if $\phi_X(l) =\True$. 
We have to precise how $\phi_X$ changes when redexes in $X$ are moved or duplicated.


\begin{definition}[Substitution, subterms and labels]
\label{definition-substitution-label}
Suppose $X$ is a decidable set of redexes of $t$ and $Y$ a decidable set of redexes of $u$
\begin{enumerate}
\item
$Z = X[Y/x]$ is a a decidable set of redexes of $t[u/x]$ defined as:
for all $l \in \universe{t}$, $m \universe{u}$ 
if $l$ is a free occurrence of $x$ we set $\phi_Z(l \conc m) = \phi_Y(m)$,
we set $\phi_Z(l) = \phi_X(l)$ otherwise.
otherwise.
\item
If $n=1,2$ and $t = c(t_1)\ldots(t_n)$ and $X$ a decidable set of redexes of $t$,
then for all $1 \le i \le n$ we define a set $X_i$ of redexes in $t_i$ by $\phi_{X_i}(l) = \phi_X((i) \conc l)$. 
\item
If $n=1,2$ and $t = c(t_1)\ldots(t_n)$ and for all $1 \le i \le n$ $X_i$ is a decidable set of redexes of $t_i$,
then we define a set $X$ of redexes in $t$ by $\phi_X(\nil)=\False$
and $\phi_X((i) \conc l) = \phi_{X_i}(l)$. 
\end{enumerate}
\end{definition}


%%%%%%%%%%%%%%%%%%%%%%%%%%%%
%   \ldots\ldots\ldots\ldots\ldots\ldots
%%%%%%%%%%%%%%%%%%%%%%%%%%%%


We define $t \reduces_X u$ as the limit of a map $\rho(t,X,n)$ for $n \rightarrow \infty$. 
$\rho(t,X,n)$ start with the undefined value $\bot$, then either holds the value $\bot$ forever,
or at some step the root of $\rho(t,X,n)$ becomes some constructor of $\LAMBDA$ and never
changes again. At each step, if $t$ itself is a redex in $X$ then we reduce it, 
if $t$ is not a redex in $X$ then we move to the subterms of $t$. 
In both case we update $X$ accordingly to some set of labels $Y$.
We update any other set $Z$ of labels in $t$ in the same way to some $\sigma(t,X,n,Z)$,
where $Y = \sigma(t,X,n,X)$.

\begin{definition}
Assume $t \in \LAMBDA_\bot$ and $X,Z$ are decidable sets of redexes of $t$.

We set $\rho(t,X,0)=\bot$. Now assume  $\phi_X(t) = \True$. Then we set:

\begin{enumerate}
\item
If $t = (\lambda x^T.b)(a)$, then $\rho(t,X,n+1) = \rho(b[a/x],Y,n)$ with $Y = (X_1)_1[X_2/x]$.
We set $\sigma(t,X,n,Z) = (Z_1)_1[Z_2/x]$.
\item
If $t = \cond(f,g)(0)$, then $\rho(t,X,n+1) = \rho(f,Y,n)$ with $Y = (X_1)_1$.
We set $\sigma(t,X,n,Z) = (Z_1)_1$.
\item
If $t = \cond(f,g)(\Succ(u))$, then $\rho(t,X,n+1) = \rho(g(u),Y,n)$ with $Y = \ap((X_1)_2, (X_2)_1)$.
We set $\sigma(t,X,n,Z) =  \ap((Z_1)_2, (Z_2)_1)$
\end{enumerate}

Assume  $\phi_X(t) = \False$ 
and $t=c(t_1)\ldots(t_h)$ for some $h=0,1,2$ some $t_1, \ldots, t_h \in \LAMBDA_\bot$.
Then we set 
$$
\rho(t,X,n+1) = c(\rho(t_1,X_1,n)\ldots(\rho(t_h,X_h,n))
$$
We define $\rho(t,X) = \lim_{n \rightarrow \infty} \rho(t,X,n)$,  
$\sigma(t,X,Z) = \lim_{n \rightarrow \infty} \sigma(t,X,n,Z)$
and $t \reduces_X \rho(t,X)$.
\end{definition}


\begin{theorem}[Infinite Church-Rosser]
\label{theorem-infinite-church-rosser}
For all $t,u,v \in \LAMBDA_\bot$, all decidable sets $X$ of redexes of $t$:
if $t \reduces_X u$ and $t \reduces_Y v$, 
then for some $w \in \LAMBDA_\bot$, for some sets
$Z=\sigma(t,X,n,Y)$, $T = \sigma(t,Y,n,X)$ of redexes of $u, v$
we have $u \reduces_Z w$ and $v  \reduces_T w$ and $t \reduces_{X \cup Y} w$.
\end{theorem}

\begin{proof}
We have to prove that $\rho(\rho(t,X),Y))  = \rho(t,X \cup Y)$.
We prove that for all $n,m \in \N$
there is some $p=n+m \in \N$ such that  $\rho(\rho(t,X,n),Y,m))  = \rho(t,X \cup Y,p)$,
and conversely that for all $p \in \N$ there are $n,m\in\N$ such that $p=n+m$ and
$\rho(\rho(t,X,n),Y,m))  = \rho(t,X \cup Y,p)$.
\end{proof}

We define the safe trunk of a term as the part of the term which we can normalize with safe reductions only.
In the rest of this section we will
prove that Infinite Church-Rosser implies that if the safe trunk exists then it is unique. 

Infinite Church-Rosser and Safe strong Normalization together imply that after finitely many steps
all safe reductions reach the same safe trunk.

\begin{definition}[Safe Trunk of a term]
\label{definition-safe-trunk}
Assume $t \in \LAMBDA$.
\begin{enumerate}
\item
The safe trunk of $t$ is any expression $u[\cond(f_1,\cdot), \ldots, \cond(f_n,\cdot)]$
such that  for some $g_1, \ldots, g_n$ we have $v = u[\cond(f_1,g_1), \ldots, \cond(f_n,g_n)]$
\emph{safe normal} and $t \reduces v$.
\item
$t$ is \emph{finite for safe reduction} if and only if all infinite reduction sequences from $t$ 
include only finitely many \quotationMarks{safe} reduction steps.  
\end{enumerate}
\end{definition}


\begin{lemma}[Safe Trunk of a term]
\label{lemma-safe-trunk}
Assume $t$ is finite for safe reductions.
If the  safe-trunk of $t$ exists then it is unique. 
\end{lemma}


\begin{proof}
Assume $t$ is finite for safe reductions in order to prove that the safe-trunk of $t$ is unique.

Assume that $u[\cond(f_1,\cdot), \ldots, \cond(f_n,\cdot)]$ and
$u'[\cond(f'_1,\cdot), \ldots, \cond(f'_{n'},\cdot)]$ are safe-trunks for $t$, in order to prove
that $u=u'$ and $n=n'$. 

Then for some $g_1, \ldots,g_n$ and some $g'_1, \ldots,g'_n$ we have that 
$v = u[\cond(f_1,g_1), \ldots, \cond(f_n,g_n)]$ and 
$v' = u'[\cond(f'_1,g'_1), \ldots, \cond(f'_n,g'_{n'})]$ 
are safe-normal forms of $t$ and all $\cond$-expressions shown are maximal. 
The decomposition of each safe-normal form $v$ is therefore unique:
if $v = u"[\cond(f"_1,g"_1), \ldots, \cond(f"_n,g"_{n"})]$ then $u=u"$ and $n=n"$
and $f_1=f"_1$, \ldots, $f_n=f"_n$.
Each reduction from $v$, $v'$ takes place in some $g_1, \ldots,g'_{n'}$. 

By Infinite Church-Rosser  
(theorem \ref{theorem-infinite-church-rosser}) we deduce that $v$ and $v'$ are confluent
outside the right-hand-side of $\cond$-expressions, therefore
for some $v"$ we have $v \reduces_Z v"$ and $v' \reduces_T v'"$. 
Since the reductions on $v, v'$ take place in some $g_1, \ldots,g'_{n'}$, 
we deduce that $v" = u[\cond(f_1,g"_1), \ldots, \cond(f_n,g"_n)]$
and $v'" = u[\cond(f'_1,g'"_1), \ldots, \cond(f'_n,g'"_{n'})]$ for some $g"_1, \ldots,g'"_{n'}$.
From the unicity of the decomposition of $v"$
with maximal $\cond$-subterms we conclude that $u=u'$ and $n=n'$
and and $f_1=f'_1$, \ldots, $f_n=f'_n$, as wished.

\end{proof}


