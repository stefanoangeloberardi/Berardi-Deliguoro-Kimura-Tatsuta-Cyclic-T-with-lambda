%!TEX root = CTL-main.tex
\section{Substitution and reduction}\label{sec:reduction}


\medskip
\begin{definition}[Substitution]\label{def:substitution}
Let $\vec{x} = x_1, \ldots , x_n$ be pairwise distinct variables, and $\vec{u} = u_1, \ldots u_n$ terms in $\Term$. 
Define the {\em simultaneous substitution} $t[u_1/x_1,\ldots,u_n/x_n]$, also written $t[\vec{u}/{\vec{x}}]$, of the $x_i$ by $u_i$ in $t$
by corecursion on $t$:
\begin{enumerate}
\item $x_i [\vec{u}/{\vec{x}}] = u_i$, $y [\vec{u}/{\vec{x}}] = y$ if $y \not\in \vec{x}$
\item $(t v)[\vec{u}/{\vec{x}}] = (t [\vec{u}/{\vec{x}}])(v [\vec{u}/{\vec{x}}])$  \vspace{1mm}
\item \label{def:substitution-lambda}
         $(\lambda y^A.t) [\vec{u}/{\vec{x}}] = % \lambda z^A. (t [z/y])  [\vec{u}/{\vec{x}}]$ for some $z \not \in \FV(\vec{u}) \cup \vec{x}$
		\left\{ \begin{array}{ll}
		\lambda y^A. (t [\Restrict{(\vec{u}/\vec{x})}{\overline{y}}]) & \mbox{if $y \not \in \FV(\vec{u})$} \\
												     % & \mbox{$\vec{x} \cap \FV(t) = \emptyset$} \\
		\lambda z^A. t [\Restrict{(\vec{u}/\vec{x})}{\overline{y}}, z/y] & \mbox{otherwise}
		\end{array}\right.$
	where $z \not \in \FV(t,\vec{u})$ is a fresh variable and $\Restrict{(\vec{u}/\vec{x})}{\overline{y}}$ 
	is obtained by removing the pair $u'/y$ from $\vec{u}/{\vec{x}}$ for if any
\item $0  [\vec{u}/{\vec{x}}] = 0$
\item $(\Succ(t)) [\vec{u}/{\vec{x}}] = \Succ (t [\vec{u}/{\vec{x}}])$
\item $(\cond(t,v)) [\vec{u}/{\vec{x}}] = \cond(t [\vec{u}/{\vec{x}}], v [\vec{u}/{\vec{x}}])$
\end{enumerate}
A {\em substitution} $\theta$ is a mapping over $\Term$ defined by $\theta(t) = t[\vec{u}/{\vec{x}}]$; $\theta$ is a {\em renaming} if all
the terms in $\vec{u}$ are pairwise distinct variables.
\end{definition}

It is straightforward to show that $t[\vec{u}/{\vec{x}}]$ is well defined and uniquely determined if the
fresh $z$ in (\ref{def:substitution}.\ref{def:substitution-lambda}) is the least variable not in $\FV(t,\vec{u})$ w.r.t. some fixed and
well-founded order of the variables of type $A$; such a $z$ exists since $\FV(t,\vec{u})$ is finite by the regularity of $t$ and of the $u_i$ and
because the set of variables of type $A$ is denumerable for all $A$. 

As for finitary $\lambda$-calculus, we identify $\alpha$-equivalent terms, namely terms that can be obtained by renaming of their bound variables, to avoid capture of free variables under substitution. This is achieved via the following coinductive definition, where double lines are used in the rules to stress that we take the greatest fixed point of the endo-function of binary relations over $\Term$ associated to the them.

\begin{definition}[$\alpha$-Equivalence]\label{def:alpha-equivalence}
The binary relation $=_\alpha$ over $\Term$ is coinductively defined by the rules:
\[\begin{array}{ccc}
 \infer=[]{x =_\alpha x}{x \in \Var} &
 \infer=[]{tu =_\alpha t'u'}{t =_\alpha t' \quad s =_\alpha s'} &
 \infer=[]{\lambda x.t =_\alpha \lambda y. s[y/x]}{t =_\alpha s \quad y \not \in \FV(t,s)} \\[2mm]
 \infer=[]{0 =_\alpha 0}{} & 
 \infer=[]{\Succ(t) =_\alpha \Succ(s)}{t =_\alpha s} &
 \infer=[]{\cond(t, s) =_\alpha \cond(t', s')}{t =_\alpha t' \quad s =_\alpha s'}
\end{array}\]
and their symmetric.
\end{definition}

%\medskip
\vspace{1mm}
\begin{lemma}\label{lem:substitution-composition} 
If $x\neq y$ and $x \not \in \FV(v)$ then 
\[t[u/x][v/y] =_\alpha t[v/y][u[v/y]/x].\]
\end{lemma}

\begin{proof} By coinduction and cases of $t$: the only interesting one is when
$t = \lambda z.t'$: by Definition \ref{def:alpha-equivalence} we can freely assume
that $z \not\in \FV(u,v)$, since otherwise it can be replaced by some fresh $z'$ in $t'$ yielding
the term $\lambda z'.t'[z'/z] =_\alpha \lambda z.t'$. That $z \not\in \FV(u,v)$ implies 
$(*)$ $z \not\in \FV(u[v/y],v)$ since substitution
does not generate new free variables. Then
\[\begin{array}{lll}
& (\lambda z.t') [u/x][v/y] \\
= & \lambda z. \ t' [\Restrict{(u/x)}{\overline{z}}][\Restrict{(v/y)}{\overline{z}}] \\
=_\alpha & \lambda z. \ t' [\Restrict{(v/y)}{\overline{z}}] [\Restrict{(u[v/y]/x)}{\overline{z}}] & \mbox{by $(*)$ and coinduction} \\
= &  (\lambda z.t') [v/y][u[v/y]/x]
\end{array}\]
\qed
\end{proof}

\begin{definition}[Reductions $\reduces_\beta$, $\reduces_\cond$]\label{def:reduction}
%\hfill 
Reductions $\reduces_\beta$ and $\reduces_\cond$ are the binary relations over $\Term$ that are the compatible closure of 
the following contraction rules:

\begin{enumerate}
\item \label{def:reduction-1} $(\lambda x.t)u \reduces_\beta t[u/x]$
\item \label{def:reduction-2} $\cond(f,g)(0) \reduces_\cond f$
\item \label{def:reduction-3} $\cond(f,g)(\Succ (t)) \reduces_\cond g(t)$
\end{enumerate}
We define $\reduces = \reduces_\beta \cup \reduces_\cond$, $\reduces^+$ and $\reduces^*$ as the transitive, and reflexive and transitive
closures of $\reduces$ respectively.
\end{definition}

The left-hand-sides in (\ref{def:reduction-1})-(\ref{def:reduction-3}) above are called {\em redexes}, the rigt-hand-sides the respective {\em contracta};
the replacement of a redex by its contractum in called a {\em contraction}.
Let $\BTree_t = (|\BTree_t|, \varphi_t)$ be the syntactic tree of $t\in \Term$;
the {\em $\cond$-depth} of a position $\sigma \in |\BTree_t|$ is the number of occurrences of $\cond$ 
in the branch from the root of $\BTree_t$ to $\sigma$ when it traverses the second argument of $\cond$ and includes its root.
Formally
\[\condDepth{\sigma} = |\set{\tau 2 \leq \sigma \mid \varphi_t(\tau) = \cond}|\]
The {\em $n$-safe level} of $t$ is the set of nodes $\set{\sigma \in |\BTree_t| \mid \condDepth{\sigma} \leq n}$.
%\newcommand{\condDepth}[1]{\textit{$\cond$-depth}(#1)}

\begin{definition}[$n$-Safe Reduction and Normal Forml]\label{def:n-safe}
\hfill
\begin{enumerate}
\item A redex $r$ is {\em $n$-safe} in $t \in \Term$ if $r = \SubTree{t}{\sigma}$ for some $\sigma$ in the $n$-safe level of $t$.

\item We say that $t$ {\em $n$-safely reduces} to $u$, written $t \nsafeReduces{n} u$, if $t \reduces u$ by contracting an $n$-safe
redex of $t$.

\item The term $t$ is $n$-safe normal if there is no $u$ such that $t \nsafeReduces{n} u$.
\end{enumerate}
\end{definition}

Hence, in a $n$-safe normal term $t$ all redexes, if any, occur at positions with $\cond$-depth greather than $n$. We call
a $0$-safe normal term just {\em safe-normal}.

