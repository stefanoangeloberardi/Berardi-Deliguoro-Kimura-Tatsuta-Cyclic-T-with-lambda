

\newpage

\section{Weak Church-Rosser for leveled safe reductions}
\label{section-n-safe-church-rosser}

In this section we prove a result for the $n$-safe part of a term, which we call 
\quotationMarks{\emph{unicity of the $n$-safe part of the safe normal form up to $\nequal{n}$}}. By this we mean:
for all $t, u, v \in \GTC$, if $t \nsafeReduces{n} u$ and $t \nsafeReduces{n} v$ and $u$, $v$ are $n$-safe-normal 
then $u \nequal{n} v$ holds. 

Our first idea (wrong) is to prove a full Church-Rosser property for $\LAMBDA$: 
for all $t,u,v \in \LAMBDA$, if $t \reduces u$ and $t \reduces v$ then for some $w \in \LAMBDA$
we have $u \reduces w$ and $v \reduces w$. This property is false: for some $t \in \LAMBDA$, finding a 
common reduction of $u$, $v$ takes infinitely many steps. This even in the case $t \in \CTlambda$,
as the next example shows.

\begin{Eg}[Failure of Church-Rosser for $\CTlambda$]
Let $b = \cond(x^{\N},b):\N \rightarrow \N$ a normal form
and $t = (\lambda x^{\N}.b)(r):\N \rightarrow \N$, 
where $r = (\lambda x^{\N}.x^{\N})(3)$ is some redex. 

We have $b[r/x](n) \reduces r \reduces 3$ for all numerals $n$, 
therefore $t$ and $\lambda \_.3$ are extensionally equal, however $t \not \reduces \lambda \_.3$. 
We have $t \in \CTlambda$. Indeed, 
\begin{enumerate}
\item
$t$ is regular by construction.
\item
We have $t \in \GTC$, because the unique infinite path of $t$ is 
$t, \lambda x^{\N}.b, b, b, b, \ldots$, and the
unique unnamed argument of $b:\N \rightarrow \N$ in the path progresses infinitely many times.
\end{enumerate}

Now consider the reductions: $t \reduces b[r/x^\N]$ and $t \reduces  (\lambda x^{\N}.b)(3)$.
We expect $b[3/x^\N]$ as common normal form. But we have $b[r/x^\N] = \cond(r,b[r/x^\N]$,
that is, we have replicated the redex $r$ infinitely many times in $b[r/x^\N]$. Therefore to reduce 
$b[r/x^\N]$ to $b[3/x^\N]$ takes infinitely many steps, and for \emph{no finite reduction we have}
$b[r/x^\N] \reduces b[3/x^\N]$. 

We proved that Church-Rosser is false for $\CTlambda$.
\end{Eg}

In the following we will work in the $n$-safe level of $t\in \GTC$. 
The global trace condition guarentees the finiteness of the $n$-safe level for each $n$. 
\begin{lemma}
  For any $n\ge 0$, the set of the $n$-safe level of $t\in \GTC$ is finite.
\end{lemma}
\begin{proof}
  Let $t\in \GTC$. We show the finiteness of any $n$-level of $t$ by proof of contradiction.
  Assume that the finiteness fails. 
  Then take the least number $n$ such that the $n$-safe level of $t$ is infinite.
  
  The case of $n=0$.
  Since the binary tree obtained from $t$ restricting to the $0$-safe level of $t$ contains infinite nodes,
  we can take an infinite path $t,t_1,t_2,\ldots$ of the tree by using K\"{o}nig's lemma.
  Then this infinite path does not contain a progressing trace since any $t_i$
  cannot be the right subterm of $\cond$. This contradicts $t\in \GTC$.
  
  The case of $n>0$. By the leastness of $n$, for any $n'<n$, all $n'$-level of $t$ are finite.
  We can also take an infinite path from 
  the binary tree that is a restriction of $t$ with nodes until the $n$-safe level of $t$. 
  The path has the form $\vec{t_0},\vec{t_1},\ldots,\vec{t_n}$,
  where each $\vec{t_i}$ is a sequence of nodes of the $i$-safe level,
  and only $\vec{t_n}$ is infinite.
  Then this infinite path does not contain a progressing trace as in the case of $n=0$. 
  This also contradicts $t\in \GTC$.
  
  Hence we have the finiteness of the $n$-safe level of $t\in\GTC$, as we wished. 
\end{proof}

We write $\Lv{n}{t}$ for the number of the $n$-safe level of $t\in\GTC$. 

We define a $n$-safe level context $\Lctx{n}$
with holes (written $\cdot$) that will be filled with terms in $\LAMBDA$,
and whose positions are at the $n$-safe level. 
\begin{definition}[Safe level context]
  We define $\Lctx{-1} = \cdot$. 
  For $n \ge 0$, the $n$-safe level context, written $\Lctx{n}$, is inductively defined as
  follows:
  \[
  \Lctx{n} ::= x^T \mid 0 \mid \Lctx{n}\Lctx{n} \mid \lambda x^T.\Lctx{n}
  \mid \Suc{\Lctx{n}} \mid \Cond{\Lctx{n}}{\Lctx{n-1}}. 
  \]
\end{definition}
Multiple holes may appear in a context and each holes are distinguished.
The resulting term obtained by filling $k$-holes in $\Lctx{n}$
with terms $t_1,\ldots,t_k$ is written $\Lctx{n}[t_1,\ldots,t_k]$. 
Note that filling holes with terms is not substitution,
but just putting terms at the positions of holes, namely, for example,
the result of filling the unique hole in $\lambda x.\Cond{0}{\cdot}$ with $x$
is $\lambda x.\Cond{0}{x}$. 

The finiteness of the $n$-safe level of $t\in\GTC$ enables 
to split $t$ by a $n$-safe level context and terms that apper at level $>n$
as stated in the next lemma. 

\begin{lemma}\label{lem:split_context}
  \begin{enumerate}
  \item\label{lem:split_context1}
    For any $\Lctx{n}$, there uniquely exists
    $(\Lctx{0},\Lctx{n-1}^{1},\ldots,\Lctx{n-1}^{k})$ such that
    $\Lctx{0}$ has $k$-holes and
    $\Lctx{n} = \Lctx{0}[\Lctx{n-1}^1,\ldots,\Lctx{n-1}^k]$, namely $\Lctx{n}[\vec{t_1},\ldots,\vec{t_k}] = \Lctx{0}[\Lctx{n-1}^1[\vec{t_1}],\ldots,\Lctx{n-1}^k[\vec{t_k}]]$ holds for any $\vec{t_1},\ldots,\vec{t_k}$. 
  \item\label{lem:split_context2}
    Let $t\in\GTC$ and $n\ge 0$.
    There uniquely exists $(\Lctx{n},t_1,\ldots,t_k)$
    such that $\Lctx{n}$ has $k$-holes, $t = \Lctx{n}[t_1,\ldots,t_k]$,
    and all $t_1,\ldots,t_k$ appear at the $(n+1)$-safe level of $t$.
  \end{enumerate}
\end{lemma}
\begin{proof}
  The point~\ref{lem:split_context1} is shown by induction on the construction of $\Lctx{n}$.
  If $\Lctx{n}$ is $x^T$ or $0$, then define $\Lctx{0}$ by the same one as $\Lctx{n}$. 
  For the case of $\Lctx{n} = \lambda x^T.\Lctx{n}'$, 
  we have a unique $(\Lctx{0}',\overrightarrow{\Lctx{n-1}'})$ such that 
  $\Lctx{n}' = \Lctx{0}'[\overrightarrow{\Lctx{n-1}'}]$.
  Then $(\lambda x^T.\Lctx{0}',\overrightarrow{\Lctx{n-1}'})$
  satisfies the expected condition for $\Lctx{n}$ of this case. 
  The cases of $\Lctx{n} = \Lctx{n}'\Lctx{n}''$ and $\Lctx{n} = \Suc{\Lctx{n}'}$
  are shown similarly by using the induction hypothesis.
  We show the case $\Lctx{n} = \Cond{\Lctx{n}'}{\Lctx{n-1}''}$. 
  By the induction hypothesis, there exists a unique
  $(\Lctx{0}',\overrightarrow{\Lctx{n-1}'})$ such that 
  $\Lctx{n}' = \Lctx{0}'[\overrightarrow{\Lctx{n-1}'}]$.
  Then $(\Cond{\Lctx{0}'}{\cdot}, \overrightarrow{\Lctx{n-1}'}, \Lctx{n-1}'')$
  satisfies the expected condition for $\Lctx{n}$ of this case. 
  
  The point~\ref{lem:split_context2} is shown
  by induction on $n$ using \ref{lem:split_context1}.

  We first show the case of $n=0$ by induction on the number of the $0$-safe level of $t$.
  If $t = x^T$ or $t = 0$, it is shown by taking $\Lctx{0}$ as $t$.
  If $t = \lambda x^T.t'$, by the induction hypothesis,
  there uniquely exists $(\Lctx{0}',\vec{t'})$ that satisfies the condition for $t'$. 
  Then $(\lambda x^T.\Lctx{0}',\vec{t'})$ satisfies
  the expected condition for $\lambda x^T.t'$. 
  If $t =t't''$ or $t = \Suc{t'}$, it is also shown by the induction hypothesis. 
  If $t = \Cond{t'}{f}$, by the induction hypothesis, 
  there uniquely exists $(\Lctx{0}',\vec{t'})$ that satisfies the condition for $t'$. 
  Then $(\Cond{\Lctx{0}'}{\cdot},\vec{t'},f)$ satisfies
  the expected condition for $\Cond{t'}{f}$. 

  Then we show the case of $n>0$.
  By the result of the case of $n=0$, there uniquely exists $(\Lctx{0},t_1,\ldots,t_k)$
  that satisfies $t = \Lctx{0}[t_1,\ldots,t_k]$ and
  each $t_i$ appears at the $1$-safe level of $t$.
  For each $i$, by applying the induction hypothesis to $t_i$,
  we have unique $(\Lctx{0}^i,\vec{t'_i})$
  satisfies $t_i = \Lctx{0}^i[\vec{t'_i}]$ and $\vec{t'_i}$ appear
  at the $(n-1)$-safe level of $t_i$. 
  Then $(\Lctx{0}[\Lctx{n-1}^1,\ldots,\Lctx{n-1}^k],\vec{t'_1},\ldots,\vec{t'_k})$
  satisfies the expected condition for $t$.
  Its uniqueness is obtained by using the uniqueness of $\Lctx{0}$ and $\Lctx{n-1}^i$,
  and the point \ref{lem:split_context1}. 
\end{proof}

We define the notion \quotationMarks{$n$-safe equality} that intuitively means
two terms are approximately equal excepting for deeper safe levels more than $n$. 

\begin{definition}[$n$-safe equal]
  $t_1 \nequal{n} t_2$, or \quotationMarks{$t_1$ and $t_2$ are $n$-safe equal}
  is defined by after possibly renaming the bound variables of $t_1$ and $t_2$, 
  they have the forms $\Lctx{n}[\vec{u}_1]$ and $\Lctx{n}[\vec{u}_2]$
  with some terms $\vec{u}_1$ and $\vec{u}_2$. 
\end{definition}

The $n$-safe equality is necessary to fill \quotationMarks{gaps}
after $n$-safe reductions from the same term.
For example, let $t$ be $(\lambda x^{\N\to\N}.\Cond{0}{x})(IS)$,
where $I = \lambda z^{\N\to\N}.z$ and $S = \lambda n^\N.\Suc{n}$.
For the two $0$-safe reducuctions 
$t \nsafeReducesAst{0} \Cond{0}{S}$ and $t \nsafeReducesAst{0} \Cond{0}{IS}$, 
we have $\Cond{0}{IS} \nequal{0} \Cond{0}{S}$
since $\Cond{0}{IS} \nsafeReduces{0} \Cond{0}{S}$ does not hold. 

The main theorem of this section is a weak form of Church-Rosser
of the $n$-safe reductions that holds up to the $n$-equalities. 

\begin{theorem}[Weak Church-Rosser of $n$-safe reduction modulo $n$-safe equality]
  Let $t\in \GTC$.
  If $t_1 \nsafeReducesAstL{n} t \nsafeReducesAst{n} t_2$, 
  then there exist $t'_1$ and $t'_2$ such that
  $t_1 \nsafeReducesAst{n} t'_1 \nequal{n} t'_2 \nsafeReducesAstL{n} t_2$. 
\end{theorem}

This theorem will be proved by a variant of the parallel reduction technique. 

\begin{definition}[Safe parallel reduction]
  Let $n\ge -1$.
  We define the $n$-safe parallel reduction relation $\nsafePReduces{n}$ on the $\GTC$ terms.
  The $\nsafePReduces{-1}$ is defined by $t\nsafePReduces{-1} t$ for all $t\in\GTC$. 
  For $n\ge 0$, the relation $\nsafePReduces{n}$ is inductively defined as follows: 
  \begin{itemize}
  \item[(id)]
    $t \nsafePReduces{n} t$.
  \item[$(\Succ)$]
    If $t \nsafePReduces{n} t'$, then $\Suc{t} \nsafePReduces{n} \Suc{t'}$.
  \item[$(\lambda)$]
    If $t \nsafePReduces{n} t'$, then $\lambda x^T.t \nsafePReduces{n} \lambda x^T.t'$.
  \item[(ap)]
    If $f \nsafePReduces{n} f'$ and $a \nsafePReduces{n} a'$,
    then $f(a) \nsafePReduces{n} f'(a')$.
  \item[$(\cond)$]
    If $a \nsafePReduces{n} a'$ and $f \nsafePReduces{n-1} f'$,
    then $\Cond{a}{f} \nsafePReduces{n} \Cond{a'}{f'}$.
  \item[$(\beta)$]
    If $t \nsafePReduces{n} t'$ and $u \nsafePReduces{n} u'$,
    then $(\lambda x^T.t)u \nsafePReduces{n} t'[u'/x]$.
  \item[$(\cond\,0)$]
    If $a \nsafePReduces{n} a'$, 
    then $\Cond{a}{f}(0) \nsafePReduces{n} a'$.
  \item[$(\cond\,\Succ)$]
    If $f \nsafePReduces{n-1} f'$ and $u \nsafePReduces{n} u'$, 
    then $\Cond{a}{f}(\Suc{u}) \nsafePReduces{n} f'(u')$.
  \end{itemize}
\end{definition}

The relation $\nsafePReduces{-1}$ is a special case to simplify
the rules $(\cond)$ and $(\cond\,\Succ)$.
For $n\ge 0$, if $t$ is reduced to $u$ by the $n$-safe parallel reduction,
redexes that appear in the $n$-safe level of $t$ can be reduced. 
For example, $\Cond{I0}{\Cond{I0}{IS}} \nsafePReduces{n} \Cond{0}{\Cond{0}{S}}$
holds if $n=2$, but does not hold if $n=1$. 
The case (id) cannot be replaced by $x^T \nsafePReduces{n} x^T$ and $0 \nsafePReduces{n} 0$ expecting (id) can derived from them, but it is not the case in our setting
because we are working on the infinite terms. 

The relation $\nsafePReduces{n}$ satisfies the following basic properties. 
\begin{lemma}\label{lem:parallel_basic}
  Let $t, u\in \GTC$. The following properties hold. 
  \begin{enumerate}
  \item\label{lem:parallel_basic1}
    $n < n'$ and $t \nsafePReduces{n} u$ implies $t \nsafePReduces{n'} u$ (monotonicity). 
  \item\label{lem:parallel_basic2}
    If $t \nsafeReduces{n} u$, then $t \nsafePReduces{n} u$. 
  \item\label{lem:parallel_basic3}
    If $t \nsafePReduces{n} u$, then $t \nsafeReducesAst{n} u$. 
  \end{enumerate}
\end{lemma}
\begin{proof}
  The point~\ref{lem:parallel_basic1} is shown by induction on $\nsafePReduces{n}$.
  We show the point~\ref{lem:parallel_basic2}.
  
  

\end{proof}
