
\newcommand{\xx}{\boldsymbol{x}}

\section{Subject Reduction for Well-Typed Infinite Lambda Terms}
\label{section-subject-reduction}

We show the subject reduction for well-typed terms of $\LAMBDA$,
and also show the global trace condition is preserved by reductions. 
We first introduce some auxiliary notations for proofs of them.

Let $X$ be a set of variables of the form $x^T$.
We write $\Gamma_X$ for the sublist of $\Gamma$ consisting of all $x^T:T$ such that $x^T\in X$. 
Let $\Gamma$ and $\Delta$ be contexts of $\LAMBDA$.
The merged context $\mergeCtx{\Gamma}{\Delta}$
is defined by $\Gamma\conc(\Delta_{\FV(\Delta)\setminus\FV(\Gamma)})$. 

Let $\theta$ be a renaming, and 
$S_1 = \Gamma_1\vdash t_1:\vec{B_1}\rightarrow N$
and $S_2 = \Gamma_2[\theta]\vdash t_2[\theta]:\vec{B_2}\rightarrow N$ be sequents. 
Let $k_1$ and $k_2$ be indexes of $N$-arguments of $t_1$ and $t_2$, respectively. 
Then we write $(k_1,S_1) \simIndex{\theta} (k_2,S_2)$ (or $(k_1,t_1) \simIndex{\theta} (k_2,t_2)$ for short)
if they are respectively indexes of named arguments for some $y$ and $\theta(y)$, or
are respectively those of unnamed arguments at the same position $i$ in $\vec{B_1}$ and $\vec{B_2}$,
namely $k_1=|\Gamma_1|+i$ and $k_2=|\Gamma_2|+i$,
where $|\Gamma_1|$ and $|\Gamma_2|$ are the sizes of $\Gamma_1$ and $\Gamma_2$. 
Note that the index equivalent to $k_1$ is unique (if it exists), namely 
$(k_1,t_1) \simIndex{\theta} (k_2,t_2)$ and $(k_1,t_1) \simIndex{\theta} (k'_2,t_2)$ implies $k_2=k'_2$.
We write $(k_1,t_1) \simIndex{} (k_2,t_2)$ if $\theta$ is the identity renaming. 

In this paper we do not implicitly identify $\alpha$-equivalent terms,
in order to make simpler to state the global trace condition.
For this reason, proofs of this section requires some delicate treatment of variables.

We consider the following restricted $\lambda$ rule (called $\lambda'$) as follows:
\begin{itemize}
\item
  $\lambda'$-rule.
  If $\Gamma, x^A:A \vdash b: B$ and $\FV(\Gamma) = \FV(\lambda x.b)$, 
  then $ \Gamma \vdash \lambda x^A.b :A \rightarrow B$.
\end{itemize}

The next lemma says that if $\Gamma\vdash t:A$ is provable,
then $t:A$ can be shown in a restricted context $\Gamma_{\FV(t)}$
even if the rule $\lambda$ is restricted to $\lambda'$. 

\begin{lemma}\label{lem:thinning}
  Assume $\Pi:\Gamma\vdash t:A$.
  Then $\Pi':\Gamma_{\FV(t)}\vdash t:A$ for some $\Pi'$ that may have instances of the rule $\lambda'$
  and not have those of the rule $\lambda$.
  Moreover if the global trace condition holds for $\Pi$, then it also holds for $\Pi'$. 
\end{lemma}
\begin{proof}
  First we call the following admissible rule $\lambda'\weak$:
  \begin{itemize}
  \item
    $\lambda'\weak$-rule.
    If $\Gamma, x^A:A \vdash b: B$, $\FV(\Gamma) = \FV(\lambda x.b)$ and $\Gamma\subseteqsim \Gamma'$, 
    then $ \Gamma' \vdash \lambda x^A.b :A \rightarrow B$.
  \end{itemize}
  We write $\Rule'$ as the set of rule instances obtained by removing
  those of $\lambda$ from $\Rule$ and adding those of $\lambda'\weak$. 
  Note that if we have a proof of a sequent with $\lambda'\weak$,
  then we also have a proof of the same sequent not with $\lambda'\weak$
  by a proof transformation that splits each $\lambda'\weak$ by $\lambda'$ and $\weak$. 
  Also note that this proof transformation preserves the global trace condition.

  Let $\Pi$ be $(T,\phi)$.
  For each $l \in T$, we write $\Gamma_l\vdash t_l:A_l$
  for the conclusion of $\phi(l) \in \Rule$. 
  To show the lemma, from a given proof $\Pi$, 
  it is enough to construct a proof $\Pi'$ of $\Restrict{\Gamma}{\FV(t)}\vdash t:A$ with $\Rule'$.
  
  We define the set of nodes of $\Pi'$ is $T$, which is the same one of $\Pi$. 
  For each $l\in T$, we define $\phi'(l)$ and $\Gamma'_l$ that satisfies
  the following requirements:
  \begin{itemize}
  \item[(a)]
    $\Gamma'_l\vdash t_l:A_l$ is the conclusion of $\phi'(l) \in \Rule'$,
  \item[(b)]
    $\Restrict{(\Gamma_l)}{\FV(t_l)} \subseteqsim \Gamma'_l \subseteqsim \Gamma_l$
    and $\Gamma'_{\nil} = \Restrict{\Gamma}{\FV(t)}$, 
  \item[(c)]
    if $\tilde{k},t_{l\conc(i)}$ is the successor of $k,t_l$ in $\Pi$
    and $k$ is an index of some unname argument
    or a named argument of some name $z\in\FV(t_l)$, 
    then there are $\tilde{k'}$ and $k'$ such that
    $\tilde{k'},t_{l\conc(i)}$ is the successor of $k',t_l$ in $\Pi'$, 
    $(k,t_l) \simIndex{} (k',t_l)$,
    and $(\tilde{k},t_{l\conc(i)}) \simIndex{} (\tilde{k'},t_{l\conc(i)})$.
    Moreover, if $k$ is an index of a progressing argument, then so is $k'$.
  \end{itemize}
  Note that the proof is done if we construct $\Pi'$ that satisfies these requirements. 
  
  We define $\Gamma'_{\nil} = \Restrict{\Gamma}{\FV(t)}$.
  Then it satisfies (b) since $t_\nil = t$. 
  Next, assuming the induction hypothesis that $\Gamma'_l$ which satisfies (b)
  is already defined, 
  we define $\phi'(l)$ and $\Gamma_{l\conc(i)}$, 
  for each $i$ such that $l\conc(i)\in T$, 
  that satisfies (a), (b), and (c). 
  It is done by the case analysis of $\phi(l)$.

  The case of $\phi(l) = \Gamma_l\vdash x:A_l$,
  which is an instance of the rule $\var$ with $x:A_l\in\Gamma_l$. 
  Then define $\phi'(l) = \Gamma'_l\vdash x:A_l$. This is an instance of $\var$
  because $x:A_l \in \Restrict{(\Gamma_l)}{\FV(x)} \subseteqsim \Gamma'_l$ by (b).

  The case of $\phi(l) = (\Gamma_{l\conc(1)}\vdash t_l:A_l,\Gamma_{l}\vdash t_l:A_l)$,
  which is an instance of the rule $\weak$ with
  $t_{l\conc(1)} = t_l$, $A_{l\conc(1)} = A_{l}$, and
  $\Gamma_{l\conc(1)}\subseteqsim \Gamma_l$.
  Let $\psi$ be the unique map that determines $\Gamma_{l\conc(1)}\subseteqsim \Gamma_l$.
  Then define $\Gamma'_{l\conc(1)}$ such that
  $\Gamma'_{l\conc(1)}\subseteqsim \Gamma'_l$ determined by
  the induced map from $\psi$ restricting the range to $\FV(\Gamma'_l)$.
  Note that $\FV(\Gamma'_{l\conc(1)}) = \FV(\Gamma'_l)\cap\FV(\Gamma_{l\conc(1)})$. 
  Then it satisfies (b) since $\FV(t_l)\subseteq \FV(\Gamma'_l) \cap \FV(\Gamma_{l\conc(1)}) = \FV(\Gamma'_{l\conc(1)}) \subseteq \FV(\Gamma_{l\conc(1)})$ by (b) for $\Gamma'_l$.
  Define $\phi'(l) = (\Gamma'_{l\conc(1)}\vdash t_l:A_l,\Gamma'_{l}\vdash t_l:A_l)$
  as an instance of the rule $\weak$. 
  Hence the requirement (a) holds.
  We can also show (c): if $k$ is an index in $\Gamma_l\vdash t_l:A_l$ for a name
  $z\in\FV(t_{l\conc(1)}) = \FV(t_l)$, then we can take an index $k'$
  in $\Gamma'_l\vdash t_l:A_l$ for $z$ by $\FV(t_l)\subseteq\FV(\Gamma'_l)$ by (b). 
  An index $\tilde{k'}$ for the name $z$ can be taken
  from $\Gamma'_{l\conc(1)}\vdash t_l:A_l$
  since $z \in \FV(t_l) \subseteq \FV(\Gamma'_{l\conc(1)})$. 
  
  The case of $\phi(l) = (\Gamma_l,z:C\vdash b:\vec{B}\to N, \Gamma_l\vdash\lambda z^C.b:C,\vec{B}\to N)$
  that is an instance of the rule $\lambda$. 
  Define $\Gamma'_{l\conc(1)} = \Restrict{(\Gamma_l)}{\FV(\lambda z.b)},z:C$ and 
  $\phi'(l) = (\Restrict{(\Gamma_l)}{\FV(\lambda z.b)},z:C\vdash b:\vec{B}\to N, \Gamma'_l\vdash \lambda z.b:C,\vec{B}\to N)$. 
  This is an instance of the rule $\lambda'\weak$, since $\Restrict{(\Gamma_l)}{\FV(\lambda z.b)}\subseteqsim \Gamma'_l$ by the induction hypothesis.
  Trivially we have (a). 
  We also have (b) by $\Restrict{(\Gamma_l,z:C)}{\FV(b)} \subseteqsim (\Restrict{(\Gamma_l)}{\FV(\lambda z.b)},z:C) \subseteqsim (\Gamma_l,z:C)$.   
  The requirement (c) holds: 
  if $k$ is an index of some named argument $y\in\FV(\lambda z.b)$ in $\Gamma_l$,
  then $k'$ can be taken as the index of $y$ in $\Gamma'_l$ by (b) for $\Gamma_l$.
  If $k$ is an index of some unnamed argument in $C,\vec{B}$,
  then $k'$ can be taken as the index of some unnamed argument. 
  In both cases, their successors $\tilde{k}$ and $\tilde{k'}$ are uniquely
  determined by $k$ and $k'$, respectively. 
  
  The case that $\phi(l)$ is an instance of the rule $\apvar$
  whose conclusion is $\Gamma_l\vdash f(x^B):\vec{C}\to N$ with $x:B\in \Gamma_l$.  
  Define $\Gamma'_{l\conc(1)} = \Gamma'_l$.
  This satisfies (b) by the induction hypothesis. 
  Then define
  $\phi'(l) = (\Gamma'_{l\conc(1)}\vdash f:B,\vec{C}\to N, \Gamma'_l\vdash f(x^B):\vec{C}\to N)$
  as an instance of the rule $\apvar$. This satisfies (a). 
  In order to show (c), 
  take an index $k$ for $\Gamma_l\vdash f(x^B):\vec{C}\to N$, which is the one for
  a named argument $y \in \FV(f(x^B))$ in $\Gamma_l$
  or for some unnamed argument of $\vec{C}$. 
  For the latter case, the indexes $\tilde{k}$, $k'$ and $\tilde{k'}$ can be taken
  as those of the unnamed argument in $\vec{C}$ at the same position as $k$.
  For the former case, we take $k'$ as the index for $y$ in $\Gamma'_l$.
  In order to take $\tilde{k'}$,
  we further have two subcases according to $\tilde{k}$: 
  The first one is $\tilde{k}$ is the index for the same named argument as $k$, 
  and the second one is $\tilde{k}$ is the index for the unnamed argument, namely $B=N$.
  In both subcases, we can take the equivalent $\tilde{k'}$ to $\tilde{k}$ as we wished.
    
  The case that $\phi(l)$ is an instance of the rule $\apnotvar$
  whose conclusion is $\Gamma_l\vdash f(b^B):A_l$. 
  Define $\Gamma'_{l\conc(1)} = \Gamma'_{l\conc(2)} = \Gamma'_l$.
  This satisfies (b) by the induction hypothesis.
  Then define $\phi'(l) = (S_1, S_2, \Gamma'_l\vdash f(b^B):\vec{C}\to N)$,
  where $S_1$ is $\Gamma'_{l}\vdash f:B,\vec{C}\to N$
  and $S_2$ is $\Gamma'_{l}\vdash b:B$, as an instance of $\apnotvar$.
  This satisfies (a).
  In order to show (c), 
  take an index $k$ for $\Gamma_l\vdash f(x^B):\vec{C}\to N$, which is the one for
  a named argument $y \in \FV(f(x^B))$ in $\Gamma_l$
  or for some unnamed argument of $\vec{C}$.
  For the former case, the successor $\tilde{k}$ of $k$ is uniquely taken.
  By $\FV(f(b)) \subseteq \FV(\Gamma'_l)$, the index $k'$ equivalent to $k$ is uniquely taken. 
  Then the successor $\tilde{k'}$ of $k'$ is also taken as we wished. 
  For the latter case, the successor $\tilde{k}$ of $k$ is uniquely taken
  as the one for unnamed argument at the same position as $k$. 
  Then $k'$ equivalent to $k$ is uniquely taken, and 
  its successor $\tilde{k'}$ is also taken as we wished. 

  The case that $\phi(l)$ is an instance of the rule $0$
  whose conclusion is $\Gamma_l\vdash 0:N$. 
  Define $\phi'(l) = \Gamma'_l\vdash 0:N$ as an instance of the rule $0$.
  This satisfies the requirements (a), (b), and (c). 

  The case that $\phi(l)$ is an instance of the rule $\Succ$
  whose conclusion is $\Gamma_l\vdash \Succ(t_{l\conc(1)}):N$ with $A_l=N$.
  Define $\Gamma'_{l\conc(1)} = \Gamma'_l$, and 
  $\phi'(l) = (\Gamma'_l\vdash t_{l\conc(1)}:N, \Gamma'_l\vdash \Succ(t_{l\conc(1)}):N)$
  as an instance of $\Succ$. They satisfy (a) and (b). 
  The requirement (c) is also satisfied:
  Take an index $k$ for $\Gamma_l\vdash \Succ(t_{l\conc(1)}): N$,
  which is the one for a named argument $y \in \FV(\Succ(t_{l\conc(1)})$ in $\Gamma_l$.
  Then the successor $\tilde{k}$ of $k$ is uniquely taken.  
  By $\FV(t_{l\conc(1)}) \subseteq \FV(\Gamma'_l)$, the index $k'$ equivalent to $k$ is uniquely taken. 
  Then the successor $\tilde{k'}$ of $k'$ is also taken as we wished. 
  
  The case that $\phi(l)$ is an instance of the rule $\cond$
  whose conclusion is $\Gamma_l\vdash \cond(f,g):N,\vec{C}\to N$. 
  Define $\Gamma'_{l\conc(1)} = \Gamma'_{l\conc(2)} = \Gamma'_l$, and 
  $\phi'(l) = (S_1,S_2, \Gamma'_l\vdash \cond(f,g):N,\vec{C}\to N)$,
  where $S_1$ is $\Gamma'_l\vdash f:\vec{C}\to N$ and $S_2$ is $\Gamma'_l \vdash g:N,\vec{C}\to N$, 
  as an instance of $\cond$. They satisfy (a) and (b). 
  To show (c), take an index $k$ for $\Gamma_l\vdash \cond(f,g): N,\vec{C}\to N$ as required. 
  We need to consider three cases:
  $k$ is the index for a named argument $y \in \FV(\cond(f,g))$ in $\Gamma_l$,
  is the one for an unnamed argument in $\vec{C}$, or
  is the one for an unnamed argument in $N$ (the first one of $N,\vec{C}\to N$). 
  For the first and second cases, we can take $\tilde{k}$, $k'$, and $\tilde{k'}$
  similar to the other cases.
  For the last case, the successor $\tilde{k}$ of $k$ should be taken
  as the index of the unnamed $N$-argument of $\Gamma'_l \vdash g:N,\vec{C}\to N$
  at the same position as $k$. 
  Then $k'$ and $\tilde{k'}$ are taken as those equivalent to $k$ and $\tilde{k}$, respectively.
  Note that this $k$ is an index of a progressing argument $N$, and so is $k'$. 
\end{proof}



\begin{lemma}[Substitution lemma]
  Assume that $\Pi_u: \Delta \vdash u:A$ and $\Pi_t:\Gamma,x:A \vdash t:B$ hold. 
  Then there exists $\Pi^\circ$ such that $\Pi^\circ:\mergeCtx{\Delta}{\Gamma} \vdash t[u/x]:B$. 
  Moreover, if both $\Pi_u$ and $\Pi_t$ satisfy the global trace condition,
  then $\Pi^\circ$ also satisfies it. 
\end{lemma}
\begin{proof}
  Assume that $\Pi_u: \Delta \vdash u:A$ and $\Pi_t:\Gamma,x:A \vdash t:B$ hold. 
  By Lemma~\ref{lem:thinning}, without loss of generality,
  we can assume $\FV(\Delta) = \FV(u)$ and $\Pi_t$ is a proof with the restricted $\lambda$ rule (the $\lambda'$ rule). 
  Let $\Pi_u=(T_u,\phi_u)$ and $\Pi_t=(T_t,\phi_t)$.
  For each $l\in T_t$, we write $\Gamma_l\vdash t_l:C_l$
  for the conclusion of $\phi_t(l)$.
  We use $\xx$ to mark occurences of $x$ in $\Pi_t$ that connects
  with the explicit $\xx$ of $\Gamma,\xx:A\vdash t:B$. 
  In the following, we construct a proof $\Pi^\circ = (T^\circ,\phi^\circ)$ of
  $\mergeCtx{\Delta}{\Gamma}\vdash t[u/x]:B$ such that 
  \begin{itemize}
  \item[(a1)]
    $T^\circ = T_t \cup T_{\var} \cup T_{\apvar}$, where
    $T_{\var} = \{l\conc(1)\conc l' \mid \text{$l\in T_t$, $\phi(l)=\var$ of $\xx$, and $l'\in T_u$} \}$
    and
    $T_{\apvar} = \{l\conc(2)\conc l' \mid \text{$u$ is not a variable, $l\in T_t$, $\phi(l)=\apvar$ of $f(\xx)$ for some $f$, and $l'\in T_u$}\}$.
    The explicit $1$ of $l\conc(1)\conc l' \in T_\var$ is called the switching point of $l\conc(1)\conc l'$.
    The explicit $2$ of $l\conc(2)\conc l' \in T_\apvar$ is also called the switching point of $l\conc(2)\conc l'$.     \item[(a2)]
    For each $l\conc(i)\conc l' \in T_\var \cup T_\apvar$ with switching point $i$,
    we have $\phi^\circ(l\conc(i)\conc l') = \phi_u(l')$. 
  \end{itemize}
  Moreover, for all $l \in T_t$,
  the rule instance $\phi^\circ(l)$ satisfies the following requirements
  with an auxiliary function $\sigma^\circ:T_t \to \Seq$ and a substitution $\theta_l$: 
  \begin{itemize}
  \item[(b1)]
    The sequent $\sigma^\circ(l)$ has the form $\Gamma^\circ_l\vdash t_l[\theta_l]:C_l$.
    The substitution $\theta_l$ is $\{u/\xx\}\cup\theta_{{\rm ren}}$
    if $\xx:A \in \Gamma_l$, and is $\theta_{{\rm ren}}$ otherwise, 
    where $\theta_{{\rm ren}}$ is some renaming.
    $\Gamma^\circ_l \sim \Delta_l\sharp\Restrict{(\Gamma_l)}{\overline{\xx}}[\theta_l]$ holds,
    where $\Delta_l$ is $\Delta$ if $\xx:A\in\Gamma_l$, and is $\emptyset$ otherwise. 
  \item[(b2)]
    $\sigma^\circ(l)$ is the conclusion of $\phi^\circ(l)$,
    and $\sigma^\circ(l\conc(i))$ is the $i$-th assumption of $\phi^\circ(l)$ if $l\conc(i) \in T_t$.
  \item[(b3)]
    Assume $\tilde{k},t_{l\conc(i)}$ is a successor of $k,t_l$ and $k$ is not a named index for $\xx$. 
    Then there exist $\tilde{k^\circ}$ and $k^\circ$ such that
    $\tilde{k^\circ},t_{l\conc(i)}[\theta_{l\conc(i)}]$ is a successor of $k^\circ,t_l[\theta_l]$ and 
    $(k,t_l) \simIndex{\Restrict{(\theta_l)}{\overline{\xx}}} (k^\circ,t_l[\theta_l])$.
    If $k$ is an index of a progressing argument, then so is $k^\circ$.    
  \end{itemize}
  
  Note that if we have $\Pi^\circ$ that satisfies the requirements, its possibly infinite path
  is a path of $\Pi_t$ or a path of $\Pi_u$ or a path $l\conc(i)\conc l'$,
  where $l$ is a path of $T_t$, $i$ is a switching point, and $l'$ is a path of $T_u$.
  Hence $\Pi^\circ$ is Almost-left-finite, since $\Pi_t$ and $\Pi_u$ are Almost-left-finite.
  In addition, if $\Pi_t$ and $\Pi_u$ satisfies the global trace condition,
  so is $\Pi^\circ$ by (b3). 
  Therefore, to complete the proof, it is enough to construct $\Pi^\circ$. 

  First, for each $l\conc(j)\conc l'\in T_\var\cup T_\apvar$ with a switching point $j$, 
  we define $\phi^\circ(l\conc(j)\conc l') = \phi_u(l')$. 

  In the following, for each $l\in T_t$, we inductively define $\Gamma^\circ_l$, $\phi^\circ(l)$, and $\sigma(l)$
  that satisty (b1), (b2), and (b3).
  The case of $l=\nil$, define $\Gamma^\circ_\nil = \Delta\sharp\Gamma$ and $\theta_\nil=\{u/\xx\}$,
  and define $\sigma^\circ(\nil)$ as $\mergeCtx{\Delta}{\Gamma}\vdash t[u/\xx]:A$. 
  Hence we have (b1) since
  $\Gamma^\circ_\nil = \mergeCtx{\Delta}{\Gamma} \sim \mergeCtx{\Delta}{\Restrict{(\Gamma,\xx:A)}{\overline{\xx}}[\theta_\nil]}$. 

  For each $l\in T_t$,
  with the induction hypothesis that $\sigma^\circ(l)$ and $\theta_l$ that satisfy (b1) are already defined, 
  we define $\phi^\circ(l)$, and define $\theta_{l\conc(i)}$ and $\sigma^\circ(l\conc(i))$ when $l\conc(i)\in T_t$, 
  such that they satisfy (b1), (b2), and (b3). 
  We perform this by the case analysis of $\phi(l)$.

  The case $\phi(l)= (\Gamma_l\vdash \xx:A)$ that is an instance of $\var$.
  We can define $\phi^\circ(l) = (\Delta\vdash u:A, \Gamma^\circ_l\vdash u:A)$ as $\weak$,
  since $\Delta\subseteqsim \mergeCtx{\Delta}{\Restrict{(\Gamma_l)}{\overline{\xx}}[\theta_l]} \sim \Gamma^\circ_l$
  holds by (b1). 
  As $l\conc(i)\not\in T_t$, (b1), (b2), and (b3) trivially hold.
  
  The case $\phi(l) = (\Gamma_l\vdash y:B)$ that is an instance of $\var$ with $y \neq \xx$. 
  We can define $\phi^\circ(l) = (\Gamma^\circ_l\vdash \theta_l(y):B)$ as $\var$,
  since $\theta_l(y):B \in \Restrict{(\Gamma_l)}{\overline{\xx}}[\theta_l] \subseteqsim \mergeCtx{\Delta_l}{\Restrict{(\Gamma_l)}{\overline{\xx}}[\theta_l]} \sim \Gamma^\circ_l$ holds by (b1).
  As $l\conc(i)\not\in T_t$, (b1), (b2), and (b3) trivially hold.

  The case $\phi(l) = (S_1,S_2,\Gamma_l\vdash f(b):C)$ that is an instance of $\apnotvar$,
  where $S_1 = \Gamma_l\vdash f:B\to C$, $S_2 = \Gamma_l\vdash b:B$, and $b$ is not a variable. 
  Define $\phi^\circ(l) = (S'_1, S'_2,\Gamma^\circ_l\vdash f[\theta_l](b[\theta_l]):C)$ as $\apnotvar$, 
  where $S'_1 = \Gamma^\circ_l\vdash f[\theta_l]:B\to C$ and $S'_2 = \Gamma^\circ_l\vdash b[\theta_l]:B$,
  since $b[\theta_l]$ is not a variable.
  Also define $\sigma^\circ(l\conc(1)) = S'_1$, $\sigma^\circ(l\conc(2)) = S'_2$,
  and $\theta_{l\conc(1)} = \theta_{l\conc(2)} = \theta_l$. 
  Then (b1) and (b2) trivially hold.
  To check (b3),
  assume that $\tilde{k},t$ is a successor of $k,f(b)$, where $t$ is $f$ or $b$,
  and $k$ is not a named index for $\xx$.
  Then (i) both $k$ and $\tilde{k}$ are named indexes for some variable $y^D\in\Dom(\Gamma_l)$,
  (ii) both $k$ and $\tilde{k}$ are unnamed indexes in $C$. 
  For the case (ii), take $k'$ and $\tilde{k'}$ as the same indexes as $k$ and $\tilde{k}$, respectively.
  For the case (i), we have
  $\theta_l(y):D \in \Restrict{(\Gamma_l)}{\overline{\xx}}[\theta_l] \subseteqsim \Gamma^\circ_l \subseteqsim \Gamma^\circ_{l\conc(1)}$ 
  by $y \neq \xx$. Then take $k'$ and $\tilde{k'}$ as named indexes for $\theta_l(y)$.
  In each case, $k'$ and $\tilde{k'}$ satisfy (b3) as expected. 
  
  The case $\phi(l) = (S_1,\Gamma_l\vdash f(\xx):C)$ that is an instance of $\apvar$, 
  where $S_1 = \Gamma_l \vdash f:A\to C$ and $\xx:A\in\Gamma_l$. 
  We need to consider the two subcases whether $u$ is a variable or not.
  \begin{itemize}
  \item
    If $u$ is a variable $y^B$, 
    we can define $\phi^\circ(l) = (S'_1, \Gamma^\circ_l\vdash f[\theta_l](y):C)$,
    where $S'_1 = \Gamma^\circ_l\vdash f[\theta_l]:A\to C$ as $\apvar$,
    since
    $y:B \in \Delta \subseteqsim \mergeCtx{\Delta}{\Restrict{(\Gamma_l)}{\overline{\xx}}[\theta_l]} \sim \Gamma^\circ_l$
    holds by (b1). 
    Also define $\sigma^\circ(l\conc(1)) = S'_1$ and $\theta_{l\conc(1)} = \theta_l$. 
    Then (b1) and (b2) trivially hold. (b3) is checked in a similar way to the case $\apnotvar$.
  \item
    If $u$ is not a variable, 
    define $\phi^\circ(l) = (S'_1, S'_2, \Gamma^\circ_l\vdash f[\theta_l](u):C)$,
    where
    $S'_1 = \Gamma^\circ_l\vdash f[\theta_l]:A\to C$ and
    $S'_2 = \Gamma^\circ_l\vdash u:A$ as $\apnotvar$. 
    Also define $\sigma^\circ(l\conc(1)) = S'_1$, $\sigma^\circ(l\conc(2)) = S'_2$,
    and $\theta_{l\conc(1)} = \theta_{l\conc(2)} = \theta_l$. 
    Then (b1) and (b2) trivially hold. (b3) is checked in a similar way to the case of $\apnotvar$. 
  \end{itemize}

  The case $\phi(l) = (S_1,\Gamma_l\vdash f(y):C)$
  that is an instance of $\apvar$, where $S_1 = \Gamma_l \vdash f:B \to C$, $y:B\in\Gamma_l$ and $y\neq \xx$. 
  We can define $\phi^\circ(l) = (S'_1, \Gamma^\circ_l\vdash f[\theta_l](\theta_l(y)):C)$ as $\apvar$,
  where $S'_1 = \Gamma^\circ_l\vdash f[\theta_l]:B\to C$, 
  since
  $\theta_l(y):B \in \Restrict{(\Gamma_l)}{\overline{\xx}}[\theta_l] \subseteqsim \mergeCtx{\Delta_l}{\Restrict{(\Gamma_l)}{\overline{\xx}}[\theta_l]} \sim \Gamma^\circ_l$
  holds by (b1). 
  Also define $\sigma^\circ(l\conc(1)) = S'_1$ and $\theta_{l\conc(1)} = \theta_l$. 
  Then (b1) and (b2) trivially hold. (b3) is checked in a similar way to the case $\apnotvar$.
  
  The case $\phi(l) = (\Gamma_l, z:C\vdash b:B,\Gamma_l\vdash \lambda z.b:C\to B)$
  that is an instance of $\lambda'$. 
  By the assumption, we have $\theta_l$
  and $\sigma^\circ(l)=\Gamma^\circ\vdash (\lambda z.b)[\theta_l]:C\to B$ that satisfy (b1).
  Let $\theta_l=\{\vec{u}/\vec{x}\}$. 
  We consider two subcases.
  \begin{itemize}
  \item
    The first subcase is when $z\not\in\FV(\vec{u})$.
    We have $(\lambda z.b)[\theta_l] = \lambda z.(b[\Restrict{(\theta_l)}{\overline{z}}])$.
    Then define $\phi^\circ(l) = (\Gamma^\circ_l, z:C\vdash b[\Restrict{(\theta_l)}{\overline{z}}]:B,\Gamma^\circ_l\vdash \lambda z.(b[\Restrict{(\theta_l)}{\overline{z}}]):C\to B)$ as $\lambda$. 
    Note that $\Gamma^\circ_l,z:C$ is a context 
    because $z:C \not\in \mergeCtx{\Delta_l}{\Restrict{(\Gamma_l)}{\overline{\xx}}[\theta_l]} \sim \Gamma^\circ_l$
    holds by (b1), $z\not\in \FV(\Gamma_l)$ and $z \not\in\FV(\vec{u}) \supseteq \FV(\Delta_l)$
    (recall that $\Delta_l$ is $\Delta$ when $\xx:A\in\Gamma_l$, and is $\emptyset$ otherwise).
    Define $\sigma^\circ(l\conc(1)) = \Gamma^\circ_l,z:C\vdash b[\Restrict{(\theta_l)}{\overline{z}}]:B$
    and $\theta_{l\conc(1)} = \Restrict{(\theta_l)}{\overline{z}}$.
    We have (b2) by the definition. 
    We also have (b1) by $(\Gamma^\circ_l,z:C) \sim (\mergeCtx{\Delta_l}{\Restrict{(\Gamma_l)}{\overline{\xx}}}[\theta_l],z:C) = \mergeCtx{\Delta_l}{\Restrict{(\Gamma_l,z:C)}{\overline{\xx}}}[\theta_{l\conc(1)}]$.
    To check (b3),
    assume that $\tilde{k},b$ is a successor of $k,\lambda z.b$ and $k$ is not a named index for $\xx$.
    Then (i) both $k$ and $\tilde{k}$ are named indexes for some variable $y^D\in\Dom(\Gamma_l)$,
    (ii) both $k$ and $\tilde{k}$ are unnamed indexes (not for $C$), or
    (iii) $k$ is an unnamed indexes for $C$ and $\tilde{k}$ is a named index for $z$.
    The cases (i) and (ii) can be checked in a similar way to the case of $\apnotvar$.
    The case (iii) is checked by taking $k'$ and $\tilde{k'}$ as the same indexes as $k$ and $\tilde{k}$,
    respectively.
  \item
    The second subcase is when $z\in\FV(\vec{u})$.
    We have $(\lambda z.b)[\theta_l] = \lambda z'.(b[\Restrict{(\theta_l)}{\overline{z}},z'/z])$, 
    where $z' \not\in\FV(b,\vec{u})$. 
    Then define $\phi^\circ(l) = (\Gamma^\circ_l, z':C\vdash b[\Restrict{(\theta_l)}{\overline{z}},z'/z]:B,\Gamma^\circ_l\vdash \lambda z'.(b[\Restrict{(\theta_l)}{\overline{z}},z'/z]):C\to B)$ as $\lambda$. 
    We need to check that $\Gamma^\circ_l,z':C$ is a context.
    It is shown by
    $z':C \not\in \mergeCtx{\Delta_l}{\Restrict{(\Gamma_l)}{\overline{\xx}}[\theta_l]} \sim \Gamma^\circ_l$
    using (b1), 
    $z'\not\in \FV(\lambda z.b) = \FV(\Gamma_l)$ and $z' \not\in\FV(\vec{u}) \supseteq \FV(\Delta_l)$. 
    Define $\sigma^\circ(l\conc(1)) = \Gamma^\circ_l,z':C\vdash b[\Restrict{(\theta_l)}{\overline{z}},z'/z]:B$
    and $\theta_{l\conc(1)} = \Restrict{(\theta_l)}{\overline{z}}\cup\{z'/z\}$.
    We have (b2) by the definition. 
    We also have (b1) by $(\Gamma^\circ_l,z':C) \sim (\mergeCtx{\Delta_l}{\Restrict{(\Gamma_l)}{\overline{\xx}}}[\theta_l],z':C) = \mergeCtx{\Delta_l}{\Restrict{(\Gamma_l,z:C)}{\overline{\xx}}}[\theta_{l\conc(1)}]$.
    Checking (b3) is done in a similar way to that of the first subcase.
  \end{itemize}

  The case $\phi(l) = (S_1,S_2,\Gamma_l\vdash \cond(f,g):\N\to C)$, where
  $S_1 = \Gamma_l\vdash f:C$, $S_2 = \Gamma_l\vdash g:\N\to C$, as an instance of $\cond$. 
  Define $\phi^\circ(l) = (S'_1,S'_2,\Gamma^\circ_l\vdash \cond(f[\theta_l],g[\theta_l]):\N\to C)$ as $\cond$,
  where $S'_1 = \Gamma^\circ_l \vdash f[\theta_l]:C$ and $S'_2 = \Gamma^\circ_l \vdash g[\theta_l]:\N\to C$.
  Also define $\sigma^\circ(l\conc(1)) = S'_1$, $\sigma^\circ(l\conc(2))  = S'_2$,
  and $\theta_{l\conc(1)} = \theta_{l\conc(2)} = \theta_l$. 
  Then (b1) and (b2) trivially hold.
  (b3) is checked in a similar way to the case of $\apnotvar$.

  The case $\phi(l) = (\Gamma_l\vdash 0:\N)$, as an instance of $0$. 
  Define $\phi^\circ(l) = (\Gamma^\circ_l\vdash 0:\N)$ as $0$. 
  Since $l\conc(i)\not\in T_t$, (b1), (b2), and (b3) trivially hold.
  
  The case $\phi(l) = (\Gamma_l\vdash t:\N, \Gamma_l\vdash \Succ(t):\N)$, as an instance of $\Succ$. 
  Define
  $\phi^\circ(l) = (\Gamma^\circ_l\vdash t[\theta_l]:\N, \Gamma^\circ_l\vdash \Succ(t[\theta_l]):\N)$ as $\Succ$. 
  Also define $\sigma^\circ(l\conc(1)) = \Gamma^\circ_l\vdash t:\N$
  and $\theta_{l\conc(1)} = \theta_l$.
  Then (b1) and (b2) trivially hold.
  (b3) is checked in a similar way to the case of $\apnotvar$.

  Therefore our construction of $\Pi^\circ$ is completed.
\end{proof}

\begin{lemma}\label{lem:inversion}
  \begin{enumerate}
  \item\label{lem:inversion1}
    If $\Pi:\Gamma\vdash f(a):A$ and $\Pi$ satisfies the global trace condition, then
    there exist $\Pi_1$, $\Pi_2$ and $B$ such that
    $\Pi_1:\Gamma\vdash f:B\to A$, $\Pi_2:\Gamma\vdash a:B$,
    and both $\Pi_1$ and $\Pi_2$ satisfy the global trace condition. 
  \item\label{lem:inversion2}
    If $\Pi:\Gamma\vdash \lambda x^T.b:A$, where $x\not\in\FV(\Gamma)$,
    and $\Pi$ satisfies the global trace condition, then
    there exist $\Pi_1$ and $B$ such that
    $\Pi_1:\Gamma,x:T\vdash b:B$ and $A = T\to B$,
    and $\Pi_1$ satisfies the global trace condition. 
  \item\label{lem:inversion3}
    If $\Pi:\Gamma\vdash \cond(f,g):A$ and $\Pi$ satisfies the global trace condition, then
    there exist $\Pi_1$, $\Pi_2$ and $B$ such that
    $\Pi_1:\Gamma \vdash f:B$, $\Pi_2:\Gamma \vdash g:N\to B$, $A = N\to B$,
    and both $\Pi_1$ and $\Pi_2$ satisfy the global trace condition. 
  \item\label{lem:inversion4}
    If $\Pi:\Gamma\vdash \Succ(t):A$ and $\Succ(t) \in \GTC$, then    
    there exists $\Pi_1$ such that $\Pi_1:\Gamma \vdash t:N$, $A=N$, 
    and $\Pi_1$ satisfies the global trace condition. 
  \end{enumerate}
\end{lemma}
\begin{proof}
  We first claim that if $t\in\GTC$ and $\Pi:\Gamma\vdash t:A$ with $\Pi=(T,\phi)$, 
  then there exists $l\in T$ such that $\phi(l) \neq \weak$ and $\phi(m) = \weak$ for all $m < l$.
  Because, if not, the only infinite path in $\Pi$ is the consective use of $\weak$,
  which does not contain progressing trace, this contradicts with $t\in\GTC$.

  We show the point \ref{lem:inversion1}.
  Assume that $\Pi:\Gamma\vdash f(a):A$ with $\Pi=(T,\phi)$ and $f(a) \in \GTC$.
  By the claim, take $l\in T$ such that $\phi(l) \neq \weak$ and $\phi(m) = \weak$ for all $m < l$.
  Let $\Gamma' \vdash f(a):A$ be the conclusion of $\phi(l)$. Then $\Gamma'\subseteqsim \Gamma$ holds
  by the transitivity of $\subseteqsim$.
  Now $\phi(l)$ is $\apvar$ or $\apnotvar$.
  In the former case $\phi(l) = \apvar$, we have $\Pi\restr l\conc(1):\Gamma'\vdash f:B\to A$ for some $B$,
  and $a = x^B \in \Gamma'$. Hence we have a proof $\Pi_1:\Gamma\vdash f:B\to A$ by $\weak$. 
  We also obtain $\Pi_2:\Gamma\vdash a:B$ by $a = x^B \in \Gamma'$ and $\weak$. 
  In the latter case $\phi(l) = \apnotvar$,
  we have $\Pi\restr l\conc(1):\Gamma'\vdash f:B\to A$ and $\Pi\restr l\conc(2):\Gamma'\vdash a:B$ for some $B$. 
  Hence we have a proof $\Pi_1:\Gamma\vdash f:B\to A$ and $\Pi_2:\Gamma\vdash a:B$ by $\weak$.
  In both cases, if $t\in\GTC$, we have $f\in\GTC$ and $a\in\GTC$ by the construction of $\Pi_1$ and $\Pi_2$. 
  
  The points \ref{lem:inversion2}, \ref{lem:inversion3}, and \ref{lem:inversion4} are shown similarly. 
\end{proof}


\begin{theorem}[Subject reduction]
  Assume that $\Pi_t:\Gamma\vdash t:A$, $\Pi_t$ satisfies the global trace condition, and $t\reduces u$.
  Then there exists $\Pi_u$ such that $\Pi_u:\Gamma\vdash u:A$ and $\Pi_u$ satisfies the global trace condition. 
\end{theorem}
\begin{proof}
  By the definition of $t\reduces u$, there is a context $\hat{t}[-]$ such that
  $t=\hat{t}[t_0]$, $u=\hat{t}[u_0]$, and $t_0\reduces_\Box u_0$, where $\Box\in\{\beta,\cond\}$. 
  Since $t_0$ is a subterm of $t$, there is $l$ such that $\Pi_t\restr l: \Gamma_0\vdash t_0:A_0$.
  Note that $\Pi_t\restr l$ satisfies the global trace condition,
  since it is a subtree of $\Pi_t$, which satisfies the global trace condition. 
  Then, if we have $\Pi'_u:\Gamma_0\vdash u_0:A_0$ that satisfies the global trace condition,
  the tree obtained from $\Pi_t$ by replacing the subtree $\Pi_t\restr l$ by $\Pi'_u$ is also
  a proof of $\Gamma\vdash \hat{t}[u_0]:A$ that satisfies the global trace condition. 
  Hence it is enough to show the following:
  \begin{itemize}
  \item[(a)]
    If $\Pi:\Gamma \vdash (\lambda x^A.b)(a): B$ and it satisfies the global trace condition,
    then $\Pi':\Gamma \vdash b[a/x]: B$ for some $\Pi'$ that satisfies the global trace condition.
  \item[(b)]
    If $\Pi:\Gamma \vdash \cond(f,g)(0): B$ and it satisfies the global trace condition,
    then $\Pi':\Gamma \vdash f: B$ for some $\Pi'$ that satisfies the global trace condition. 
  \item[(c)]
    If $\Pi:\Gamma \vdash \cond(f,g)(\Succ(t)): B$ and it satisfies the global trace condition,
    then $\Pi':\Gamma \vdash g(t): B$ for some $\Pi'$ that satisfies the global trace condition. 
  \end{itemize}
  (b) is shown immediately by Lemma~\ref{lem:inversion}~\ref{lem:inversion3}.
  We show (c).
  Assume $\Pi:\Gamma \vdash \cond(f,g)(\Succ(t)): B$.
  Then by \ref{lem:inversion1}, \ref{lem:inversion3}, and \ref{lem:inversion4} of Lemma~\ref{lem:inversion},
  we have
  $\Pi_1:\Gamma \vdash g:N\to B$ and $\Pi_2:\Gamma \vdash t:N$,
  where $\Pi_1$ and $\Pi_2$ satisfy the global trace condition. 
  Hence, by applying $\apnotvar$ or $\apvar$ to $\Pi_1$ and $\Pi_2$, 
  we have a proof $\Pi':\Gamma\vdash g(t):B$ that satisfies the global trace condition. 
  In order to show (a), assume $\Pi:\Gamma \vdash (\lambda x^A.b)(a): B$.
  Then by Lemma~\ref{lem:inversion}~\ref{lem:inversion1},
  we have $\Pi_1:\Gamma \vdash \lambda x^A.b:A\to B$ and $\Pi_2:\Gamma \vdash a:A$,
  where $\Pi_1$ and $\Pi_2$ satisty the global trace condition. 
  Then, by Lemma~\ref{lem:thinning},
  we have $\Pi'_1:\Restrict{\Gamma}{\FV(\lambda x.b)} \vdash \lambda x^A.b:A\to B$,
  where $\Pi'$ satisfies the global trace condition. 
  By Lemma~\ref{lem:inversion}~\ref{lem:inversion2},
  we have a proof $\Pi''_1:\Restrict{\Gamma}{\FV(\lambda x.b)},x:A \vdash b:B$
  that satisfies the global trace condition. 
  Hence, by the substitution lemma, we have a proof $\Pi':\Gamma\vdash b[a/x]:B$
  that satisfies the global trace condition, as we wished. 
\end{proof}

Using the subject reduction theorem, variable renaming is shown to be admissible in our system. 

\begin{proposition}\label{prop:renaming}
  \begin{enumerate}
  \item\label{prop:renaming1}
    Let $\theta$ be a renaming.
    If $\Pi:\Gamma\vdash t:A$ and $\Gamma[\theta]$ is a context, 
    then $\Pi':\Gamma[\theta]\vdash t[\theta]:A$ for some $\Pi'$.
    Moreover, if $\Pi$ satisfies the global trace condition, so is $\Pi'$.
  \item\label{prop:renaming2}
    If $\Pi:\Gamma\vdash t:A$ and $t'$ is an $\alpha$-equivalent term of $t$,
    then $\Pi':\Gamma\vdash t':A$ for some $\Pi'$.
    Moreover, if $\Pi$ satisfies the global trace condition, so is $\Pi'$.
  \end{enumerate}
\end{proposition}


\begin{proof}
  We show the point \ref{prop:renaming1}. 
  It is enough to show the claim for a single renaming $\theta = \{y'/y\}$.
  Then, by the assumption, we have a proof $\Pi_1$ of $\Gamma[y'/y] \vdash (\lambda y^B.t)y':A$
  such that $\Pi_1$ satisfies the global trace condition if $\Pi$ satisfies it.  
  Hence we have $\Pi': \Gamma[y'/y] \vdash t[y'/y]:A$ by the subject reduction theorem
  such that $\Pi'$ satisfies the global trace condition if $\Pi$ satisfies it.

  Next we show the point \ref{prop:renaming2}.
  It is enough to show when $t = \lambda x.b$ and $t' = \lambda x'.(b[x'/x])$, where $x'\not\in\FV(\lambda x.b)$. 
  Let $\Pi$ be a proof of $\Gamma\vdash \lambda x.b:A\to B$.
  Then we have $\Pi_1: \Restrict{\Gamma}{\FV(\lambda x.b)} \vdash \lambda x.b:A\to B$
  by Lemma~\ref{lem:thinning}, and
  also have $\Pi_2: \Restrict{\Gamma}{\FV(\lambda x.b)},x':B \vdash b[x'/x]:B$
  by the subject reduction theorem. 
  Hence we have $\Pi': \Gamma \vdash \lambda x'.b[x'/x]:A\to B$ by the rules $\lambda$ and $\weak$. 
  Note that $\Pi'$ satisfies the global trace condition if $\Pi$ satisfies it. 
\end{proof}

