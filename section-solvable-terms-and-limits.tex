\newcommand\Lim[1]{{\tt Lim}(#1)}
\newcommand{\NFZ}{{\tt NF}_0}

\section{Solvable terms and Limits}

In this section we formally define the notion of limit of a reduction on a cyclic 
$\lambda$-term and we prove its main properties. The notion is akin to the notion
of B\"{o}hm tree. However, there is some important difference with B\"{o}hm trees,
due to the fact that we are modeling equality of functions and functionals over $\N$. 
For this reason the maximal consistent equational theory of limits not equating two 
unsolvable has a higher arithmetical complexity than in the case of B\"{o}hm trees.


\subsection{Solvable Terms of LAMBDA}
      Let us define a main path as the unique path moving left except in $\cond(a,b)(t)$, where it 
moves right. We define a head redex as the first redex in the main path. We say that t is in head 
normal form if the main path is finite and with no head redex. t is solvable if some redex of t is in 
head normal form. t is unsolvable if t is not solvable. The main path is non-progressing,
because in $\cond(a,b)$ it moves to $a$. Thus, if a term satisfies GTC then its main path is finite,
and the term is head normal if and only if it has no head redex.
 
      Solvability is related to the notion of limit. The limit of $t$ is bottom (undefined) if $t$ is unsolvable, otherwise the limit is obtained by reducing $t$ to some head normal form, taking the first symbol of the head normal form and then taking the limit of all immediate subterms, cyclically. Limit is not unique because we can reduce infinitely many times a term while avoiding to contract one particular redex. However, as remarked by Klop, at least one limit is normal. Infinite Church-rosser implies that the normal limit is unique. 
      Having the same normal limit can be expressed by your maximal bisimulation (see later
      in this section).
 
      Without GTC, the limit can be bottom, or can include some bottom subterms, as it happens for B\"{o}hm trees in lambda calculus.  In our paper, we can prove that all limit of all terms with GTC include no bottom. This is a corollary of our normalization result: head-reduction is 0-safe and therefore it always terminates in a head-normal form. GTC is preserved by reductions and subterms. If we recursively head-normalize immediate subterms of a head-normal form we always reduce terms with GTC, therefore we always obtain solvable subterms.
 
 
 
\subsection{A co-inductive Definition of Equality between normal limits of terms}

\vspace{10pt}

\begin{remark}
  We assume the following (consequence of limit Church-Rosser):
  \begin{center}
    Any $t\in \GTC$ has a unique normal form $\Lim{t}$ (limit of safe-reductions).
  \end{center}
\end{remark}

\noindent{\bf Purpose of this sub-section}\quad
We want to characterize the limit equivalence of terms in $\GTC$, 
which means two terms have the same limit: $\Lim{t} = \Lim{u}$. 
More formally, we try to coinductively define a binary relation $\approx$ that satisfies:
\[
t \approx u \quad \Longleftrightarrow \quad \Lim{t} = \Lim{u}. 
\]

A {\em 1-context} is a finite context with several number of distinguished holes $[-]$.

\begin{definition}[0-context]
  \begin{center}
    $C ::= x \mid \lambda x.C \mid CC \mid 0 \mid \Suc{C} \mid \Cond{C}{[-]}$
  \end{center}
\end{definition}

We write $C[-,-,-]$ if $C$ has three holes, and write $C[t,u,v]$ for the resulting term
obtained by filling terms $t,u,v$ for the holes. 

A term can be divided into the 0-context part and the terms part filling the holes of the 0-context,
namely the following lemma holds. 
\begin{lemma}
  For any $t\in \LAMBDA$, there uniquely exists $(C,\vec{u})$ such that $t = C[\vec{u}]$. 
\end{lemma}

For a binary relation $R$ on terms,
we write $(t_1,\ldots,t_n)R(u_1,\ldots,u_n)$ if $t_iRu_i$ holds for all $i$. 

Let $\NFZ$ be the set of $0$-safe normal forms. 
Note that if both $C_1[\vec{t_1}]$ and $C_2[\vec{t_2}]$ are $0$-safe normal forms of $t$, then $C_1=C_2$.
  
\begin{definition}[0-bisimulation]\rm
  A binary relation $R$ on $\LAMBDA$ is said to be a {\em 0-bisimulation} if it satisfies the following:
  \begin{center}
    $tRu$ and $t \nsafeReducesAst{0} C_t[\vec{t}] \in \NFZ$ and $u \nsafeReducesAst{0} C_u[\vec{u}] \in \NFZ$
    \\
    \quad
    implies
    \quad
    $C_t = C_u$ and $\vec{t}R\vec{u}$. 
  \end{center}
  
  We define $\approx$ by the largest $0$-bisimulation.
\end{definition}

\begin{lemma}
  $\approx$ is an equivalence relation.
\end{lemma}
\begin{proof}
  Reflexivity for $\approx$ holds by $=\,\subseteq\,\approx$, since $=$ is a $0$-bisimulation.
  Symmetricty holds by $\approx^{-1}\,\subseteq\,\approx$, since $\approx^{-1}$ is a $0$-bisimulation.
  Transitivity holds by $(\approx\circ\approx)\,\subseteq\,\approx$, since $\approx\circ\approx$ is a $0$-bisimulation.
  \hfill$\Box$
\end{proof}

Finally we show the relation $\approx$ characterizes the limit equivalence. 
\begin{proposition}
  $t \approx u$ and $\Lim{t} = \Lim{u}$ are equivalent. 
\end{proposition}
\begin{proof}
  $(\Leftarrow)$: Let $R$ be the limit equivalence relation, namely $tRu \Leftrightarrow \Lim{t}=\Lim{u}$. 
  Then $R$ is a $0$-bisimulation.
  Hence $\Lim{t} = \Lim{u}$ implies $t \approx u$ by $R \subseteq \approx$. 
  
  $(\Rightarrow)$:
  We show this direction by proof by contradiction.
  Define
  \[
  X = \{(t,u) \mid \text{$t \approx u$ and $\Lim{t} \neq \Lim{u}$}\,\}.
  \]
  Assume that $X \neq \emptyset$ (there is a counter example of the $(\Rightarrow)$-direction). 
  For $(t,u)\in X$, we define the difference depth $\sharp(t,u)$
  by $\min\{\,|\pi| \mid (\Lim{t})_{\pi} \neq (\Lim{u})_{\pi}\,\}$. 
  Fix a $(t_0,u_0) \in X$ that has the least difference depth.
  Take $0$-safe normal forms $C_t[\vec{t'}]$ and $C_u[\vec{u'}]$ of $t_0$ and $u_0$, respectively. 
  Then, by $t_0\approx u_0$, we have $C_t = C_u$ and $\vec{t'} \approx \vec{u'}$. 
  Also we have $C_t[\overrightarrow{\Lim{t'}}] = \Lim{t_0}$ and $C_u[\overrightarrow{\Lim{u'}}] = \Lim{u_0}$. 
  By $\Lim{t_0}\neq \Lim{u_0}$ and $C_t=C_u$,
  there is $j$ such that $\Lim{t'_j} \neq \Lim{u'_j}$. 
  Hence we have $(t'_j,u'_j) \in X$. 
  However we also have $\sharp(t'_j,u'_j) < \sharp(t_0,u_0)$,
  since, when we consider the path to $t'_j$ from the root of $t_0$,
  there must be at least one ${\tt cond}$ on the path.
  This contradicts the leastness of $(t_0,u_0)$.
  Therefore we have $X = \emptyset$, namely the $(\Rightarrow)$-direction holds.
  \hfill$\Box$
\end{proof}


\subsection{Comparing with B\"{o}hm trees}
We only consider equational theories equating all unsolvable. We consider all unsolvable 
denoting a unique \quotationMarks{undefined} value.
      
In $\lambda$-calculus two solvable lambda terms can be equated in a consistent 
theory equating all unsolvable if and only if have the same  
$\eta$-bohm-tree, that is, have the same limit normal $\beta\eta$-form. 
 
Instead, we found no simple way to characterize the terms
we may consistently equate in $\LAMBDA$, or in $\CTlambda$ by adding a reduction
like $\eta$. 
We include some example, showing that this problem is difficult.
    
\begin{enumerate}
\item
An example of terms of $\LAMBDA$ we can equate:
$$\lambda x.\Succ(x), 
\ \ \ 
\cond(1, \lambda \_. \cond(2, \lambda \_. \cond(3, \ldots))), 
\ \ \
\cond(\Succ(0),\lambda x.\Succ^2(x)) \  (z)
$$
All are notations for the successor map. 

\item
As an opposite example, let $I = \lambda x.x$ and 
$f = \cond(I(0),f)$ and $g = \cond(I(0),\cond(1,g))$. 
$f$ and $g$ are non-terminating terms of $\CTlambda$: they include infinitely many
disjoint redexes of the form $I(0)$. They are solvable terms with limits 
representing different map, which cannot be equated. 
The limit normal form of $f$ is $\cond(0,\cond(0, \cond(0, ...)))$: the map $f$ is always $0$.
The limit normal form of $g$ is $\cond(0,\cond(1, \cond(0, ...)))$: the map $g$ return $0$ or $1$
according if the argument is even or odd. We cannot equate $f=g$,
otherwise we obtain $0 = f(1) = g(1) = 1$, and from it the equality of all terms.
\end{enumerate}
 
We could not find finitely many computable reduction rules such that two terms have the same
limit normal form if and only if they represent the same map on $\N$.
 
  
